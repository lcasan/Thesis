\subthesischapter{Tecnologías y herramientas}
\subsubthesischapter{Motor gráfico(Unity3D)}
Actualmente podemos encontrar una amplia gama de motores de videojuegos, con diferentes tipos de licencias y orientados a cumplir distintos tipos de propósitos. Se pueden encontrar motores comerciales y gratuitos, con metodologías 2D o 3D, que incluso brindan soluciones de juegos a variadas plataformas (Windows, Linux, Android, etc). La Tabla~\ref{tab: graphics-engines} muestra detalles comparativos de los motores gráficos más usados del mercado.

Para el propósito de la tesis se considera Unity como unos de los motores más populares a la hora del desarrollo de videojuegos, para las distintas plataformas, una gran ventaja que tiene Unity a la hora de desarrollar un videojuego para Android, es su modo de configurar el entorno, porque solo se necesita descargar los módulos de Android, y ya está listo para crear juegos.

Principales características de este motor gráfico que son de utilidad en el sistema:~\cite{unity3d}
\begin{itemize}
    \item Programación con C\#.
    \item Soporte parcial de .NET (incluye soporte a .NET Sockets).
    \item Soporte de plugins para código nativo.
    \item Incluye el motor físico PhysX de Nvidia.
    \item Carga de modelados y texturas de varios formatos de programas externos como Blender, Maya, Adobe Suite, 3D Max, Cinema 4D, entre otros.
    \item Despliegue gratis sobre Android.
    \item Inspector para clases personalizadas en tiempo de ejecución.
\end{itemize}

\begin{table}[ht]
    \hspace*{-25pt}
    \centering
    \begin{tabularx}{1.1\textwidth}{|X|X|X|X|X|X|X|}
        \hline
        \textbf{Motor gráfico  \& Atributos} & \textbf{Unity} & \textbf{Unreal Engine} & \textbf{GameMaker Studio} & \textbf{Godot} & \textbf{CryEngine}\\\hline
        Orientación                 & 2D y 3D                                                       &2D y 3D                                                          & 2D                            & 2D y 3D               & 3D \\\hline
        Sistema Operativo           & Windows, Linux, MacOS                                           & Windows                                                         & Windows                       & Windows, Linux, MacOS               & Windows, Linux, MacOS\\\hline
        Plataformas                 & iOS, Android, Window, PS4, MacOS,  Linux, Xbox One y 360, HTML5 & iOS, Android, Window, PS4, MacOS,  Linux, Xbox One y 360, HTML5   & Windows, MacOS, Linux, HTML5, Android, iOS, PlayStation 4, Xbox One & Windows, MacOS, Linux    & Windows, Linux, Nintendo Switch, Xbox One, PlayStation 4, Xbox 360, PlayStation 3, Wii U\\\hline
        Lenguaje de Programación    & C\#                                                           & C++                                                             & Lenguaje propio GML           & Lenguaje scripting & Programación basada en nodos \\\hline
        Licencia                    & Gratuita                                                      & De pago                                                         & Versión gratuita limitada     & Gratuita              &  Gratuita en la versión 5\\\hline
        Documentación               & Muy amplia                                                    & Escasa                                                          & Amplia                        & Limitada              & Amplia\\\hline
        Curva de aprendizaje        & Sencilla                                                      & Complicada                                                      & Media                         & Media                 & Media \\\hline
    \end{tabularx}
    \caption{Comparación de motores gráficos en función de sus atributos \\ (Adaptado de~\cite{gonjar2019desarrollo})}
    \label{tab: graphics-engines}
\end{table}

\subsubthesischapter{Visual Studio Code}
Visual Studio Code es un editor de código fuente desarrollado por Microsoft para Windows, Linux, macOS y Web. Incluye soporte para la depuración, control integrado de Git, resaltado de sintaxis, finalización inteligente de código, fragmentos y refactorización de código, características que lo convierten en una herramienta perfecta para manipular los scripts en Unity \cite{vscode}.

\subsubthesischapter{NET Framework 4.7}
El framework .NET provee un modelo de programación global que permite el desarrollo de todo tipo de aplicaciones móviles, web o de escritorio. Técnicamente el framework .NET es un ambiente de ejecución que administra las aplicaciones que corren sobre este. En él se brinda un conjunto extensivo de bibliotecas para dar solución a las principales áreas del desarrollo de software.

El framework en sí consiste de dos grandes componentes: motor de ejecución CLR (Common Language Runtime, por sus siglas en inglés) el cual se encarga de la ejecución de las aplicaciones; y la biblioteca de clases (nombre en inglés .NET Framework Class Library)~\cite{netframework}

Características principales:
\begin{itemize}
    \item Manejo de memoria automático.
    \item Una biblioteca bien vasta para operaciones generales de bajo nivel.
    \item Operatividad entre lenguajes. O sea, se puede acceder a rutinas y bibliotecas (dinámicas y estáticas) escritas en otros lenguajes si fueron compiladas con compiladores para esos lenguajes que soporten el framework .NET.
    \item Gran número de lenguajes disponibles: Visual Basic, C\#, Visual F\#, C++, entre otros.
    \item Es posible crear compilados que funcionen en múltiples plataformas de la .NET.
\end{itemize}