\subthesischapter{Juegos Serios}
Los videojuegos interactivos han surgido como nuevos enfoques de tratamiento en la rehabilitación del accidente cerebrovascular. Estos enfoques pueden ser ventajosos ya que dan la oportunidad de practicar actividades que no se pueden realizar dentro del entorno clínico. Además, los programas de realidad virtual están diseñados para ser más interesantes y agradables que las terapias tradicionales, alentando al paciente para que realice un mayor número de repeticiones de los ejercicios~\cite{laver2018virtual,alfageme2002aprendiendo}.

En el área de la rehabilitación existen diferentes contribuciones dirigidas a la recuperación y rehabilitación de las diferentes habilidades y competencias psicomotrices. A continuación se presentan las propuestas más relevantes que existen actualmente usando juegos serio:
\begin{itemize}
    \item En el 2019, Adrián Rodríguez y cols~\cite{rodriguez2019design} presentaron un juego serio con propósito de rehabilitación a personas con ACV. El juego utiliza el software de desarrollo para entornos Unity 3D en su versión libre. El sensor que realiza el seguimiento de los movimientos de la persona es el sensor Kinect\footnote{Controlador de juegos desarrollado por Microsoft para la videoconsola Xbox 360.}. El objetivo del juego consiste en que el usuario agarre esferas de un color específico. Los puntajes alcanzados en el juego y los parámetros de configuración colocados por el kinesiólogo son almacenados junto con la información del paciente. De esta manera el fisioterapeuta puede recuperar los datos almacenados y realizar un análisis de la evaluación de la evolución de la persona en su rehabilitación. 
    
    \item Alberto y Edwin Daniel~\cite{morales2019desarrollo} proponen una aplicación de juego serio para tratar de aumentar y de evaluar el límite de estabilidad de la población envejecida. Tras esta idea, se hacen uso del sensor Kinect V1 y el motor de juego Unity3D, compatible mediante el uso de la librería \textit{Kinect with MS-SDK}, para poder registrar los movimientos del paciente durante la terapia. El juego basará su fundamento clínico en diversos test de evaluación del balance, como son el test Fugl-Meyer o el Choice Stepping Reaction Time, y tomará como mecánica de referencia la de uno de los videojuegos clásicos, el Tetris. Blocks Rehab, la adaptación clínica de este juego de los años 80, consistirá en tres modalidades de juego (Bloques, Puzzle y E.T. Tris), cumpliendo cada uno de ellos una función específica. El paciente deberá desplazarse de manera frontal, lateral y mantener el equilibrio sobre una pierna para llevar a cabo los objetivos que el videojuego les plantea, con el propósito de extraer métricas de salida que evalúen la actuación del usuario en términos de estabilidad. 
    
    \item En la revisión realizada por Gerdienke y cols~\cite{doi:10.1177/1545968314535985}, se comparó el efecto de realizar el entrenamiento con un soporte de brazo, combinado con ejercicios mediante un juego de realidad virtual y la rehabilitación convencional en pacientes con ictus.\\ El dispositivo que utiliza el soporte para el brazo se llama ArmeoBoom. Está compuesto de un sistema de suspensión en cabestrillo elevado, que proporciona un soporte para la muñeca. Este elemento proporciona un buen soporte en un espacio tridimensional que te permite realizar movimientos funcionales sin ninguna restricción. El mecanismo se adapta a la situación física de cada paciente. ArmeoBoom está compuesto por una webcam y un ordenador portátil que permiten al usuario interaccionar con los videojuegos incorporados en el ordenador portátil y jugar moviendo la extremidad afectada en un ambiente tridimensional ajustado al grado de movimiento funcional de cada paciente. Los movimientos, tanto en el plano horizontal como el plano vertical, son controlados y grabados con la finalidad de premiar con puntos dependiendo de la actuación del usuario y del tiempo de ejecución.
    
    Entrenar con el soporte de brazo consiste en realizar los movimientos con el brazo afectado, con el objetivo de maximizar la habilidad de los ejercicios usando el mínimo soporte de brazo posible.
    
    El entrenamiento de rehabilitación convencional consiste en la realización de unos ejercicios ejecutados con los brazos, dirigidos por el terapeuta, para reflejar la terapia física y ocupacional. El objetivo de este entrenamiento es que el paciente incremente el rango de movimiento del brazo, principalmente del hombro y el codo con el mínimo soporte posible de una superficie como una mesa. Todos los ejercicios son análogos en esencia y consisten en alcanzar objetos colocados en una mesa, en mover o apilar vasos, colocar discos o transportar bloques de piezas sin la ayuda de ningún ordenador ni soporte informático. Conforme el paciente va superando los juegos propuestos, los movimientos se van complicando un poco más.

    Un total de 70 pacientes que habían padecido un ictus hemorrágico o isquémico en las últimas 12 semanas y que estaban estables con su medicación fueron elegidos para la rehabilitación. Los participantes pasaron por un programa de entrenamiento de seis semanas, divididas en tres sesiones de 30 minutos cada una.
    
    Un grupo de 35 pacientes se formó para la rehabilitación de las extremidades superiores en combinación con el soporte de brazo y el ejercicio, mientras que 33 pacientes realizaron la rehabilitación con los ejercicios convencionales.
    
    \item En el 2014  autores como Gerdienke~\cite{10.3233/NRE-141105}, comparan el efecto que se produce al entrenar con un videojuego de realidad virtual Kinect Xbox 360 Paddle Panic Mini Game, en pacientes que han padecido un ictus. El objetivo es mejorar el movimiento de la extremidad superior, observando el hemisferio cerebral afectado de estos pacientes.

    La evaluación se lleva a cabo mediante la acción de beber un vaso de agua antes y después de entrenar. Se seleccionaron un total de 40 participantes para realizar el estudio. El grupo inicial se subdividió en cuatro grupos atendiendo a los siguientes criterios:

    \begin{itemize}
        \item Pacientes con afectación derecha del cerebro (hemiparesia izquierda).
        \item Pacientes con afectación izquierda del cerebro (hemiparesia derecha).
        \item Pacientes sanos que entrenaron con la extremidad superior derecha en el videojuego.
        \item Pacientes sanos que entrenaron con la extremidad superior izquierda en el videojuego.
    \end{itemize}

    Los resultados del estudio muestran que los pacientes con la afectación de la  superior por ictus presentaron una mejoría del entrenamiento con el videojuego, demostrado un incremento de la puntuación en cada jugada.
\end{itemize}

Las terapias con juegos serios son una alternativa complementaria a las terapias tradicionales. Tanto los estudios que se apoyan en tecnologías de realidad virtual, como Kinect y Nintendo, las cuales no tienen propósitos clínicos, como los que se centran en juegos serios diseñados específicamente para la rehabilitación física de diferentes patologías, indican que estos juegos tienen el potencial de mejorar el equilibrio, el control postural y la condición física de los pacientes, sin embargo los estudios son pocos y demasiado pequeños para llegar a una conclusión completamente fiable. La falta de eventos adversos reportados (como mareos, dolor de cabeza o náuseas) sugiere que este enfoque de la terapia es relativamente seguro, aunque esto puede variar dependiendo de las características de la persona, el hardware y el software de la realidad virtual y la tarea. 