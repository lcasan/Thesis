\subthesischapter{Metodología de trabajo}
Existen diferentes modelos y metodologías que han sido utilizados en los últimos años como herramientas de apoyo para el desarrollo del software. La interrogante principal está en conocer cuál modelo utilizar en el proceso de desarrollo de software en un proyecto \cite{DELGADOOLIVERA2021}.

Para el desarrollo de nuestra aplicación se hará uso del modelo RAD que permite la construcción de software basada en módulos, utilizando herramientas de software que permitan de forma ágil y efectiva realizar una aplicación con altos estándares de calidad en un corto período de tiempo. Esto se debe a que un sistema completamente integrado e interdependiente entre cada una de sus partes era muy difícil de adaptar a los distintos cambios de requisitos. Además, el diseño modular del mismo brinda la posibilidad de migrar el sistema a otras tecnologías distintas o más modernas, con relativa facilidad.

El sistema se desarrollará de manera iterativa, donde en cada iteración se realizará un acercamiento a una versión más definitiva de un componente o módulo en específico. De esta forma siempre habrá una sección del producto parcialmente terminada que podía ser mostrada, para adaptarla a las sugerencias y peticiones de los usuarios.