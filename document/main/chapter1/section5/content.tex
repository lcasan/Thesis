\subthesischapter{Metodología de trabajo}
Existen diferentes modelos y metodologías que han sido utilizados en los últimos años como herramientas de apoyo para el desarrollo del software. La interrogante principal está en conocer cuál modelo utilizar en el proceso de desarrollo de software en un proyecto \cite{DELGADOOLIVERA2021}.

Para el desarrollo de nuestra aplicación, se empleará el modelo de desarrollo rápido de aplicaciones (RAD, por sus siglas en inglés), una metodología que se caracteriza por su enfoque iterativo e incremental. Este modelo facilita la rápida prototipificación y retroalimentación continua, lo que resulta fundamental en entornos dinámicos donde los requisitos del sistema pueden evolucionar o aclararse a lo largo del proceso de desarrollo. La metodología RAD posibilita una mayor flexibilidad y adaptabilidad al cambio, permitiendo a los desarrolladores identificar y abordar prontamente posibles ajustes o mejoras en la funcionalidad del sistema. Este enfoque colaborativo y ágil contribuirá a la eficiencia en el desarrollo de la aplicación, al tiempo que asegurará la alineación con las expectativas y requisitos del usuario final.

La aplicación se desarrollará de manera iterativa, donde en cada iteración se realizará un acercamiento a una versión más definitiva de un componente o módulo en específico. De esta forma siempre habrá una sección del producto parcialmente terminada que podía ser mostrada, para adaptarla a las sugerencias y peticiones de los usuarios.