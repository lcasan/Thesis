\begin{introduction}
    Desde la perspectiva de la rehabilitación, el accidente cerebrovascular 
    es un gran generador de discapacidad, tanto física como cognitiva. De la población que sufre 
    accidentes cerebrovasculares (ACV), un 15 a 30 porciento resulta con un deterioro funcional severo a 
    largo plazo, lo que implica un alto grado de dependencia de terceros. Además, el ACV se ha
    establecido como la segunda causa de demencia a nivel mundial~\cite{moyano2010accidente} .Gracias 
    al advenimiento de nuevas terapias, la mortalidad por ACV ha disminuido notablemente en los últimos 
    años~\cite{cuadrado2009rehabilitacion,harold2007guidelines}, lo que deja un número cada vez más 
    alto de sobrevivientes con mayor probabilidad de recurrencia. Si a esto se suma una población 
    envejecida (por el aumento en la esperanza de vida), se genera un importante impacto sanitario: 
    más población con déficit funcional, quienes tienen más probabilidad de presentar complicaciones 
    asociadas y que éstas sean de mayor gravedad.

    \vspace{5pt}
    Las personas afectadas suelen presentar dificultades para realizar actividades de la vida
    diaria como caminar. Entre los problemas motores más frecuentes que se manifiestan en
    estas enfermedades se encuentran: rigidez muscular, espasticidad, hemiparesia,
    parálisis, dolor muscular crónico, lentitud de los movimientos y problemas de
    coordinación y equilibrio. Estos síntomas traen consigo alteraciones de la marcha y
    dificultades para la activación de los músculos y la movilidad de las articulaciones en los
    miembros inferiores y superiores. Además, a medida que se agravan aparecen
    complicaciones respiratorias y cardiovasculares ~\cite{barbosa2015application, miner2020therapeutic}, 
    limitando aún más la calidad de vida. Debido a ello, las investigaciones recientes se han enfocado en 
    terapias de rehabilitación que permitan no solo restaurar las capacidades motoras, sino también,
    disminuir los factores de riesgo prevalentes en estos pacientes ~\cite{barbosa2015application}. 
    Los ejercicios de pedaleo se encuentran entre las terapias encaminadas a brindar tales efectos. Diversos estudios en pacientes con trastornos neurológicos han demostrado su eficacia en la 
    recuperación del equilibrio y una mejora funcional en los parámetros espaciotemporales de la marcha
    ~\cite{quiles2020lessons, el2021effect}. También se ha evidenciado mantenimiento del rango de movimiento (RDM) de 
    articulaciones como la rodilla y la cadera, aumento de la fuerza muscular e incremento del consumo máximo 
    de oxígeno, lo que contribuye a mejorar la función cardiorespiratoria~\cite{el2021effect, ashadi2016pengaruh}. Las 
    intervenciones con ejercicios de pedaleo no solo han demostrado beneficios para la función motora, sino también a nivel del sistema nervioso
    central (SNC)~\cite{linder2019forced, alberts2011not}; pues estimulan procesos que inducen la neuroplasticidad como la angiogénesis, la neurogénesis 
    y la sinaptogénesis ~\cite{linder2019forced, el2021effect}. Por estos motivos se han desarrollado terapias de neurorrehabilitación que aprovechan 
    los mecanismos neuronales y de activación muscular que se manifiestan durante los movimientos cíclicos del pedaleo. 
    En este sentido se han aplicado los sistemas de interacción Humano-Robot; los cuales, a partir de señales de 
    electromiografía (EMG), decodifican las intenciones motoras o cognitivas en comandos de control para dispositivos robóticos.
    
    \vspace{5pt}
    En la bibliografía científica existen diversos programas informáticos que se han empleado en la 
    rehabilitación cognitiva asistida por computadora (CACR). A pesar de la existencia de estos, muchos 
    autores han empleado otras variantes como videojuegos comerciales y juegos serios aplicados a la 
    neurorrehabilitación. El principal propósito de esta variante de rehabilitación es incrementar la 
    motivación de los pacientes en las sesiones terapéuticas. La motivación a menudo puede verse afectada 
    por diversos factores como la monotonía de algunas de las tareas cognitivas, el rechazo de algunos 
    pacientes al uso de materiales como el lápiz y el papel y la insuficiencia de los materiales para 
    realizar estas sesiones~\cite{regalon12019juegos}. 
    
    \vspace{5pt}
    Además de la rehabilitación presencial paciente-especialista como valor 
    agregado el juego serio para plataformas accesibles por el usuario representa una vía y una forma
    sencilla y asequible de rehabilitación desde el hogar que incluso rompe con la barreras establecidas 
    por la secuelas físicas y psicológicas como: el depender de un medio de transportación, pena de inserción 
    en un grupo social debido a las deformaciones físicas concebidas por la enfermedad, baja autoestima, 
    entre otras.
    
    \vspace{5pt}
    Por lo antes expuesto, los juegos serios constituyen un campo aplicado en desarrollo que requiere ser incluido 
    como parte de un sistema de rehabilitación por pedaleo que favorezca la motivación del paciente y minimice el tiempo total 
    del tratamiento. La viabilidad y los beneficios de su implementación, guía al problema de investigación: 
    Ineficiente infraestructura tecnológica con aplicación de juego serio, para proveer rehabilitación neuromuscular en el tratamiento de
    enfermedades cerebrovasculares de los centros médicos cubanos . 

    \vspace{5pt}
    Para dar solución al problema se plantea como objetivo general el desarrollo de una aplicación de juego serio integrada a un sistema de 
    rehabilitación neuromuscular para pacientes con enfermedades cerebrevasculares.
    
    \vspace{5pt}
    Para el logro del objetivo general, se requiere el cumplimiento de las siguientes tareas:
    \begin{itemize}
        \item Diseño e implementación de un modelo cliente-servidor de red que permita la
        recepción de las señales capturadas por los sensores del pedal motorizado
        y la supervisión a un cliente remoto corriendo sobre plataforma Android.
        \item  Diseño y selección conjuntamente con expertos, de las rutinas de juegos que
        puedan desarrollar las capacidades físicas: fuerza, resistencia, velocidad, y que
        permitan la supervisión del desarrollo de la rehabilitación.
        \item Diseñar un sistema de base de datos para gestionar la información relacionada con
        el paciente, las rutinas diseñadas por el fisioterapeuta, y el proceso de
        rehabilitación.
        \item Desarrollo de un modelo de inteligencia artificial para la predición de los valores de MCV
    \end{itemize}
\end{introduction}

