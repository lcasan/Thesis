\begin{introduction}
    Desde la perspectiva de la rehabilitación, el accidente cerebrovascular es un gran generador de discapacidad, tanto física como cognitiva. De la población que sufre accidentes cerebrovasculares (ACV), un 15 a 30 porciento resulta con un deterioro funcional severo a largo plazo, lo que implica un alto grado de dependencia de terceros. Además, el ACV se ha establecido como la segunda causa de demencia a nivel mundial~\cite{moyano2010accidente} .Gracias al advenimiento de nuevas terapias, la mortalidad por ACV ha disminuido notablemente en los últimos años~\cite{cuadrado2009rehabilitacion,harold2007guidelines}, lo que deja un número cada vez más alto de sobrevivientes con mayor probabilidad de recurrencia. Si a lo anteriormente expresado, se suma una población envejecida (por el aumento en la esperanza de vida), se genera un importante impacto sanitario: más población con déficit funcional, quienes tienen más probabilidad de presentar complicaciones asociadas y que éstas sean de mayor gravedad.

    Las personas afectadas suelen presentar dificultades para realizar actividades de la vida diaria como caminar. Entre los problemas motores más frecuentes que se manifiestan por estas enfermedades se encuentran: rigidez muscular, espasticidad, hemiparesia, parálisis, dolor muscular crónico, lentitud en los movimientos y problemas de coordinación y equilibrio. Estos síntomas traen consigo alteraciones de la marcha y dificultades para la activación de los músculos y la movilidad de las articulaciones en los miembros inferiores y superiores. Además, a medida que se agravan aparecen complicaciones respiratorias y cardiovasculares ~\cite{barbosa2015application, miner2020therapeutic}, limitando aún más la calidad de vida. Debido a ello, las investigaciones recientes se han enfocado en terapias de rehabilitación que permitan no solo restaurar las capacidades motoras, sino también, disminuir los factores de riesgo que prevalecen en estos pacientes ~\cite{barbosa2015application}. Los ejercicios de pedaleo se encuentran entre las terapias encaminadas a brindar tales efectos. Diversos estudios en pacientes con trastornos neurológicos han demostrado su eficacia en la recuperación del equilibrio y una mejora funcional en los parámetros espaciotemporales de la marcha~\cite{quiles2020lessons, el2021effect}. También se ha evidenciado mantenimiento del rango de movimiento (RDM) de articulaciones como la rodilla y la cadera, aumento de la fuerza muscular e incremento del consumo máximo de oxígeno, lo que contribuye a mejorar la función cardiorespiratoria~\cite{el2021effect, ashadi2016pengaruh}. Las intervenciones con ejercicios de pedaleo no solo han demostrado beneficios para la función motora, sino también a nivel del sistema nervioso central (SNC)~\cite{linder2019forced, alberts2011not}; pues estimulan procesos que inducen la neuroplasticidad como la angiogénesis, la neurogénesis y la sinaptogénesis ~\cite{linder2019forced, el2021effect}. Por estos motivos se han desarrollado terapias de neurorrehabilitación que aprovechan 
    los mecanismos neuronales y de activación muscular que se manifiestan durante los movimientos cíclicos del pedaleo. 
    
    La globalización ha permitido estrechas relaciones entre diferentes áreas, como las ciencias para la salud (CPS) y las nuevas tecnologías mediante la interrelación de conceptos, programas y dispositivos~\cite{federal2008older}. Un claro ejemplo es la aplicación de los videojuegos en el entorno clínico. 
    
    Un videojuego es un software que por medio de un controlador permite la interacción con un dispositivo de video, típicamente para entretenimiento y diversión de participantes autónomos ~\cite{studenski2010interactive,gomez2012videojuegos, gonzalez2016mooc}. Generalmente, el uso de videojuegos, lejos de aportar a procesos de aprendizaje, se ha asociado a actividades improductivas, de ocio y entretenimiento que pueden generar un impacto negativo en el desempeño cotidiano~\cite{maldonado2014videojuegos}. Ya en los últimos años se ha logrado encontrar una utilidad de los videojuego diferentes de las actividades solo con fines recreativos; como es el caso de las CPS, en las que se han aplicado en procesos de aprendizaje, prevención de la enfermedad, diagnóstico de patologías, tratamiento y rehabilitación~\cite{ladino2021uso}. 

    Un videojuego serio utiliza la tecnología del entretenimiento por computador u otra interfaz para enseñar, entrenar, o cambiar el comportamiento . De ahí que su objetivo está centrado principalmente en la educación o formación, para aplicar lecciones aprendidas en la cotidianidad~\cite{graafland2014serious}, que incluyen la participación activa, la solución de problemas difíciles y la retroalimentación~\cite{gee2004learning}.
    
    La connotación de videojuego, aunque serio, se puede introducir como una experiencia agradable para la prevención de la enfermedad, el diagnóstico, el tratamiento y rehabilitación; sin desconocer las limitaciones de acceso a este tipo de plataformas por factores interpersonales como estado socioeconómico, edad y habilidades~\cite{ladino2021uso}. Además, como valor agregado, a diferencia de la rehabilitación presencial paciente-especialista dicha tecnología representa una forma sencilla y asequible de rehabilitación desde el hogar, la cual rompe con la barreras establecidas por la secuelas físicas y psicológicas como: la dependencia de un medio de transportación, dificultad de integración en entornos sociales, baja autoestima, entre otras.

    El principal propósito de esta variante de rehabilitación es incrementar la motivación de los pacientes en las sesiones terapéuticas. La motivación a menudo puede verse afectada por diversos factores como la monotonía de algunas de las tareas cognitivas, el rechazo de algunos pacientes al uso de materiales como el lápiz y el papel y la insuficiencia de los materiales para realizar estas sesiones~\cite{regalon12019juegos}. 

    Esta integración innovadora no solo transforma la experiencia de rehabilitación de los pacientes, sino que también propone tener un impacto significativo en los resultados a largo plazo, mejorando la calidad de vida y promoviendo una recuperación más efectiva para individuos afectados por enfermedades como las cerebrovasculares.

    Lo antes expuesto nos guía al problema de investigación: ¿Cómo puede la aplicación de juego serio integrada a un sistema de rehabilitación neuromuscular con pedal motorizado beneficiar a los pacientes con enfermedades cerebrovasculares?

    Para dar solución al problema se plantea como objetivo general el desarrollo de una aplicación de juego serio integrada a un sistema de rehabilitación neuromuscular con pedal motorizado para pacientes con enfermedades cerebrovasculares.
    
    Para el logro del objetivo general, se requiere el cumplimiento de los siguientes objetivos específicos:

    \begin{itemize}
        \item Elaboración del marco teórico relacionado con las tecnologías de rehabilitación clínica y de juego serio.
        \item Análisis, diseño e implementación del sistema.
        \item Diseño e implementación de un modelo cliente-servidor de red que permita el envío y recepción de información entre el pedal motorizado y el juego serio ejecutado sobre plataforma android.
        \item  Diseño y selección conjuntamente con expertos, de las rutinas de juegos que puedan desarrollar las capacidades físicas: fuerza, resistencia, velocidad, y que permitan la supervisión del desarrollo de la rehabilitación.
        \item Diseño de un sistema de base de datos para gestionar la información relacionada con el paciente, las rutinas diseñadas por el usuario, y el proceso de rehabilitación.
        \item Validación de los requisitos funcionales y no funcionales del sistema.
    \end{itemize}

    La hipótesis sustentada en este trabajo es la siguiente:

    Si se desarrolla una aplicación de juego serio integrada a un sistema de rehabilitación neuromuscular con pedal motorizado para pacientes con enfermedades cerebrovasculares se contribuirá a la adherencia del tratamiento brindando: retroalimentación inmediata sobre el rendimiento, estimulación de las habilidades tanto motoras como cognitivas, personalización del tratamiento según las necesidades individuales, monitoreo del progreso del paciente, reducción del estrés y la ansiedad asociados con la rehabilitación, y el fomento de la socialización entre los pacientes.

    En el desarrollo de este proyecto fueron empleados métodos de investigación que permitieron el cumplimiento de los objetivos y las tareas desarrolladas. El método análisis-síntesis permitió la extracción del conjunto de conocimientos que fundamentan la investigación y de las contribuciones de los sistemas de rehabilitación que actualmente se comercializan y/o desarrollan al diseño de la aplicación. El método inductivo-deductivo se puso de manifiesto en toda la programación del código tras la arquitectura de la aplicación del juego serio. El método hipotético-deductivo en relación con el método de observación (método empírico), permitió verificar la hipótesis planteada en cada tarea a implementar. La entrevista a especialistas en terapia rehabilitación, posibilitó la adquisición de la información relacionada con los requerimientos para el diseño del software.
\end{introduction}