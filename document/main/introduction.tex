\begin{introduction}
    Desde la perspectiva de la rehabilitación, el accidente cerebrovascular 
    es un gran generador de discapacidad, tanto física como cognitiva. De la población que sufre 
    accidentes cerebrovasculares (ACV), un 15 a 30 porciento resulta con un deterioro funcional severo a 
    largo plazo, lo que implica un alto grado de dependencia de terceros. Además, el ACV se ha
    establecido como la segunda causa de demencia a nivel mundial~\cite{moyano2010accidente} .Gracias 
    al advenimiento de nuevas terapias, la mortalidad por ACV ha disminuido notablemente en los últimos 
    años~\cite{cuadrado2009rehabilitacion,harold2007guidelines}, lo que deja un número cada vez más 
    alto de sobrevivientes con mayor probabilidad de recurrencia. Si a esto se suma una población 
    envejecida (por el aumento en la esperanza de vida), se genera un importante impacto sanitario: 
    más población con déficit funcional, quienes tienen más probabilidad de presentar complicaciones 
    asociadas y que éstas sean de mayor gravedad.

    \vspace{5pt}
    La rehabilitación cognitiva como parte de la rehabilitación neuropsicológica, consiste en la 
    aplicación de técnicas y procedimientos para mejorar las habilidades intelectuales y perceptuales 
    de pacientes que presentan un daño cerebral. Es un proceso sistemático enfocado principalmente a 
    que la persona retome de manera segura e independiente, las actividades cotidianas que desarrollaba 
    antes de una lesión, a través de la adaptación a ambientes familiares, sociales y de trabajo~\cite{regalon12019juegos}.
    
    \vspace{5pt}
    En la bibliografía científica existen diversos programas informáticos que se han empleado en la 
    rehabilitación cognitiva asistida por computadora (CACR). A pesar de la existencia de estos, muchos 
    autores han empleado otras variantes como videojuegos comerciales y juegos serios aplicados a la 
    neurorrehabilitación. El principal propósito de esta variante de rehabilitación es incrementar la 
    motivación de los pacientes en las sesiones terapéuticas. La motivación a menudo puede verse afectada 
    por diversos factores como la monotonía de algunas de las tareas cognitivas, el rechazo de algunos 
    pacientes al uso de materiales como el lápiz y el papel y la insuficiencia de los materiales para 
    realizar estas sesiones~\cite{regalon12019juegos}. 
    
    \vspace{5pt}
    Además de la rehabilitación presencial paciente-especialista como valor 
    agregado el juego serio para plataformas accesibles por el usuario representa una vía y una forma
    sencilla y asequible de rehabilitación desde el hogar que incluso rompe con la barreras establecidas 
    por la secuelas físicas y psicológicas como: el depender de un medio de transportación, pena de inserción 
    en un grupo social debido a las deformaciones físicas concebidas por la enfermedad, baja autoestima, 
    entre otras.
    
    \vspace{5pt}
    Por lo antes expuesto, los juegos serios constituyen un campo aplicado en desarrollo que requiere ser incluido 
    como parte de un sistema de rehabilitación que favorezca la motivación del paciente y minimice el tiempo total 
    del tratamiento. La viabilidad y los beneficios de su implementación, guía al problema de investigación: 
    Ineficiente infraestructura tecnológica con aplicación de juego serio, para proveer rehabilitación neuromuscular en el tratamiento de
    enfermedades cerebrovasculares de los centros médicos cubanos . 

    \vspace{5pt}
    Para dar solución al problema se plantea como objetivo general el desarrollo de una
    aplicación de juego serio para proveer rehabilitación neuromuscular a pacientes con
    enfermedades cerebrovasculares.
    
    \vspace{5pt}
    Para el logro del objetivo general, se requiere el cumplimiento de las siguientes tareas:
    \begin{itemize}
        \item Diseño e implementación de un modelo cliente-servidor de red que permita la
        recepción de las señales EMG capturadas por los electrodos del pedal motorizado
        y la supervisión a un cliente remoto corriendo sobre plataforma Android.
        \item  Diseño y selección conjuntamente con expertos, de las rutinas de juegos que
        puedan desarrollar las capacidades físicas: fuerza, resistencia, velocidad, y que
        permitan la supervisión del desarrollo de la rehabilitación.
    \end{itemize}
\end{introduction}