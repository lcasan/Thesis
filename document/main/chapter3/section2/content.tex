\subthesischapter{Requisitos de hardware y software}
Los requisitos de hardware y software para el uso de la aplicación en dispositivos móviles se definen a través de una evaluación de rendimiento. Dichos requisitos, para garantizar un funcionamiento eficiente y óptimo, se detallan a continuación:

\vspace{5pt}
\textbf{Hardware:}
\begin{itemize}
    \item \underline{Procesador (CPU)}: Se requiere un procesador compatible con la arquitectura ARMv7 con soporte de Neon (32 bits) o ARM64
    \item  \underline{Memoria RAM}: Para un rendimiento fluido y sin contratiempos, se estipula un requisito mínimo de 1GB de memoria RAM.  
    \item \underline{Almacenamiento Interno}: La capacidad de almacenamiento interno debe ser de 55 MB o superior. Este espacio garantiza la instalación adecuada de la aplicación en el dispositivo móvil del usuario, permitiendo la correcta disposición de archivos y recursos necesarios para su funcionamiento.   
\end{itemize}

\vspace{5pt}
\textbf{Software:}
\begin{itemize}
    \item \underline{Sistema Operativo}: La aplicación está diseñada para ejecutarse en dispositivos móviles basados en el sistema operativo Android.
    \item \underline{Versión android}: 5.1 (API 22)+.
    \item \underline{Compatibilidad de Software Gráfico:} Asegurarse de que el dispositivo móvil cuente con el soporte necesario para las API gráficas mencionadas (OpenGL ES 2.0+, OpenGL ES 3.0+ o Vulkan) es esencial para garantizar una representación visual óptima. 
\end{itemize} 