\subthesischapter{Pruebas del sistema}
% SUBSECTION 1: FUNCTIONAL REQUIREMENTS
\subsubthesischapter{Comprobación de los requisitos funcionales del sistema}
La comprobación de los requisitos funcionales del sistema se realizó a partir de una prueba de caja negra. Dicha prueba puede definirse como una técnica donde se busca la verificación de las funcionalidades del software o aplicación analizada, sin tomar como referente la estructura del código interno, las rutas de tipo internas ni la información referente a la implementación. Esto quiere decir que se llevan a cabo con desconocimiento del funcionamiento del sistema interno, debido a que se enfoca en la entrada y salida de un software, tomando como base sus especificaciones y requisitos. A modo de ejemplo en la Tabla \ref{tab: darkbox1} se muestra la  aplicación de dicha prueba para el requisito funcional inicio de sesión.    

\begin{table}[!ht]
    \centering
    \begin{tabularx}{\textwidth}{|c|X|X|X|}
        \hline
        \textbf{No} & \textbf{Entrada} & \textbf{Respuesta esperada} & \textbf{Respuesta del sistema}\\\hline
        % 1
        1
        &
        \begin{minipage}{0.3\columnwidth}
            \textbf{campo correo} (cadena de caracteres asociada con un correo registrado en la base de datos).
            
            \textbf{campo contraseña} (la contraseña correspondiente al correo registrado).
        \end{minipage}  
        & 
        Permite al usuario acceder al sistema. 
        & 
        Se accede al sistema correctamente.
        \\\hline
        
        % 2
        2
        &
        \begin{minipage}{0.3\columnwidth}
            \textbf{campo correo} (cadena de caracteres que no posea formato de correo). \\\\ \textbf{campo contraseña} (Cualquier cadena de caracteres).
        \end{minipage}  
        & 
        El sistema notifica con el mensaje: Correo inválido.
        & 
        Se notifica correctamente el mensaje: Correo inválido. (ver Figura \ref{annex: 1}a)
        \\\hline

        % 3
        3
        &
        \begin{minipage}{0.3\columnwidth}
            \textbf{campo correo} (cadena de caracteres asociada con un correo registrado en la base de datos).
            
            \textbf{campo contraseña} (Contraseña distinta a la correspondiente con el correo registrado).
        \end{minipage}  
        & 
        El sistema notifica con el mensaje: Contraseña incorrecta.
        & 
        Se notifica correctamente el mensaje: Contraseña incorrecta. (ver Figura \ref{annex: 1}c)
        \\\hline
        % 4
        4
        &
        \begin{minipage}{0.3\columnwidth}
            \textbf{campo correo} (cadena de caracteres asociada con un correo que no esté registrado en la base de datos). 
            
            \textbf{campo contraseña} (Contraseña distinta a la correspondiente con el correo registrado).
        \end{minipage}  
        & 
        El sistema notifica con el mensaje: Correo incorrecta.
        & 
        \cellcolor{red!75} Error
        \\\hline

    \end{tabularx}
    \caption{Prueba de caja negra, inicio de sesión.}
    \label{tab: darkbox1}
\end{table}

\newpage
La prueba de caja negra para la funcionalidad de inicio de sesión se llevó a cabo mediante la implementación de varios casos de prueba, cada uno de los cuales involucraba múltiples iteraciones para abordar distintas combinaciones de entradas. Estas iteraciones se realizaron con el objetivo de evaluar exhaustivamente el comportamiento del sistema en diversas situaciones. En el caso 4 se evidenció un error, para su solución se realizó una revisión del código fuente para determinar la causa subyacente. Luego, se implementaron medidas correctivas, como ajustes en el código y en la lógica de manejo de errores, con el objetivo de resolver y evitar incidentes similares en el futuros.

% SUBSECTION 2
\subsubthesischapter{Encuesta a especialistas sobre la percepción del sistema}
Para la evaluación de los requisitos no funcionales usabilidad e interoperabilidad, además de los evaluados por \cite{franco2016sistema} (ergonomía, seguridad y concordancia de los ejercicios con los de la terapia tradicional) fue aplicada una encuesta a 10 profesionales del área de la investigación, 4 de ellos doctores en ciencia, 3 ingenieros (entre ellos 1 informático, 1 ingeniero de control automático y 1 biomédico) especialistas en desarrollo en el tema de la rehabilitación y 3 especialistas del área médica. Todos con más de 5 años de experiencia. Los resultados de dicha encuesta se presentan agrupados en una escala dicotómica en la Tabla \ref{table:test}.

\begin{table}[ht]
    \centering
    \begin{tabular}{p{3cm} c c c c}
        \hline
        Característica      &  Alto(\#/Total)   &  Alto(\%) & Bajo(\#/Total)  & Bajo(\%) \\\hline
        Usabilidad          &  7/10   &  70   & 3/10  &   30\\
        Interoperabilidad   &  8/10   &  80   & 2/10  &   20\\
        Ergonomía           &  10/10  &  100  & 0/10  &   0\\
        Concordancia        &  8/10   &  80   & 2/10  &   20\\
        Seguridad           &  10/10  &  100  & 0/10  &   0\\
        \hline        
    \end{tabular}        
    \caption{Resultados de percepción de los especialistas.}
    \label{table:test}
\end{table}

Dichos resultados siguen el patrón lógico esperado:
\begin{itemize}
    \item \textbf{Usabilidad (70\% de satisfacción alta):} Un alto nivel de usabilidad significa que los profesionales evaluados encontraron el juego serio fácil de usar y comprender. Esto indica que la interfaz y las interacciones del juego son intuitivas, lo que facilita su adopción por parte de los usuarios.
    
    La facilidad de uso es fundamental para los pacientes de neurorrehabilitación, especialmente para aquellos que pueden tener limitaciones cognitivas o motoras. Un juego fácil de entender y jugar fomenta la participación y la adherencia al programa de rehabilitación.


    \item \textbf{Interoperabilidad (80\% de satisfacción alta):} Una alta interoperabilidad sugiere que el juego serio puede integrarse sin problemas con el sistema de adquisición de datos.
    
    La interoperabilidad es crucial en un entorno de neurorrehabilitación donde se utilizan varios dispositivos y sistemas para monitorear y apoyar la recuperación del paciente. Un juego serio interoperable permite una recopilación de datos sin problemas y una comunicación eficaz entre diferentes componentes del sistema.


    \item \textbf{Ergonomía y Seguridad (100\% de satisfacción alta):} En el caso de la seguridad aplican condiciones similares a la ergonomía, dado que el sistema por sí mismo incluye las condiciones asociadas con una bicicleta estática que ya es usada en los procesos de rehabilitación de miembro inferior. Además, la ergonomía también demuestra la capacidad de la aplicación para adaptarse a las necesidades y habilidades cambiantes del usuario. 
    
    La seguridad total indica que los profesionales evaluados consideran que el juego serio no representa ningún riesgo para la salud o la integridad de los usuarios.
    
    \item \textbf{Concordancia (80\% de satisfacción alta):} Implica que los ejercicios en el juego serio se alinean adecuadamente con las prácticas de terapia tradicional, lo que sugiere coherencia en el enfoque de rehabilitación. Los ejercicios en el juego deben complementar y respaldar las técnicas utilizadas en la terapia tradicional. 
    
    La concordancia asegura que los pacientes practiquen movimientos y habilidades relevantes para su recuperación.
\end{itemize}

Las observaciones generales proporcionadas por los especialistas, permitieron identificar otros usos potenciales en el proceso de rehabilitación de víctimas de ACV: resistencia, fuerza y coordinación,actividad física independiente de su condición, complemento al proceso de entrenamiento de marcha, coordinación, autoconfianza, motivación.

Por otra parte, teniendo en cuenta que el sistema de adquisición de datos no es invasivo y, por el contrario, presentó aceptación por parte de los especialistas por incluir un videojuego, se propone, en un futuro, realizar pruebas a un número determinado de pacientes víctimas de ACV y de esta manera evaluar el funcionamiento del sistema de adquisición de datos durante ejercicios de rehabilitación resistida y activa-asistida, en los que el paciente moverá los pedales continuamente y realizará por su propia acción o con ayuda de especialista los ejercicios establecidos. Adicionalmente, se valida que el sistema no es exclusivo para rehabilitación de personas víctimas de ACV, por lo tanto se plantea que en un trabajo futuro se podrá probar dicha aplicación en la modalidad ligero, por ejemplo, con deportistas de alto rendimiento con lesiones deportivas.

Si bien se recibieron comentarios  positivos otros también fueron negativos respecto a la usabilidad y la concordancia, requisitos estrechamente relacionados en el juego serio. En ambos se encontró como déficit el no cubrir toda la gama de ejercicios definidos en las rutinas de rehabilitación siendo propuestos para futuras versiones la inclusión de las modalidades resistida y activo-asistida con el fin de complementar los ejercicios de rehabilitación que se pueden realizar actualmente con el sistema de adquisición de datos, los cuales son solamente ejercicios de rehabilitación libre.
