\begin{thesischapter}{3} {Análisis de resultados}
En este capítulo se detallan los requisitos tanto a nivel de hardware como de software del sistema, y se examina minuciosamente el impacto social de la aplicación presentando datos estadísticos pertinentes sobre la prevalencia de discapacidades físicas relacionadas con las condiciones que el sistema busca abordar. Se proporciona una descripción exhaustiva junto con representaciones visuales del flujo de trabajo estándar, el cual será ampliamente utilizado en el sistema. Se expone con gran detalle la estrategia de prueba empleada para evaluar tanto la eficacia como la calidad del juego serio implementado. Esta evaluación se fundamenta en métodos cualitativos y cuantitativos e incluye pruebas con especialista y encuestas detalladas. Los resultados obtenidos de estas evaluaciones se presentan en una escala dicotómica para cada uno de los requisitos no funcionales previamente establecidos. Por último, se exploran las observaciones generales proporcionadas por los especialistas, quienes han identificado posibles aplicaciones adicionales de la aplicación en el proceso de rehabilitación, proporcionando así indicadores cualitativos.
    
% SECTION 1: SOFTWARE AND HARDWARE REQUIREMENT 
\begin{thesischapter}{2} {Diseño e implementación del Juego Serio}
En este capítulo se discuten los detalles de desarrollo de los aspectos citados en el capítulo anterior. Este comienza con una descripción y caracterización general del sistema, donde se  abordan cada uno de los componentes requeridos para su completo funcionamiento. Posteriormente se detalla la ingienría de software requerida en la etapa de conceptualización de la aplicación, se explican de forma detallada los aspectos teóricos y de implementación de la base de datos, el funcionamiento del protocolo de comunicación y por último los escenarios de juegos requeridos en las rutinas de entrenamiento ligero y clínico, y las estadísticas generadas por estos. Como herramienta de desarrollo se utilizó c\#.

% SYSTEM DESCRIPTION AND CHARACTERIZATION TO APPLY
\begin{thesischapter}{2} {Diseño e implementación del Juego Serio}
En este capítulo se discuten los detalles de desarrollo de los aspectos citados en el capítulo anterior. Este comienza con una descripción y caracterización general del sistema, donde se  abordan cada uno de los componentes requeridos para su completo funcionamiento. Posteriormente se detalla la ingienría de software requerida en la etapa de conceptualización de la aplicación, se explican de forma detallada los aspectos teóricos y de implementación de la base de datos, el funcionamiento del protocolo de comunicación y por último los escenarios de juegos requeridos en las rutinas de entrenamiento ligero y clínico, y las estadísticas generadas por estos. Como herramienta de desarrollo se utilizó c\#.

% SYSTEM DESCRIPTION AND CHARACTERIZATION TO APPLY
\begin{thesischapter}{2} {Diseño e implementación del Juego Serio}
En este capítulo se discuten los detalles de desarrollo de los aspectos citados en el capítulo anterior. Este comienza con una descripción y caracterización general del sistema, donde se  abordan cada uno de los componentes requeridos para su completo funcionamiento. Posteriormente se detalla la ingienría de software requerida en la etapa de conceptualización de la aplicación, se explican de forma detallada los aspectos teóricos y de implementación de la base de datos, el funcionamiento del protocolo de comunicación y por último los escenarios de juegos requeridos en las rutinas de entrenamiento ligero y clínico, y las estadísticas generadas por estos. Como herramienta de desarrollo se utilizó c\#.

% SYSTEM DESCRIPTION AND CHARACTERIZATION TO APPLY
\input{main/chapter2/section1/content.tex}
     
% SERIOUS GAME REQUIREMENTS
\input{main/chapter2/section2/content.tex}    

% USE CASE DEFINITION
\input{main/chapter2/section3/content.tex}

% USE CASE REALIZATION
\input{main/chapter2/section4/content.tex}

% DATABASE DESIGN 
\input{main/chapter2/section5/content.tex}

% DATA MANIPULATION
\input{main/chapter2/section6/content.tex}

% COMMUNICATION 
\input{main/chapter2/section7/content.tex}

% TRAINING SCENARIOS
\input{main/chapter2/section8/content.tex}

% EMG GRAPHIC
\input{main/chapter2/section9/content.tex}

% STATICAL REPORTS GENERATION
\input{main/chapter2/section10/content.tex}

\subthesischapter{Conclusiones del capítulo}
Se presentó una descripción del sistema de adquisición de datos para rehabilitación, sus componentes, características distintivas y su funcionamiento. Se identificaron y definieron los requisitos del juego  serio, tanto funcionales como no funcionales, así como los actores y casos de usos del sistema que establecieron las bases fundamentales para el desarrollo de la aplicación. Se realizó el diseño de la base de datos, abarcando tanto el modelo lógico como el físico, lo que aseguró una estructura robusta y eficiente para el almacenamiento de los datos. La manipulación de los datos se abordó de manera integral, desde la conexión con la base de datos hasta la persistencia de los resultados estadísticos. Se diseñó e implementó la comunicación con el pedal motorizado y la implementación de la interfaz gráfica para la representación de los datos EMG. Se definieron los escenarios de entrenamiento para las modalidades Ligero y Clínico, asegurando una cobertura completa de las necesidad de entrenamiento del usuario. Por último en el ámbito estadístico se desarrolló una serie de gráficos para el seguimiento de los resultados en las rutinas de entrenamiento.   
    
\end{thesischapter}
     
% SERIOUS GAME REQUIREMENTS
\begin{thesischapter}{2} {Diseño e implementación del Juego Serio}
En este capítulo se discuten los detalles de desarrollo de los aspectos citados en el capítulo anterior. Este comienza con una descripción y caracterización general del sistema, donde se  abordan cada uno de los componentes requeridos para su completo funcionamiento. Posteriormente se detalla la ingienría de software requerida en la etapa de conceptualización de la aplicación, se explican de forma detallada los aspectos teóricos y de implementación de la base de datos, el funcionamiento del protocolo de comunicación y por último los escenarios de juegos requeridos en las rutinas de entrenamiento ligero y clínico, y las estadísticas generadas por estos. Como herramienta de desarrollo se utilizó c\#.

% SYSTEM DESCRIPTION AND CHARACTERIZATION TO APPLY
\input{main/chapter2/section1/content.tex}
     
% SERIOUS GAME REQUIREMENTS
\input{main/chapter2/section2/content.tex}    

% USE CASE DEFINITION
\input{main/chapter2/section3/content.tex}

% USE CASE REALIZATION
\input{main/chapter2/section4/content.tex}

% DATABASE DESIGN 
\input{main/chapter2/section5/content.tex}

% DATA MANIPULATION
\input{main/chapter2/section6/content.tex}

% COMMUNICATION 
\input{main/chapter2/section7/content.tex}

% TRAINING SCENARIOS
\input{main/chapter2/section8/content.tex}

% EMG GRAPHIC
\input{main/chapter2/section9/content.tex}

% STATICAL REPORTS GENERATION
\input{main/chapter2/section10/content.tex}

\subthesischapter{Conclusiones del capítulo}
Se presentó una descripción del sistema de adquisición de datos para rehabilitación, sus componentes, características distintivas y su funcionamiento. Se identificaron y definieron los requisitos del juego  serio, tanto funcionales como no funcionales, así como los actores y casos de usos del sistema que establecieron las bases fundamentales para el desarrollo de la aplicación. Se realizó el diseño de la base de datos, abarcando tanto el modelo lógico como el físico, lo que aseguró una estructura robusta y eficiente para el almacenamiento de los datos. La manipulación de los datos se abordó de manera integral, desde la conexión con la base de datos hasta la persistencia de los resultados estadísticos. Se diseñó e implementó la comunicación con el pedal motorizado y la implementación de la interfaz gráfica para la representación de los datos EMG. Se definieron los escenarios de entrenamiento para las modalidades Ligero y Clínico, asegurando una cobertura completa de las necesidad de entrenamiento del usuario. Por último en el ámbito estadístico se desarrolló una serie de gráficos para el seguimiento de los resultados en las rutinas de entrenamiento.   
    
\end{thesischapter}    

% USE CASE DEFINITION
\begin{thesischapter}{2} {Diseño e implementación del Juego Serio}
En este capítulo se discuten los detalles de desarrollo de los aspectos citados en el capítulo anterior. Este comienza con una descripción y caracterización general del sistema, donde se  abordan cada uno de los componentes requeridos para su completo funcionamiento. Posteriormente se detalla la ingienría de software requerida en la etapa de conceptualización de la aplicación, se explican de forma detallada los aspectos teóricos y de implementación de la base de datos, el funcionamiento del protocolo de comunicación y por último los escenarios de juegos requeridos en las rutinas de entrenamiento ligero y clínico, y las estadísticas generadas por estos. Como herramienta de desarrollo se utilizó c\#.

% SYSTEM DESCRIPTION AND CHARACTERIZATION TO APPLY
\input{main/chapter2/section1/content.tex}
     
% SERIOUS GAME REQUIREMENTS
\input{main/chapter2/section2/content.tex}    

% USE CASE DEFINITION
\input{main/chapter2/section3/content.tex}

% USE CASE REALIZATION
\input{main/chapter2/section4/content.tex}

% DATABASE DESIGN 
\input{main/chapter2/section5/content.tex}

% DATA MANIPULATION
\input{main/chapter2/section6/content.tex}

% COMMUNICATION 
\input{main/chapter2/section7/content.tex}

% TRAINING SCENARIOS
\input{main/chapter2/section8/content.tex}

% EMG GRAPHIC
\input{main/chapter2/section9/content.tex}

% STATICAL REPORTS GENERATION
\input{main/chapter2/section10/content.tex}

\subthesischapter{Conclusiones del capítulo}
Se presentó una descripción del sistema de adquisición de datos para rehabilitación, sus componentes, características distintivas y su funcionamiento. Se identificaron y definieron los requisitos del juego  serio, tanto funcionales como no funcionales, así como los actores y casos de usos del sistema que establecieron las bases fundamentales para el desarrollo de la aplicación. Se realizó el diseño de la base de datos, abarcando tanto el modelo lógico como el físico, lo que aseguró una estructura robusta y eficiente para el almacenamiento de los datos. La manipulación de los datos se abordó de manera integral, desde la conexión con la base de datos hasta la persistencia de los resultados estadísticos. Se diseñó e implementó la comunicación con el pedal motorizado y la implementación de la interfaz gráfica para la representación de los datos EMG. Se definieron los escenarios de entrenamiento para las modalidades Ligero y Clínico, asegurando una cobertura completa de las necesidad de entrenamiento del usuario. Por último en el ámbito estadístico se desarrolló una serie de gráficos para el seguimiento de los resultados en las rutinas de entrenamiento.   
    
\end{thesischapter}

% USE CASE REALIZATION
\begin{thesischapter}{2} {Diseño e implementación del Juego Serio}
En este capítulo se discuten los detalles de desarrollo de los aspectos citados en el capítulo anterior. Este comienza con una descripción y caracterización general del sistema, donde se  abordan cada uno de los componentes requeridos para su completo funcionamiento. Posteriormente se detalla la ingienría de software requerida en la etapa de conceptualización de la aplicación, se explican de forma detallada los aspectos teóricos y de implementación de la base de datos, el funcionamiento del protocolo de comunicación y por último los escenarios de juegos requeridos en las rutinas de entrenamiento ligero y clínico, y las estadísticas generadas por estos. Como herramienta de desarrollo se utilizó c\#.

% SYSTEM DESCRIPTION AND CHARACTERIZATION TO APPLY
\input{main/chapter2/section1/content.tex}
     
% SERIOUS GAME REQUIREMENTS
\input{main/chapter2/section2/content.tex}    

% USE CASE DEFINITION
\input{main/chapter2/section3/content.tex}

% USE CASE REALIZATION
\input{main/chapter2/section4/content.tex}

% DATABASE DESIGN 
\input{main/chapter2/section5/content.tex}

% DATA MANIPULATION
\input{main/chapter2/section6/content.tex}

% COMMUNICATION 
\input{main/chapter2/section7/content.tex}

% TRAINING SCENARIOS
\input{main/chapter2/section8/content.tex}

% EMG GRAPHIC
\input{main/chapter2/section9/content.tex}

% STATICAL REPORTS GENERATION
\input{main/chapter2/section10/content.tex}

\subthesischapter{Conclusiones del capítulo}
Se presentó una descripción del sistema de adquisición de datos para rehabilitación, sus componentes, características distintivas y su funcionamiento. Se identificaron y definieron los requisitos del juego  serio, tanto funcionales como no funcionales, así como los actores y casos de usos del sistema que establecieron las bases fundamentales para el desarrollo de la aplicación. Se realizó el diseño de la base de datos, abarcando tanto el modelo lógico como el físico, lo que aseguró una estructura robusta y eficiente para el almacenamiento de los datos. La manipulación de los datos se abordó de manera integral, desde la conexión con la base de datos hasta la persistencia de los resultados estadísticos. Se diseñó e implementó la comunicación con el pedal motorizado y la implementación de la interfaz gráfica para la representación de los datos EMG. Se definieron los escenarios de entrenamiento para las modalidades Ligero y Clínico, asegurando una cobertura completa de las necesidad de entrenamiento del usuario. Por último en el ámbito estadístico se desarrolló una serie de gráficos para el seguimiento de los resultados en las rutinas de entrenamiento.   
    
\end{thesischapter}

% DATABASE DESIGN 
\begin{thesischapter}{2} {Diseño e implementación del Juego Serio}
En este capítulo se discuten los detalles de desarrollo de los aspectos citados en el capítulo anterior. Este comienza con una descripción y caracterización general del sistema, donde se  abordan cada uno de los componentes requeridos para su completo funcionamiento. Posteriormente se detalla la ingienría de software requerida en la etapa de conceptualización de la aplicación, se explican de forma detallada los aspectos teóricos y de implementación de la base de datos, el funcionamiento del protocolo de comunicación y por último los escenarios de juegos requeridos en las rutinas de entrenamiento ligero y clínico, y las estadísticas generadas por estos. Como herramienta de desarrollo se utilizó c\#.

% SYSTEM DESCRIPTION AND CHARACTERIZATION TO APPLY
\input{main/chapter2/section1/content.tex}
     
% SERIOUS GAME REQUIREMENTS
\input{main/chapter2/section2/content.tex}    

% USE CASE DEFINITION
\input{main/chapter2/section3/content.tex}

% USE CASE REALIZATION
\input{main/chapter2/section4/content.tex}

% DATABASE DESIGN 
\input{main/chapter2/section5/content.tex}

% DATA MANIPULATION
\input{main/chapter2/section6/content.tex}

% COMMUNICATION 
\input{main/chapter2/section7/content.tex}

% TRAINING SCENARIOS
\input{main/chapter2/section8/content.tex}

% EMG GRAPHIC
\input{main/chapter2/section9/content.tex}

% STATICAL REPORTS GENERATION
\input{main/chapter2/section10/content.tex}

\subthesischapter{Conclusiones del capítulo}
Se presentó una descripción del sistema de adquisición de datos para rehabilitación, sus componentes, características distintivas y su funcionamiento. Se identificaron y definieron los requisitos del juego  serio, tanto funcionales como no funcionales, así como los actores y casos de usos del sistema que establecieron las bases fundamentales para el desarrollo de la aplicación. Se realizó el diseño de la base de datos, abarcando tanto el modelo lógico como el físico, lo que aseguró una estructura robusta y eficiente para el almacenamiento de los datos. La manipulación de los datos se abordó de manera integral, desde la conexión con la base de datos hasta la persistencia de los resultados estadísticos. Se diseñó e implementó la comunicación con el pedal motorizado y la implementación de la interfaz gráfica para la representación de los datos EMG. Se definieron los escenarios de entrenamiento para las modalidades Ligero y Clínico, asegurando una cobertura completa de las necesidad de entrenamiento del usuario. Por último en el ámbito estadístico se desarrolló una serie de gráficos para el seguimiento de los resultados en las rutinas de entrenamiento.   
    
\end{thesischapter}

% DATA MANIPULATION
\begin{thesischapter}{2} {Diseño e implementación del Juego Serio}
En este capítulo se discuten los detalles de desarrollo de los aspectos citados en el capítulo anterior. Este comienza con una descripción y caracterización general del sistema, donde se  abordan cada uno de los componentes requeridos para su completo funcionamiento. Posteriormente se detalla la ingienría de software requerida en la etapa de conceptualización de la aplicación, se explican de forma detallada los aspectos teóricos y de implementación de la base de datos, el funcionamiento del protocolo de comunicación y por último los escenarios de juegos requeridos en las rutinas de entrenamiento ligero y clínico, y las estadísticas generadas por estos. Como herramienta de desarrollo se utilizó c\#.

% SYSTEM DESCRIPTION AND CHARACTERIZATION TO APPLY
\input{main/chapter2/section1/content.tex}
     
% SERIOUS GAME REQUIREMENTS
\input{main/chapter2/section2/content.tex}    

% USE CASE DEFINITION
\input{main/chapter2/section3/content.tex}

% USE CASE REALIZATION
\input{main/chapter2/section4/content.tex}

% DATABASE DESIGN 
\input{main/chapter2/section5/content.tex}

% DATA MANIPULATION
\input{main/chapter2/section6/content.tex}

% COMMUNICATION 
\input{main/chapter2/section7/content.tex}

% TRAINING SCENARIOS
\input{main/chapter2/section8/content.tex}

% EMG GRAPHIC
\input{main/chapter2/section9/content.tex}

% STATICAL REPORTS GENERATION
\input{main/chapter2/section10/content.tex}

\subthesischapter{Conclusiones del capítulo}
Se presentó una descripción del sistema de adquisición de datos para rehabilitación, sus componentes, características distintivas y su funcionamiento. Se identificaron y definieron los requisitos del juego  serio, tanto funcionales como no funcionales, así como los actores y casos de usos del sistema que establecieron las bases fundamentales para el desarrollo de la aplicación. Se realizó el diseño de la base de datos, abarcando tanto el modelo lógico como el físico, lo que aseguró una estructura robusta y eficiente para el almacenamiento de los datos. La manipulación de los datos se abordó de manera integral, desde la conexión con la base de datos hasta la persistencia de los resultados estadísticos. Se diseñó e implementó la comunicación con el pedal motorizado y la implementación de la interfaz gráfica para la representación de los datos EMG. Se definieron los escenarios de entrenamiento para las modalidades Ligero y Clínico, asegurando una cobertura completa de las necesidad de entrenamiento del usuario. Por último en el ámbito estadístico se desarrolló una serie de gráficos para el seguimiento de los resultados en las rutinas de entrenamiento.   
    
\end{thesischapter}

% COMMUNICATION 
\begin{thesischapter}{2} {Diseño e implementación del Juego Serio}
En este capítulo se discuten los detalles de desarrollo de los aspectos citados en el capítulo anterior. Este comienza con una descripción y caracterización general del sistema, donde se  abordan cada uno de los componentes requeridos para su completo funcionamiento. Posteriormente se detalla la ingienría de software requerida en la etapa de conceptualización de la aplicación, se explican de forma detallada los aspectos teóricos y de implementación de la base de datos, el funcionamiento del protocolo de comunicación y por último los escenarios de juegos requeridos en las rutinas de entrenamiento ligero y clínico, y las estadísticas generadas por estos. Como herramienta de desarrollo se utilizó c\#.

% SYSTEM DESCRIPTION AND CHARACTERIZATION TO APPLY
\input{main/chapter2/section1/content.tex}
     
% SERIOUS GAME REQUIREMENTS
\input{main/chapter2/section2/content.tex}    

% USE CASE DEFINITION
\input{main/chapter2/section3/content.tex}

% USE CASE REALIZATION
\input{main/chapter2/section4/content.tex}

% DATABASE DESIGN 
\input{main/chapter2/section5/content.tex}

% DATA MANIPULATION
\input{main/chapter2/section6/content.tex}

% COMMUNICATION 
\input{main/chapter2/section7/content.tex}

% TRAINING SCENARIOS
\input{main/chapter2/section8/content.tex}

% EMG GRAPHIC
\input{main/chapter2/section9/content.tex}

% STATICAL REPORTS GENERATION
\input{main/chapter2/section10/content.tex}

\subthesischapter{Conclusiones del capítulo}
Se presentó una descripción del sistema de adquisición de datos para rehabilitación, sus componentes, características distintivas y su funcionamiento. Se identificaron y definieron los requisitos del juego  serio, tanto funcionales como no funcionales, así como los actores y casos de usos del sistema que establecieron las bases fundamentales para el desarrollo de la aplicación. Se realizó el diseño de la base de datos, abarcando tanto el modelo lógico como el físico, lo que aseguró una estructura robusta y eficiente para el almacenamiento de los datos. La manipulación de los datos se abordó de manera integral, desde la conexión con la base de datos hasta la persistencia de los resultados estadísticos. Se diseñó e implementó la comunicación con el pedal motorizado y la implementación de la interfaz gráfica para la representación de los datos EMG. Se definieron los escenarios de entrenamiento para las modalidades Ligero y Clínico, asegurando una cobertura completa de las necesidad de entrenamiento del usuario. Por último en el ámbito estadístico se desarrolló una serie de gráficos para el seguimiento de los resultados en las rutinas de entrenamiento.   
    
\end{thesischapter}

% TRAINING SCENARIOS
\begin{thesischapter}{2} {Diseño e implementación del Juego Serio}
En este capítulo se discuten los detalles de desarrollo de los aspectos citados en el capítulo anterior. Este comienza con una descripción y caracterización general del sistema, donde se  abordan cada uno de los componentes requeridos para su completo funcionamiento. Posteriormente se detalla la ingienría de software requerida en la etapa de conceptualización de la aplicación, se explican de forma detallada los aspectos teóricos y de implementación de la base de datos, el funcionamiento del protocolo de comunicación y por último los escenarios de juegos requeridos en las rutinas de entrenamiento ligero y clínico, y las estadísticas generadas por estos. Como herramienta de desarrollo se utilizó c\#.

% SYSTEM DESCRIPTION AND CHARACTERIZATION TO APPLY
\input{main/chapter2/section1/content.tex}
     
% SERIOUS GAME REQUIREMENTS
\input{main/chapter2/section2/content.tex}    

% USE CASE DEFINITION
\input{main/chapter2/section3/content.tex}

% USE CASE REALIZATION
\input{main/chapter2/section4/content.tex}

% DATABASE DESIGN 
\input{main/chapter2/section5/content.tex}

% DATA MANIPULATION
\input{main/chapter2/section6/content.tex}

% COMMUNICATION 
\input{main/chapter2/section7/content.tex}

% TRAINING SCENARIOS
\input{main/chapter2/section8/content.tex}

% EMG GRAPHIC
\input{main/chapter2/section9/content.tex}

% STATICAL REPORTS GENERATION
\input{main/chapter2/section10/content.tex}

\subthesischapter{Conclusiones del capítulo}
Se presentó una descripción del sistema de adquisición de datos para rehabilitación, sus componentes, características distintivas y su funcionamiento. Se identificaron y definieron los requisitos del juego  serio, tanto funcionales como no funcionales, así como los actores y casos de usos del sistema que establecieron las bases fundamentales para el desarrollo de la aplicación. Se realizó el diseño de la base de datos, abarcando tanto el modelo lógico como el físico, lo que aseguró una estructura robusta y eficiente para el almacenamiento de los datos. La manipulación de los datos se abordó de manera integral, desde la conexión con la base de datos hasta la persistencia de los resultados estadísticos. Se diseñó e implementó la comunicación con el pedal motorizado y la implementación de la interfaz gráfica para la representación de los datos EMG. Se definieron los escenarios de entrenamiento para las modalidades Ligero y Clínico, asegurando una cobertura completa de las necesidad de entrenamiento del usuario. Por último en el ámbito estadístico se desarrolló una serie de gráficos para el seguimiento de los resultados en las rutinas de entrenamiento.   
    
\end{thesischapter}

% EMG GRAPHIC
\begin{thesischapter}{2} {Diseño e implementación del Juego Serio}
En este capítulo se discuten los detalles de desarrollo de los aspectos citados en el capítulo anterior. Este comienza con una descripción y caracterización general del sistema, donde se  abordan cada uno de los componentes requeridos para su completo funcionamiento. Posteriormente se detalla la ingienría de software requerida en la etapa de conceptualización de la aplicación, se explican de forma detallada los aspectos teóricos y de implementación de la base de datos, el funcionamiento del protocolo de comunicación y por último los escenarios de juegos requeridos en las rutinas de entrenamiento ligero y clínico, y las estadísticas generadas por estos. Como herramienta de desarrollo se utilizó c\#.

% SYSTEM DESCRIPTION AND CHARACTERIZATION TO APPLY
\input{main/chapter2/section1/content.tex}
     
% SERIOUS GAME REQUIREMENTS
\input{main/chapter2/section2/content.tex}    

% USE CASE DEFINITION
\input{main/chapter2/section3/content.tex}

% USE CASE REALIZATION
\input{main/chapter2/section4/content.tex}

% DATABASE DESIGN 
\input{main/chapter2/section5/content.tex}

% DATA MANIPULATION
\input{main/chapter2/section6/content.tex}

% COMMUNICATION 
\input{main/chapter2/section7/content.tex}

% TRAINING SCENARIOS
\input{main/chapter2/section8/content.tex}

% EMG GRAPHIC
\input{main/chapter2/section9/content.tex}

% STATICAL REPORTS GENERATION
\input{main/chapter2/section10/content.tex}

\subthesischapter{Conclusiones del capítulo}
Se presentó una descripción del sistema de adquisición de datos para rehabilitación, sus componentes, características distintivas y su funcionamiento. Se identificaron y definieron los requisitos del juego  serio, tanto funcionales como no funcionales, así como los actores y casos de usos del sistema que establecieron las bases fundamentales para el desarrollo de la aplicación. Se realizó el diseño de la base de datos, abarcando tanto el modelo lógico como el físico, lo que aseguró una estructura robusta y eficiente para el almacenamiento de los datos. La manipulación de los datos se abordó de manera integral, desde la conexión con la base de datos hasta la persistencia de los resultados estadísticos. Se diseñó e implementó la comunicación con el pedal motorizado y la implementación de la interfaz gráfica para la representación de los datos EMG. Se definieron los escenarios de entrenamiento para las modalidades Ligero y Clínico, asegurando una cobertura completa de las necesidad de entrenamiento del usuario. Por último en el ámbito estadístico se desarrolló una serie de gráficos para el seguimiento de los resultados en las rutinas de entrenamiento.   
    
\end{thesischapter}

% STATICAL REPORTS GENERATION
\begin{thesischapter}{2} {Diseño e implementación del Juego Serio}
En este capítulo se discuten los detalles de desarrollo de los aspectos citados en el capítulo anterior. Este comienza con una descripción y caracterización general del sistema, donde se  abordan cada uno de los componentes requeridos para su completo funcionamiento. Posteriormente se detalla la ingienría de software requerida en la etapa de conceptualización de la aplicación, se explican de forma detallada los aspectos teóricos y de implementación de la base de datos, el funcionamiento del protocolo de comunicación y por último los escenarios de juegos requeridos en las rutinas de entrenamiento ligero y clínico, y las estadísticas generadas por estos. Como herramienta de desarrollo se utilizó c\#.

% SYSTEM DESCRIPTION AND CHARACTERIZATION TO APPLY
\input{main/chapter2/section1/content.tex}
     
% SERIOUS GAME REQUIREMENTS
\input{main/chapter2/section2/content.tex}    

% USE CASE DEFINITION
\input{main/chapter2/section3/content.tex}

% USE CASE REALIZATION
\input{main/chapter2/section4/content.tex}

% DATABASE DESIGN 
\input{main/chapter2/section5/content.tex}

% DATA MANIPULATION
\input{main/chapter2/section6/content.tex}

% COMMUNICATION 
\input{main/chapter2/section7/content.tex}

% TRAINING SCENARIOS
\input{main/chapter2/section8/content.tex}

% EMG GRAPHIC
\input{main/chapter2/section9/content.tex}

% STATICAL REPORTS GENERATION
\input{main/chapter2/section10/content.tex}

\subthesischapter{Conclusiones del capítulo}
Se presentó una descripción del sistema de adquisición de datos para rehabilitación, sus componentes, características distintivas y su funcionamiento. Se identificaron y definieron los requisitos del juego  serio, tanto funcionales como no funcionales, así como los actores y casos de usos del sistema que establecieron las bases fundamentales para el desarrollo de la aplicación. Se realizó el diseño de la base de datos, abarcando tanto el modelo lógico como el físico, lo que aseguró una estructura robusta y eficiente para el almacenamiento de los datos. La manipulación de los datos se abordó de manera integral, desde la conexión con la base de datos hasta la persistencia de los resultados estadísticos. Se diseñó e implementó la comunicación con el pedal motorizado y la implementación de la interfaz gráfica para la representación de los datos EMG. Se definieron los escenarios de entrenamiento para las modalidades Ligero y Clínico, asegurando una cobertura completa de las necesidad de entrenamiento del usuario. Por último en el ámbito estadístico se desarrolló una serie de gráficos para el seguimiento de los resultados en las rutinas de entrenamiento.   
    
\end{thesischapter}

\subthesischapter{Conclusiones del capítulo}
Se presentó una descripción del sistema de adquisición de datos para rehabilitación, sus componentes, características distintivas y su funcionamiento. Se identificaron y definieron los requisitos del juego  serio, tanto funcionales como no funcionales, así como los actores y casos de usos del sistema que establecieron las bases fundamentales para el desarrollo de la aplicación. Se realizó el diseño de la base de datos, abarcando tanto el modelo lógico como el físico, lo que aseguró una estructura robusta y eficiente para el almacenamiento de los datos. La manipulación de los datos se abordó de manera integral, desde la conexión con la base de datos hasta la persistencia de los resultados estadísticos. Se diseñó e implementó la comunicación con el pedal motorizado y la implementación de la interfaz gráfica para la representación de los datos EMG. Se definieron los escenarios de entrenamiento para las modalidades Ligero y Clínico, asegurando una cobertura completa de las necesidad de entrenamiento del usuario. Por último en el ámbito estadístico se desarrolló una serie de gráficos para el seguimiento de los resultados en las rutinas de entrenamiento.   
    
\end{thesischapter}
     
% SERIOUS GAME REQUIREMENTS
\begin{thesischapter}{2} {Diseño e implementación del Juego Serio}
En este capítulo se discuten los detalles de desarrollo de los aspectos citados en el capítulo anterior. Este comienza con una descripción y caracterización general del sistema, donde se  abordan cada uno de los componentes requeridos para su completo funcionamiento. Posteriormente se detalla la ingienría de software requerida en la etapa de conceptualización de la aplicación, se explican de forma detallada los aspectos teóricos y de implementación de la base de datos, el funcionamiento del protocolo de comunicación y por último los escenarios de juegos requeridos en las rutinas de entrenamiento ligero y clínico, y las estadísticas generadas por estos. Como herramienta de desarrollo se utilizó c\#.

% SYSTEM DESCRIPTION AND CHARACTERIZATION TO APPLY
\begin{thesischapter}{2} {Diseño e implementación del Juego Serio}
En este capítulo se discuten los detalles de desarrollo de los aspectos citados en el capítulo anterior. Este comienza con una descripción y caracterización general del sistema, donde se  abordan cada uno de los componentes requeridos para su completo funcionamiento. Posteriormente se detalla la ingienría de software requerida en la etapa de conceptualización de la aplicación, se explican de forma detallada los aspectos teóricos y de implementación de la base de datos, el funcionamiento del protocolo de comunicación y por último los escenarios de juegos requeridos en las rutinas de entrenamiento ligero y clínico, y las estadísticas generadas por estos. Como herramienta de desarrollo se utilizó c\#.

% SYSTEM DESCRIPTION AND CHARACTERIZATION TO APPLY
\input{main/chapter2/section1/content.tex}
     
% SERIOUS GAME REQUIREMENTS
\input{main/chapter2/section2/content.tex}    

% USE CASE DEFINITION
\input{main/chapter2/section3/content.tex}

% USE CASE REALIZATION
\input{main/chapter2/section4/content.tex}

% DATABASE DESIGN 
\input{main/chapter2/section5/content.tex}

% DATA MANIPULATION
\input{main/chapter2/section6/content.tex}

% COMMUNICATION 
\input{main/chapter2/section7/content.tex}

% TRAINING SCENARIOS
\input{main/chapter2/section8/content.tex}

% EMG GRAPHIC
\input{main/chapter2/section9/content.tex}

% STATICAL REPORTS GENERATION
\input{main/chapter2/section10/content.tex}

\subthesischapter{Conclusiones del capítulo}
Se presentó una descripción del sistema de adquisición de datos para rehabilitación, sus componentes, características distintivas y su funcionamiento. Se identificaron y definieron los requisitos del juego  serio, tanto funcionales como no funcionales, así como los actores y casos de usos del sistema que establecieron las bases fundamentales para el desarrollo de la aplicación. Se realizó el diseño de la base de datos, abarcando tanto el modelo lógico como el físico, lo que aseguró una estructura robusta y eficiente para el almacenamiento de los datos. La manipulación de los datos se abordó de manera integral, desde la conexión con la base de datos hasta la persistencia de los resultados estadísticos. Se diseñó e implementó la comunicación con el pedal motorizado y la implementación de la interfaz gráfica para la representación de los datos EMG. Se definieron los escenarios de entrenamiento para las modalidades Ligero y Clínico, asegurando una cobertura completa de las necesidad de entrenamiento del usuario. Por último en el ámbito estadístico se desarrolló una serie de gráficos para el seguimiento de los resultados en las rutinas de entrenamiento.   
    
\end{thesischapter}
     
% SERIOUS GAME REQUIREMENTS
\begin{thesischapter}{2} {Diseño e implementación del Juego Serio}
En este capítulo se discuten los detalles de desarrollo de los aspectos citados en el capítulo anterior. Este comienza con una descripción y caracterización general del sistema, donde se  abordan cada uno de los componentes requeridos para su completo funcionamiento. Posteriormente se detalla la ingienría de software requerida en la etapa de conceptualización de la aplicación, se explican de forma detallada los aspectos teóricos y de implementación de la base de datos, el funcionamiento del protocolo de comunicación y por último los escenarios de juegos requeridos en las rutinas de entrenamiento ligero y clínico, y las estadísticas generadas por estos. Como herramienta de desarrollo se utilizó c\#.

% SYSTEM DESCRIPTION AND CHARACTERIZATION TO APPLY
\input{main/chapter2/section1/content.tex}
     
% SERIOUS GAME REQUIREMENTS
\input{main/chapter2/section2/content.tex}    

% USE CASE DEFINITION
\input{main/chapter2/section3/content.tex}

% USE CASE REALIZATION
\input{main/chapter2/section4/content.tex}

% DATABASE DESIGN 
\input{main/chapter2/section5/content.tex}

% DATA MANIPULATION
\input{main/chapter2/section6/content.tex}

% COMMUNICATION 
\input{main/chapter2/section7/content.tex}

% TRAINING SCENARIOS
\input{main/chapter2/section8/content.tex}

% EMG GRAPHIC
\input{main/chapter2/section9/content.tex}

% STATICAL REPORTS GENERATION
\input{main/chapter2/section10/content.tex}

\subthesischapter{Conclusiones del capítulo}
Se presentó una descripción del sistema de adquisición de datos para rehabilitación, sus componentes, características distintivas y su funcionamiento. Se identificaron y definieron los requisitos del juego  serio, tanto funcionales como no funcionales, así como los actores y casos de usos del sistema que establecieron las bases fundamentales para el desarrollo de la aplicación. Se realizó el diseño de la base de datos, abarcando tanto el modelo lógico como el físico, lo que aseguró una estructura robusta y eficiente para el almacenamiento de los datos. La manipulación de los datos se abordó de manera integral, desde la conexión con la base de datos hasta la persistencia de los resultados estadísticos. Se diseñó e implementó la comunicación con el pedal motorizado y la implementación de la interfaz gráfica para la representación de los datos EMG. Se definieron los escenarios de entrenamiento para las modalidades Ligero y Clínico, asegurando una cobertura completa de las necesidad de entrenamiento del usuario. Por último en el ámbito estadístico se desarrolló una serie de gráficos para el seguimiento de los resultados en las rutinas de entrenamiento.   
    
\end{thesischapter}    

% USE CASE DEFINITION
\begin{thesischapter}{2} {Diseño e implementación del Juego Serio}
En este capítulo se discuten los detalles de desarrollo de los aspectos citados en el capítulo anterior. Este comienza con una descripción y caracterización general del sistema, donde se  abordan cada uno de los componentes requeridos para su completo funcionamiento. Posteriormente se detalla la ingienría de software requerida en la etapa de conceptualización de la aplicación, se explican de forma detallada los aspectos teóricos y de implementación de la base de datos, el funcionamiento del protocolo de comunicación y por último los escenarios de juegos requeridos en las rutinas de entrenamiento ligero y clínico, y las estadísticas generadas por estos. Como herramienta de desarrollo se utilizó c\#.

% SYSTEM DESCRIPTION AND CHARACTERIZATION TO APPLY
\input{main/chapter2/section1/content.tex}
     
% SERIOUS GAME REQUIREMENTS
\input{main/chapter2/section2/content.tex}    

% USE CASE DEFINITION
\input{main/chapter2/section3/content.tex}

% USE CASE REALIZATION
\input{main/chapter2/section4/content.tex}

% DATABASE DESIGN 
\input{main/chapter2/section5/content.tex}

% DATA MANIPULATION
\input{main/chapter2/section6/content.tex}

% COMMUNICATION 
\input{main/chapter2/section7/content.tex}

% TRAINING SCENARIOS
\input{main/chapter2/section8/content.tex}

% EMG GRAPHIC
\input{main/chapter2/section9/content.tex}

% STATICAL REPORTS GENERATION
\input{main/chapter2/section10/content.tex}

\subthesischapter{Conclusiones del capítulo}
Se presentó una descripción del sistema de adquisición de datos para rehabilitación, sus componentes, características distintivas y su funcionamiento. Se identificaron y definieron los requisitos del juego  serio, tanto funcionales como no funcionales, así como los actores y casos de usos del sistema que establecieron las bases fundamentales para el desarrollo de la aplicación. Se realizó el diseño de la base de datos, abarcando tanto el modelo lógico como el físico, lo que aseguró una estructura robusta y eficiente para el almacenamiento de los datos. La manipulación de los datos se abordó de manera integral, desde la conexión con la base de datos hasta la persistencia de los resultados estadísticos. Se diseñó e implementó la comunicación con el pedal motorizado y la implementación de la interfaz gráfica para la representación de los datos EMG. Se definieron los escenarios de entrenamiento para las modalidades Ligero y Clínico, asegurando una cobertura completa de las necesidad de entrenamiento del usuario. Por último en el ámbito estadístico se desarrolló una serie de gráficos para el seguimiento de los resultados en las rutinas de entrenamiento.   
    
\end{thesischapter}

% USE CASE REALIZATION
\begin{thesischapter}{2} {Diseño e implementación del Juego Serio}
En este capítulo se discuten los detalles de desarrollo de los aspectos citados en el capítulo anterior. Este comienza con una descripción y caracterización general del sistema, donde se  abordan cada uno de los componentes requeridos para su completo funcionamiento. Posteriormente se detalla la ingienría de software requerida en la etapa de conceptualización de la aplicación, se explican de forma detallada los aspectos teóricos y de implementación de la base de datos, el funcionamiento del protocolo de comunicación y por último los escenarios de juegos requeridos en las rutinas de entrenamiento ligero y clínico, y las estadísticas generadas por estos. Como herramienta de desarrollo se utilizó c\#.

% SYSTEM DESCRIPTION AND CHARACTERIZATION TO APPLY
\input{main/chapter2/section1/content.tex}
     
% SERIOUS GAME REQUIREMENTS
\input{main/chapter2/section2/content.tex}    

% USE CASE DEFINITION
\input{main/chapter2/section3/content.tex}

% USE CASE REALIZATION
\input{main/chapter2/section4/content.tex}

% DATABASE DESIGN 
\input{main/chapter2/section5/content.tex}

% DATA MANIPULATION
\input{main/chapter2/section6/content.tex}

% COMMUNICATION 
\input{main/chapter2/section7/content.tex}

% TRAINING SCENARIOS
\input{main/chapter2/section8/content.tex}

% EMG GRAPHIC
\input{main/chapter2/section9/content.tex}

% STATICAL REPORTS GENERATION
\input{main/chapter2/section10/content.tex}

\subthesischapter{Conclusiones del capítulo}
Se presentó una descripción del sistema de adquisición de datos para rehabilitación, sus componentes, características distintivas y su funcionamiento. Se identificaron y definieron los requisitos del juego  serio, tanto funcionales como no funcionales, así como los actores y casos de usos del sistema que establecieron las bases fundamentales para el desarrollo de la aplicación. Se realizó el diseño de la base de datos, abarcando tanto el modelo lógico como el físico, lo que aseguró una estructura robusta y eficiente para el almacenamiento de los datos. La manipulación de los datos se abordó de manera integral, desde la conexión con la base de datos hasta la persistencia de los resultados estadísticos. Se diseñó e implementó la comunicación con el pedal motorizado y la implementación de la interfaz gráfica para la representación de los datos EMG. Se definieron los escenarios de entrenamiento para las modalidades Ligero y Clínico, asegurando una cobertura completa de las necesidad de entrenamiento del usuario. Por último en el ámbito estadístico se desarrolló una serie de gráficos para el seguimiento de los resultados en las rutinas de entrenamiento.   
    
\end{thesischapter}

% DATABASE DESIGN 
\begin{thesischapter}{2} {Diseño e implementación del Juego Serio}
En este capítulo se discuten los detalles de desarrollo de los aspectos citados en el capítulo anterior. Este comienza con una descripción y caracterización general del sistema, donde se  abordan cada uno de los componentes requeridos para su completo funcionamiento. Posteriormente se detalla la ingienría de software requerida en la etapa de conceptualización de la aplicación, se explican de forma detallada los aspectos teóricos y de implementación de la base de datos, el funcionamiento del protocolo de comunicación y por último los escenarios de juegos requeridos en las rutinas de entrenamiento ligero y clínico, y las estadísticas generadas por estos. Como herramienta de desarrollo se utilizó c\#.

% SYSTEM DESCRIPTION AND CHARACTERIZATION TO APPLY
\input{main/chapter2/section1/content.tex}
     
% SERIOUS GAME REQUIREMENTS
\input{main/chapter2/section2/content.tex}    

% USE CASE DEFINITION
\input{main/chapter2/section3/content.tex}

% USE CASE REALIZATION
\input{main/chapter2/section4/content.tex}

% DATABASE DESIGN 
\input{main/chapter2/section5/content.tex}

% DATA MANIPULATION
\input{main/chapter2/section6/content.tex}

% COMMUNICATION 
\input{main/chapter2/section7/content.tex}

% TRAINING SCENARIOS
\input{main/chapter2/section8/content.tex}

% EMG GRAPHIC
\input{main/chapter2/section9/content.tex}

% STATICAL REPORTS GENERATION
\input{main/chapter2/section10/content.tex}

\subthesischapter{Conclusiones del capítulo}
Se presentó una descripción del sistema de adquisición de datos para rehabilitación, sus componentes, características distintivas y su funcionamiento. Se identificaron y definieron los requisitos del juego  serio, tanto funcionales como no funcionales, así como los actores y casos de usos del sistema que establecieron las bases fundamentales para el desarrollo de la aplicación. Se realizó el diseño de la base de datos, abarcando tanto el modelo lógico como el físico, lo que aseguró una estructura robusta y eficiente para el almacenamiento de los datos. La manipulación de los datos se abordó de manera integral, desde la conexión con la base de datos hasta la persistencia de los resultados estadísticos. Se diseñó e implementó la comunicación con el pedal motorizado y la implementación de la interfaz gráfica para la representación de los datos EMG. Se definieron los escenarios de entrenamiento para las modalidades Ligero y Clínico, asegurando una cobertura completa de las necesidad de entrenamiento del usuario. Por último en el ámbito estadístico se desarrolló una serie de gráficos para el seguimiento de los resultados en las rutinas de entrenamiento.   
    
\end{thesischapter}

% DATA MANIPULATION
\begin{thesischapter}{2} {Diseño e implementación del Juego Serio}
En este capítulo se discuten los detalles de desarrollo de los aspectos citados en el capítulo anterior. Este comienza con una descripción y caracterización general del sistema, donde se  abordan cada uno de los componentes requeridos para su completo funcionamiento. Posteriormente se detalla la ingienría de software requerida en la etapa de conceptualización de la aplicación, se explican de forma detallada los aspectos teóricos y de implementación de la base de datos, el funcionamiento del protocolo de comunicación y por último los escenarios de juegos requeridos en las rutinas de entrenamiento ligero y clínico, y las estadísticas generadas por estos. Como herramienta de desarrollo se utilizó c\#.

% SYSTEM DESCRIPTION AND CHARACTERIZATION TO APPLY
\input{main/chapter2/section1/content.tex}
     
% SERIOUS GAME REQUIREMENTS
\input{main/chapter2/section2/content.tex}    

% USE CASE DEFINITION
\input{main/chapter2/section3/content.tex}

% USE CASE REALIZATION
\input{main/chapter2/section4/content.tex}

% DATABASE DESIGN 
\input{main/chapter2/section5/content.tex}

% DATA MANIPULATION
\input{main/chapter2/section6/content.tex}

% COMMUNICATION 
\input{main/chapter2/section7/content.tex}

% TRAINING SCENARIOS
\input{main/chapter2/section8/content.tex}

% EMG GRAPHIC
\input{main/chapter2/section9/content.tex}

% STATICAL REPORTS GENERATION
\input{main/chapter2/section10/content.tex}

\subthesischapter{Conclusiones del capítulo}
Se presentó una descripción del sistema de adquisición de datos para rehabilitación, sus componentes, características distintivas y su funcionamiento. Se identificaron y definieron los requisitos del juego  serio, tanto funcionales como no funcionales, así como los actores y casos de usos del sistema que establecieron las bases fundamentales para el desarrollo de la aplicación. Se realizó el diseño de la base de datos, abarcando tanto el modelo lógico como el físico, lo que aseguró una estructura robusta y eficiente para el almacenamiento de los datos. La manipulación de los datos se abordó de manera integral, desde la conexión con la base de datos hasta la persistencia de los resultados estadísticos. Se diseñó e implementó la comunicación con el pedal motorizado y la implementación de la interfaz gráfica para la representación de los datos EMG. Se definieron los escenarios de entrenamiento para las modalidades Ligero y Clínico, asegurando una cobertura completa de las necesidad de entrenamiento del usuario. Por último en el ámbito estadístico se desarrolló una serie de gráficos para el seguimiento de los resultados en las rutinas de entrenamiento.   
    
\end{thesischapter}

% COMMUNICATION 
\begin{thesischapter}{2} {Diseño e implementación del Juego Serio}
En este capítulo se discuten los detalles de desarrollo de los aspectos citados en el capítulo anterior. Este comienza con una descripción y caracterización general del sistema, donde se  abordan cada uno de los componentes requeridos para su completo funcionamiento. Posteriormente se detalla la ingienría de software requerida en la etapa de conceptualización de la aplicación, se explican de forma detallada los aspectos teóricos y de implementación de la base de datos, el funcionamiento del protocolo de comunicación y por último los escenarios de juegos requeridos en las rutinas de entrenamiento ligero y clínico, y las estadísticas generadas por estos. Como herramienta de desarrollo se utilizó c\#.

% SYSTEM DESCRIPTION AND CHARACTERIZATION TO APPLY
\input{main/chapter2/section1/content.tex}
     
% SERIOUS GAME REQUIREMENTS
\input{main/chapter2/section2/content.tex}    

% USE CASE DEFINITION
\input{main/chapter2/section3/content.tex}

% USE CASE REALIZATION
\input{main/chapter2/section4/content.tex}

% DATABASE DESIGN 
\input{main/chapter2/section5/content.tex}

% DATA MANIPULATION
\input{main/chapter2/section6/content.tex}

% COMMUNICATION 
\input{main/chapter2/section7/content.tex}

% TRAINING SCENARIOS
\input{main/chapter2/section8/content.tex}

% EMG GRAPHIC
\input{main/chapter2/section9/content.tex}

% STATICAL REPORTS GENERATION
\input{main/chapter2/section10/content.tex}

\subthesischapter{Conclusiones del capítulo}
Se presentó una descripción del sistema de adquisición de datos para rehabilitación, sus componentes, características distintivas y su funcionamiento. Se identificaron y definieron los requisitos del juego  serio, tanto funcionales como no funcionales, así como los actores y casos de usos del sistema que establecieron las bases fundamentales para el desarrollo de la aplicación. Se realizó el diseño de la base de datos, abarcando tanto el modelo lógico como el físico, lo que aseguró una estructura robusta y eficiente para el almacenamiento de los datos. La manipulación de los datos se abordó de manera integral, desde la conexión con la base de datos hasta la persistencia de los resultados estadísticos. Se diseñó e implementó la comunicación con el pedal motorizado y la implementación de la interfaz gráfica para la representación de los datos EMG. Se definieron los escenarios de entrenamiento para las modalidades Ligero y Clínico, asegurando una cobertura completa de las necesidad de entrenamiento del usuario. Por último en el ámbito estadístico se desarrolló una serie de gráficos para el seguimiento de los resultados en las rutinas de entrenamiento.   
    
\end{thesischapter}

% TRAINING SCENARIOS
\begin{thesischapter}{2} {Diseño e implementación del Juego Serio}
En este capítulo se discuten los detalles de desarrollo de los aspectos citados en el capítulo anterior. Este comienza con una descripción y caracterización general del sistema, donde se  abordan cada uno de los componentes requeridos para su completo funcionamiento. Posteriormente se detalla la ingienría de software requerida en la etapa de conceptualización de la aplicación, se explican de forma detallada los aspectos teóricos y de implementación de la base de datos, el funcionamiento del protocolo de comunicación y por último los escenarios de juegos requeridos en las rutinas de entrenamiento ligero y clínico, y las estadísticas generadas por estos. Como herramienta de desarrollo se utilizó c\#.

% SYSTEM DESCRIPTION AND CHARACTERIZATION TO APPLY
\input{main/chapter2/section1/content.tex}
     
% SERIOUS GAME REQUIREMENTS
\input{main/chapter2/section2/content.tex}    

% USE CASE DEFINITION
\input{main/chapter2/section3/content.tex}

% USE CASE REALIZATION
\input{main/chapter2/section4/content.tex}

% DATABASE DESIGN 
\input{main/chapter2/section5/content.tex}

% DATA MANIPULATION
\input{main/chapter2/section6/content.tex}

% COMMUNICATION 
\input{main/chapter2/section7/content.tex}

% TRAINING SCENARIOS
\input{main/chapter2/section8/content.tex}

% EMG GRAPHIC
\input{main/chapter2/section9/content.tex}

% STATICAL REPORTS GENERATION
\input{main/chapter2/section10/content.tex}

\subthesischapter{Conclusiones del capítulo}
Se presentó una descripción del sistema de adquisición de datos para rehabilitación, sus componentes, características distintivas y su funcionamiento. Se identificaron y definieron los requisitos del juego  serio, tanto funcionales como no funcionales, así como los actores y casos de usos del sistema que establecieron las bases fundamentales para el desarrollo de la aplicación. Se realizó el diseño de la base de datos, abarcando tanto el modelo lógico como el físico, lo que aseguró una estructura robusta y eficiente para el almacenamiento de los datos. La manipulación de los datos se abordó de manera integral, desde la conexión con la base de datos hasta la persistencia de los resultados estadísticos. Se diseñó e implementó la comunicación con el pedal motorizado y la implementación de la interfaz gráfica para la representación de los datos EMG. Se definieron los escenarios de entrenamiento para las modalidades Ligero y Clínico, asegurando una cobertura completa de las necesidad de entrenamiento del usuario. Por último en el ámbito estadístico se desarrolló una serie de gráficos para el seguimiento de los resultados en las rutinas de entrenamiento.   
    
\end{thesischapter}

% EMG GRAPHIC
\begin{thesischapter}{2} {Diseño e implementación del Juego Serio}
En este capítulo se discuten los detalles de desarrollo de los aspectos citados en el capítulo anterior. Este comienza con una descripción y caracterización general del sistema, donde se  abordan cada uno de los componentes requeridos para su completo funcionamiento. Posteriormente se detalla la ingienría de software requerida en la etapa de conceptualización de la aplicación, se explican de forma detallada los aspectos teóricos y de implementación de la base de datos, el funcionamiento del protocolo de comunicación y por último los escenarios de juegos requeridos en las rutinas de entrenamiento ligero y clínico, y las estadísticas generadas por estos. Como herramienta de desarrollo se utilizó c\#.

% SYSTEM DESCRIPTION AND CHARACTERIZATION TO APPLY
\input{main/chapter2/section1/content.tex}
     
% SERIOUS GAME REQUIREMENTS
\input{main/chapter2/section2/content.tex}    

% USE CASE DEFINITION
\input{main/chapter2/section3/content.tex}

% USE CASE REALIZATION
\input{main/chapter2/section4/content.tex}

% DATABASE DESIGN 
\input{main/chapter2/section5/content.tex}

% DATA MANIPULATION
\input{main/chapter2/section6/content.tex}

% COMMUNICATION 
\input{main/chapter2/section7/content.tex}

% TRAINING SCENARIOS
\input{main/chapter2/section8/content.tex}

% EMG GRAPHIC
\input{main/chapter2/section9/content.tex}

% STATICAL REPORTS GENERATION
\input{main/chapter2/section10/content.tex}

\subthesischapter{Conclusiones del capítulo}
Se presentó una descripción del sistema de adquisición de datos para rehabilitación, sus componentes, características distintivas y su funcionamiento. Se identificaron y definieron los requisitos del juego  serio, tanto funcionales como no funcionales, así como los actores y casos de usos del sistema que establecieron las bases fundamentales para el desarrollo de la aplicación. Se realizó el diseño de la base de datos, abarcando tanto el modelo lógico como el físico, lo que aseguró una estructura robusta y eficiente para el almacenamiento de los datos. La manipulación de los datos se abordó de manera integral, desde la conexión con la base de datos hasta la persistencia de los resultados estadísticos. Se diseñó e implementó la comunicación con el pedal motorizado y la implementación de la interfaz gráfica para la representación de los datos EMG. Se definieron los escenarios de entrenamiento para las modalidades Ligero y Clínico, asegurando una cobertura completa de las necesidad de entrenamiento del usuario. Por último en el ámbito estadístico se desarrolló una serie de gráficos para el seguimiento de los resultados en las rutinas de entrenamiento.   
    
\end{thesischapter}

% STATICAL REPORTS GENERATION
\begin{thesischapter}{2} {Diseño e implementación del Juego Serio}
En este capítulo se discuten los detalles de desarrollo de los aspectos citados en el capítulo anterior. Este comienza con una descripción y caracterización general del sistema, donde se  abordan cada uno de los componentes requeridos para su completo funcionamiento. Posteriormente se detalla la ingienría de software requerida en la etapa de conceptualización de la aplicación, se explican de forma detallada los aspectos teóricos y de implementación de la base de datos, el funcionamiento del protocolo de comunicación y por último los escenarios de juegos requeridos en las rutinas de entrenamiento ligero y clínico, y las estadísticas generadas por estos. Como herramienta de desarrollo se utilizó c\#.

% SYSTEM DESCRIPTION AND CHARACTERIZATION TO APPLY
\input{main/chapter2/section1/content.tex}
     
% SERIOUS GAME REQUIREMENTS
\input{main/chapter2/section2/content.tex}    

% USE CASE DEFINITION
\input{main/chapter2/section3/content.tex}

% USE CASE REALIZATION
\input{main/chapter2/section4/content.tex}

% DATABASE DESIGN 
\input{main/chapter2/section5/content.tex}

% DATA MANIPULATION
\input{main/chapter2/section6/content.tex}

% COMMUNICATION 
\input{main/chapter2/section7/content.tex}

% TRAINING SCENARIOS
\input{main/chapter2/section8/content.tex}

% EMG GRAPHIC
\input{main/chapter2/section9/content.tex}

% STATICAL REPORTS GENERATION
\input{main/chapter2/section10/content.tex}

\subthesischapter{Conclusiones del capítulo}
Se presentó una descripción del sistema de adquisición de datos para rehabilitación, sus componentes, características distintivas y su funcionamiento. Se identificaron y definieron los requisitos del juego  serio, tanto funcionales como no funcionales, así como los actores y casos de usos del sistema que establecieron las bases fundamentales para el desarrollo de la aplicación. Se realizó el diseño de la base de datos, abarcando tanto el modelo lógico como el físico, lo que aseguró una estructura robusta y eficiente para el almacenamiento de los datos. La manipulación de los datos se abordó de manera integral, desde la conexión con la base de datos hasta la persistencia de los resultados estadísticos. Se diseñó e implementó la comunicación con el pedal motorizado y la implementación de la interfaz gráfica para la representación de los datos EMG. Se definieron los escenarios de entrenamiento para las modalidades Ligero y Clínico, asegurando una cobertura completa de las necesidad de entrenamiento del usuario. Por último en el ámbito estadístico se desarrolló una serie de gráficos para el seguimiento de los resultados en las rutinas de entrenamiento.   
    
\end{thesischapter}

\subthesischapter{Conclusiones del capítulo}
Se presentó una descripción del sistema de adquisición de datos para rehabilitación, sus componentes, características distintivas y su funcionamiento. Se identificaron y definieron los requisitos del juego  serio, tanto funcionales como no funcionales, así como los actores y casos de usos del sistema que establecieron las bases fundamentales para el desarrollo de la aplicación. Se realizó el diseño de la base de datos, abarcando tanto el modelo lógico como el físico, lo que aseguró una estructura robusta y eficiente para el almacenamiento de los datos. La manipulación de los datos se abordó de manera integral, desde la conexión con la base de datos hasta la persistencia de los resultados estadísticos. Se diseñó e implementó la comunicación con el pedal motorizado y la implementación de la interfaz gráfica para la representación de los datos EMG. Se definieron los escenarios de entrenamiento para las modalidades Ligero y Clínico, asegurando una cobertura completa de las necesidad de entrenamiento del usuario. Por último en el ámbito estadístico se desarrolló una serie de gráficos para el seguimiento de los resultados en las rutinas de entrenamiento.   
    
\end{thesischapter}    

% USE CASE DEFINITION
\begin{thesischapter}{2} {Diseño e implementación del Juego Serio}
En este capítulo se discuten los detalles de desarrollo de los aspectos citados en el capítulo anterior. Este comienza con una descripción y caracterización general del sistema, donde se  abordan cada uno de los componentes requeridos para su completo funcionamiento. Posteriormente se detalla la ingienría de software requerida en la etapa de conceptualización de la aplicación, se explican de forma detallada los aspectos teóricos y de implementación de la base de datos, el funcionamiento del protocolo de comunicación y por último los escenarios de juegos requeridos en las rutinas de entrenamiento ligero y clínico, y las estadísticas generadas por estos. Como herramienta de desarrollo se utilizó c\#.

% SYSTEM DESCRIPTION AND CHARACTERIZATION TO APPLY
\begin{thesischapter}{2} {Diseño e implementación del Juego Serio}
En este capítulo se discuten los detalles de desarrollo de los aspectos citados en el capítulo anterior. Este comienza con una descripción y caracterización general del sistema, donde se  abordan cada uno de los componentes requeridos para su completo funcionamiento. Posteriormente se detalla la ingienría de software requerida en la etapa de conceptualización de la aplicación, se explican de forma detallada los aspectos teóricos y de implementación de la base de datos, el funcionamiento del protocolo de comunicación y por último los escenarios de juegos requeridos en las rutinas de entrenamiento ligero y clínico, y las estadísticas generadas por estos. Como herramienta de desarrollo se utilizó c\#.

% SYSTEM DESCRIPTION AND CHARACTERIZATION TO APPLY
\input{main/chapter2/section1/content.tex}
     
% SERIOUS GAME REQUIREMENTS
\input{main/chapter2/section2/content.tex}    

% USE CASE DEFINITION
\input{main/chapter2/section3/content.tex}

% USE CASE REALIZATION
\input{main/chapter2/section4/content.tex}

% DATABASE DESIGN 
\input{main/chapter2/section5/content.tex}

% DATA MANIPULATION
\input{main/chapter2/section6/content.tex}

% COMMUNICATION 
\input{main/chapter2/section7/content.tex}

% TRAINING SCENARIOS
\input{main/chapter2/section8/content.tex}

% EMG GRAPHIC
\input{main/chapter2/section9/content.tex}

% STATICAL REPORTS GENERATION
\input{main/chapter2/section10/content.tex}

\subthesischapter{Conclusiones del capítulo}
Se presentó una descripción del sistema de adquisición de datos para rehabilitación, sus componentes, características distintivas y su funcionamiento. Se identificaron y definieron los requisitos del juego  serio, tanto funcionales como no funcionales, así como los actores y casos de usos del sistema que establecieron las bases fundamentales para el desarrollo de la aplicación. Se realizó el diseño de la base de datos, abarcando tanto el modelo lógico como el físico, lo que aseguró una estructura robusta y eficiente para el almacenamiento de los datos. La manipulación de los datos se abordó de manera integral, desde la conexión con la base de datos hasta la persistencia de los resultados estadísticos. Se diseñó e implementó la comunicación con el pedal motorizado y la implementación de la interfaz gráfica para la representación de los datos EMG. Se definieron los escenarios de entrenamiento para las modalidades Ligero y Clínico, asegurando una cobertura completa de las necesidad de entrenamiento del usuario. Por último en el ámbito estadístico se desarrolló una serie de gráficos para el seguimiento de los resultados en las rutinas de entrenamiento.   
    
\end{thesischapter}
     
% SERIOUS GAME REQUIREMENTS
\begin{thesischapter}{2} {Diseño e implementación del Juego Serio}
En este capítulo se discuten los detalles de desarrollo de los aspectos citados en el capítulo anterior. Este comienza con una descripción y caracterización general del sistema, donde se  abordan cada uno de los componentes requeridos para su completo funcionamiento. Posteriormente se detalla la ingienría de software requerida en la etapa de conceptualización de la aplicación, se explican de forma detallada los aspectos teóricos y de implementación de la base de datos, el funcionamiento del protocolo de comunicación y por último los escenarios de juegos requeridos en las rutinas de entrenamiento ligero y clínico, y las estadísticas generadas por estos. Como herramienta de desarrollo se utilizó c\#.

% SYSTEM DESCRIPTION AND CHARACTERIZATION TO APPLY
\input{main/chapter2/section1/content.tex}
     
% SERIOUS GAME REQUIREMENTS
\input{main/chapter2/section2/content.tex}    

% USE CASE DEFINITION
\input{main/chapter2/section3/content.tex}

% USE CASE REALIZATION
\input{main/chapter2/section4/content.tex}

% DATABASE DESIGN 
\input{main/chapter2/section5/content.tex}

% DATA MANIPULATION
\input{main/chapter2/section6/content.tex}

% COMMUNICATION 
\input{main/chapter2/section7/content.tex}

% TRAINING SCENARIOS
\input{main/chapter2/section8/content.tex}

% EMG GRAPHIC
\input{main/chapter2/section9/content.tex}

% STATICAL REPORTS GENERATION
\input{main/chapter2/section10/content.tex}

\subthesischapter{Conclusiones del capítulo}
Se presentó una descripción del sistema de adquisición de datos para rehabilitación, sus componentes, características distintivas y su funcionamiento. Se identificaron y definieron los requisitos del juego  serio, tanto funcionales como no funcionales, así como los actores y casos de usos del sistema que establecieron las bases fundamentales para el desarrollo de la aplicación. Se realizó el diseño de la base de datos, abarcando tanto el modelo lógico como el físico, lo que aseguró una estructura robusta y eficiente para el almacenamiento de los datos. La manipulación de los datos se abordó de manera integral, desde la conexión con la base de datos hasta la persistencia de los resultados estadísticos. Se diseñó e implementó la comunicación con el pedal motorizado y la implementación de la interfaz gráfica para la representación de los datos EMG. Se definieron los escenarios de entrenamiento para las modalidades Ligero y Clínico, asegurando una cobertura completa de las necesidad de entrenamiento del usuario. Por último en el ámbito estadístico se desarrolló una serie de gráficos para el seguimiento de los resultados en las rutinas de entrenamiento.   
    
\end{thesischapter}    

% USE CASE DEFINITION
\begin{thesischapter}{2} {Diseño e implementación del Juego Serio}
En este capítulo se discuten los detalles de desarrollo de los aspectos citados en el capítulo anterior. Este comienza con una descripción y caracterización general del sistema, donde se  abordan cada uno de los componentes requeridos para su completo funcionamiento. Posteriormente se detalla la ingienría de software requerida en la etapa de conceptualización de la aplicación, se explican de forma detallada los aspectos teóricos y de implementación de la base de datos, el funcionamiento del protocolo de comunicación y por último los escenarios de juegos requeridos en las rutinas de entrenamiento ligero y clínico, y las estadísticas generadas por estos. Como herramienta de desarrollo se utilizó c\#.

% SYSTEM DESCRIPTION AND CHARACTERIZATION TO APPLY
\input{main/chapter2/section1/content.tex}
     
% SERIOUS GAME REQUIREMENTS
\input{main/chapter2/section2/content.tex}    

% USE CASE DEFINITION
\input{main/chapter2/section3/content.tex}

% USE CASE REALIZATION
\input{main/chapter2/section4/content.tex}

% DATABASE DESIGN 
\input{main/chapter2/section5/content.tex}

% DATA MANIPULATION
\input{main/chapter2/section6/content.tex}

% COMMUNICATION 
\input{main/chapter2/section7/content.tex}

% TRAINING SCENARIOS
\input{main/chapter2/section8/content.tex}

% EMG GRAPHIC
\input{main/chapter2/section9/content.tex}

% STATICAL REPORTS GENERATION
\input{main/chapter2/section10/content.tex}

\subthesischapter{Conclusiones del capítulo}
Se presentó una descripción del sistema de adquisición de datos para rehabilitación, sus componentes, características distintivas y su funcionamiento. Se identificaron y definieron los requisitos del juego  serio, tanto funcionales como no funcionales, así como los actores y casos de usos del sistema que establecieron las bases fundamentales para el desarrollo de la aplicación. Se realizó el diseño de la base de datos, abarcando tanto el modelo lógico como el físico, lo que aseguró una estructura robusta y eficiente para el almacenamiento de los datos. La manipulación de los datos se abordó de manera integral, desde la conexión con la base de datos hasta la persistencia de los resultados estadísticos. Se diseñó e implementó la comunicación con el pedal motorizado y la implementación de la interfaz gráfica para la representación de los datos EMG. Se definieron los escenarios de entrenamiento para las modalidades Ligero y Clínico, asegurando una cobertura completa de las necesidad de entrenamiento del usuario. Por último en el ámbito estadístico se desarrolló una serie de gráficos para el seguimiento de los resultados en las rutinas de entrenamiento.   
    
\end{thesischapter}

% USE CASE REALIZATION
\begin{thesischapter}{2} {Diseño e implementación del Juego Serio}
En este capítulo se discuten los detalles de desarrollo de los aspectos citados en el capítulo anterior. Este comienza con una descripción y caracterización general del sistema, donde se  abordan cada uno de los componentes requeridos para su completo funcionamiento. Posteriormente se detalla la ingienría de software requerida en la etapa de conceptualización de la aplicación, se explican de forma detallada los aspectos teóricos y de implementación de la base de datos, el funcionamiento del protocolo de comunicación y por último los escenarios de juegos requeridos en las rutinas de entrenamiento ligero y clínico, y las estadísticas generadas por estos. Como herramienta de desarrollo se utilizó c\#.

% SYSTEM DESCRIPTION AND CHARACTERIZATION TO APPLY
\input{main/chapter2/section1/content.tex}
     
% SERIOUS GAME REQUIREMENTS
\input{main/chapter2/section2/content.tex}    

% USE CASE DEFINITION
\input{main/chapter2/section3/content.tex}

% USE CASE REALIZATION
\input{main/chapter2/section4/content.tex}

% DATABASE DESIGN 
\input{main/chapter2/section5/content.tex}

% DATA MANIPULATION
\input{main/chapter2/section6/content.tex}

% COMMUNICATION 
\input{main/chapter2/section7/content.tex}

% TRAINING SCENARIOS
\input{main/chapter2/section8/content.tex}

% EMG GRAPHIC
\input{main/chapter2/section9/content.tex}

% STATICAL REPORTS GENERATION
\input{main/chapter2/section10/content.tex}

\subthesischapter{Conclusiones del capítulo}
Se presentó una descripción del sistema de adquisición de datos para rehabilitación, sus componentes, características distintivas y su funcionamiento. Se identificaron y definieron los requisitos del juego  serio, tanto funcionales como no funcionales, así como los actores y casos de usos del sistema que establecieron las bases fundamentales para el desarrollo de la aplicación. Se realizó el diseño de la base de datos, abarcando tanto el modelo lógico como el físico, lo que aseguró una estructura robusta y eficiente para el almacenamiento de los datos. La manipulación de los datos se abordó de manera integral, desde la conexión con la base de datos hasta la persistencia de los resultados estadísticos. Se diseñó e implementó la comunicación con el pedal motorizado y la implementación de la interfaz gráfica para la representación de los datos EMG. Se definieron los escenarios de entrenamiento para las modalidades Ligero y Clínico, asegurando una cobertura completa de las necesidad de entrenamiento del usuario. Por último en el ámbito estadístico se desarrolló una serie de gráficos para el seguimiento de los resultados en las rutinas de entrenamiento.   
    
\end{thesischapter}

% DATABASE DESIGN 
\begin{thesischapter}{2} {Diseño e implementación del Juego Serio}
En este capítulo se discuten los detalles de desarrollo de los aspectos citados en el capítulo anterior. Este comienza con una descripción y caracterización general del sistema, donde se  abordan cada uno de los componentes requeridos para su completo funcionamiento. Posteriormente se detalla la ingienría de software requerida en la etapa de conceptualización de la aplicación, se explican de forma detallada los aspectos teóricos y de implementación de la base de datos, el funcionamiento del protocolo de comunicación y por último los escenarios de juegos requeridos en las rutinas de entrenamiento ligero y clínico, y las estadísticas generadas por estos. Como herramienta de desarrollo se utilizó c\#.

% SYSTEM DESCRIPTION AND CHARACTERIZATION TO APPLY
\input{main/chapter2/section1/content.tex}
     
% SERIOUS GAME REQUIREMENTS
\input{main/chapter2/section2/content.tex}    

% USE CASE DEFINITION
\input{main/chapter2/section3/content.tex}

% USE CASE REALIZATION
\input{main/chapter2/section4/content.tex}

% DATABASE DESIGN 
\input{main/chapter2/section5/content.tex}

% DATA MANIPULATION
\input{main/chapter2/section6/content.tex}

% COMMUNICATION 
\input{main/chapter2/section7/content.tex}

% TRAINING SCENARIOS
\input{main/chapter2/section8/content.tex}

% EMG GRAPHIC
\input{main/chapter2/section9/content.tex}

% STATICAL REPORTS GENERATION
\input{main/chapter2/section10/content.tex}

\subthesischapter{Conclusiones del capítulo}
Se presentó una descripción del sistema de adquisición de datos para rehabilitación, sus componentes, características distintivas y su funcionamiento. Se identificaron y definieron los requisitos del juego  serio, tanto funcionales como no funcionales, así como los actores y casos de usos del sistema que establecieron las bases fundamentales para el desarrollo de la aplicación. Se realizó el diseño de la base de datos, abarcando tanto el modelo lógico como el físico, lo que aseguró una estructura robusta y eficiente para el almacenamiento de los datos. La manipulación de los datos se abordó de manera integral, desde la conexión con la base de datos hasta la persistencia de los resultados estadísticos. Se diseñó e implementó la comunicación con el pedal motorizado y la implementación de la interfaz gráfica para la representación de los datos EMG. Se definieron los escenarios de entrenamiento para las modalidades Ligero y Clínico, asegurando una cobertura completa de las necesidad de entrenamiento del usuario. Por último en el ámbito estadístico se desarrolló una serie de gráficos para el seguimiento de los resultados en las rutinas de entrenamiento.   
    
\end{thesischapter}

% DATA MANIPULATION
\begin{thesischapter}{2} {Diseño e implementación del Juego Serio}
En este capítulo se discuten los detalles de desarrollo de los aspectos citados en el capítulo anterior. Este comienza con una descripción y caracterización general del sistema, donde se  abordan cada uno de los componentes requeridos para su completo funcionamiento. Posteriormente se detalla la ingienría de software requerida en la etapa de conceptualización de la aplicación, se explican de forma detallada los aspectos teóricos y de implementación de la base de datos, el funcionamiento del protocolo de comunicación y por último los escenarios de juegos requeridos en las rutinas de entrenamiento ligero y clínico, y las estadísticas generadas por estos. Como herramienta de desarrollo se utilizó c\#.

% SYSTEM DESCRIPTION AND CHARACTERIZATION TO APPLY
\input{main/chapter2/section1/content.tex}
     
% SERIOUS GAME REQUIREMENTS
\input{main/chapter2/section2/content.tex}    

% USE CASE DEFINITION
\input{main/chapter2/section3/content.tex}

% USE CASE REALIZATION
\input{main/chapter2/section4/content.tex}

% DATABASE DESIGN 
\input{main/chapter2/section5/content.tex}

% DATA MANIPULATION
\input{main/chapter2/section6/content.tex}

% COMMUNICATION 
\input{main/chapter2/section7/content.tex}

% TRAINING SCENARIOS
\input{main/chapter2/section8/content.tex}

% EMG GRAPHIC
\input{main/chapter2/section9/content.tex}

% STATICAL REPORTS GENERATION
\input{main/chapter2/section10/content.tex}

\subthesischapter{Conclusiones del capítulo}
Se presentó una descripción del sistema de adquisición de datos para rehabilitación, sus componentes, características distintivas y su funcionamiento. Se identificaron y definieron los requisitos del juego  serio, tanto funcionales como no funcionales, así como los actores y casos de usos del sistema que establecieron las bases fundamentales para el desarrollo de la aplicación. Se realizó el diseño de la base de datos, abarcando tanto el modelo lógico como el físico, lo que aseguró una estructura robusta y eficiente para el almacenamiento de los datos. La manipulación de los datos se abordó de manera integral, desde la conexión con la base de datos hasta la persistencia de los resultados estadísticos. Se diseñó e implementó la comunicación con el pedal motorizado y la implementación de la interfaz gráfica para la representación de los datos EMG. Se definieron los escenarios de entrenamiento para las modalidades Ligero y Clínico, asegurando una cobertura completa de las necesidad de entrenamiento del usuario. Por último en el ámbito estadístico se desarrolló una serie de gráficos para el seguimiento de los resultados en las rutinas de entrenamiento.   
    
\end{thesischapter}

% COMMUNICATION 
\begin{thesischapter}{2} {Diseño e implementación del Juego Serio}
En este capítulo se discuten los detalles de desarrollo de los aspectos citados en el capítulo anterior. Este comienza con una descripción y caracterización general del sistema, donde se  abordan cada uno de los componentes requeridos para su completo funcionamiento. Posteriormente se detalla la ingienría de software requerida en la etapa de conceptualización de la aplicación, se explican de forma detallada los aspectos teóricos y de implementación de la base de datos, el funcionamiento del protocolo de comunicación y por último los escenarios de juegos requeridos en las rutinas de entrenamiento ligero y clínico, y las estadísticas generadas por estos. Como herramienta de desarrollo se utilizó c\#.

% SYSTEM DESCRIPTION AND CHARACTERIZATION TO APPLY
\input{main/chapter2/section1/content.tex}
     
% SERIOUS GAME REQUIREMENTS
\input{main/chapter2/section2/content.tex}    

% USE CASE DEFINITION
\input{main/chapter2/section3/content.tex}

% USE CASE REALIZATION
\input{main/chapter2/section4/content.tex}

% DATABASE DESIGN 
\input{main/chapter2/section5/content.tex}

% DATA MANIPULATION
\input{main/chapter2/section6/content.tex}

% COMMUNICATION 
\input{main/chapter2/section7/content.tex}

% TRAINING SCENARIOS
\input{main/chapter2/section8/content.tex}

% EMG GRAPHIC
\input{main/chapter2/section9/content.tex}

% STATICAL REPORTS GENERATION
\input{main/chapter2/section10/content.tex}

\subthesischapter{Conclusiones del capítulo}
Se presentó una descripción del sistema de adquisición de datos para rehabilitación, sus componentes, características distintivas y su funcionamiento. Se identificaron y definieron los requisitos del juego  serio, tanto funcionales como no funcionales, así como los actores y casos de usos del sistema que establecieron las bases fundamentales para el desarrollo de la aplicación. Se realizó el diseño de la base de datos, abarcando tanto el modelo lógico como el físico, lo que aseguró una estructura robusta y eficiente para el almacenamiento de los datos. La manipulación de los datos se abordó de manera integral, desde la conexión con la base de datos hasta la persistencia de los resultados estadísticos. Se diseñó e implementó la comunicación con el pedal motorizado y la implementación de la interfaz gráfica para la representación de los datos EMG. Se definieron los escenarios de entrenamiento para las modalidades Ligero y Clínico, asegurando una cobertura completa de las necesidad de entrenamiento del usuario. Por último en el ámbito estadístico se desarrolló una serie de gráficos para el seguimiento de los resultados en las rutinas de entrenamiento.   
    
\end{thesischapter}

% TRAINING SCENARIOS
\begin{thesischapter}{2} {Diseño e implementación del Juego Serio}
En este capítulo se discuten los detalles de desarrollo de los aspectos citados en el capítulo anterior. Este comienza con una descripción y caracterización general del sistema, donde se  abordan cada uno de los componentes requeridos para su completo funcionamiento. Posteriormente se detalla la ingienría de software requerida en la etapa de conceptualización de la aplicación, se explican de forma detallada los aspectos teóricos y de implementación de la base de datos, el funcionamiento del protocolo de comunicación y por último los escenarios de juegos requeridos en las rutinas de entrenamiento ligero y clínico, y las estadísticas generadas por estos. Como herramienta de desarrollo se utilizó c\#.

% SYSTEM DESCRIPTION AND CHARACTERIZATION TO APPLY
\input{main/chapter2/section1/content.tex}
     
% SERIOUS GAME REQUIREMENTS
\input{main/chapter2/section2/content.tex}    

% USE CASE DEFINITION
\input{main/chapter2/section3/content.tex}

% USE CASE REALIZATION
\input{main/chapter2/section4/content.tex}

% DATABASE DESIGN 
\input{main/chapter2/section5/content.tex}

% DATA MANIPULATION
\input{main/chapter2/section6/content.tex}

% COMMUNICATION 
\input{main/chapter2/section7/content.tex}

% TRAINING SCENARIOS
\input{main/chapter2/section8/content.tex}

% EMG GRAPHIC
\input{main/chapter2/section9/content.tex}

% STATICAL REPORTS GENERATION
\input{main/chapter2/section10/content.tex}

\subthesischapter{Conclusiones del capítulo}
Se presentó una descripción del sistema de adquisición de datos para rehabilitación, sus componentes, características distintivas y su funcionamiento. Se identificaron y definieron los requisitos del juego  serio, tanto funcionales como no funcionales, así como los actores y casos de usos del sistema que establecieron las bases fundamentales para el desarrollo de la aplicación. Se realizó el diseño de la base de datos, abarcando tanto el modelo lógico como el físico, lo que aseguró una estructura robusta y eficiente para el almacenamiento de los datos. La manipulación de los datos se abordó de manera integral, desde la conexión con la base de datos hasta la persistencia de los resultados estadísticos. Se diseñó e implementó la comunicación con el pedal motorizado y la implementación de la interfaz gráfica para la representación de los datos EMG. Se definieron los escenarios de entrenamiento para las modalidades Ligero y Clínico, asegurando una cobertura completa de las necesidad de entrenamiento del usuario. Por último en el ámbito estadístico se desarrolló una serie de gráficos para el seguimiento de los resultados en las rutinas de entrenamiento.   
    
\end{thesischapter}

% EMG GRAPHIC
\begin{thesischapter}{2} {Diseño e implementación del Juego Serio}
En este capítulo se discuten los detalles de desarrollo de los aspectos citados en el capítulo anterior. Este comienza con una descripción y caracterización general del sistema, donde se  abordan cada uno de los componentes requeridos para su completo funcionamiento. Posteriormente se detalla la ingienría de software requerida en la etapa de conceptualización de la aplicación, se explican de forma detallada los aspectos teóricos y de implementación de la base de datos, el funcionamiento del protocolo de comunicación y por último los escenarios de juegos requeridos en las rutinas de entrenamiento ligero y clínico, y las estadísticas generadas por estos. Como herramienta de desarrollo se utilizó c\#.

% SYSTEM DESCRIPTION AND CHARACTERIZATION TO APPLY
\input{main/chapter2/section1/content.tex}
     
% SERIOUS GAME REQUIREMENTS
\input{main/chapter2/section2/content.tex}    

% USE CASE DEFINITION
\input{main/chapter2/section3/content.tex}

% USE CASE REALIZATION
\input{main/chapter2/section4/content.tex}

% DATABASE DESIGN 
\input{main/chapter2/section5/content.tex}

% DATA MANIPULATION
\input{main/chapter2/section6/content.tex}

% COMMUNICATION 
\input{main/chapter2/section7/content.tex}

% TRAINING SCENARIOS
\input{main/chapter2/section8/content.tex}

% EMG GRAPHIC
\input{main/chapter2/section9/content.tex}

% STATICAL REPORTS GENERATION
\input{main/chapter2/section10/content.tex}

\subthesischapter{Conclusiones del capítulo}
Se presentó una descripción del sistema de adquisición de datos para rehabilitación, sus componentes, características distintivas y su funcionamiento. Se identificaron y definieron los requisitos del juego  serio, tanto funcionales como no funcionales, así como los actores y casos de usos del sistema que establecieron las bases fundamentales para el desarrollo de la aplicación. Se realizó el diseño de la base de datos, abarcando tanto el modelo lógico como el físico, lo que aseguró una estructura robusta y eficiente para el almacenamiento de los datos. La manipulación de los datos se abordó de manera integral, desde la conexión con la base de datos hasta la persistencia de los resultados estadísticos. Se diseñó e implementó la comunicación con el pedal motorizado y la implementación de la interfaz gráfica para la representación de los datos EMG. Se definieron los escenarios de entrenamiento para las modalidades Ligero y Clínico, asegurando una cobertura completa de las necesidad de entrenamiento del usuario. Por último en el ámbito estadístico se desarrolló una serie de gráficos para el seguimiento de los resultados en las rutinas de entrenamiento.   
    
\end{thesischapter}

% STATICAL REPORTS GENERATION
\begin{thesischapter}{2} {Diseño e implementación del Juego Serio}
En este capítulo se discuten los detalles de desarrollo de los aspectos citados en el capítulo anterior. Este comienza con una descripción y caracterización general del sistema, donde se  abordan cada uno de los componentes requeridos para su completo funcionamiento. Posteriormente se detalla la ingienría de software requerida en la etapa de conceptualización de la aplicación, se explican de forma detallada los aspectos teóricos y de implementación de la base de datos, el funcionamiento del protocolo de comunicación y por último los escenarios de juegos requeridos en las rutinas de entrenamiento ligero y clínico, y las estadísticas generadas por estos. Como herramienta de desarrollo se utilizó c\#.

% SYSTEM DESCRIPTION AND CHARACTERIZATION TO APPLY
\input{main/chapter2/section1/content.tex}
     
% SERIOUS GAME REQUIREMENTS
\input{main/chapter2/section2/content.tex}    

% USE CASE DEFINITION
\input{main/chapter2/section3/content.tex}

% USE CASE REALIZATION
\input{main/chapter2/section4/content.tex}

% DATABASE DESIGN 
\input{main/chapter2/section5/content.tex}

% DATA MANIPULATION
\input{main/chapter2/section6/content.tex}

% COMMUNICATION 
\input{main/chapter2/section7/content.tex}

% TRAINING SCENARIOS
\input{main/chapter2/section8/content.tex}

% EMG GRAPHIC
\input{main/chapter2/section9/content.tex}

% STATICAL REPORTS GENERATION
\input{main/chapter2/section10/content.tex}

\subthesischapter{Conclusiones del capítulo}
Se presentó una descripción del sistema de adquisición de datos para rehabilitación, sus componentes, características distintivas y su funcionamiento. Se identificaron y definieron los requisitos del juego  serio, tanto funcionales como no funcionales, así como los actores y casos de usos del sistema que establecieron las bases fundamentales para el desarrollo de la aplicación. Se realizó el diseño de la base de datos, abarcando tanto el modelo lógico como el físico, lo que aseguró una estructura robusta y eficiente para el almacenamiento de los datos. La manipulación de los datos se abordó de manera integral, desde la conexión con la base de datos hasta la persistencia de los resultados estadísticos. Se diseñó e implementó la comunicación con el pedal motorizado y la implementación de la interfaz gráfica para la representación de los datos EMG. Se definieron los escenarios de entrenamiento para las modalidades Ligero y Clínico, asegurando una cobertura completa de las necesidad de entrenamiento del usuario. Por último en el ámbito estadístico se desarrolló una serie de gráficos para el seguimiento de los resultados en las rutinas de entrenamiento.   
    
\end{thesischapter}

\subthesischapter{Conclusiones del capítulo}
Se presentó una descripción del sistema de adquisición de datos para rehabilitación, sus componentes, características distintivas y su funcionamiento. Se identificaron y definieron los requisitos del juego  serio, tanto funcionales como no funcionales, así como los actores y casos de usos del sistema que establecieron las bases fundamentales para el desarrollo de la aplicación. Se realizó el diseño de la base de datos, abarcando tanto el modelo lógico como el físico, lo que aseguró una estructura robusta y eficiente para el almacenamiento de los datos. La manipulación de los datos se abordó de manera integral, desde la conexión con la base de datos hasta la persistencia de los resultados estadísticos. Se diseñó e implementó la comunicación con el pedal motorizado y la implementación de la interfaz gráfica para la representación de los datos EMG. Se definieron los escenarios de entrenamiento para las modalidades Ligero y Clínico, asegurando una cobertura completa de las necesidad de entrenamiento del usuario. Por último en el ámbito estadístico se desarrolló una serie de gráficos para el seguimiento de los resultados en las rutinas de entrenamiento.   
    
\end{thesischapter}

% USE CASE REALIZATION
\begin{thesischapter}{2} {Diseño e implementación del Juego Serio}
En este capítulo se discuten los detalles de desarrollo de los aspectos citados en el capítulo anterior. Este comienza con una descripción y caracterización general del sistema, donde se  abordan cada uno de los componentes requeridos para su completo funcionamiento. Posteriormente se detalla la ingienría de software requerida en la etapa de conceptualización de la aplicación, se explican de forma detallada los aspectos teóricos y de implementación de la base de datos, el funcionamiento del protocolo de comunicación y por último los escenarios de juegos requeridos en las rutinas de entrenamiento ligero y clínico, y las estadísticas generadas por estos. Como herramienta de desarrollo se utilizó c\#.

% SYSTEM DESCRIPTION AND CHARACTERIZATION TO APPLY
\begin{thesischapter}{2} {Diseño e implementación del Juego Serio}
En este capítulo se discuten los detalles de desarrollo de los aspectos citados en el capítulo anterior. Este comienza con una descripción y caracterización general del sistema, donde se  abordan cada uno de los componentes requeridos para su completo funcionamiento. Posteriormente se detalla la ingienría de software requerida en la etapa de conceptualización de la aplicación, se explican de forma detallada los aspectos teóricos y de implementación de la base de datos, el funcionamiento del protocolo de comunicación y por último los escenarios de juegos requeridos en las rutinas de entrenamiento ligero y clínico, y las estadísticas generadas por estos. Como herramienta de desarrollo se utilizó c\#.

% SYSTEM DESCRIPTION AND CHARACTERIZATION TO APPLY
\input{main/chapter2/section1/content.tex}
     
% SERIOUS GAME REQUIREMENTS
\input{main/chapter2/section2/content.tex}    

% USE CASE DEFINITION
\input{main/chapter2/section3/content.tex}

% USE CASE REALIZATION
\input{main/chapter2/section4/content.tex}

% DATABASE DESIGN 
\input{main/chapter2/section5/content.tex}

% DATA MANIPULATION
\input{main/chapter2/section6/content.tex}

% COMMUNICATION 
\input{main/chapter2/section7/content.tex}

% TRAINING SCENARIOS
\input{main/chapter2/section8/content.tex}

% EMG GRAPHIC
\input{main/chapter2/section9/content.tex}

% STATICAL REPORTS GENERATION
\input{main/chapter2/section10/content.tex}

\subthesischapter{Conclusiones del capítulo}
Se presentó una descripción del sistema de adquisición de datos para rehabilitación, sus componentes, características distintivas y su funcionamiento. Se identificaron y definieron los requisitos del juego  serio, tanto funcionales como no funcionales, así como los actores y casos de usos del sistema que establecieron las bases fundamentales para el desarrollo de la aplicación. Se realizó el diseño de la base de datos, abarcando tanto el modelo lógico como el físico, lo que aseguró una estructura robusta y eficiente para el almacenamiento de los datos. La manipulación de los datos se abordó de manera integral, desde la conexión con la base de datos hasta la persistencia de los resultados estadísticos. Se diseñó e implementó la comunicación con el pedal motorizado y la implementación de la interfaz gráfica para la representación de los datos EMG. Se definieron los escenarios de entrenamiento para las modalidades Ligero y Clínico, asegurando una cobertura completa de las necesidad de entrenamiento del usuario. Por último en el ámbito estadístico se desarrolló una serie de gráficos para el seguimiento de los resultados en las rutinas de entrenamiento.   
    
\end{thesischapter}
     
% SERIOUS GAME REQUIREMENTS
\begin{thesischapter}{2} {Diseño e implementación del Juego Serio}
En este capítulo se discuten los detalles de desarrollo de los aspectos citados en el capítulo anterior. Este comienza con una descripción y caracterización general del sistema, donde se  abordan cada uno de los componentes requeridos para su completo funcionamiento. Posteriormente se detalla la ingienría de software requerida en la etapa de conceptualización de la aplicación, se explican de forma detallada los aspectos teóricos y de implementación de la base de datos, el funcionamiento del protocolo de comunicación y por último los escenarios de juegos requeridos en las rutinas de entrenamiento ligero y clínico, y las estadísticas generadas por estos. Como herramienta de desarrollo se utilizó c\#.

% SYSTEM DESCRIPTION AND CHARACTERIZATION TO APPLY
\input{main/chapter2/section1/content.tex}
     
% SERIOUS GAME REQUIREMENTS
\input{main/chapter2/section2/content.tex}    

% USE CASE DEFINITION
\input{main/chapter2/section3/content.tex}

% USE CASE REALIZATION
\input{main/chapter2/section4/content.tex}

% DATABASE DESIGN 
\input{main/chapter2/section5/content.tex}

% DATA MANIPULATION
\input{main/chapter2/section6/content.tex}

% COMMUNICATION 
\input{main/chapter2/section7/content.tex}

% TRAINING SCENARIOS
\input{main/chapter2/section8/content.tex}

% EMG GRAPHIC
\input{main/chapter2/section9/content.tex}

% STATICAL REPORTS GENERATION
\input{main/chapter2/section10/content.tex}

\subthesischapter{Conclusiones del capítulo}
Se presentó una descripción del sistema de adquisición de datos para rehabilitación, sus componentes, características distintivas y su funcionamiento. Se identificaron y definieron los requisitos del juego  serio, tanto funcionales como no funcionales, así como los actores y casos de usos del sistema que establecieron las bases fundamentales para el desarrollo de la aplicación. Se realizó el diseño de la base de datos, abarcando tanto el modelo lógico como el físico, lo que aseguró una estructura robusta y eficiente para el almacenamiento de los datos. La manipulación de los datos se abordó de manera integral, desde la conexión con la base de datos hasta la persistencia de los resultados estadísticos. Se diseñó e implementó la comunicación con el pedal motorizado y la implementación de la interfaz gráfica para la representación de los datos EMG. Se definieron los escenarios de entrenamiento para las modalidades Ligero y Clínico, asegurando una cobertura completa de las necesidad de entrenamiento del usuario. Por último en el ámbito estadístico se desarrolló una serie de gráficos para el seguimiento de los resultados en las rutinas de entrenamiento.   
    
\end{thesischapter}    

% USE CASE DEFINITION
\begin{thesischapter}{2} {Diseño e implementación del Juego Serio}
En este capítulo se discuten los detalles de desarrollo de los aspectos citados en el capítulo anterior. Este comienza con una descripción y caracterización general del sistema, donde se  abordan cada uno de los componentes requeridos para su completo funcionamiento. Posteriormente se detalla la ingienría de software requerida en la etapa de conceptualización de la aplicación, se explican de forma detallada los aspectos teóricos y de implementación de la base de datos, el funcionamiento del protocolo de comunicación y por último los escenarios de juegos requeridos en las rutinas de entrenamiento ligero y clínico, y las estadísticas generadas por estos. Como herramienta de desarrollo se utilizó c\#.

% SYSTEM DESCRIPTION AND CHARACTERIZATION TO APPLY
\input{main/chapter2/section1/content.tex}
     
% SERIOUS GAME REQUIREMENTS
\input{main/chapter2/section2/content.tex}    

% USE CASE DEFINITION
\input{main/chapter2/section3/content.tex}

% USE CASE REALIZATION
\input{main/chapter2/section4/content.tex}

% DATABASE DESIGN 
\input{main/chapter2/section5/content.tex}

% DATA MANIPULATION
\input{main/chapter2/section6/content.tex}

% COMMUNICATION 
\input{main/chapter2/section7/content.tex}

% TRAINING SCENARIOS
\input{main/chapter2/section8/content.tex}

% EMG GRAPHIC
\input{main/chapter2/section9/content.tex}

% STATICAL REPORTS GENERATION
\input{main/chapter2/section10/content.tex}

\subthesischapter{Conclusiones del capítulo}
Se presentó una descripción del sistema de adquisición de datos para rehabilitación, sus componentes, características distintivas y su funcionamiento. Se identificaron y definieron los requisitos del juego  serio, tanto funcionales como no funcionales, así como los actores y casos de usos del sistema que establecieron las bases fundamentales para el desarrollo de la aplicación. Se realizó el diseño de la base de datos, abarcando tanto el modelo lógico como el físico, lo que aseguró una estructura robusta y eficiente para el almacenamiento de los datos. La manipulación de los datos se abordó de manera integral, desde la conexión con la base de datos hasta la persistencia de los resultados estadísticos. Se diseñó e implementó la comunicación con el pedal motorizado y la implementación de la interfaz gráfica para la representación de los datos EMG. Se definieron los escenarios de entrenamiento para las modalidades Ligero y Clínico, asegurando una cobertura completa de las necesidad de entrenamiento del usuario. Por último en el ámbito estadístico se desarrolló una serie de gráficos para el seguimiento de los resultados en las rutinas de entrenamiento.   
    
\end{thesischapter}

% USE CASE REALIZATION
\begin{thesischapter}{2} {Diseño e implementación del Juego Serio}
En este capítulo se discuten los detalles de desarrollo de los aspectos citados en el capítulo anterior. Este comienza con una descripción y caracterización general del sistema, donde se  abordan cada uno de los componentes requeridos para su completo funcionamiento. Posteriormente se detalla la ingienría de software requerida en la etapa de conceptualización de la aplicación, se explican de forma detallada los aspectos teóricos y de implementación de la base de datos, el funcionamiento del protocolo de comunicación y por último los escenarios de juegos requeridos en las rutinas de entrenamiento ligero y clínico, y las estadísticas generadas por estos. Como herramienta de desarrollo se utilizó c\#.

% SYSTEM DESCRIPTION AND CHARACTERIZATION TO APPLY
\input{main/chapter2/section1/content.tex}
     
% SERIOUS GAME REQUIREMENTS
\input{main/chapter2/section2/content.tex}    

% USE CASE DEFINITION
\input{main/chapter2/section3/content.tex}

% USE CASE REALIZATION
\input{main/chapter2/section4/content.tex}

% DATABASE DESIGN 
\input{main/chapter2/section5/content.tex}

% DATA MANIPULATION
\input{main/chapter2/section6/content.tex}

% COMMUNICATION 
\input{main/chapter2/section7/content.tex}

% TRAINING SCENARIOS
\input{main/chapter2/section8/content.tex}

% EMG GRAPHIC
\input{main/chapter2/section9/content.tex}

% STATICAL REPORTS GENERATION
\input{main/chapter2/section10/content.tex}

\subthesischapter{Conclusiones del capítulo}
Se presentó una descripción del sistema de adquisición de datos para rehabilitación, sus componentes, características distintivas y su funcionamiento. Se identificaron y definieron los requisitos del juego  serio, tanto funcionales como no funcionales, así como los actores y casos de usos del sistema que establecieron las bases fundamentales para el desarrollo de la aplicación. Se realizó el diseño de la base de datos, abarcando tanto el modelo lógico como el físico, lo que aseguró una estructura robusta y eficiente para el almacenamiento de los datos. La manipulación de los datos se abordó de manera integral, desde la conexión con la base de datos hasta la persistencia de los resultados estadísticos. Se diseñó e implementó la comunicación con el pedal motorizado y la implementación de la interfaz gráfica para la representación de los datos EMG. Se definieron los escenarios de entrenamiento para las modalidades Ligero y Clínico, asegurando una cobertura completa de las necesidad de entrenamiento del usuario. Por último en el ámbito estadístico se desarrolló una serie de gráficos para el seguimiento de los resultados en las rutinas de entrenamiento.   
    
\end{thesischapter}

% DATABASE DESIGN 
\begin{thesischapter}{2} {Diseño e implementación del Juego Serio}
En este capítulo se discuten los detalles de desarrollo de los aspectos citados en el capítulo anterior. Este comienza con una descripción y caracterización general del sistema, donde se  abordan cada uno de los componentes requeridos para su completo funcionamiento. Posteriormente se detalla la ingienría de software requerida en la etapa de conceptualización de la aplicación, se explican de forma detallada los aspectos teóricos y de implementación de la base de datos, el funcionamiento del protocolo de comunicación y por último los escenarios de juegos requeridos en las rutinas de entrenamiento ligero y clínico, y las estadísticas generadas por estos. Como herramienta de desarrollo se utilizó c\#.

% SYSTEM DESCRIPTION AND CHARACTERIZATION TO APPLY
\input{main/chapter2/section1/content.tex}
     
% SERIOUS GAME REQUIREMENTS
\input{main/chapter2/section2/content.tex}    

% USE CASE DEFINITION
\input{main/chapter2/section3/content.tex}

% USE CASE REALIZATION
\input{main/chapter2/section4/content.tex}

% DATABASE DESIGN 
\input{main/chapter2/section5/content.tex}

% DATA MANIPULATION
\input{main/chapter2/section6/content.tex}

% COMMUNICATION 
\input{main/chapter2/section7/content.tex}

% TRAINING SCENARIOS
\input{main/chapter2/section8/content.tex}

% EMG GRAPHIC
\input{main/chapter2/section9/content.tex}

% STATICAL REPORTS GENERATION
\input{main/chapter2/section10/content.tex}

\subthesischapter{Conclusiones del capítulo}
Se presentó una descripción del sistema de adquisición de datos para rehabilitación, sus componentes, características distintivas y su funcionamiento. Se identificaron y definieron los requisitos del juego  serio, tanto funcionales como no funcionales, así como los actores y casos de usos del sistema que establecieron las bases fundamentales para el desarrollo de la aplicación. Se realizó el diseño de la base de datos, abarcando tanto el modelo lógico como el físico, lo que aseguró una estructura robusta y eficiente para el almacenamiento de los datos. La manipulación de los datos se abordó de manera integral, desde la conexión con la base de datos hasta la persistencia de los resultados estadísticos. Se diseñó e implementó la comunicación con el pedal motorizado y la implementación de la interfaz gráfica para la representación de los datos EMG. Se definieron los escenarios de entrenamiento para las modalidades Ligero y Clínico, asegurando una cobertura completa de las necesidad de entrenamiento del usuario. Por último en el ámbito estadístico se desarrolló una serie de gráficos para el seguimiento de los resultados en las rutinas de entrenamiento.   
    
\end{thesischapter}

% DATA MANIPULATION
\begin{thesischapter}{2} {Diseño e implementación del Juego Serio}
En este capítulo se discuten los detalles de desarrollo de los aspectos citados en el capítulo anterior. Este comienza con una descripción y caracterización general del sistema, donde se  abordan cada uno de los componentes requeridos para su completo funcionamiento. Posteriormente se detalla la ingienría de software requerida en la etapa de conceptualización de la aplicación, se explican de forma detallada los aspectos teóricos y de implementación de la base de datos, el funcionamiento del protocolo de comunicación y por último los escenarios de juegos requeridos en las rutinas de entrenamiento ligero y clínico, y las estadísticas generadas por estos. Como herramienta de desarrollo se utilizó c\#.

% SYSTEM DESCRIPTION AND CHARACTERIZATION TO APPLY
\input{main/chapter2/section1/content.tex}
     
% SERIOUS GAME REQUIREMENTS
\input{main/chapter2/section2/content.tex}    

% USE CASE DEFINITION
\input{main/chapter2/section3/content.tex}

% USE CASE REALIZATION
\input{main/chapter2/section4/content.tex}

% DATABASE DESIGN 
\input{main/chapter2/section5/content.tex}

% DATA MANIPULATION
\input{main/chapter2/section6/content.tex}

% COMMUNICATION 
\input{main/chapter2/section7/content.tex}

% TRAINING SCENARIOS
\input{main/chapter2/section8/content.tex}

% EMG GRAPHIC
\input{main/chapter2/section9/content.tex}

% STATICAL REPORTS GENERATION
\input{main/chapter2/section10/content.tex}

\subthesischapter{Conclusiones del capítulo}
Se presentó una descripción del sistema de adquisición de datos para rehabilitación, sus componentes, características distintivas y su funcionamiento. Se identificaron y definieron los requisitos del juego  serio, tanto funcionales como no funcionales, así como los actores y casos de usos del sistema que establecieron las bases fundamentales para el desarrollo de la aplicación. Se realizó el diseño de la base de datos, abarcando tanto el modelo lógico como el físico, lo que aseguró una estructura robusta y eficiente para el almacenamiento de los datos. La manipulación de los datos se abordó de manera integral, desde la conexión con la base de datos hasta la persistencia de los resultados estadísticos. Se diseñó e implementó la comunicación con el pedal motorizado y la implementación de la interfaz gráfica para la representación de los datos EMG. Se definieron los escenarios de entrenamiento para las modalidades Ligero y Clínico, asegurando una cobertura completa de las necesidad de entrenamiento del usuario. Por último en el ámbito estadístico se desarrolló una serie de gráficos para el seguimiento de los resultados en las rutinas de entrenamiento.   
    
\end{thesischapter}

% COMMUNICATION 
\begin{thesischapter}{2} {Diseño e implementación del Juego Serio}
En este capítulo se discuten los detalles de desarrollo de los aspectos citados en el capítulo anterior. Este comienza con una descripción y caracterización general del sistema, donde se  abordan cada uno de los componentes requeridos para su completo funcionamiento. Posteriormente se detalla la ingienría de software requerida en la etapa de conceptualización de la aplicación, se explican de forma detallada los aspectos teóricos y de implementación de la base de datos, el funcionamiento del protocolo de comunicación y por último los escenarios de juegos requeridos en las rutinas de entrenamiento ligero y clínico, y las estadísticas generadas por estos. Como herramienta de desarrollo se utilizó c\#.

% SYSTEM DESCRIPTION AND CHARACTERIZATION TO APPLY
\input{main/chapter2/section1/content.tex}
     
% SERIOUS GAME REQUIREMENTS
\input{main/chapter2/section2/content.tex}    

% USE CASE DEFINITION
\input{main/chapter2/section3/content.tex}

% USE CASE REALIZATION
\input{main/chapter2/section4/content.tex}

% DATABASE DESIGN 
\input{main/chapter2/section5/content.tex}

% DATA MANIPULATION
\input{main/chapter2/section6/content.tex}

% COMMUNICATION 
\input{main/chapter2/section7/content.tex}

% TRAINING SCENARIOS
\input{main/chapter2/section8/content.tex}

% EMG GRAPHIC
\input{main/chapter2/section9/content.tex}

% STATICAL REPORTS GENERATION
\input{main/chapter2/section10/content.tex}

\subthesischapter{Conclusiones del capítulo}
Se presentó una descripción del sistema de adquisición de datos para rehabilitación, sus componentes, características distintivas y su funcionamiento. Se identificaron y definieron los requisitos del juego  serio, tanto funcionales como no funcionales, así como los actores y casos de usos del sistema que establecieron las bases fundamentales para el desarrollo de la aplicación. Se realizó el diseño de la base de datos, abarcando tanto el modelo lógico como el físico, lo que aseguró una estructura robusta y eficiente para el almacenamiento de los datos. La manipulación de los datos se abordó de manera integral, desde la conexión con la base de datos hasta la persistencia de los resultados estadísticos. Se diseñó e implementó la comunicación con el pedal motorizado y la implementación de la interfaz gráfica para la representación de los datos EMG. Se definieron los escenarios de entrenamiento para las modalidades Ligero y Clínico, asegurando una cobertura completa de las necesidad de entrenamiento del usuario. Por último en el ámbito estadístico se desarrolló una serie de gráficos para el seguimiento de los resultados en las rutinas de entrenamiento.   
    
\end{thesischapter}

% TRAINING SCENARIOS
\begin{thesischapter}{2} {Diseño e implementación del Juego Serio}
En este capítulo se discuten los detalles de desarrollo de los aspectos citados en el capítulo anterior. Este comienza con una descripción y caracterización general del sistema, donde se  abordan cada uno de los componentes requeridos para su completo funcionamiento. Posteriormente se detalla la ingienría de software requerida en la etapa de conceptualización de la aplicación, se explican de forma detallada los aspectos teóricos y de implementación de la base de datos, el funcionamiento del protocolo de comunicación y por último los escenarios de juegos requeridos en las rutinas de entrenamiento ligero y clínico, y las estadísticas generadas por estos. Como herramienta de desarrollo se utilizó c\#.

% SYSTEM DESCRIPTION AND CHARACTERIZATION TO APPLY
\input{main/chapter2/section1/content.tex}
     
% SERIOUS GAME REQUIREMENTS
\input{main/chapter2/section2/content.tex}    

% USE CASE DEFINITION
\input{main/chapter2/section3/content.tex}

% USE CASE REALIZATION
\input{main/chapter2/section4/content.tex}

% DATABASE DESIGN 
\input{main/chapter2/section5/content.tex}

% DATA MANIPULATION
\input{main/chapter2/section6/content.tex}

% COMMUNICATION 
\input{main/chapter2/section7/content.tex}

% TRAINING SCENARIOS
\input{main/chapter2/section8/content.tex}

% EMG GRAPHIC
\input{main/chapter2/section9/content.tex}

% STATICAL REPORTS GENERATION
\input{main/chapter2/section10/content.tex}

\subthesischapter{Conclusiones del capítulo}
Se presentó una descripción del sistema de adquisición de datos para rehabilitación, sus componentes, características distintivas y su funcionamiento. Se identificaron y definieron los requisitos del juego  serio, tanto funcionales como no funcionales, así como los actores y casos de usos del sistema que establecieron las bases fundamentales para el desarrollo de la aplicación. Se realizó el diseño de la base de datos, abarcando tanto el modelo lógico como el físico, lo que aseguró una estructura robusta y eficiente para el almacenamiento de los datos. La manipulación de los datos se abordó de manera integral, desde la conexión con la base de datos hasta la persistencia de los resultados estadísticos. Se diseñó e implementó la comunicación con el pedal motorizado y la implementación de la interfaz gráfica para la representación de los datos EMG. Se definieron los escenarios de entrenamiento para las modalidades Ligero y Clínico, asegurando una cobertura completa de las necesidad de entrenamiento del usuario. Por último en el ámbito estadístico se desarrolló una serie de gráficos para el seguimiento de los resultados en las rutinas de entrenamiento.   
    
\end{thesischapter}

% EMG GRAPHIC
\begin{thesischapter}{2} {Diseño e implementación del Juego Serio}
En este capítulo se discuten los detalles de desarrollo de los aspectos citados en el capítulo anterior. Este comienza con una descripción y caracterización general del sistema, donde se  abordan cada uno de los componentes requeridos para su completo funcionamiento. Posteriormente se detalla la ingienría de software requerida en la etapa de conceptualización de la aplicación, se explican de forma detallada los aspectos teóricos y de implementación de la base de datos, el funcionamiento del protocolo de comunicación y por último los escenarios de juegos requeridos en las rutinas de entrenamiento ligero y clínico, y las estadísticas generadas por estos. Como herramienta de desarrollo se utilizó c\#.

% SYSTEM DESCRIPTION AND CHARACTERIZATION TO APPLY
\input{main/chapter2/section1/content.tex}
     
% SERIOUS GAME REQUIREMENTS
\input{main/chapter2/section2/content.tex}    

% USE CASE DEFINITION
\input{main/chapter2/section3/content.tex}

% USE CASE REALIZATION
\input{main/chapter2/section4/content.tex}

% DATABASE DESIGN 
\input{main/chapter2/section5/content.tex}

% DATA MANIPULATION
\input{main/chapter2/section6/content.tex}

% COMMUNICATION 
\input{main/chapter2/section7/content.tex}

% TRAINING SCENARIOS
\input{main/chapter2/section8/content.tex}

% EMG GRAPHIC
\input{main/chapter2/section9/content.tex}

% STATICAL REPORTS GENERATION
\input{main/chapter2/section10/content.tex}

\subthesischapter{Conclusiones del capítulo}
Se presentó una descripción del sistema de adquisición de datos para rehabilitación, sus componentes, características distintivas y su funcionamiento. Se identificaron y definieron los requisitos del juego  serio, tanto funcionales como no funcionales, así como los actores y casos de usos del sistema que establecieron las bases fundamentales para el desarrollo de la aplicación. Se realizó el diseño de la base de datos, abarcando tanto el modelo lógico como el físico, lo que aseguró una estructura robusta y eficiente para el almacenamiento de los datos. La manipulación de los datos se abordó de manera integral, desde la conexión con la base de datos hasta la persistencia de los resultados estadísticos. Se diseñó e implementó la comunicación con el pedal motorizado y la implementación de la interfaz gráfica para la representación de los datos EMG. Se definieron los escenarios de entrenamiento para las modalidades Ligero y Clínico, asegurando una cobertura completa de las necesidad de entrenamiento del usuario. Por último en el ámbito estadístico se desarrolló una serie de gráficos para el seguimiento de los resultados en las rutinas de entrenamiento.   
    
\end{thesischapter}

% STATICAL REPORTS GENERATION
\begin{thesischapter}{2} {Diseño e implementación del Juego Serio}
En este capítulo se discuten los detalles de desarrollo de los aspectos citados en el capítulo anterior. Este comienza con una descripción y caracterización general del sistema, donde se  abordan cada uno de los componentes requeridos para su completo funcionamiento. Posteriormente se detalla la ingienría de software requerida en la etapa de conceptualización de la aplicación, se explican de forma detallada los aspectos teóricos y de implementación de la base de datos, el funcionamiento del protocolo de comunicación y por último los escenarios de juegos requeridos en las rutinas de entrenamiento ligero y clínico, y las estadísticas generadas por estos. Como herramienta de desarrollo se utilizó c\#.

% SYSTEM DESCRIPTION AND CHARACTERIZATION TO APPLY
\input{main/chapter2/section1/content.tex}
     
% SERIOUS GAME REQUIREMENTS
\input{main/chapter2/section2/content.tex}    

% USE CASE DEFINITION
\input{main/chapter2/section3/content.tex}

% USE CASE REALIZATION
\input{main/chapter2/section4/content.tex}

% DATABASE DESIGN 
\input{main/chapter2/section5/content.tex}

% DATA MANIPULATION
\input{main/chapter2/section6/content.tex}

% COMMUNICATION 
\input{main/chapter2/section7/content.tex}

% TRAINING SCENARIOS
\input{main/chapter2/section8/content.tex}

% EMG GRAPHIC
\input{main/chapter2/section9/content.tex}

% STATICAL REPORTS GENERATION
\input{main/chapter2/section10/content.tex}

\subthesischapter{Conclusiones del capítulo}
Se presentó una descripción del sistema de adquisición de datos para rehabilitación, sus componentes, características distintivas y su funcionamiento. Se identificaron y definieron los requisitos del juego  serio, tanto funcionales como no funcionales, así como los actores y casos de usos del sistema que establecieron las bases fundamentales para el desarrollo de la aplicación. Se realizó el diseño de la base de datos, abarcando tanto el modelo lógico como el físico, lo que aseguró una estructura robusta y eficiente para el almacenamiento de los datos. La manipulación de los datos se abordó de manera integral, desde la conexión con la base de datos hasta la persistencia de los resultados estadísticos. Se diseñó e implementó la comunicación con el pedal motorizado y la implementación de la interfaz gráfica para la representación de los datos EMG. Se definieron los escenarios de entrenamiento para las modalidades Ligero y Clínico, asegurando una cobertura completa de las necesidad de entrenamiento del usuario. Por último en el ámbito estadístico se desarrolló una serie de gráficos para el seguimiento de los resultados en las rutinas de entrenamiento.   
    
\end{thesischapter}

\subthesischapter{Conclusiones del capítulo}
Se presentó una descripción del sistema de adquisición de datos para rehabilitación, sus componentes, características distintivas y su funcionamiento. Se identificaron y definieron los requisitos del juego  serio, tanto funcionales como no funcionales, así como los actores y casos de usos del sistema que establecieron las bases fundamentales para el desarrollo de la aplicación. Se realizó el diseño de la base de datos, abarcando tanto el modelo lógico como el físico, lo que aseguró una estructura robusta y eficiente para el almacenamiento de los datos. La manipulación de los datos se abordó de manera integral, desde la conexión con la base de datos hasta la persistencia de los resultados estadísticos. Se diseñó e implementó la comunicación con el pedal motorizado y la implementación de la interfaz gráfica para la representación de los datos EMG. Se definieron los escenarios de entrenamiento para las modalidades Ligero y Clínico, asegurando una cobertura completa de las necesidad de entrenamiento del usuario. Por último en el ámbito estadístico se desarrolló una serie de gráficos para el seguimiento de los resultados en las rutinas de entrenamiento.   
    
\end{thesischapter}

% DATABASE DESIGN 
\begin{thesischapter}{2} {Diseño e implementación del Juego Serio}
En este capítulo se discuten los detalles de desarrollo de los aspectos citados en el capítulo anterior. Este comienza con una descripción y caracterización general del sistema, donde se  abordan cada uno de los componentes requeridos para su completo funcionamiento. Posteriormente se detalla la ingienría de software requerida en la etapa de conceptualización de la aplicación, se explican de forma detallada los aspectos teóricos y de implementación de la base de datos, el funcionamiento del protocolo de comunicación y por último los escenarios de juegos requeridos en las rutinas de entrenamiento ligero y clínico, y las estadísticas generadas por estos. Como herramienta de desarrollo se utilizó c\#.

% SYSTEM DESCRIPTION AND CHARACTERIZATION TO APPLY
\begin{thesischapter}{2} {Diseño e implementación del Juego Serio}
En este capítulo se discuten los detalles de desarrollo de los aspectos citados en el capítulo anterior. Este comienza con una descripción y caracterización general del sistema, donde se  abordan cada uno de los componentes requeridos para su completo funcionamiento. Posteriormente se detalla la ingienría de software requerida en la etapa de conceptualización de la aplicación, se explican de forma detallada los aspectos teóricos y de implementación de la base de datos, el funcionamiento del protocolo de comunicación y por último los escenarios de juegos requeridos en las rutinas de entrenamiento ligero y clínico, y las estadísticas generadas por estos. Como herramienta de desarrollo se utilizó c\#.

% SYSTEM DESCRIPTION AND CHARACTERIZATION TO APPLY
\input{main/chapter2/section1/content.tex}
     
% SERIOUS GAME REQUIREMENTS
\input{main/chapter2/section2/content.tex}    

% USE CASE DEFINITION
\input{main/chapter2/section3/content.tex}

% USE CASE REALIZATION
\input{main/chapter2/section4/content.tex}

% DATABASE DESIGN 
\input{main/chapter2/section5/content.tex}

% DATA MANIPULATION
\input{main/chapter2/section6/content.tex}

% COMMUNICATION 
\input{main/chapter2/section7/content.tex}

% TRAINING SCENARIOS
\input{main/chapter2/section8/content.tex}

% EMG GRAPHIC
\input{main/chapter2/section9/content.tex}

% STATICAL REPORTS GENERATION
\input{main/chapter2/section10/content.tex}

\subthesischapter{Conclusiones del capítulo}
Se presentó una descripción del sistema de adquisición de datos para rehabilitación, sus componentes, características distintivas y su funcionamiento. Se identificaron y definieron los requisitos del juego  serio, tanto funcionales como no funcionales, así como los actores y casos de usos del sistema que establecieron las bases fundamentales para el desarrollo de la aplicación. Se realizó el diseño de la base de datos, abarcando tanto el modelo lógico como el físico, lo que aseguró una estructura robusta y eficiente para el almacenamiento de los datos. La manipulación de los datos se abordó de manera integral, desde la conexión con la base de datos hasta la persistencia de los resultados estadísticos. Se diseñó e implementó la comunicación con el pedal motorizado y la implementación de la interfaz gráfica para la representación de los datos EMG. Se definieron los escenarios de entrenamiento para las modalidades Ligero y Clínico, asegurando una cobertura completa de las necesidad de entrenamiento del usuario. Por último en el ámbito estadístico se desarrolló una serie de gráficos para el seguimiento de los resultados en las rutinas de entrenamiento.   
    
\end{thesischapter}
     
% SERIOUS GAME REQUIREMENTS
\begin{thesischapter}{2} {Diseño e implementación del Juego Serio}
En este capítulo se discuten los detalles de desarrollo de los aspectos citados en el capítulo anterior. Este comienza con una descripción y caracterización general del sistema, donde se  abordan cada uno de los componentes requeridos para su completo funcionamiento. Posteriormente se detalla la ingienría de software requerida en la etapa de conceptualización de la aplicación, se explican de forma detallada los aspectos teóricos y de implementación de la base de datos, el funcionamiento del protocolo de comunicación y por último los escenarios de juegos requeridos en las rutinas de entrenamiento ligero y clínico, y las estadísticas generadas por estos. Como herramienta de desarrollo se utilizó c\#.

% SYSTEM DESCRIPTION AND CHARACTERIZATION TO APPLY
\input{main/chapter2/section1/content.tex}
     
% SERIOUS GAME REQUIREMENTS
\input{main/chapter2/section2/content.tex}    

% USE CASE DEFINITION
\input{main/chapter2/section3/content.tex}

% USE CASE REALIZATION
\input{main/chapter2/section4/content.tex}

% DATABASE DESIGN 
\input{main/chapter2/section5/content.tex}

% DATA MANIPULATION
\input{main/chapter2/section6/content.tex}

% COMMUNICATION 
\input{main/chapter2/section7/content.tex}

% TRAINING SCENARIOS
\input{main/chapter2/section8/content.tex}

% EMG GRAPHIC
\input{main/chapter2/section9/content.tex}

% STATICAL REPORTS GENERATION
\input{main/chapter2/section10/content.tex}

\subthesischapter{Conclusiones del capítulo}
Se presentó una descripción del sistema de adquisición de datos para rehabilitación, sus componentes, características distintivas y su funcionamiento. Se identificaron y definieron los requisitos del juego  serio, tanto funcionales como no funcionales, así como los actores y casos de usos del sistema que establecieron las bases fundamentales para el desarrollo de la aplicación. Se realizó el diseño de la base de datos, abarcando tanto el modelo lógico como el físico, lo que aseguró una estructura robusta y eficiente para el almacenamiento de los datos. La manipulación de los datos se abordó de manera integral, desde la conexión con la base de datos hasta la persistencia de los resultados estadísticos. Se diseñó e implementó la comunicación con el pedal motorizado y la implementación de la interfaz gráfica para la representación de los datos EMG. Se definieron los escenarios de entrenamiento para las modalidades Ligero y Clínico, asegurando una cobertura completa de las necesidad de entrenamiento del usuario. Por último en el ámbito estadístico se desarrolló una serie de gráficos para el seguimiento de los resultados en las rutinas de entrenamiento.   
    
\end{thesischapter}    

% USE CASE DEFINITION
\begin{thesischapter}{2} {Diseño e implementación del Juego Serio}
En este capítulo se discuten los detalles de desarrollo de los aspectos citados en el capítulo anterior. Este comienza con una descripción y caracterización general del sistema, donde se  abordan cada uno de los componentes requeridos para su completo funcionamiento. Posteriormente se detalla la ingienría de software requerida en la etapa de conceptualización de la aplicación, se explican de forma detallada los aspectos teóricos y de implementación de la base de datos, el funcionamiento del protocolo de comunicación y por último los escenarios de juegos requeridos en las rutinas de entrenamiento ligero y clínico, y las estadísticas generadas por estos. Como herramienta de desarrollo se utilizó c\#.

% SYSTEM DESCRIPTION AND CHARACTERIZATION TO APPLY
\input{main/chapter2/section1/content.tex}
     
% SERIOUS GAME REQUIREMENTS
\input{main/chapter2/section2/content.tex}    

% USE CASE DEFINITION
\input{main/chapter2/section3/content.tex}

% USE CASE REALIZATION
\input{main/chapter2/section4/content.tex}

% DATABASE DESIGN 
\input{main/chapter2/section5/content.tex}

% DATA MANIPULATION
\input{main/chapter2/section6/content.tex}

% COMMUNICATION 
\input{main/chapter2/section7/content.tex}

% TRAINING SCENARIOS
\input{main/chapter2/section8/content.tex}

% EMG GRAPHIC
\input{main/chapter2/section9/content.tex}

% STATICAL REPORTS GENERATION
\input{main/chapter2/section10/content.tex}

\subthesischapter{Conclusiones del capítulo}
Se presentó una descripción del sistema de adquisición de datos para rehabilitación, sus componentes, características distintivas y su funcionamiento. Se identificaron y definieron los requisitos del juego  serio, tanto funcionales como no funcionales, así como los actores y casos de usos del sistema que establecieron las bases fundamentales para el desarrollo de la aplicación. Se realizó el diseño de la base de datos, abarcando tanto el modelo lógico como el físico, lo que aseguró una estructura robusta y eficiente para el almacenamiento de los datos. La manipulación de los datos se abordó de manera integral, desde la conexión con la base de datos hasta la persistencia de los resultados estadísticos. Se diseñó e implementó la comunicación con el pedal motorizado y la implementación de la interfaz gráfica para la representación de los datos EMG. Se definieron los escenarios de entrenamiento para las modalidades Ligero y Clínico, asegurando una cobertura completa de las necesidad de entrenamiento del usuario. Por último en el ámbito estadístico se desarrolló una serie de gráficos para el seguimiento de los resultados en las rutinas de entrenamiento.   
    
\end{thesischapter}

% USE CASE REALIZATION
\begin{thesischapter}{2} {Diseño e implementación del Juego Serio}
En este capítulo se discuten los detalles de desarrollo de los aspectos citados en el capítulo anterior. Este comienza con una descripción y caracterización general del sistema, donde se  abordan cada uno de los componentes requeridos para su completo funcionamiento. Posteriormente se detalla la ingienría de software requerida en la etapa de conceptualización de la aplicación, se explican de forma detallada los aspectos teóricos y de implementación de la base de datos, el funcionamiento del protocolo de comunicación y por último los escenarios de juegos requeridos en las rutinas de entrenamiento ligero y clínico, y las estadísticas generadas por estos. Como herramienta de desarrollo se utilizó c\#.

% SYSTEM DESCRIPTION AND CHARACTERIZATION TO APPLY
\input{main/chapter2/section1/content.tex}
     
% SERIOUS GAME REQUIREMENTS
\input{main/chapter2/section2/content.tex}    

% USE CASE DEFINITION
\input{main/chapter2/section3/content.tex}

% USE CASE REALIZATION
\input{main/chapter2/section4/content.tex}

% DATABASE DESIGN 
\input{main/chapter2/section5/content.tex}

% DATA MANIPULATION
\input{main/chapter2/section6/content.tex}

% COMMUNICATION 
\input{main/chapter2/section7/content.tex}

% TRAINING SCENARIOS
\input{main/chapter2/section8/content.tex}

% EMG GRAPHIC
\input{main/chapter2/section9/content.tex}

% STATICAL REPORTS GENERATION
\input{main/chapter2/section10/content.tex}

\subthesischapter{Conclusiones del capítulo}
Se presentó una descripción del sistema de adquisición de datos para rehabilitación, sus componentes, características distintivas y su funcionamiento. Se identificaron y definieron los requisitos del juego  serio, tanto funcionales como no funcionales, así como los actores y casos de usos del sistema que establecieron las bases fundamentales para el desarrollo de la aplicación. Se realizó el diseño de la base de datos, abarcando tanto el modelo lógico como el físico, lo que aseguró una estructura robusta y eficiente para el almacenamiento de los datos. La manipulación de los datos se abordó de manera integral, desde la conexión con la base de datos hasta la persistencia de los resultados estadísticos. Se diseñó e implementó la comunicación con el pedal motorizado y la implementación de la interfaz gráfica para la representación de los datos EMG. Se definieron los escenarios de entrenamiento para las modalidades Ligero y Clínico, asegurando una cobertura completa de las necesidad de entrenamiento del usuario. Por último en el ámbito estadístico se desarrolló una serie de gráficos para el seguimiento de los resultados en las rutinas de entrenamiento.   
    
\end{thesischapter}

% DATABASE DESIGN 
\begin{thesischapter}{2} {Diseño e implementación del Juego Serio}
En este capítulo se discuten los detalles de desarrollo de los aspectos citados en el capítulo anterior. Este comienza con una descripción y caracterización general del sistema, donde se  abordan cada uno de los componentes requeridos para su completo funcionamiento. Posteriormente se detalla la ingienría de software requerida en la etapa de conceptualización de la aplicación, se explican de forma detallada los aspectos teóricos y de implementación de la base de datos, el funcionamiento del protocolo de comunicación y por último los escenarios de juegos requeridos en las rutinas de entrenamiento ligero y clínico, y las estadísticas generadas por estos. Como herramienta de desarrollo se utilizó c\#.

% SYSTEM DESCRIPTION AND CHARACTERIZATION TO APPLY
\input{main/chapter2/section1/content.tex}
     
% SERIOUS GAME REQUIREMENTS
\input{main/chapter2/section2/content.tex}    

% USE CASE DEFINITION
\input{main/chapter2/section3/content.tex}

% USE CASE REALIZATION
\input{main/chapter2/section4/content.tex}

% DATABASE DESIGN 
\input{main/chapter2/section5/content.tex}

% DATA MANIPULATION
\input{main/chapter2/section6/content.tex}

% COMMUNICATION 
\input{main/chapter2/section7/content.tex}

% TRAINING SCENARIOS
\input{main/chapter2/section8/content.tex}

% EMG GRAPHIC
\input{main/chapter2/section9/content.tex}

% STATICAL REPORTS GENERATION
\input{main/chapter2/section10/content.tex}

\subthesischapter{Conclusiones del capítulo}
Se presentó una descripción del sistema de adquisición de datos para rehabilitación, sus componentes, características distintivas y su funcionamiento. Se identificaron y definieron los requisitos del juego  serio, tanto funcionales como no funcionales, así como los actores y casos de usos del sistema que establecieron las bases fundamentales para el desarrollo de la aplicación. Se realizó el diseño de la base de datos, abarcando tanto el modelo lógico como el físico, lo que aseguró una estructura robusta y eficiente para el almacenamiento de los datos. La manipulación de los datos se abordó de manera integral, desde la conexión con la base de datos hasta la persistencia de los resultados estadísticos. Se diseñó e implementó la comunicación con el pedal motorizado y la implementación de la interfaz gráfica para la representación de los datos EMG. Se definieron los escenarios de entrenamiento para las modalidades Ligero y Clínico, asegurando una cobertura completa de las necesidad de entrenamiento del usuario. Por último en el ámbito estadístico se desarrolló una serie de gráficos para el seguimiento de los resultados en las rutinas de entrenamiento.   
    
\end{thesischapter}

% DATA MANIPULATION
\begin{thesischapter}{2} {Diseño e implementación del Juego Serio}
En este capítulo se discuten los detalles de desarrollo de los aspectos citados en el capítulo anterior. Este comienza con una descripción y caracterización general del sistema, donde se  abordan cada uno de los componentes requeridos para su completo funcionamiento. Posteriormente se detalla la ingienría de software requerida en la etapa de conceptualización de la aplicación, se explican de forma detallada los aspectos teóricos y de implementación de la base de datos, el funcionamiento del protocolo de comunicación y por último los escenarios de juegos requeridos en las rutinas de entrenamiento ligero y clínico, y las estadísticas generadas por estos. Como herramienta de desarrollo se utilizó c\#.

% SYSTEM DESCRIPTION AND CHARACTERIZATION TO APPLY
\input{main/chapter2/section1/content.tex}
     
% SERIOUS GAME REQUIREMENTS
\input{main/chapter2/section2/content.tex}    

% USE CASE DEFINITION
\input{main/chapter2/section3/content.tex}

% USE CASE REALIZATION
\input{main/chapter2/section4/content.tex}

% DATABASE DESIGN 
\input{main/chapter2/section5/content.tex}

% DATA MANIPULATION
\input{main/chapter2/section6/content.tex}

% COMMUNICATION 
\input{main/chapter2/section7/content.tex}

% TRAINING SCENARIOS
\input{main/chapter2/section8/content.tex}

% EMG GRAPHIC
\input{main/chapter2/section9/content.tex}

% STATICAL REPORTS GENERATION
\input{main/chapter2/section10/content.tex}

\subthesischapter{Conclusiones del capítulo}
Se presentó una descripción del sistema de adquisición de datos para rehabilitación, sus componentes, características distintivas y su funcionamiento. Se identificaron y definieron los requisitos del juego  serio, tanto funcionales como no funcionales, así como los actores y casos de usos del sistema que establecieron las bases fundamentales para el desarrollo de la aplicación. Se realizó el diseño de la base de datos, abarcando tanto el modelo lógico como el físico, lo que aseguró una estructura robusta y eficiente para el almacenamiento de los datos. La manipulación de los datos se abordó de manera integral, desde la conexión con la base de datos hasta la persistencia de los resultados estadísticos. Se diseñó e implementó la comunicación con el pedal motorizado y la implementación de la interfaz gráfica para la representación de los datos EMG. Se definieron los escenarios de entrenamiento para las modalidades Ligero y Clínico, asegurando una cobertura completa de las necesidad de entrenamiento del usuario. Por último en el ámbito estadístico se desarrolló una serie de gráficos para el seguimiento de los resultados en las rutinas de entrenamiento.   
    
\end{thesischapter}

% COMMUNICATION 
\begin{thesischapter}{2} {Diseño e implementación del Juego Serio}
En este capítulo se discuten los detalles de desarrollo de los aspectos citados en el capítulo anterior. Este comienza con una descripción y caracterización general del sistema, donde se  abordan cada uno de los componentes requeridos para su completo funcionamiento. Posteriormente se detalla la ingienría de software requerida en la etapa de conceptualización de la aplicación, se explican de forma detallada los aspectos teóricos y de implementación de la base de datos, el funcionamiento del protocolo de comunicación y por último los escenarios de juegos requeridos en las rutinas de entrenamiento ligero y clínico, y las estadísticas generadas por estos. Como herramienta de desarrollo se utilizó c\#.

% SYSTEM DESCRIPTION AND CHARACTERIZATION TO APPLY
\input{main/chapter2/section1/content.tex}
     
% SERIOUS GAME REQUIREMENTS
\input{main/chapter2/section2/content.tex}    

% USE CASE DEFINITION
\input{main/chapter2/section3/content.tex}

% USE CASE REALIZATION
\input{main/chapter2/section4/content.tex}

% DATABASE DESIGN 
\input{main/chapter2/section5/content.tex}

% DATA MANIPULATION
\input{main/chapter2/section6/content.tex}

% COMMUNICATION 
\input{main/chapter2/section7/content.tex}

% TRAINING SCENARIOS
\input{main/chapter2/section8/content.tex}

% EMG GRAPHIC
\input{main/chapter2/section9/content.tex}

% STATICAL REPORTS GENERATION
\input{main/chapter2/section10/content.tex}

\subthesischapter{Conclusiones del capítulo}
Se presentó una descripción del sistema de adquisición de datos para rehabilitación, sus componentes, características distintivas y su funcionamiento. Se identificaron y definieron los requisitos del juego  serio, tanto funcionales como no funcionales, así como los actores y casos de usos del sistema que establecieron las bases fundamentales para el desarrollo de la aplicación. Se realizó el diseño de la base de datos, abarcando tanto el modelo lógico como el físico, lo que aseguró una estructura robusta y eficiente para el almacenamiento de los datos. La manipulación de los datos se abordó de manera integral, desde la conexión con la base de datos hasta la persistencia de los resultados estadísticos. Se diseñó e implementó la comunicación con el pedal motorizado y la implementación de la interfaz gráfica para la representación de los datos EMG. Se definieron los escenarios de entrenamiento para las modalidades Ligero y Clínico, asegurando una cobertura completa de las necesidad de entrenamiento del usuario. Por último en el ámbito estadístico se desarrolló una serie de gráficos para el seguimiento de los resultados en las rutinas de entrenamiento.   
    
\end{thesischapter}

% TRAINING SCENARIOS
\begin{thesischapter}{2} {Diseño e implementación del Juego Serio}
En este capítulo se discuten los detalles de desarrollo de los aspectos citados en el capítulo anterior. Este comienza con una descripción y caracterización general del sistema, donde se  abordan cada uno de los componentes requeridos para su completo funcionamiento. Posteriormente se detalla la ingienría de software requerida en la etapa de conceptualización de la aplicación, se explican de forma detallada los aspectos teóricos y de implementación de la base de datos, el funcionamiento del protocolo de comunicación y por último los escenarios de juegos requeridos en las rutinas de entrenamiento ligero y clínico, y las estadísticas generadas por estos. Como herramienta de desarrollo se utilizó c\#.

% SYSTEM DESCRIPTION AND CHARACTERIZATION TO APPLY
\input{main/chapter2/section1/content.tex}
     
% SERIOUS GAME REQUIREMENTS
\input{main/chapter2/section2/content.tex}    

% USE CASE DEFINITION
\input{main/chapter2/section3/content.tex}

% USE CASE REALIZATION
\input{main/chapter2/section4/content.tex}

% DATABASE DESIGN 
\input{main/chapter2/section5/content.tex}

% DATA MANIPULATION
\input{main/chapter2/section6/content.tex}

% COMMUNICATION 
\input{main/chapter2/section7/content.tex}

% TRAINING SCENARIOS
\input{main/chapter2/section8/content.tex}

% EMG GRAPHIC
\input{main/chapter2/section9/content.tex}

% STATICAL REPORTS GENERATION
\input{main/chapter2/section10/content.tex}

\subthesischapter{Conclusiones del capítulo}
Se presentó una descripción del sistema de adquisición de datos para rehabilitación, sus componentes, características distintivas y su funcionamiento. Se identificaron y definieron los requisitos del juego  serio, tanto funcionales como no funcionales, así como los actores y casos de usos del sistema que establecieron las bases fundamentales para el desarrollo de la aplicación. Se realizó el diseño de la base de datos, abarcando tanto el modelo lógico como el físico, lo que aseguró una estructura robusta y eficiente para el almacenamiento de los datos. La manipulación de los datos se abordó de manera integral, desde la conexión con la base de datos hasta la persistencia de los resultados estadísticos. Se diseñó e implementó la comunicación con el pedal motorizado y la implementación de la interfaz gráfica para la representación de los datos EMG. Se definieron los escenarios de entrenamiento para las modalidades Ligero y Clínico, asegurando una cobertura completa de las necesidad de entrenamiento del usuario. Por último en el ámbito estadístico se desarrolló una serie de gráficos para el seguimiento de los resultados en las rutinas de entrenamiento.   
    
\end{thesischapter}

% EMG GRAPHIC
\begin{thesischapter}{2} {Diseño e implementación del Juego Serio}
En este capítulo se discuten los detalles de desarrollo de los aspectos citados en el capítulo anterior. Este comienza con una descripción y caracterización general del sistema, donde se  abordan cada uno de los componentes requeridos para su completo funcionamiento. Posteriormente se detalla la ingienría de software requerida en la etapa de conceptualización de la aplicación, se explican de forma detallada los aspectos teóricos y de implementación de la base de datos, el funcionamiento del protocolo de comunicación y por último los escenarios de juegos requeridos en las rutinas de entrenamiento ligero y clínico, y las estadísticas generadas por estos. Como herramienta de desarrollo se utilizó c\#.

% SYSTEM DESCRIPTION AND CHARACTERIZATION TO APPLY
\input{main/chapter2/section1/content.tex}
     
% SERIOUS GAME REQUIREMENTS
\input{main/chapter2/section2/content.tex}    

% USE CASE DEFINITION
\input{main/chapter2/section3/content.tex}

% USE CASE REALIZATION
\input{main/chapter2/section4/content.tex}

% DATABASE DESIGN 
\input{main/chapter2/section5/content.tex}

% DATA MANIPULATION
\input{main/chapter2/section6/content.tex}

% COMMUNICATION 
\input{main/chapter2/section7/content.tex}

% TRAINING SCENARIOS
\input{main/chapter2/section8/content.tex}

% EMG GRAPHIC
\input{main/chapter2/section9/content.tex}

% STATICAL REPORTS GENERATION
\input{main/chapter2/section10/content.tex}

\subthesischapter{Conclusiones del capítulo}
Se presentó una descripción del sistema de adquisición de datos para rehabilitación, sus componentes, características distintivas y su funcionamiento. Se identificaron y definieron los requisitos del juego  serio, tanto funcionales como no funcionales, así como los actores y casos de usos del sistema que establecieron las bases fundamentales para el desarrollo de la aplicación. Se realizó el diseño de la base de datos, abarcando tanto el modelo lógico como el físico, lo que aseguró una estructura robusta y eficiente para el almacenamiento de los datos. La manipulación de los datos se abordó de manera integral, desde la conexión con la base de datos hasta la persistencia de los resultados estadísticos. Se diseñó e implementó la comunicación con el pedal motorizado y la implementación de la interfaz gráfica para la representación de los datos EMG. Se definieron los escenarios de entrenamiento para las modalidades Ligero y Clínico, asegurando una cobertura completa de las necesidad de entrenamiento del usuario. Por último en el ámbito estadístico se desarrolló una serie de gráficos para el seguimiento de los resultados en las rutinas de entrenamiento.   
    
\end{thesischapter}

% STATICAL REPORTS GENERATION
\begin{thesischapter}{2} {Diseño e implementación del Juego Serio}
En este capítulo se discuten los detalles de desarrollo de los aspectos citados en el capítulo anterior. Este comienza con una descripción y caracterización general del sistema, donde se  abordan cada uno de los componentes requeridos para su completo funcionamiento. Posteriormente se detalla la ingienría de software requerida en la etapa de conceptualización de la aplicación, se explican de forma detallada los aspectos teóricos y de implementación de la base de datos, el funcionamiento del protocolo de comunicación y por último los escenarios de juegos requeridos en las rutinas de entrenamiento ligero y clínico, y las estadísticas generadas por estos. Como herramienta de desarrollo se utilizó c\#.

% SYSTEM DESCRIPTION AND CHARACTERIZATION TO APPLY
\input{main/chapter2/section1/content.tex}
     
% SERIOUS GAME REQUIREMENTS
\input{main/chapter2/section2/content.tex}    

% USE CASE DEFINITION
\input{main/chapter2/section3/content.tex}

% USE CASE REALIZATION
\input{main/chapter2/section4/content.tex}

% DATABASE DESIGN 
\input{main/chapter2/section5/content.tex}

% DATA MANIPULATION
\input{main/chapter2/section6/content.tex}

% COMMUNICATION 
\input{main/chapter2/section7/content.tex}

% TRAINING SCENARIOS
\input{main/chapter2/section8/content.tex}

% EMG GRAPHIC
\input{main/chapter2/section9/content.tex}

% STATICAL REPORTS GENERATION
\input{main/chapter2/section10/content.tex}

\subthesischapter{Conclusiones del capítulo}
Se presentó una descripción del sistema de adquisición de datos para rehabilitación, sus componentes, características distintivas y su funcionamiento. Se identificaron y definieron los requisitos del juego  serio, tanto funcionales como no funcionales, así como los actores y casos de usos del sistema que establecieron las bases fundamentales para el desarrollo de la aplicación. Se realizó el diseño de la base de datos, abarcando tanto el modelo lógico como el físico, lo que aseguró una estructura robusta y eficiente para el almacenamiento de los datos. La manipulación de los datos se abordó de manera integral, desde la conexión con la base de datos hasta la persistencia de los resultados estadísticos. Se diseñó e implementó la comunicación con el pedal motorizado y la implementación de la interfaz gráfica para la representación de los datos EMG. Se definieron los escenarios de entrenamiento para las modalidades Ligero y Clínico, asegurando una cobertura completa de las necesidad de entrenamiento del usuario. Por último en el ámbito estadístico se desarrolló una serie de gráficos para el seguimiento de los resultados en las rutinas de entrenamiento.   
    
\end{thesischapter}

\subthesischapter{Conclusiones del capítulo}
Se presentó una descripción del sistema de adquisición de datos para rehabilitación, sus componentes, características distintivas y su funcionamiento. Se identificaron y definieron los requisitos del juego  serio, tanto funcionales como no funcionales, así como los actores y casos de usos del sistema que establecieron las bases fundamentales para el desarrollo de la aplicación. Se realizó el diseño de la base de datos, abarcando tanto el modelo lógico como el físico, lo que aseguró una estructura robusta y eficiente para el almacenamiento de los datos. La manipulación de los datos se abordó de manera integral, desde la conexión con la base de datos hasta la persistencia de los resultados estadísticos. Se diseñó e implementó la comunicación con el pedal motorizado y la implementación de la interfaz gráfica para la representación de los datos EMG. Se definieron los escenarios de entrenamiento para las modalidades Ligero y Clínico, asegurando una cobertura completa de las necesidad de entrenamiento del usuario. Por último en el ámbito estadístico se desarrolló una serie de gráficos para el seguimiento de los resultados en las rutinas de entrenamiento.   
    
\end{thesischapter}

% DATA MANIPULATION
\begin{thesischapter}{2} {Diseño e implementación del Juego Serio}
En este capítulo se discuten los detalles de desarrollo de los aspectos citados en el capítulo anterior. Este comienza con una descripción y caracterización general del sistema, donde se  abordan cada uno de los componentes requeridos para su completo funcionamiento. Posteriormente se detalla la ingienría de software requerida en la etapa de conceptualización de la aplicación, se explican de forma detallada los aspectos teóricos y de implementación de la base de datos, el funcionamiento del protocolo de comunicación y por último los escenarios de juegos requeridos en las rutinas de entrenamiento ligero y clínico, y las estadísticas generadas por estos. Como herramienta de desarrollo se utilizó c\#.

% SYSTEM DESCRIPTION AND CHARACTERIZATION TO APPLY
\begin{thesischapter}{2} {Diseño e implementación del Juego Serio}
En este capítulo se discuten los detalles de desarrollo de los aspectos citados en el capítulo anterior. Este comienza con una descripción y caracterización general del sistema, donde se  abordan cada uno de los componentes requeridos para su completo funcionamiento. Posteriormente se detalla la ingienría de software requerida en la etapa de conceptualización de la aplicación, se explican de forma detallada los aspectos teóricos y de implementación de la base de datos, el funcionamiento del protocolo de comunicación y por último los escenarios de juegos requeridos en las rutinas de entrenamiento ligero y clínico, y las estadísticas generadas por estos. Como herramienta de desarrollo se utilizó c\#.

% SYSTEM DESCRIPTION AND CHARACTERIZATION TO APPLY
\input{main/chapter2/section1/content.tex}
     
% SERIOUS GAME REQUIREMENTS
\input{main/chapter2/section2/content.tex}    

% USE CASE DEFINITION
\input{main/chapter2/section3/content.tex}

% USE CASE REALIZATION
\input{main/chapter2/section4/content.tex}

% DATABASE DESIGN 
\input{main/chapter2/section5/content.tex}

% DATA MANIPULATION
\input{main/chapter2/section6/content.tex}

% COMMUNICATION 
\input{main/chapter2/section7/content.tex}

% TRAINING SCENARIOS
\input{main/chapter2/section8/content.tex}

% EMG GRAPHIC
\input{main/chapter2/section9/content.tex}

% STATICAL REPORTS GENERATION
\input{main/chapter2/section10/content.tex}

\subthesischapter{Conclusiones del capítulo}
Se presentó una descripción del sistema de adquisición de datos para rehabilitación, sus componentes, características distintivas y su funcionamiento. Se identificaron y definieron los requisitos del juego  serio, tanto funcionales como no funcionales, así como los actores y casos de usos del sistema que establecieron las bases fundamentales para el desarrollo de la aplicación. Se realizó el diseño de la base de datos, abarcando tanto el modelo lógico como el físico, lo que aseguró una estructura robusta y eficiente para el almacenamiento de los datos. La manipulación de los datos se abordó de manera integral, desde la conexión con la base de datos hasta la persistencia de los resultados estadísticos. Se diseñó e implementó la comunicación con el pedal motorizado y la implementación de la interfaz gráfica para la representación de los datos EMG. Se definieron los escenarios de entrenamiento para las modalidades Ligero y Clínico, asegurando una cobertura completa de las necesidad de entrenamiento del usuario. Por último en el ámbito estadístico se desarrolló una serie de gráficos para el seguimiento de los resultados en las rutinas de entrenamiento.   
    
\end{thesischapter}
     
% SERIOUS GAME REQUIREMENTS
\begin{thesischapter}{2} {Diseño e implementación del Juego Serio}
En este capítulo se discuten los detalles de desarrollo de los aspectos citados en el capítulo anterior. Este comienza con una descripción y caracterización general del sistema, donde se  abordan cada uno de los componentes requeridos para su completo funcionamiento. Posteriormente se detalla la ingienría de software requerida en la etapa de conceptualización de la aplicación, se explican de forma detallada los aspectos teóricos y de implementación de la base de datos, el funcionamiento del protocolo de comunicación y por último los escenarios de juegos requeridos en las rutinas de entrenamiento ligero y clínico, y las estadísticas generadas por estos. Como herramienta de desarrollo se utilizó c\#.

% SYSTEM DESCRIPTION AND CHARACTERIZATION TO APPLY
\input{main/chapter2/section1/content.tex}
     
% SERIOUS GAME REQUIREMENTS
\input{main/chapter2/section2/content.tex}    

% USE CASE DEFINITION
\input{main/chapter2/section3/content.tex}

% USE CASE REALIZATION
\input{main/chapter2/section4/content.tex}

% DATABASE DESIGN 
\input{main/chapter2/section5/content.tex}

% DATA MANIPULATION
\input{main/chapter2/section6/content.tex}

% COMMUNICATION 
\input{main/chapter2/section7/content.tex}

% TRAINING SCENARIOS
\input{main/chapter2/section8/content.tex}

% EMG GRAPHIC
\input{main/chapter2/section9/content.tex}

% STATICAL REPORTS GENERATION
\input{main/chapter2/section10/content.tex}

\subthesischapter{Conclusiones del capítulo}
Se presentó una descripción del sistema de adquisición de datos para rehabilitación, sus componentes, características distintivas y su funcionamiento. Se identificaron y definieron los requisitos del juego  serio, tanto funcionales como no funcionales, así como los actores y casos de usos del sistema que establecieron las bases fundamentales para el desarrollo de la aplicación. Se realizó el diseño de la base de datos, abarcando tanto el modelo lógico como el físico, lo que aseguró una estructura robusta y eficiente para el almacenamiento de los datos. La manipulación de los datos se abordó de manera integral, desde la conexión con la base de datos hasta la persistencia de los resultados estadísticos. Se diseñó e implementó la comunicación con el pedal motorizado y la implementación de la interfaz gráfica para la representación de los datos EMG. Se definieron los escenarios de entrenamiento para las modalidades Ligero y Clínico, asegurando una cobertura completa de las necesidad de entrenamiento del usuario. Por último en el ámbito estadístico se desarrolló una serie de gráficos para el seguimiento de los resultados en las rutinas de entrenamiento.   
    
\end{thesischapter}    

% USE CASE DEFINITION
\begin{thesischapter}{2} {Diseño e implementación del Juego Serio}
En este capítulo se discuten los detalles de desarrollo de los aspectos citados en el capítulo anterior. Este comienza con una descripción y caracterización general del sistema, donde se  abordan cada uno de los componentes requeridos para su completo funcionamiento. Posteriormente se detalla la ingienría de software requerida en la etapa de conceptualización de la aplicación, se explican de forma detallada los aspectos teóricos y de implementación de la base de datos, el funcionamiento del protocolo de comunicación y por último los escenarios de juegos requeridos en las rutinas de entrenamiento ligero y clínico, y las estadísticas generadas por estos. Como herramienta de desarrollo se utilizó c\#.

% SYSTEM DESCRIPTION AND CHARACTERIZATION TO APPLY
\input{main/chapter2/section1/content.tex}
     
% SERIOUS GAME REQUIREMENTS
\input{main/chapter2/section2/content.tex}    

% USE CASE DEFINITION
\input{main/chapter2/section3/content.tex}

% USE CASE REALIZATION
\input{main/chapter2/section4/content.tex}

% DATABASE DESIGN 
\input{main/chapter2/section5/content.tex}

% DATA MANIPULATION
\input{main/chapter2/section6/content.tex}

% COMMUNICATION 
\input{main/chapter2/section7/content.tex}

% TRAINING SCENARIOS
\input{main/chapter2/section8/content.tex}

% EMG GRAPHIC
\input{main/chapter2/section9/content.tex}

% STATICAL REPORTS GENERATION
\input{main/chapter2/section10/content.tex}

\subthesischapter{Conclusiones del capítulo}
Se presentó una descripción del sistema de adquisición de datos para rehabilitación, sus componentes, características distintivas y su funcionamiento. Se identificaron y definieron los requisitos del juego  serio, tanto funcionales como no funcionales, así como los actores y casos de usos del sistema que establecieron las bases fundamentales para el desarrollo de la aplicación. Se realizó el diseño de la base de datos, abarcando tanto el modelo lógico como el físico, lo que aseguró una estructura robusta y eficiente para el almacenamiento de los datos. La manipulación de los datos se abordó de manera integral, desde la conexión con la base de datos hasta la persistencia de los resultados estadísticos. Se diseñó e implementó la comunicación con el pedal motorizado y la implementación de la interfaz gráfica para la representación de los datos EMG. Se definieron los escenarios de entrenamiento para las modalidades Ligero y Clínico, asegurando una cobertura completa de las necesidad de entrenamiento del usuario. Por último en el ámbito estadístico se desarrolló una serie de gráficos para el seguimiento de los resultados en las rutinas de entrenamiento.   
    
\end{thesischapter}

% USE CASE REALIZATION
\begin{thesischapter}{2} {Diseño e implementación del Juego Serio}
En este capítulo se discuten los detalles de desarrollo de los aspectos citados en el capítulo anterior. Este comienza con una descripción y caracterización general del sistema, donde se  abordan cada uno de los componentes requeridos para su completo funcionamiento. Posteriormente se detalla la ingienría de software requerida en la etapa de conceptualización de la aplicación, se explican de forma detallada los aspectos teóricos y de implementación de la base de datos, el funcionamiento del protocolo de comunicación y por último los escenarios de juegos requeridos en las rutinas de entrenamiento ligero y clínico, y las estadísticas generadas por estos. Como herramienta de desarrollo se utilizó c\#.

% SYSTEM DESCRIPTION AND CHARACTERIZATION TO APPLY
\input{main/chapter2/section1/content.tex}
     
% SERIOUS GAME REQUIREMENTS
\input{main/chapter2/section2/content.tex}    

% USE CASE DEFINITION
\input{main/chapter2/section3/content.tex}

% USE CASE REALIZATION
\input{main/chapter2/section4/content.tex}

% DATABASE DESIGN 
\input{main/chapter2/section5/content.tex}

% DATA MANIPULATION
\input{main/chapter2/section6/content.tex}

% COMMUNICATION 
\input{main/chapter2/section7/content.tex}

% TRAINING SCENARIOS
\input{main/chapter2/section8/content.tex}

% EMG GRAPHIC
\input{main/chapter2/section9/content.tex}

% STATICAL REPORTS GENERATION
\input{main/chapter2/section10/content.tex}

\subthesischapter{Conclusiones del capítulo}
Se presentó una descripción del sistema de adquisición de datos para rehabilitación, sus componentes, características distintivas y su funcionamiento. Se identificaron y definieron los requisitos del juego  serio, tanto funcionales como no funcionales, así como los actores y casos de usos del sistema que establecieron las bases fundamentales para el desarrollo de la aplicación. Se realizó el diseño de la base de datos, abarcando tanto el modelo lógico como el físico, lo que aseguró una estructura robusta y eficiente para el almacenamiento de los datos. La manipulación de los datos se abordó de manera integral, desde la conexión con la base de datos hasta la persistencia de los resultados estadísticos. Se diseñó e implementó la comunicación con el pedal motorizado y la implementación de la interfaz gráfica para la representación de los datos EMG. Se definieron los escenarios de entrenamiento para las modalidades Ligero y Clínico, asegurando una cobertura completa de las necesidad de entrenamiento del usuario. Por último en el ámbito estadístico se desarrolló una serie de gráficos para el seguimiento de los resultados en las rutinas de entrenamiento.   
    
\end{thesischapter}

% DATABASE DESIGN 
\begin{thesischapter}{2} {Diseño e implementación del Juego Serio}
En este capítulo se discuten los detalles de desarrollo de los aspectos citados en el capítulo anterior. Este comienza con una descripción y caracterización general del sistema, donde se  abordan cada uno de los componentes requeridos para su completo funcionamiento. Posteriormente se detalla la ingienría de software requerida en la etapa de conceptualización de la aplicación, se explican de forma detallada los aspectos teóricos y de implementación de la base de datos, el funcionamiento del protocolo de comunicación y por último los escenarios de juegos requeridos en las rutinas de entrenamiento ligero y clínico, y las estadísticas generadas por estos. Como herramienta de desarrollo se utilizó c\#.

% SYSTEM DESCRIPTION AND CHARACTERIZATION TO APPLY
\input{main/chapter2/section1/content.tex}
     
% SERIOUS GAME REQUIREMENTS
\input{main/chapter2/section2/content.tex}    

% USE CASE DEFINITION
\input{main/chapter2/section3/content.tex}

% USE CASE REALIZATION
\input{main/chapter2/section4/content.tex}

% DATABASE DESIGN 
\input{main/chapter2/section5/content.tex}

% DATA MANIPULATION
\input{main/chapter2/section6/content.tex}

% COMMUNICATION 
\input{main/chapter2/section7/content.tex}

% TRAINING SCENARIOS
\input{main/chapter2/section8/content.tex}

% EMG GRAPHIC
\input{main/chapter2/section9/content.tex}

% STATICAL REPORTS GENERATION
\input{main/chapter2/section10/content.tex}

\subthesischapter{Conclusiones del capítulo}
Se presentó una descripción del sistema de adquisición de datos para rehabilitación, sus componentes, características distintivas y su funcionamiento. Se identificaron y definieron los requisitos del juego  serio, tanto funcionales como no funcionales, así como los actores y casos de usos del sistema que establecieron las bases fundamentales para el desarrollo de la aplicación. Se realizó el diseño de la base de datos, abarcando tanto el modelo lógico como el físico, lo que aseguró una estructura robusta y eficiente para el almacenamiento de los datos. La manipulación de los datos se abordó de manera integral, desde la conexión con la base de datos hasta la persistencia de los resultados estadísticos. Se diseñó e implementó la comunicación con el pedal motorizado y la implementación de la interfaz gráfica para la representación de los datos EMG. Se definieron los escenarios de entrenamiento para las modalidades Ligero y Clínico, asegurando una cobertura completa de las necesidad de entrenamiento del usuario. Por último en el ámbito estadístico se desarrolló una serie de gráficos para el seguimiento de los resultados en las rutinas de entrenamiento.   
    
\end{thesischapter}

% DATA MANIPULATION
\begin{thesischapter}{2} {Diseño e implementación del Juego Serio}
En este capítulo se discuten los detalles de desarrollo de los aspectos citados en el capítulo anterior. Este comienza con una descripción y caracterización general del sistema, donde se  abordan cada uno de los componentes requeridos para su completo funcionamiento. Posteriormente se detalla la ingienría de software requerida en la etapa de conceptualización de la aplicación, se explican de forma detallada los aspectos teóricos y de implementación de la base de datos, el funcionamiento del protocolo de comunicación y por último los escenarios de juegos requeridos en las rutinas de entrenamiento ligero y clínico, y las estadísticas generadas por estos. Como herramienta de desarrollo se utilizó c\#.

% SYSTEM DESCRIPTION AND CHARACTERIZATION TO APPLY
\input{main/chapter2/section1/content.tex}
     
% SERIOUS GAME REQUIREMENTS
\input{main/chapter2/section2/content.tex}    

% USE CASE DEFINITION
\input{main/chapter2/section3/content.tex}

% USE CASE REALIZATION
\input{main/chapter2/section4/content.tex}

% DATABASE DESIGN 
\input{main/chapter2/section5/content.tex}

% DATA MANIPULATION
\input{main/chapter2/section6/content.tex}

% COMMUNICATION 
\input{main/chapter2/section7/content.tex}

% TRAINING SCENARIOS
\input{main/chapter2/section8/content.tex}

% EMG GRAPHIC
\input{main/chapter2/section9/content.tex}

% STATICAL REPORTS GENERATION
\input{main/chapter2/section10/content.tex}

\subthesischapter{Conclusiones del capítulo}
Se presentó una descripción del sistema de adquisición de datos para rehabilitación, sus componentes, características distintivas y su funcionamiento. Se identificaron y definieron los requisitos del juego  serio, tanto funcionales como no funcionales, así como los actores y casos de usos del sistema que establecieron las bases fundamentales para el desarrollo de la aplicación. Se realizó el diseño de la base de datos, abarcando tanto el modelo lógico como el físico, lo que aseguró una estructura robusta y eficiente para el almacenamiento de los datos. La manipulación de los datos se abordó de manera integral, desde la conexión con la base de datos hasta la persistencia de los resultados estadísticos. Se diseñó e implementó la comunicación con el pedal motorizado y la implementación de la interfaz gráfica para la representación de los datos EMG. Se definieron los escenarios de entrenamiento para las modalidades Ligero y Clínico, asegurando una cobertura completa de las necesidad de entrenamiento del usuario. Por último en el ámbito estadístico se desarrolló una serie de gráficos para el seguimiento de los resultados en las rutinas de entrenamiento.   
    
\end{thesischapter}

% COMMUNICATION 
\begin{thesischapter}{2} {Diseño e implementación del Juego Serio}
En este capítulo se discuten los detalles de desarrollo de los aspectos citados en el capítulo anterior. Este comienza con una descripción y caracterización general del sistema, donde se  abordan cada uno de los componentes requeridos para su completo funcionamiento. Posteriormente se detalla la ingienría de software requerida en la etapa de conceptualización de la aplicación, se explican de forma detallada los aspectos teóricos y de implementación de la base de datos, el funcionamiento del protocolo de comunicación y por último los escenarios de juegos requeridos en las rutinas de entrenamiento ligero y clínico, y las estadísticas generadas por estos. Como herramienta de desarrollo se utilizó c\#.

% SYSTEM DESCRIPTION AND CHARACTERIZATION TO APPLY
\input{main/chapter2/section1/content.tex}
     
% SERIOUS GAME REQUIREMENTS
\input{main/chapter2/section2/content.tex}    

% USE CASE DEFINITION
\input{main/chapter2/section3/content.tex}

% USE CASE REALIZATION
\input{main/chapter2/section4/content.tex}

% DATABASE DESIGN 
\input{main/chapter2/section5/content.tex}

% DATA MANIPULATION
\input{main/chapter2/section6/content.tex}

% COMMUNICATION 
\input{main/chapter2/section7/content.tex}

% TRAINING SCENARIOS
\input{main/chapter2/section8/content.tex}

% EMG GRAPHIC
\input{main/chapter2/section9/content.tex}

% STATICAL REPORTS GENERATION
\input{main/chapter2/section10/content.tex}

\subthesischapter{Conclusiones del capítulo}
Se presentó una descripción del sistema de adquisición de datos para rehabilitación, sus componentes, características distintivas y su funcionamiento. Se identificaron y definieron los requisitos del juego  serio, tanto funcionales como no funcionales, así como los actores y casos de usos del sistema que establecieron las bases fundamentales para el desarrollo de la aplicación. Se realizó el diseño de la base de datos, abarcando tanto el modelo lógico como el físico, lo que aseguró una estructura robusta y eficiente para el almacenamiento de los datos. La manipulación de los datos se abordó de manera integral, desde la conexión con la base de datos hasta la persistencia de los resultados estadísticos. Se diseñó e implementó la comunicación con el pedal motorizado y la implementación de la interfaz gráfica para la representación de los datos EMG. Se definieron los escenarios de entrenamiento para las modalidades Ligero y Clínico, asegurando una cobertura completa de las necesidad de entrenamiento del usuario. Por último en el ámbito estadístico se desarrolló una serie de gráficos para el seguimiento de los resultados en las rutinas de entrenamiento.   
    
\end{thesischapter}

% TRAINING SCENARIOS
\begin{thesischapter}{2} {Diseño e implementación del Juego Serio}
En este capítulo se discuten los detalles de desarrollo de los aspectos citados en el capítulo anterior. Este comienza con una descripción y caracterización general del sistema, donde se  abordan cada uno de los componentes requeridos para su completo funcionamiento. Posteriormente se detalla la ingienría de software requerida en la etapa de conceptualización de la aplicación, se explican de forma detallada los aspectos teóricos y de implementación de la base de datos, el funcionamiento del protocolo de comunicación y por último los escenarios de juegos requeridos en las rutinas de entrenamiento ligero y clínico, y las estadísticas generadas por estos. Como herramienta de desarrollo se utilizó c\#.

% SYSTEM DESCRIPTION AND CHARACTERIZATION TO APPLY
\input{main/chapter2/section1/content.tex}
     
% SERIOUS GAME REQUIREMENTS
\input{main/chapter2/section2/content.tex}    

% USE CASE DEFINITION
\input{main/chapter2/section3/content.tex}

% USE CASE REALIZATION
\input{main/chapter2/section4/content.tex}

% DATABASE DESIGN 
\input{main/chapter2/section5/content.tex}

% DATA MANIPULATION
\input{main/chapter2/section6/content.tex}

% COMMUNICATION 
\input{main/chapter2/section7/content.tex}

% TRAINING SCENARIOS
\input{main/chapter2/section8/content.tex}

% EMG GRAPHIC
\input{main/chapter2/section9/content.tex}

% STATICAL REPORTS GENERATION
\input{main/chapter2/section10/content.tex}

\subthesischapter{Conclusiones del capítulo}
Se presentó una descripción del sistema de adquisición de datos para rehabilitación, sus componentes, características distintivas y su funcionamiento. Se identificaron y definieron los requisitos del juego  serio, tanto funcionales como no funcionales, así como los actores y casos de usos del sistema que establecieron las bases fundamentales para el desarrollo de la aplicación. Se realizó el diseño de la base de datos, abarcando tanto el modelo lógico como el físico, lo que aseguró una estructura robusta y eficiente para el almacenamiento de los datos. La manipulación de los datos se abordó de manera integral, desde la conexión con la base de datos hasta la persistencia de los resultados estadísticos. Se diseñó e implementó la comunicación con el pedal motorizado y la implementación de la interfaz gráfica para la representación de los datos EMG. Se definieron los escenarios de entrenamiento para las modalidades Ligero y Clínico, asegurando una cobertura completa de las necesidad de entrenamiento del usuario. Por último en el ámbito estadístico se desarrolló una serie de gráficos para el seguimiento de los resultados en las rutinas de entrenamiento.   
    
\end{thesischapter}

% EMG GRAPHIC
\begin{thesischapter}{2} {Diseño e implementación del Juego Serio}
En este capítulo se discuten los detalles de desarrollo de los aspectos citados en el capítulo anterior. Este comienza con una descripción y caracterización general del sistema, donde se  abordan cada uno de los componentes requeridos para su completo funcionamiento. Posteriormente se detalla la ingienría de software requerida en la etapa de conceptualización de la aplicación, se explican de forma detallada los aspectos teóricos y de implementación de la base de datos, el funcionamiento del protocolo de comunicación y por último los escenarios de juegos requeridos en las rutinas de entrenamiento ligero y clínico, y las estadísticas generadas por estos. Como herramienta de desarrollo se utilizó c\#.

% SYSTEM DESCRIPTION AND CHARACTERIZATION TO APPLY
\input{main/chapter2/section1/content.tex}
     
% SERIOUS GAME REQUIREMENTS
\input{main/chapter2/section2/content.tex}    

% USE CASE DEFINITION
\input{main/chapter2/section3/content.tex}

% USE CASE REALIZATION
\input{main/chapter2/section4/content.tex}

% DATABASE DESIGN 
\input{main/chapter2/section5/content.tex}

% DATA MANIPULATION
\input{main/chapter2/section6/content.tex}

% COMMUNICATION 
\input{main/chapter2/section7/content.tex}

% TRAINING SCENARIOS
\input{main/chapter2/section8/content.tex}

% EMG GRAPHIC
\input{main/chapter2/section9/content.tex}

% STATICAL REPORTS GENERATION
\input{main/chapter2/section10/content.tex}

\subthesischapter{Conclusiones del capítulo}
Se presentó una descripción del sistema de adquisición de datos para rehabilitación, sus componentes, características distintivas y su funcionamiento. Se identificaron y definieron los requisitos del juego  serio, tanto funcionales como no funcionales, así como los actores y casos de usos del sistema que establecieron las bases fundamentales para el desarrollo de la aplicación. Se realizó el diseño de la base de datos, abarcando tanto el modelo lógico como el físico, lo que aseguró una estructura robusta y eficiente para el almacenamiento de los datos. La manipulación de los datos se abordó de manera integral, desde la conexión con la base de datos hasta la persistencia de los resultados estadísticos. Se diseñó e implementó la comunicación con el pedal motorizado y la implementación de la interfaz gráfica para la representación de los datos EMG. Se definieron los escenarios de entrenamiento para las modalidades Ligero y Clínico, asegurando una cobertura completa de las necesidad de entrenamiento del usuario. Por último en el ámbito estadístico se desarrolló una serie de gráficos para el seguimiento de los resultados en las rutinas de entrenamiento.   
    
\end{thesischapter}

% STATICAL REPORTS GENERATION
\begin{thesischapter}{2} {Diseño e implementación del Juego Serio}
En este capítulo se discuten los detalles de desarrollo de los aspectos citados en el capítulo anterior. Este comienza con una descripción y caracterización general del sistema, donde se  abordan cada uno de los componentes requeridos para su completo funcionamiento. Posteriormente se detalla la ingienría de software requerida en la etapa de conceptualización de la aplicación, se explican de forma detallada los aspectos teóricos y de implementación de la base de datos, el funcionamiento del protocolo de comunicación y por último los escenarios de juegos requeridos en las rutinas de entrenamiento ligero y clínico, y las estadísticas generadas por estos. Como herramienta de desarrollo se utilizó c\#.

% SYSTEM DESCRIPTION AND CHARACTERIZATION TO APPLY
\input{main/chapter2/section1/content.tex}
     
% SERIOUS GAME REQUIREMENTS
\input{main/chapter2/section2/content.tex}    

% USE CASE DEFINITION
\input{main/chapter2/section3/content.tex}

% USE CASE REALIZATION
\input{main/chapter2/section4/content.tex}

% DATABASE DESIGN 
\input{main/chapter2/section5/content.tex}

% DATA MANIPULATION
\input{main/chapter2/section6/content.tex}

% COMMUNICATION 
\input{main/chapter2/section7/content.tex}

% TRAINING SCENARIOS
\input{main/chapter2/section8/content.tex}

% EMG GRAPHIC
\input{main/chapter2/section9/content.tex}

% STATICAL REPORTS GENERATION
\input{main/chapter2/section10/content.tex}

\subthesischapter{Conclusiones del capítulo}
Se presentó una descripción del sistema de adquisición de datos para rehabilitación, sus componentes, características distintivas y su funcionamiento. Se identificaron y definieron los requisitos del juego  serio, tanto funcionales como no funcionales, así como los actores y casos de usos del sistema que establecieron las bases fundamentales para el desarrollo de la aplicación. Se realizó el diseño de la base de datos, abarcando tanto el modelo lógico como el físico, lo que aseguró una estructura robusta y eficiente para el almacenamiento de los datos. La manipulación de los datos se abordó de manera integral, desde la conexión con la base de datos hasta la persistencia de los resultados estadísticos. Se diseñó e implementó la comunicación con el pedal motorizado y la implementación de la interfaz gráfica para la representación de los datos EMG. Se definieron los escenarios de entrenamiento para las modalidades Ligero y Clínico, asegurando una cobertura completa de las necesidad de entrenamiento del usuario. Por último en el ámbito estadístico se desarrolló una serie de gráficos para el seguimiento de los resultados en las rutinas de entrenamiento.   
    
\end{thesischapter}

\subthesischapter{Conclusiones del capítulo}
Se presentó una descripción del sistema de adquisición de datos para rehabilitación, sus componentes, características distintivas y su funcionamiento. Se identificaron y definieron los requisitos del juego  serio, tanto funcionales como no funcionales, así como los actores y casos de usos del sistema que establecieron las bases fundamentales para el desarrollo de la aplicación. Se realizó el diseño de la base de datos, abarcando tanto el modelo lógico como el físico, lo que aseguró una estructura robusta y eficiente para el almacenamiento de los datos. La manipulación de los datos se abordó de manera integral, desde la conexión con la base de datos hasta la persistencia de los resultados estadísticos. Se diseñó e implementó la comunicación con el pedal motorizado y la implementación de la interfaz gráfica para la representación de los datos EMG. Se definieron los escenarios de entrenamiento para las modalidades Ligero y Clínico, asegurando una cobertura completa de las necesidad de entrenamiento del usuario. Por último en el ámbito estadístico se desarrolló una serie de gráficos para el seguimiento de los resultados en las rutinas de entrenamiento.   
    
\end{thesischapter}

% COMMUNICATION 
\begin{thesischapter}{2} {Diseño e implementación del Juego Serio}
En este capítulo se discuten los detalles de desarrollo de los aspectos citados en el capítulo anterior. Este comienza con una descripción y caracterización general del sistema, donde se  abordan cada uno de los componentes requeridos para su completo funcionamiento. Posteriormente se detalla la ingienría de software requerida en la etapa de conceptualización de la aplicación, se explican de forma detallada los aspectos teóricos y de implementación de la base de datos, el funcionamiento del protocolo de comunicación y por último los escenarios de juegos requeridos en las rutinas de entrenamiento ligero y clínico, y las estadísticas generadas por estos. Como herramienta de desarrollo se utilizó c\#.

% SYSTEM DESCRIPTION AND CHARACTERIZATION TO APPLY
\begin{thesischapter}{2} {Diseño e implementación del Juego Serio}
En este capítulo se discuten los detalles de desarrollo de los aspectos citados en el capítulo anterior. Este comienza con una descripción y caracterización general del sistema, donde se  abordan cada uno de los componentes requeridos para su completo funcionamiento. Posteriormente se detalla la ingienría de software requerida en la etapa de conceptualización de la aplicación, se explican de forma detallada los aspectos teóricos y de implementación de la base de datos, el funcionamiento del protocolo de comunicación y por último los escenarios de juegos requeridos en las rutinas de entrenamiento ligero y clínico, y las estadísticas generadas por estos. Como herramienta de desarrollo se utilizó c\#.

% SYSTEM DESCRIPTION AND CHARACTERIZATION TO APPLY
\input{main/chapter2/section1/content.tex}
     
% SERIOUS GAME REQUIREMENTS
\input{main/chapter2/section2/content.tex}    

% USE CASE DEFINITION
\input{main/chapter2/section3/content.tex}

% USE CASE REALIZATION
\input{main/chapter2/section4/content.tex}

% DATABASE DESIGN 
\input{main/chapter2/section5/content.tex}

% DATA MANIPULATION
\input{main/chapter2/section6/content.tex}

% COMMUNICATION 
\input{main/chapter2/section7/content.tex}

% TRAINING SCENARIOS
\input{main/chapter2/section8/content.tex}

% EMG GRAPHIC
\input{main/chapter2/section9/content.tex}

% STATICAL REPORTS GENERATION
\input{main/chapter2/section10/content.tex}

\subthesischapter{Conclusiones del capítulo}
Se presentó una descripción del sistema de adquisición de datos para rehabilitación, sus componentes, características distintivas y su funcionamiento. Se identificaron y definieron los requisitos del juego  serio, tanto funcionales como no funcionales, así como los actores y casos de usos del sistema que establecieron las bases fundamentales para el desarrollo de la aplicación. Se realizó el diseño de la base de datos, abarcando tanto el modelo lógico como el físico, lo que aseguró una estructura robusta y eficiente para el almacenamiento de los datos. La manipulación de los datos se abordó de manera integral, desde la conexión con la base de datos hasta la persistencia de los resultados estadísticos. Se diseñó e implementó la comunicación con el pedal motorizado y la implementación de la interfaz gráfica para la representación de los datos EMG. Se definieron los escenarios de entrenamiento para las modalidades Ligero y Clínico, asegurando una cobertura completa de las necesidad de entrenamiento del usuario. Por último en el ámbito estadístico se desarrolló una serie de gráficos para el seguimiento de los resultados en las rutinas de entrenamiento.   
    
\end{thesischapter}
     
% SERIOUS GAME REQUIREMENTS
\begin{thesischapter}{2} {Diseño e implementación del Juego Serio}
En este capítulo se discuten los detalles de desarrollo de los aspectos citados en el capítulo anterior. Este comienza con una descripción y caracterización general del sistema, donde se  abordan cada uno de los componentes requeridos para su completo funcionamiento. Posteriormente se detalla la ingienría de software requerida en la etapa de conceptualización de la aplicación, se explican de forma detallada los aspectos teóricos y de implementación de la base de datos, el funcionamiento del protocolo de comunicación y por último los escenarios de juegos requeridos en las rutinas de entrenamiento ligero y clínico, y las estadísticas generadas por estos. Como herramienta de desarrollo se utilizó c\#.

% SYSTEM DESCRIPTION AND CHARACTERIZATION TO APPLY
\input{main/chapter2/section1/content.tex}
     
% SERIOUS GAME REQUIREMENTS
\input{main/chapter2/section2/content.tex}    

% USE CASE DEFINITION
\input{main/chapter2/section3/content.tex}

% USE CASE REALIZATION
\input{main/chapter2/section4/content.tex}

% DATABASE DESIGN 
\input{main/chapter2/section5/content.tex}

% DATA MANIPULATION
\input{main/chapter2/section6/content.tex}

% COMMUNICATION 
\input{main/chapter2/section7/content.tex}

% TRAINING SCENARIOS
\input{main/chapter2/section8/content.tex}

% EMG GRAPHIC
\input{main/chapter2/section9/content.tex}

% STATICAL REPORTS GENERATION
\input{main/chapter2/section10/content.tex}

\subthesischapter{Conclusiones del capítulo}
Se presentó una descripción del sistema de adquisición de datos para rehabilitación, sus componentes, características distintivas y su funcionamiento. Se identificaron y definieron los requisitos del juego  serio, tanto funcionales como no funcionales, así como los actores y casos de usos del sistema que establecieron las bases fundamentales para el desarrollo de la aplicación. Se realizó el diseño de la base de datos, abarcando tanto el modelo lógico como el físico, lo que aseguró una estructura robusta y eficiente para el almacenamiento de los datos. La manipulación de los datos se abordó de manera integral, desde la conexión con la base de datos hasta la persistencia de los resultados estadísticos. Se diseñó e implementó la comunicación con el pedal motorizado y la implementación de la interfaz gráfica para la representación de los datos EMG. Se definieron los escenarios de entrenamiento para las modalidades Ligero y Clínico, asegurando una cobertura completa de las necesidad de entrenamiento del usuario. Por último en el ámbito estadístico se desarrolló una serie de gráficos para el seguimiento de los resultados en las rutinas de entrenamiento.   
    
\end{thesischapter}    

% USE CASE DEFINITION
\begin{thesischapter}{2} {Diseño e implementación del Juego Serio}
En este capítulo se discuten los detalles de desarrollo de los aspectos citados en el capítulo anterior. Este comienza con una descripción y caracterización general del sistema, donde se  abordan cada uno de los componentes requeridos para su completo funcionamiento. Posteriormente se detalla la ingienría de software requerida en la etapa de conceptualización de la aplicación, se explican de forma detallada los aspectos teóricos y de implementación de la base de datos, el funcionamiento del protocolo de comunicación y por último los escenarios de juegos requeridos en las rutinas de entrenamiento ligero y clínico, y las estadísticas generadas por estos. Como herramienta de desarrollo se utilizó c\#.

% SYSTEM DESCRIPTION AND CHARACTERIZATION TO APPLY
\input{main/chapter2/section1/content.tex}
     
% SERIOUS GAME REQUIREMENTS
\input{main/chapter2/section2/content.tex}    

% USE CASE DEFINITION
\input{main/chapter2/section3/content.tex}

% USE CASE REALIZATION
\input{main/chapter2/section4/content.tex}

% DATABASE DESIGN 
\input{main/chapter2/section5/content.tex}

% DATA MANIPULATION
\input{main/chapter2/section6/content.tex}

% COMMUNICATION 
\input{main/chapter2/section7/content.tex}

% TRAINING SCENARIOS
\input{main/chapter2/section8/content.tex}

% EMG GRAPHIC
\input{main/chapter2/section9/content.tex}

% STATICAL REPORTS GENERATION
\input{main/chapter2/section10/content.tex}

\subthesischapter{Conclusiones del capítulo}
Se presentó una descripción del sistema de adquisición de datos para rehabilitación, sus componentes, características distintivas y su funcionamiento. Se identificaron y definieron los requisitos del juego  serio, tanto funcionales como no funcionales, así como los actores y casos de usos del sistema que establecieron las bases fundamentales para el desarrollo de la aplicación. Se realizó el diseño de la base de datos, abarcando tanto el modelo lógico como el físico, lo que aseguró una estructura robusta y eficiente para el almacenamiento de los datos. La manipulación de los datos se abordó de manera integral, desde la conexión con la base de datos hasta la persistencia de los resultados estadísticos. Se diseñó e implementó la comunicación con el pedal motorizado y la implementación de la interfaz gráfica para la representación de los datos EMG. Se definieron los escenarios de entrenamiento para las modalidades Ligero y Clínico, asegurando una cobertura completa de las necesidad de entrenamiento del usuario. Por último en el ámbito estadístico se desarrolló una serie de gráficos para el seguimiento de los resultados en las rutinas de entrenamiento.   
    
\end{thesischapter}

% USE CASE REALIZATION
\begin{thesischapter}{2} {Diseño e implementación del Juego Serio}
En este capítulo se discuten los detalles de desarrollo de los aspectos citados en el capítulo anterior. Este comienza con una descripción y caracterización general del sistema, donde se  abordan cada uno de los componentes requeridos para su completo funcionamiento. Posteriormente se detalla la ingienría de software requerida en la etapa de conceptualización de la aplicación, se explican de forma detallada los aspectos teóricos y de implementación de la base de datos, el funcionamiento del protocolo de comunicación y por último los escenarios de juegos requeridos en las rutinas de entrenamiento ligero y clínico, y las estadísticas generadas por estos. Como herramienta de desarrollo se utilizó c\#.

% SYSTEM DESCRIPTION AND CHARACTERIZATION TO APPLY
\input{main/chapter2/section1/content.tex}
     
% SERIOUS GAME REQUIREMENTS
\input{main/chapter2/section2/content.tex}    

% USE CASE DEFINITION
\input{main/chapter2/section3/content.tex}

% USE CASE REALIZATION
\input{main/chapter2/section4/content.tex}

% DATABASE DESIGN 
\input{main/chapter2/section5/content.tex}

% DATA MANIPULATION
\input{main/chapter2/section6/content.tex}

% COMMUNICATION 
\input{main/chapter2/section7/content.tex}

% TRAINING SCENARIOS
\input{main/chapter2/section8/content.tex}

% EMG GRAPHIC
\input{main/chapter2/section9/content.tex}

% STATICAL REPORTS GENERATION
\input{main/chapter2/section10/content.tex}

\subthesischapter{Conclusiones del capítulo}
Se presentó una descripción del sistema de adquisición de datos para rehabilitación, sus componentes, características distintivas y su funcionamiento. Se identificaron y definieron los requisitos del juego  serio, tanto funcionales como no funcionales, así como los actores y casos de usos del sistema que establecieron las bases fundamentales para el desarrollo de la aplicación. Se realizó el diseño de la base de datos, abarcando tanto el modelo lógico como el físico, lo que aseguró una estructura robusta y eficiente para el almacenamiento de los datos. La manipulación de los datos se abordó de manera integral, desde la conexión con la base de datos hasta la persistencia de los resultados estadísticos. Se diseñó e implementó la comunicación con el pedal motorizado y la implementación de la interfaz gráfica para la representación de los datos EMG. Se definieron los escenarios de entrenamiento para las modalidades Ligero y Clínico, asegurando una cobertura completa de las necesidad de entrenamiento del usuario. Por último en el ámbito estadístico se desarrolló una serie de gráficos para el seguimiento de los resultados en las rutinas de entrenamiento.   
    
\end{thesischapter}

% DATABASE DESIGN 
\begin{thesischapter}{2} {Diseño e implementación del Juego Serio}
En este capítulo se discuten los detalles de desarrollo de los aspectos citados en el capítulo anterior. Este comienza con una descripción y caracterización general del sistema, donde se  abordan cada uno de los componentes requeridos para su completo funcionamiento. Posteriormente se detalla la ingienría de software requerida en la etapa de conceptualización de la aplicación, se explican de forma detallada los aspectos teóricos y de implementación de la base de datos, el funcionamiento del protocolo de comunicación y por último los escenarios de juegos requeridos en las rutinas de entrenamiento ligero y clínico, y las estadísticas generadas por estos. Como herramienta de desarrollo se utilizó c\#.

% SYSTEM DESCRIPTION AND CHARACTERIZATION TO APPLY
\input{main/chapter2/section1/content.tex}
     
% SERIOUS GAME REQUIREMENTS
\input{main/chapter2/section2/content.tex}    

% USE CASE DEFINITION
\input{main/chapter2/section3/content.tex}

% USE CASE REALIZATION
\input{main/chapter2/section4/content.tex}

% DATABASE DESIGN 
\input{main/chapter2/section5/content.tex}

% DATA MANIPULATION
\input{main/chapter2/section6/content.tex}

% COMMUNICATION 
\input{main/chapter2/section7/content.tex}

% TRAINING SCENARIOS
\input{main/chapter2/section8/content.tex}

% EMG GRAPHIC
\input{main/chapter2/section9/content.tex}

% STATICAL REPORTS GENERATION
\input{main/chapter2/section10/content.tex}

\subthesischapter{Conclusiones del capítulo}
Se presentó una descripción del sistema de adquisición de datos para rehabilitación, sus componentes, características distintivas y su funcionamiento. Se identificaron y definieron los requisitos del juego  serio, tanto funcionales como no funcionales, así como los actores y casos de usos del sistema que establecieron las bases fundamentales para el desarrollo de la aplicación. Se realizó el diseño de la base de datos, abarcando tanto el modelo lógico como el físico, lo que aseguró una estructura robusta y eficiente para el almacenamiento de los datos. La manipulación de los datos se abordó de manera integral, desde la conexión con la base de datos hasta la persistencia de los resultados estadísticos. Se diseñó e implementó la comunicación con el pedal motorizado y la implementación de la interfaz gráfica para la representación de los datos EMG. Se definieron los escenarios de entrenamiento para las modalidades Ligero y Clínico, asegurando una cobertura completa de las necesidad de entrenamiento del usuario. Por último en el ámbito estadístico se desarrolló una serie de gráficos para el seguimiento de los resultados en las rutinas de entrenamiento.   
    
\end{thesischapter}

% DATA MANIPULATION
\begin{thesischapter}{2} {Diseño e implementación del Juego Serio}
En este capítulo se discuten los detalles de desarrollo de los aspectos citados en el capítulo anterior. Este comienza con una descripción y caracterización general del sistema, donde se  abordan cada uno de los componentes requeridos para su completo funcionamiento. Posteriormente se detalla la ingienría de software requerida en la etapa de conceptualización de la aplicación, se explican de forma detallada los aspectos teóricos y de implementación de la base de datos, el funcionamiento del protocolo de comunicación y por último los escenarios de juegos requeridos en las rutinas de entrenamiento ligero y clínico, y las estadísticas generadas por estos. Como herramienta de desarrollo se utilizó c\#.

% SYSTEM DESCRIPTION AND CHARACTERIZATION TO APPLY
\input{main/chapter2/section1/content.tex}
     
% SERIOUS GAME REQUIREMENTS
\input{main/chapter2/section2/content.tex}    

% USE CASE DEFINITION
\input{main/chapter2/section3/content.tex}

% USE CASE REALIZATION
\input{main/chapter2/section4/content.tex}

% DATABASE DESIGN 
\input{main/chapter2/section5/content.tex}

% DATA MANIPULATION
\input{main/chapter2/section6/content.tex}

% COMMUNICATION 
\input{main/chapter2/section7/content.tex}

% TRAINING SCENARIOS
\input{main/chapter2/section8/content.tex}

% EMG GRAPHIC
\input{main/chapter2/section9/content.tex}

% STATICAL REPORTS GENERATION
\input{main/chapter2/section10/content.tex}

\subthesischapter{Conclusiones del capítulo}
Se presentó una descripción del sistema de adquisición de datos para rehabilitación, sus componentes, características distintivas y su funcionamiento. Se identificaron y definieron los requisitos del juego  serio, tanto funcionales como no funcionales, así como los actores y casos de usos del sistema que establecieron las bases fundamentales para el desarrollo de la aplicación. Se realizó el diseño de la base de datos, abarcando tanto el modelo lógico como el físico, lo que aseguró una estructura robusta y eficiente para el almacenamiento de los datos. La manipulación de los datos se abordó de manera integral, desde la conexión con la base de datos hasta la persistencia de los resultados estadísticos. Se diseñó e implementó la comunicación con el pedal motorizado y la implementación de la interfaz gráfica para la representación de los datos EMG. Se definieron los escenarios de entrenamiento para las modalidades Ligero y Clínico, asegurando una cobertura completa de las necesidad de entrenamiento del usuario. Por último en el ámbito estadístico se desarrolló una serie de gráficos para el seguimiento de los resultados en las rutinas de entrenamiento.   
    
\end{thesischapter}

% COMMUNICATION 
\begin{thesischapter}{2} {Diseño e implementación del Juego Serio}
En este capítulo se discuten los detalles de desarrollo de los aspectos citados en el capítulo anterior. Este comienza con una descripción y caracterización general del sistema, donde se  abordan cada uno de los componentes requeridos para su completo funcionamiento. Posteriormente se detalla la ingienría de software requerida en la etapa de conceptualización de la aplicación, se explican de forma detallada los aspectos teóricos y de implementación de la base de datos, el funcionamiento del protocolo de comunicación y por último los escenarios de juegos requeridos en las rutinas de entrenamiento ligero y clínico, y las estadísticas generadas por estos. Como herramienta de desarrollo se utilizó c\#.

% SYSTEM DESCRIPTION AND CHARACTERIZATION TO APPLY
\input{main/chapter2/section1/content.tex}
     
% SERIOUS GAME REQUIREMENTS
\input{main/chapter2/section2/content.tex}    

% USE CASE DEFINITION
\input{main/chapter2/section3/content.tex}

% USE CASE REALIZATION
\input{main/chapter2/section4/content.tex}

% DATABASE DESIGN 
\input{main/chapter2/section5/content.tex}

% DATA MANIPULATION
\input{main/chapter2/section6/content.tex}

% COMMUNICATION 
\input{main/chapter2/section7/content.tex}

% TRAINING SCENARIOS
\input{main/chapter2/section8/content.tex}

% EMG GRAPHIC
\input{main/chapter2/section9/content.tex}

% STATICAL REPORTS GENERATION
\input{main/chapter2/section10/content.tex}

\subthesischapter{Conclusiones del capítulo}
Se presentó una descripción del sistema de adquisición de datos para rehabilitación, sus componentes, características distintivas y su funcionamiento. Se identificaron y definieron los requisitos del juego  serio, tanto funcionales como no funcionales, así como los actores y casos de usos del sistema que establecieron las bases fundamentales para el desarrollo de la aplicación. Se realizó el diseño de la base de datos, abarcando tanto el modelo lógico como el físico, lo que aseguró una estructura robusta y eficiente para el almacenamiento de los datos. La manipulación de los datos se abordó de manera integral, desde la conexión con la base de datos hasta la persistencia de los resultados estadísticos. Se diseñó e implementó la comunicación con el pedal motorizado y la implementación de la interfaz gráfica para la representación de los datos EMG. Se definieron los escenarios de entrenamiento para las modalidades Ligero y Clínico, asegurando una cobertura completa de las necesidad de entrenamiento del usuario. Por último en el ámbito estadístico se desarrolló una serie de gráficos para el seguimiento de los resultados en las rutinas de entrenamiento.   
    
\end{thesischapter}

% TRAINING SCENARIOS
\begin{thesischapter}{2} {Diseño e implementación del Juego Serio}
En este capítulo se discuten los detalles de desarrollo de los aspectos citados en el capítulo anterior. Este comienza con una descripción y caracterización general del sistema, donde se  abordan cada uno de los componentes requeridos para su completo funcionamiento. Posteriormente se detalla la ingienría de software requerida en la etapa de conceptualización de la aplicación, se explican de forma detallada los aspectos teóricos y de implementación de la base de datos, el funcionamiento del protocolo de comunicación y por último los escenarios de juegos requeridos en las rutinas de entrenamiento ligero y clínico, y las estadísticas generadas por estos. Como herramienta de desarrollo se utilizó c\#.

% SYSTEM DESCRIPTION AND CHARACTERIZATION TO APPLY
\input{main/chapter2/section1/content.tex}
     
% SERIOUS GAME REQUIREMENTS
\input{main/chapter2/section2/content.tex}    

% USE CASE DEFINITION
\input{main/chapter2/section3/content.tex}

% USE CASE REALIZATION
\input{main/chapter2/section4/content.tex}

% DATABASE DESIGN 
\input{main/chapter2/section5/content.tex}

% DATA MANIPULATION
\input{main/chapter2/section6/content.tex}

% COMMUNICATION 
\input{main/chapter2/section7/content.tex}

% TRAINING SCENARIOS
\input{main/chapter2/section8/content.tex}

% EMG GRAPHIC
\input{main/chapter2/section9/content.tex}

% STATICAL REPORTS GENERATION
\input{main/chapter2/section10/content.tex}

\subthesischapter{Conclusiones del capítulo}
Se presentó una descripción del sistema de adquisición de datos para rehabilitación, sus componentes, características distintivas y su funcionamiento. Se identificaron y definieron los requisitos del juego  serio, tanto funcionales como no funcionales, así como los actores y casos de usos del sistema que establecieron las bases fundamentales para el desarrollo de la aplicación. Se realizó el diseño de la base de datos, abarcando tanto el modelo lógico como el físico, lo que aseguró una estructura robusta y eficiente para el almacenamiento de los datos. La manipulación de los datos se abordó de manera integral, desde la conexión con la base de datos hasta la persistencia de los resultados estadísticos. Se diseñó e implementó la comunicación con el pedal motorizado y la implementación de la interfaz gráfica para la representación de los datos EMG. Se definieron los escenarios de entrenamiento para las modalidades Ligero y Clínico, asegurando una cobertura completa de las necesidad de entrenamiento del usuario. Por último en el ámbito estadístico se desarrolló una serie de gráficos para el seguimiento de los resultados en las rutinas de entrenamiento.   
    
\end{thesischapter}

% EMG GRAPHIC
\begin{thesischapter}{2} {Diseño e implementación del Juego Serio}
En este capítulo se discuten los detalles de desarrollo de los aspectos citados en el capítulo anterior. Este comienza con una descripción y caracterización general del sistema, donde se  abordan cada uno de los componentes requeridos para su completo funcionamiento. Posteriormente se detalla la ingienría de software requerida en la etapa de conceptualización de la aplicación, se explican de forma detallada los aspectos teóricos y de implementación de la base de datos, el funcionamiento del protocolo de comunicación y por último los escenarios de juegos requeridos en las rutinas de entrenamiento ligero y clínico, y las estadísticas generadas por estos. Como herramienta de desarrollo se utilizó c\#.

% SYSTEM DESCRIPTION AND CHARACTERIZATION TO APPLY
\input{main/chapter2/section1/content.tex}
     
% SERIOUS GAME REQUIREMENTS
\input{main/chapter2/section2/content.tex}    

% USE CASE DEFINITION
\input{main/chapter2/section3/content.tex}

% USE CASE REALIZATION
\input{main/chapter2/section4/content.tex}

% DATABASE DESIGN 
\input{main/chapter2/section5/content.tex}

% DATA MANIPULATION
\input{main/chapter2/section6/content.tex}

% COMMUNICATION 
\input{main/chapter2/section7/content.tex}

% TRAINING SCENARIOS
\input{main/chapter2/section8/content.tex}

% EMG GRAPHIC
\input{main/chapter2/section9/content.tex}

% STATICAL REPORTS GENERATION
\input{main/chapter2/section10/content.tex}

\subthesischapter{Conclusiones del capítulo}
Se presentó una descripción del sistema de adquisición de datos para rehabilitación, sus componentes, características distintivas y su funcionamiento. Se identificaron y definieron los requisitos del juego  serio, tanto funcionales como no funcionales, así como los actores y casos de usos del sistema que establecieron las bases fundamentales para el desarrollo de la aplicación. Se realizó el diseño de la base de datos, abarcando tanto el modelo lógico como el físico, lo que aseguró una estructura robusta y eficiente para el almacenamiento de los datos. La manipulación de los datos se abordó de manera integral, desde la conexión con la base de datos hasta la persistencia de los resultados estadísticos. Se diseñó e implementó la comunicación con el pedal motorizado y la implementación de la interfaz gráfica para la representación de los datos EMG. Se definieron los escenarios de entrenamiento para las modalidades Ligero y Clínico, asegurando una cobertura completa de las necesidad de entrenamiento del usuario. Por último en el ámbito estadístico se desarrolló una serie de gráficos para el seguimiento de los resultados en las rutinas de entrenamiento.   
    
\end{thesischapter}

% STATICAL REPORTS GENERATION
\begin{thesischapter}{2} {Diseño e implementación del Juego Serio}
En este capítulo se discuten los detalles de desarrollo de los aspectos citados en el capítulo anterior. Este comienza con una descripción y caracterización general del sistema, donde se  abordan cada uno de los componentes requeridos para su completo funcionamiento. Posteriormente se detalla la ingienría de software requerida en la etapa de conceptualización de la aplicación, se explican de forma detallada los aspectos teóricos y de implementación de la base de datos, el funcionamiento del protocolo de comunicación y por último los escenarios de juegos requeridos en las rutinas de entrenamiento ligero y clínico, y las estadísticas generadas por estos. Como herramienta de desarrollo se utilizó c\#.

% SYSTEM DESCRIPTION AND CHARACTERIZATION TO APPLY
\input{main/chapter2/section1/content.tex}
     
% SERIOUS GAME REQUIREMENTS
\input{main/chapter2/section2/content.tex}    

% USE CASE DEFINITION
\input{main/chapter2/section3/content.tex}

% USE CASE REALIZATION
\input{main/chapter2/section4/content.tex}

% DATABASE DESIGN 
\input{main/chapter2/section5/content.tex}

% DATA MANIPULATION
\input{main/chapter2/section6/content.tex}

% COMMUNICATION 
\input{main/chapter2/section7/content.tex}

% TRAINING SCENARIOS
\input{main/chapter2/section8/content.tex}

% EMG GRAPHIC
\input{main/chapter2/section9/content.tex}

% STATICAL REPORTS GENERATION
\input{main/chapter2/section10/content.tex}

\subthesischapter{Conclusiones del capítulo}
Se presentó una descripción del sistema de adquisición de datos para rehabilitación, sus componentes, características distintivas y su funcionamiento. Se identificaron y definieron los requisitos del juego  serio, tanto funcionales como no funcionales, así como los actores y casos de usos del sistema que establecieron las bases fundamentales para el desarrollo de la aplicación. Se realizó el diseño de la base de datos, abarcando tanto el modelo lógico como el físico, lo que aseguró una estructura robusta y eficiente para el almacenamiento de los datos. La manipulación de los datos se abordó de manera integral, desde la conexión con la base de datos hasta la persistencia de los resultados estadísticos. Se diseñó e implementó la comunicación con el pedal motorizado y la implementación de la interfaz gráfica para la representación de los datos EMG. Se definieron los escenarios de entrenamiento para las modalidades Ligero y Clínico, asegurando una cobertura completa de las necesidad de entrenamiento del usuario. Por último en el ámbito estadístico se desarrolló una serie de gráficos para el seguimiento de los resultados en las rutinas de entrenamiento.   
    
\end{thesischapter}

\subthesischapter{Conclusiones del capítulo}
Se presentó una descripción del sistema de adquisición de datos para rehabilitación, sus componentes, características distintivas y su funcionamiento. Se identificaron y definieron los requisitos del juego  serio, tanto funcionales como no funcionales, así como los actores y casos de usos del sistema que establecieron las bases fundamentales para el desarrollo de la aplicación. Se realizó el diseño de la base de datos, abarcando tanto el modelo lógico como el físico, lo que aseguró una estructura robusta y eficiente para el almacenamiento de los datos. La manipulación de los datos se abordó de manera integral, desde la conexión con la base de datos hasta la persistencia de los resultados estadísticos. Se diseñó e implementó la comunicación con el pedal motorizado y la implementación de la interfaz gráfica para la representación de los datos EMG. Se definieron los escenarios de entrenamiento para las modalidades Ligero y Clínico, asegurando una cobertura completa de las necesidad de entrenamiento del usuario. Por último en el ámbito estadístico se desarrolló una serie de gráficos para el seguimiento de los resultados en las rutinas de entrenamiento.   
    
\end{thesischapter}

% TRAINING SCENARIOS
\begin{thesischapter}{2} {Diseño e implementación del Juego Serio}
En este capítulo se discuten los detalles de desarrollo de los aspectos citados en el capítulo anterior. Este comienza con una descripción y caracterización general del sistema, donde se  abordan cada uno de los componentes requeridos para su completo funcionamiento. Posteriormente se detalla la ingienría de software requerida en la etapa de conceptualización de la aplicación, se explican de forma detallada los aspectos teóricos y de implementación de la base de datos, el funcionamiento del protocolo de comunicación y por último los escenarios de juegos requeridos en las rutinas de entrenamiento ligero y clínico, y las estadísticas generadas por estos. Como herramienta de desarrollo se utilizó c\#.

% SYSTEM DESCRIPTION AND CHARACTERIZATION TO APPLY
\begin{thesischapter}{2} {Diseño e implementación del Juego Serio}
En este capítulo se discuten los detalles de desarrollo de los aspectos citados en el capítulo anterior. Este comienza con una descripción y caracterización general del sistema, donde se  abordan cada uno de los componentes requeridos para su completo funcionamiento. Posteriormente se detalla la ingienría de software requerida en la etapa de conceptualización de la aplicación, se explican de forma detallada los aspectos teóricos y de implementación de la base de datos, el funcionamiento del protocolo de comunicación y por último los escenarios de juegos requeridos en las rutinas de entrenamiento ligero y clínico, y las estadísticas generadas por estos. Como herramienta de desarrollo se utilizó c\#.

% SYSTEM DESCRIPTION AND CHARACTERIZATION TO APPLY
\input{main/chapter2/section1/content.tex}
     
% SERIOUS GAME REQUIREMENTS
\input{main/chapter2/section2/content.tex}    

% USE CASE DEFINITION
\input{main/chapter2/section3/content.tex}

% USE CASE REALIZATION
\input{main/chapter2/section4/content.tex}

% DATABASE DESIGN 
\input{main/chapter2/section5/content.tex}

% DATA MANIPULATION
\input{main/chapter2/section6/content.tex}

% COMMUNICATION 
\input{main/chapter2/section7/content.tex}

% TRAINING SCENARIOS
\input{main/chapter2/section8/content.tex}

% EMG GRAPHIC
\input{main/chapter2/section9/content.tex}

% STATICAL REPORTS GENERATION
\input{main/chapter2/section10/content.tex}

\subthesischapter{Conclusiones del capítulo}
Se presentó una descripción del sistema de adquisición de datos para rehabilitación, sus componentes, características distintivas y su funcionamiento. Se identificaron y definieron los requisitos del juego  serio, tanto funcionales como no funcionales, así como los actores y casos de usos del sistema que establecieron las bases fundamentales para el desarrollo de la aplicación. Se realizó el diseño de la base de datos, abarcando tanto el modelo lógico como el físico, lo que aseguró una estructura robusta y eficiente para el almacenamiento de los datos. La manipulación de los datos se abordó de manera integral, desde la conexión con la base de datos hasta la persistencia de los resultados estadísticos. Se diseñó e implementó la comunicación con el pedal motorizado y la implementación de la interfaz gráfica para la representación de los datos EMG. Se definieron los escenarios de entrenamiento para las modalidades Ligero y Clínico, asegurando una cobertura completa de las necesidad de entrenamiento del usuario. Por último en el ámbito estadístico se desarrolló una serie de gráficos para el seguimiento de los resultados en las rutinas de entrenamiento.   
    
\end{thesischapter}
     
% SERIOUS GAME REQUIREMENTS
\begin{thesischapter}{2} {Diseño e implementación del Juego Serio}
En este capítulo se discuten los detalles de desarrollo de los aspectos citados en el capítulo anterior. Este comienza con una descripción y caracterización general del sistema, donde se  abordan cada uno de los componentes requeridos para su completo funcionamiento. Posteriormente se detalla la ingienría de software requerida en la etapa de conceptualización de la aplicación, se explican de forma detallada los aspectos teóricos y de implementación de la base de datos, el funcionamiento del protocolo de comunicación y por último los escenarios de juegos requeridos en las rutinas de entrenamiento ligero y clínico, y las estadísticas generadas por estos. Como herramienta de desarrollo se utilizó c\#.

% SYSTEM DESCRIPTION AND CHARACTERIZATION TO APPLY
\input{main/chapter2/section1/content.tex}
     
% SERIOUS GAME REQUIREMENTS
\input{main/chapter2/section2/content.tex}    

% USE CASE DEFINITION
\input{main/chapter2/section3/content.tex}

% USE CASE REALIZATION
\input{main/chapter2/section4/content.tex}

% DATABASE DESIGN 
\input{main/chapter2/section5/content.tex}

% DATA MANIPULATION
\input{main/chapter2/section6/content.tex}

% COMMUNICATION 
\input{main/chapter2/section7/content.tex}

% TRAINING SCENARIOS
\input{main/chapter2/section8/content.tex}

% EMG GRAPHIC
\input{main/chapter2/section9/content.tex}

% STATICAL REPORTS GENERATION
\input{main/chapter2/section10/content.tex}

\subthesischapter{Conclusiones del capítulo}
Se presentó una descripción del sistema de adquisición de datos para rehabilitación, sus componentes, características distintivas y su funcionamiento. Se identificaron y definieron los requisitos del juego  serio, tanto funcionales como no funcionales, así como los actores y casos de usos del sistema que establecieron las bases fundamentales para el desarrollo de la aplicación. Se realizó el diseño de la base de datos, abarcando tanto el modelo lógico como el físico, lo que aseguró una estructura robusta y eficiente para el almacenamiento de los datos. La manipulación de los datos se abordó de manera integral, desde la conexión con la base de datos hasta la persistencia de los resultados estadísticos. Se diseñó e implementó la comunicación con el pedal motorizado y la implementación de la interfaz gráfica para la representación de los datos EMG. Se definieron los escenarios de entrenamiento para las modalidades Ligero y Clínico, asegurando una cobertura completa de las necesidad de entrenamiento del usuario. Por último en el ámbito estadístico se desarrolló una serie de gráficos para el seguimiento de los resultados en las rutinas de entrenamiento.   
    
\end{thesischapter}    

% USE CASE DEFINITION
\begin{thesischapter}{2} {Diseño e implementación del Juego Serio}
En este capítulo se discuten los detalles de desarrollo de los aspectos citados en el capítulo anterior. Este comienza con una descripción y caracterización general del sistema, donde se  abordan cada uno de los componentes requeridos para su completo funcionamiento. Posteriormente se detalla la ingienría de software requerida en la etapa de conceptualización de la aplicación, se explican de forma detallada los aspectos teóricos y de implementación de la base de datos, el funcionamiento del protocolo de comunicación y por último los escenarios de juegos requeridos en las rutinas de entrenamiento ligero y clínico, y las estadísticas generadas por estos. Como herramienta de desarrollo se utilizó c\#.

% SYSTEM DESCRIPTION AND CHARACTERIZATION TO APPLY
\input{main/chapter2/section1/content.tex}
     
% SERIOUS GAME REQUIREMENTS
\input{main/chapter2/section2/content.tex}    

% USE CASE DEFINITION
\input{main/chapter2/section3/content.tex}

% USE CASE REALIZATION
\input{main/chapter2/section4/content.tex}

% DATABASE DESIGN 
\input{main/chapter2/section5/content.tex}

% DATA MANIPULATION
\input{main/chapter2/section6/content.tex}

% COMMUNICATION 
\input{main/chapter2/section7/content.tex}

% TRAINING SCENARIOS
\input{main/chapter2/section8/content.tex}

% EMG GRAPHIC
\input{main/chapter2/section9/content.tex}

% STATICAL REPORTS GENERATION
\input{main/chapter2/section10/content.tex}

\subthesischapter{Conclusiones del capítulo}
Se presentó una descripción del sistema de adquisición de datos para rehabilitación, sus componentes, características distintivas y su funcionamiento. Se identificaron y definieron los requisitos del juego  serio, tanto funcionales como no funcionales, así como los actores y casos de usos del sistema que establecieron las bases fundamentales para el desarrollo de la aplicación. Se realizó el diseño de la base de datos, abarcando tanto el modelo lógico como el físico, lo que aseguró una estructura robusta y eficiente para el almacenamiento de los datos. La manipulación de los datos se abordó de manera integral, desde la conexión con la base de datos hasta la persistencia de los resultados estadísticos. Se diseñó e implementó la comunicación con el pedal motorizado y la implementación de la interfaz gráfica para la representación de los datos EMG. Se definieron los escenarios de entrenamiento para las modalidades Ligero y Clínico, asegurando una cobertura completa de las necesidad de entrenamiento del usuario. Por último en el ámbito estadístico se desarrolló una serie de gráficos para el seguimiento de los resultados en las rutinas de entrenamiento.   
    
\end{thesischapter}

% USE CASE REALIZATION
\begin{thesischapter}{2} {Diseño e implementación del Juego Serio}
En este capítulo se discuten los detalles de desarrollo de los aspectos citados en el capítulo anterior. Este comienza con una descripción y caracterización general del sistema, donde se  abordan cada uno de los componentes requeridos para su completo funcionamiento. Posteriormente se detalla la ingienría de software requerida en la etapa de conceptualización de la aplicación, se explican de forma detallada los aspectos teóricos y de implementación de la base de datos, el funcionamiento del protocolo de comunicación y por último los escenarios de juegos requeridos en las rutinas de entrenamiento ligero y clínico, y las estadísticas generadas por estos. Como herramienta de desarrollo se utilizó c\#.

% SYSTEM DESCRIPTION AND CHARACTERIZATION TO APPLY
\input{main/chapter2/section1/content.tex}
     
% SERIOUS GAME REQUIREMENTS
\input{main/chapter2/section2/content.tex}    

% USE CASE DEFINITION
\input{main/chapter2/section3/content.tex}

% USE CASE REALIZATION
\input{main/chapter2/section4/content.tex}

% DATABASE DESIGN 
\input{main/chapter2/section5/content.tex}

% DATA MANIPULATION
\input{main/chapter2/section6/content.tex}

% COMMUNICATION 
\input{main/chapter2/section7/content.tex}

% TRAINING SCENARIOS
\input{main/chapter2/section8/content.tex}

% EMG GRAPHIC
\input{main/chapter2/section9/content.tex}

% STATICAL REPORTS GENERATION
\input{main/chapter2/section10/content.tex}

\subthesischapter{Conclusiones del capítulo}
Se presentó una descripción del sistema de adquisición de datos para rehabilitación, sus componentes, características distintivas y su funcionamiento. Se identificaron y definieron los requisitos del juego  serio, tanto funcionales como no funcionales, así como los actores y casos de usos del sistema que establecieron las bases fundamentales para el desarrollo de la aplicación. Se realizó el diseño de la base de datos, abarcando tanto el modelo lógico como el físico, lo que aseguró una estructura robusta y eficiente para el almacenamiento de los datos. La manipulación de los datos se abordó de manera integral, desde la conexión con la base de datos hasta la persistencia de los resultados estadísticos. Se diseñó e implementó la comunicación con el pedal motorizado y la implementación de la interfaz gráfica para la representación de los datos EMG. Se definieron los escenarios de entrenamiento para las modalidades Ligero y Clínico, asegurando una cobertura completa de las necesidad de entrenamiento del usuario. Por último en el ámbito estadístico se desarrolló una serie de gráficos para el seguimiento de los resultados en las rutinas de entrenamiento.   
    
\end{thesischapter}

% DATABASE DESIGN 
\begin{thesischapter}{2} {Diseño e implementación del Juego Serio}
En este capítulo se discuten los detalles de desarrollo de los aspectos citados en el capítulo anterior. Este comienza con una descripción y caracterización general del sistema, donde se  abordan cada uno de los componentes requeridos para su completo funcionamiento. Posteriormente se detalla la ingienría de software requerida en la etapa de conceptualización de la aplicación, se explican de forma detallada los aspectos teóricos y de implementación de la base de datos, el funcionamiento del protocolo de comunicación y por último los escenarios de juegos requeridos en las rutinas de entrenamiento ligero y clínico, y las estadísticas generadas por estos. Como herramienta de desarrollo se utilizó c\#.

% SYSTEM DESCRIPTION AND CHARACTERIZATION TO APPLY
\input{main/chapter2/section1/content.tex}
     
% SERIOUS GAME REQUIREMENTS
\input{main/chapter2/section2/content.tex}    

% USE CASE DEFINITION
\input{main/chapter2/section3/content.tex}

% USE CASE REALIZATION
\input{main/chapter2/section4/content.tex}

% DATABASE DESIGN 
\input{main/chapter2/section5/content.tex}

% DATA MANIPULATION
\input{main/chapter2/section6/content.tex}

% COMMUNICATION 
\input{main/chapter2/section7/content.tex}

% TRAINING SCENARIOS
\input{main/chapter2/section8/content.tex}

% EMG GRAPHIC
\input{main/chapter2/section9/content.tex}

% STATICAL REPORTS GENERATION
\input{main/chapter2/section10/content.tex}

\subthesischapter{Conclusiones del capítulo}
Se presentó una descripción del sistema de adquisición de datos para rehabilitación, sus componentes, características distintivas y su funcionamiento. Se identificaron y definieron los requisitos del juego  serio, tanto funcionales como no funcionales, así como los actores y casos de usos del sistema que establecieron las bases fundamentales para el desarrollo de la aplicación. Se realizó el diseño de la base de datos, abarcando tanto el modelo lógico como el físico, lo que aseguró una estructura robusta y eficiente para el almacenamiento de los datos. La manipulación de los datos se abordó de manera integral, desde la conexión con la base de datos hasta la persistencia de los resultados estadísticos. Se diseñó e implementó la comunicación con el pedal motorizado y la implementación de la interfaz gráfica para la representación de los datos EMG. Se definieron los escenarios de entrenamiento para las modalidades Ligero y Clínico, asegurando una cobertura completa de las necesidad de entrenamiento del usuario. Por último en el ámbito estadístico se desarrolló una serie de gráficos para el seguimiento de los resultados en las rutinas de entrenamiento.   
    
\end{thesischapter}

% DATA MANIPULATION
\begin{thesischapter}{2} {Diseño e implementación del Juego Serio}
En este capítulo se discuten los detalles de desarrollo de los aspectos citados en el capítulo anterior. Este comienza con una descripción y caracterización general del sistema, donde se  abordan cada uno de los componentes requeridos para su completo funcionamiento. Posteriormente se detalla la ingienría de software requerida en la etapa de conceptualización de la aplicación, se explican de forma detallada los aspectos teóricos y de implementación de la base de datos, el funcionamiento del protocolo de comunicación y por último los escenarios de juegos requeridos en las rutinas de entrenamiento ligero y clínico, y las estadísticas generadas por estos. Como herramienta de desarrollo se utilizó c\#.

% SYSTEM DESCRIPTION AND CHARACTERIZATION TO APPLY
\input{main/chapter2/section1/content.tex}
     
% SERIOUS GAME REQUIREMENTS
\input{main/chapter2/section2/content.tex}    

% USE CASE DEFINITION
\input{main/chapter2/section3/content.tex}

% USE CASE REALIZATION
\input{main/chapter2/section4/content.tex}

% DATABASE DESIGN 
\input{main/chapter2/section5/content.tex}

% DATA MANIPULATION
\input{main/chapter2/section6/content.tex}

% COMMUNICATION 
\input{main/chapter2/section7/content.tex}

% TRAINING SCENARIOS
\input{main/chapter2/section8/content.tex}

% EMG GRAPHIC
\input{main/chapter2/section9/content.tex}

% STATICAL REPORTS GENERATION
\input{main/chapter2/section10/content.tex}

\subthesischapter{Conclusiones del capítulo}
Se presentó una descripción del sistema de adquisición de datos para rehabilitación, sus componentes, características distintivas y su funcionamiento. Se identificaron y definieron los requisitos del juego  serio, tanto funcionales como no funcionales, así como los actores y casos de usos del sistema que establecieron las bases fundamentales para el desarrollo de la aplicación. Se realizó el diseño de la base de datos, abarcando tanto el modelo lógico como el físico, lo que aseguró una estructura robusta y eficiente para el almacenamiento de los datos. La manipulación de los datos se abordó de manera integral, desde la conexión con la base de datos hasta la persistencia de los resultados estadísticos. Se diseñó e implementó la comunicación con el pedal motorizado y la implementación de la interfaz gráfica para la representación de los datos EMG. Se definieron los escenarios de entrenamiento para las modalidades Ligero y Clínico, asegurando una cobertura completa de las necesidad de entrenamiento del usuario. Por último en el ámbito estadístico se desarrolló una serie de gráficos para el seguimiento de los resultados en las rutinas de entrenamiento.   
    
\end{thesischapter}

% COMMUNICATION 
\begin{thesischapter}{2} {Diseño e implementación del Juego Serio}
En este capítulo se discuten los detalles de desarrollo de los aspectos citados en el capítulo anterior. Este comienza con una descripción y caracterización general del sistema, donde se  abordan cada uno de los componentes requeridos para su completo funcionamiento. Posteriormente se detalla la ingienría de software requerida en la etapa de conceptualización de la aplicación, se explican de forma detallada los aspectos teóricos y de implementación de la base de datos, el funcionamiento del protocolo de comunicación y por último los escenarios de juegos requeridos en las rutinas de entrenamiento ligero y clínico, y las estadísticas generadas por estos. Como herramienta de desarrollo se utilizó c\#.

% SYSTEM DESCRIPTION AND CHARACTERIZATION TO APPLY
\input{main/chapter2/section1/content.tex}
     
% SERIOUS GAME REQUIREMENTS
\input{main/chapter2/section2/content.tex}    

% USE CASE DEFINITION
\input{main/chapter2/section3/content.tex}

% USE CASE REALIZATION
\input{main/chapter2/section4/content.tex}

% DATABASE DESIGN 
\input{main/chapter2/section5/content.tex}

% DATA MANIPULATION
\input{main/chapter2/section6/content.tex}

% COMMUNICATION 
\input{main/chapter2/section7/content.tex}

% TRAINING SCENARIOS
\input{main/chapter2/section8/content.tex}

% EMG GRAPHIC
\input{main/chapter2/section9/content.tex}

% STATICAL REPORTS GENERATION
\input{main/chapter2/section10/content.tex}

\subthesischapter{Conclusiones del capítulo}
Se presentó una descripción del sistema de adquisición de datos para rehabilitación, sus componentes, características distintivas y su funcionamiento. Se identificaron y definieron los requisitos del juego  serio, tanto funcionales como no funcionales, así como los actores y casos de usos del sistema que establecieron las bases fundamentales para el desarrollo de la aplicación. Se realizó el diseño de la base de datos, abarcando tanto el modelo lógico como el físico, lo que aseguró una estructura robusta y eficiente para el almacenamiento de los datos. La manipulación de los datos se abordó de manera integral, desde la conexión con la base de datos hasta la persistencia de los resultados estadísticos. Se diseñó e implementó la comunicación con el pedal motorizado y la implementación de la interfaz gráfica para la representación de los datos EMG. Se definieron los escenarios de entrenamiento para las modalidades Ligero y Clínico, asegurando una cobertura completa de las necesidad de entrenamiento del usuario. Por último en el ámbito estadístico se desarrolló una serie de gráficos para el seguimiento de los resultados en las rutinas de entrenamiento.   
    
\end{thesischapter}

% TRAINING SCENARIOS
\begin{thesischapter}{2} {Diseño e implementación del Juego Serio}
En este capítulo se discuten los detalles de desarrollo de los aspectos citados en el capítulo anterior. Este comienza con una descripción y caracterización general del sistema, donde se  abordan cada uno de los componentes requeridos para su completo funcionamiento. Posteriormente se detalla la ingienría de software requerida en la etapa de conceptualización de la aplicación, se explican de forma detallada los aspectos teóricos y de implementación de la base de datos, el funcionamiento del protocolo de comunicación y por último los escenarios de juegos requeridos en las rutinas de entrenamiento ligero y clínico, y las estadísticas generadas por estos. Como herramienta de desarrollo se utilizó c\#.

% SYSTEM DESCRIPTION AND CHARACTERIZATION TO APPLY
\input{main/chapter2/section1/content.tex}
     
% SERIOUS GAME REQUIREMENTS
\input{main/chapter2/section2/content.tex}    

% USE CASE DEFINITION
\input{main/chapter2/section3/content.tex}

% USE CASE REALIZATION
\input{main/chapter2/section4/content.tex}

% DATABASE DESIGN 
\input{main/chapter2/section5/content.tex}

% DATA MANIPULATION
\input{main/chapter2/section6/content.tex}

% COMMUNICATION 
\input{main/chapter2/section7/content.tex}

% TRAINING SCENARIOS
\input{main/chapter2/section8/content.tex}

% EMG GRAPHIC
\input{main/chapter2/section9/content.tex}

% STATICAL REPORTS GENERATION
\input{main/chapter2/section10/content.tex}

\subthesischapter{Conclusiones del capítulo}
Se presentó una descripción del sistema de adquisición de datos para rehabilitación, sus componentes, características distintivas y su funcionamiento. Se identificaron y definieron los requisitos del juego  serio, tanto funcionales como no funcionales, así como los actores y casos de usos del sistema que establecieron las bases fundamentales para el desarrollo de la aplicación. Se realizó el diseño de la base de datos, abarcando tanto el modelo lógico como el físico, lo que aseguró una estructura robusta y eficiente para el almacenamiento de los datos. La manipulación de los datos se abordó de manera integral, desde la conexión con la base de datos hasta la persistencia de los resultados estadísticos. Se diseñó e implementó la comunicación con el pedal motorizado y la implementación de la interfaz gráfica para la representación de los datos EMG. Se definieron los escenarios de entrenamiento para las modalidades Ligero y Clínico, asegurando una cobertura completa de las necesidad de entrenamiento del usuario. Por último en el ámbito estadístico se desarrolló una serie de gráficos para el seguimiento de los resultados en las rutinas de entrenamiento.   
    
\end{thesischapter}

% EMG GRAPHIC
\begin{thesischapter}{2} {Diseño e implementación del Juego Serio}
En este capítulo se discuten los detalles de desarrollo de los aspectos citados en el capítulo anterior. Este comienza con una descripción y caracterización general del sistema, donde se  abordan cada uno de los componentes requeridos para su completo funcionamiento. Posteriormente se detalla la ingienría de software requerida en la etapa de conceptualización de la aplicación, se explican de forma detallada los aspectos teóricos y de implementación de la base de datos, el funcionamiento del protocolo de comunicación y por último los escenarios de juegos requeridos en las rutinas de entrenamiento ligero y clínico, y las estadísticas generadas por estos. Como herramienta de desarrollo se utilizó c\#.

% SYSTEM DESCRIPTION AND CHARACTERIZATION TO APPLY
\input{main/chapter2/section1/content.tex}
     
% SERIOUS GAME REQUIREMENTS
\input{main/chapter2/section2/content.tex}    

% USE CASE DEFINITION
\input{main/chapter2/section3/content.tex}

% USE CASE REALIZATION
\input{main/chapter2/section4/content.tex}

% DATABASE DESIGN 
\input{main/chapter2/section5/content.tex}

% DATA MANIPULATION
\input{main/chapter2/section6/content.tex}

% COMMUNICATION 
\input{main/chapter2/section7/content.tex}

% TRAINING SCENARIOS
\input{main/chapter2/section8/content.tex}

% EMG GRAPHIC
\input{main/chapter2/section9/content.tex}

% STATICAL REPORTS GENERATION
\input{main/chapter2/section10/content.tex}

\subthesischapter{Conclusiones del capítulo}
Se presentó una descripción del sistema de adquisición de datos para rehabilitación, sus componentes, características distintivas y su funcionamiento. Se identificaron y definieron los requisitos del juego  serio, tanto funcionales como no funcionales, así como los actores y casos de usos del sistema que establecieron las bases fundamentales para el desarrollo de la aplicación. Se realizó el diseño de la base de datos, abarcando tanto el modelo lógico como el físico, lo que aseguró una estructura robusta y eficiente para el almacenamiento de los datos. La manipulación de los datos se abordó de manera integral, desde la conexión con la base de datos hasta la persistencia de los resultados estadísticos. Se diseñó e implementó la comunicación con el pedal motorizado y la implementación de la interfaz gráfica para la representación de los datos EMG. Se definieron los escenarios de entrenamiento para las modalidades Ligero y Clínico, asegurando una cobertura completa de las necesidad de entrenamiento del usuario. Por último en el ámbito estadístico se desarrolló una serie de gráficos para el seguimiento de los resultados en las rutinas de entrenamiento.   
    
\end{thesischapter}

% STATICAL REPORTS GENERATION
\begin{thesischapter}{2} {Diseño e implementación del Juego Serio}
En este capítulo se discuten los detalles de desarrollo de los aspectos citados en el capítulo anterior. Este comienza con una descripción y caracterización general del sistema, donde se  abordan cada uno de los componentes requeridos para su completo funcionamiento. Posteriormente se detalla la ingienría de software requerida en la etapa de conceptualización de la aplicación, se explican de forma detallada los aspectos teóricos y de implementación de la base de datos, el funcionamiento del protocolo de comunicación y por último los escenarios de juegos requeridos en las rutinas de entrenamiento ligero y clínico, y las estadísticas generadas por estos. Como herramienta de desarrollo se utilizó c\#.

% SYSTEM DESCRIPTION AND CHARACTERIZATION TO APPLY
\input{main/chapter2/section1/content.tex}
     
% SERIOUS GAME REQUIREMENTS
\input{main/chapter2/section2/content.tex}    

% USE CASE DEFINITION
\input{main/chapter2/section3/content.tex}

% USE CASE REALIZATION
\input{main/chapter2/section4/content.tex}

% DATABASE DESIGN 
\input{main/chapter2/section5/content.tex}

% DATA MANIPULATION
\input{main/chapter2/section6/content.tex}

% COMMUNICATION 
\input{main/chapter2/section7/content.tex}

% TRAINING SCENARIOS
\input{main/chapter2/section8/content.tex}

% EMG GRAPHIC
\input{main/chapter2/section9/content.tex}

% STATICAL REPORTS GENERATION
\input{main/chapter2/section10/content.tex}

\subthesischapter{Conclusiones del capítulo}
Se presentó una descripción del sistema de adquisición de datos para rehabilitación, sus componentes, características distintivas y su funcionamiento. Se identificaron y definieron los requisitos del juego  serio, tanto funcionales como no funcionales, así como los actores y casos de usos del sistema que establecieron las bases fundamentales para el desarrollo de la aplicación. Se realizó el diseño de la base de datos, abarcando tanto el modelo lógico como el físico, lo que aseguró una estructura robusta y eficiente para el almacenamiento de los datos. La manipulación de los datos se abordó de manera integral, desde la conexión con la base de datos hasta la persistencia de los resultados estadísticos. Se diseñó e implementó la comunicación con el pedal motorizado y la implementación de la interfaz gráfica para la representación de los datos EMG. Se definieron los escenarios de entrenamiento para las modalidades Ligero y Clínico, asegurando una cobertura completa de las necesidad de entrenamiento del usuario. Por último en el ámbito estadístico se desarrolló una serie de gráficos para el seguimiento de los resultados en las rutinas de entrenamiento.   
    
\end{thesischapter}

\subthesischapter{Conclusiones del capítulo}
Se presentó una descripción del sistema de adquisición de datos para rehabilitación, sus componentes, características distintivas y su funcionamiento. Se identificaron y definieron los requisitos del juego  serio, tanto funcionales como no funcionales, así como los actores y casos de usos del sistema que establecieron las bases fundamentales para el desarrollo de la aplicación. Se realizó el diseño de la base de datos, abarcando tanto el modelo lógico como el físico, lo que aseguró una estructura robusta y eficiente para el almacenamiento de los datos. La manipulación de los datos se abordó de manera integral, desde la conexión con la base de datos hasta la persistencia de los resultados estadísticos. Se diseñó e implementó la comunicación con el pedal motorizado y la implementación de la interfaz gráfica para la representación de los datos EMG. Se definieron los escenarios de entrenamiento para las modalidades Ligero y Clínico, asegurando una cobertura completa de las necesidad de entrenamiento del usuario. Por último en el ámbito estadístico se desarrolló una serie de gráficos para el seguimiento de los resultados en las rutinas de entrenamiento.   
    
\end{thesischapter}

% EMG GRAPHIC
\begin{thesischapter}{2} {Diseño e implementación del Juego Serio}
En este capítulo se discuten los detalles de desarrollo de los aspectos citados en el capítulo anterior. Este comienza con una descripción y caracterización general del sistema, donde se  abordan cada uno de los componentes requeridos para su completo funcionamiento. Posteriormente se detalla la ingienría de software requerida en la etapa de conceptualización de la aplicación, se explican de forma detallada los aspectos teóricos y de implementación de la base de datos, el funcionamiento del protocolo de comunicación y por último los escenarios de juegos requeridos en las rutinas de entrenamiento ligero y clínico, y las estadísticas generadas por estos. Como herramienta de desarrollo se utilizó c\#.

% SYSTEM DESCRIPTION AND CHARACTERIZATION TO APPLY
\begin{thesischapter}{2} {Diseño e implementación del Juego Serio}
En este capítulo se discuten los detalles de desarrollo de los aspectos citados en el capítulo anterior. Este comienza con una descripción y caracterización general del sistema, donde se  abordan cada uno de los componentes requeridos para su completo funcionamiento. Posteriormente se detalla la ingienría de software requerida en la etapa de conceptualización de la aplicación, se explican de forma detallada los aspectos teóricos y de implementación de la base de datos, el funcionamiento del protocolo de comunicación y por último los escenarios de juegos requeridos en las rutinas de entrenamiento ligero y clínico, y las estadísticas generadas por estos. Como herramienta de desarrollo se utilizó c\#.

% SYSTEM DESCRIPTION AND CHARACTERIZATION TO APPLY
\input{main/chapter2/section1/content.tex}
     
% SERIOUS GAME REQUIREMENTS
\input{main/chapter2/section2/content.tex}    

% USE CASE DEFINITION
\input{main/chapter2/section3/content.tex}

% USE CASE REALIZATION
\input{main/chapter2/section4/content.tex}

% DATABASE DESIGN 
\input{main/chapter2/section5/content.tex}

% DATA MANIPULATION
\input{main/chapter2/section6/content.tex}

% COMMUNICATION 
\input{main/chapter2/section7/content.tex}

% TRAINING SCENARIOS
\input{main/chapter2/section8/content.tex}

% EMG GRAPHIC
\input{main/chapter2/section9/content.tex}

% STATICAL REPORTS GENERATION
\input{main/chapter2/section10/content.tex}

\subthesischapter{Conclusiones del capítulo}
Se presentó una descripción del sistema de adquisición de datos para rehabilitación, sus componentes, características distintivas y su funcionamiento. Se identificaron y definieron los requisitos del juego  serio, tanto funcionales como no funcionales, así como los actores y casos de usos del sistema que establecieron las bases fundamentales para el desarrollo de la aplicación. Se realizó el diseño de la base de datos, abarcando tanto el modelo lógico como el físico, lo que aseguró una estructura robusta y eficiente para el almacenamiento de los datos. La manipulación de los datos se abordó de manera integral, desde la conexión con la base de datos hasta la persistencia de los resultados estadísticos. Se diseñó e implementó la comunicación con el pedal motorizado y la implementación de la interfaz gráfica para la representación de los datos EMG. Se definieron los escenarios de entrenamiento para las modalidades Ligero y Clínico, asegurando una cobertura completa de las necesidad de entrenamiento del usuario. Por último en el ámbito estadístico se desarrolló una serie de gráficos para el seguimiento de los resultados en las rutinas de entrenamiento.   
    
\end{thesischapter}
     
% SERIOUS GAME REQUIREMENTS
\begin{thesischapter}{2} {Diseño e implementación del Juego Serio}
En este capítulo se discuten los detalles de desarrollo de los aspectos citados en el capítulo anterior. Este comienza con una descripción y caracterización general del sistema, donde se  abordan cada uno de los componentes requeridos para su completo funcionamiento. Posteriormente se detalla la ingienría de software requerida en la etapa de conceptualización de la aplicación, se explican de forma detallada los aspectos teóricos y de implementación de la base de datos, el funcionamiento del protocolo de comunicación y por último los escenarios de juegos requeridos en las rutinas de entrenamiento ligero y clínico, y las estadísticas generadas por estos. Como herramienta de desarrollo se utilizó c\#.

% SYSTEM DESCRIPTION AND CHARACTERIZATION TO APPLY
\input{main/chapter2/section1/content.tex}
     
% SERIOUS GAME REQUIREMENTS
\input{main/chapter2/section2/content.tex}    

% USE CASE DEFINITION
\input{main/chapter2/section3/content.tex}

% USE CASE REALIZATION
\input{main/chapter2/section4/content.tex}

% DATABASE DESIGN 
\input{main/chapter2/section5/content.tex}

% DATA MANIPULATION
\input{main/chapter2/section6/content.tex}

% COMMUNICATION 
\input{main/chapter2/section7/content.tex}

% TRAINING SCENARIOS
\input{main/chapter2/section8/content.tex}

% EMG GRAPHIC
\input{main/chapter2/section9/content.tex}

% STATICAL REPORTS GENERATION
\input{main/chapter2/section10/content.tex}

\subthesischapter{Conclusiones del capítulo}
Se presentó una descripción del sistema de adquisición de datos para rehabilitación, sus componentes, características distintivas y su funcionamiento. Se identificaron y definieron los requisitos del juego  serio, tanto funcionales como no funcionales, así como los actores y casos de usos del sistema que establecieron las bases fundamentales para el desarrollo de la aplicación. Se realizó el diseño de la base de datos, abarcando tanto el modelo lógico como el físico, lo que aseguró una estructura robusta y eficiente para el almacenamiento de los datos. La manipulación de los datos se abordó de manera integral, desde la conexión con la base de datos hasta la persistencia de los resultados estadísticos. Se diseñó e implementó la comunicación con el pedal motorizado y la implementación de la interfaz gráfica para la representación de los datos EMG. Se definieron los escenarios de entrenamiento para las modalidades Ligero y Clínico, asegurando una cobertura completa de las necesidad de entrenamiento del usuario. Por último en el ámbito estadístico se desarrolló una serie de gráficos para el seguimiento de los resultados en las rutinas de entrenamiento.   
    
\end{thesischapter}    

% USE CASE DEFINITION
\begin{thesischapter}{2} {Diseño e implementación del Juego Serio}
En este capítulo se discuten los detalles de desarrollo de los aspectos citados en el capítulo anterior. Este comienza con una descripción y caracterización general del sistema, donde se  abordan cada uno de los componentes requeridos para su completo funcionamiento. Posteriormente se detalla la ingienría de software requerida en la etapa de conceptualización de la aplicación, se explican de forma detallada los aspectos teóricos y de implementación de la base de datos, el funcionamiento del protocolo de comunicación y por último los escenarios de juegos requeridos en las rutinas de entrenamiento ligero y clínico, y las estadísticas generadas por estos. Como herramienta de desarrollo se utilizó c\#.

% SYSTEM DESCRIPTION AND CHARACTERIZATION TO APPLY
\input{main/chapter2/section1/content.tex}
     
% SERIOUS GAME REQUIREMENTS
\input{main/chapter2/section2/content.tex}    

% USE CASE DEFINITION
\input{main/chapter2/section3/content.tex}

% USE CASE REALIZATION
\input{main/chapter2/section4/content.tex}

% DATABASE DESIGN 
\input{main/chapter2/section5/content.tex}

% DATA MANIPULATION
\input{main/chapter2/section6/content.tex}

% COMMUNICATION 
\input{main/chapter2/section7/content.tex}

% TRAINING SCENARIOS
\input{main/chapter2/section8/content.tex}

% EMG GRAPHIC
\input{main/chapter2/section9/content.tex}

% STATICAL REPORTS GENERATION
\input{main/chapter2/section10/content.tex}

\subthesischapter{Conclusiones del capítulo}
Se presentó una descripción del sistema de adquisición de datos para rehabilitación, sus componentes, características distintivas y su funcionamiento. Se identificaron y definieron los requisitos del juego  serio, tanto funcionales como no funcionales, así como los actores y casos de usos del sistema que establecieron las bases fundamentales para el desarrollo de la aplicación. Se realizó el diseño de la base de datos, abarcando tanto el modelo lógico como el físico, lo que aseguró una estructura robusta y eficiente para el almacenamiento de los datos. La manipulación de los datos se abordó de manera integral, desde la conexión con la base de datos hasta la persistencia de los resultados estadísticos. Se diseñó e implementó la comunicación con el pedal motorizado y la implementación de la interfaz gráfica para la representación de los datos EMG. Se definieron los escenarios de entrenamiento para las modalidades Ligero y Clínico, asegurando una cobertura completa de las necesidad de entrenamiento del usuario. Por último en el ámbito estadístico se desarrolló una serie de gráficos para el seguimiento de los resultados en las rutinas de entrenamiento.   
    
\end{thesischapter}

% USE CASE REALIZATION
\begin{thesischapter}{2} {Diseño e implementación del Juego Serio}
En este capítulo se discuten los detalles de desarrollo de los aspectos citados en el capítulo anterior. Este comienza con una descripción y caracterización general del sistema, donde se  abordan cada uno de los componentes requeridos para su completo funcionamiento. Posteriormente se detalla la ingienría de software requerida en la etapa de conceptualización de la aplicación, se explican de forma detallada los aspectos teóricos y de implementación de la base de datos, el funcionamiento del protocolo de comunicación y por último los escenarios de juegos requeridos en las rutinas de entrenamiento ligero y clínico, y las estadísticas generadas por estos. Como herramienta de desarrollo se utilizó c\#.

% SYSTEM DESCRIPTION AND CHARACTERIZATION TO APPLY
\input{main/chapter2/section1/content.tex}
     
% SERIOUS GAME REQUIREMENTS
\input{main/chapter2/section2/content.tex}    

% USE CASE DEFINITION
\input{main/chapter2/section3/content.tex}

% USE CASE REALIZATION
\input{main/chapter2/section4/content.tex}

% DATABASE DESIGN 
\input{main/chapter2/section5/content.tex}

% DATA MANIPULATION
\input{main/chapter2/section6/content.tex}

% COMMUNICATION 
\input{main/chapter2/section7/content.tex}

% TRAINING SCENARIOS
\input{main/chapter2/section8/content.tex}

% EMG GRAPHIC
\input{main/chapter2/section9/content.tex}

% STATICAL REPORTS GENERATION
\input{main/chapter2/section10/content.tex}

\subthesischapter{Conclusiones del capítulo}
Se presentó una descripción del sistema de adquisición de datos para rehabilitación, sus componentes, características distintivas y su funcionamiento. Se identificaron y definieron los requisitos del juego  serio, tanto funcionales como no funcionales, así como los actores y casos de usos del sistema que establecieron las bases fundamentales para el desarrollo de la aplicación. Se realizó el diseño de la base de datos, abarcando tanto el modelo lógico como el físico, lo que aseguró una estructura robusta y eficiente para el almacenamiento de los datos. La manipulación de los datos se abordó de manera integral, desde la conexión con la base de datos hasta la persistencia de los resultados estadísticos. Se diseñó e implementó la comunicación con el pedal motorizado y la implementación de la interfaz gráfica para la representación de los datos EMG. Se definieron los escenarios de entrenamiento para las modalidades Ligero y Clínico, asegurando una cobertura completa de las necesidad de entrenamiento del usuario. Por último en el ámbito estadístico se desarrolló una serie de gráficos para el seguimiento de los resultados en las rutinas de entrenamiento.   
    
\end{thesischapter}

% DATABASE DESIGN 
\begin{thesischapter}{2} {Diseño e implementación del Juego Serio}
En este capítulo se discuten los detalles de desarrollo de los aspectos citados en el capítulo anterior. Este comienza con una descripción y caracterización general del sistema, donde se  abordan cada uno de los componentes requeridos para su completo funcionamiento. Posteriormente se detalla la ingienría de software requerida en la etapa de conceptualización de la aplicación, se explican de forma detallada los aspectos teóricos y de implementación de la base de datos, el funcionamiento del protocolo de comunicación y por último los escenarios de juegos requeridos en las rutinas de entrenamiento ligero y clínico, y las estadísticas generadas por estos. Como herramienta de desarrollo se utilizó c\#.

% SYSTEM DESCRIPTION AND CHARACTERIZATION TO APPLY
\input{main/chapter2/section1/content.tex}
     
% SERIOUS GAME REQUIREMENTS
\input{main/chapter2/section2/content.tex}    

% USE CASE DEFINITION
\input{main/chapter2/section3/content.tex}

% USE CASE REALIZATION
\input{main/chapter2/section4/content.tex}

% DATABASE DESIGN 
\input{main/chapter2/section5/content.tex}

% DATA MANIPULATION
\input{main/chapter2/section6/content.tex}

% COMMUNICATION 
\input{main/chapter2/section7/content.tex}

% TRAINING SCENARIOS
\input{main/chapter2/section8/content.tex}

% EMG GRAPHIC
\input{main/chapter2/section9/content.tex}

% STATICAL REPORTS GENERATION
\input{main/chapter2/section10/content.tex}

\subthesischapter{Conclusiones del capítulo}
Se presentó una descripción del sistema de adquisición de datos para rehabilitación, sus componentes, características distintivas y su funcionamiento. Se identificaron y definieron los requisitos del juego  serio, tanto funcionales como no funcionales, así como los actores y casos de usos del sistema que establecieron las bases fundamentales para el desarrollo de la aplicación. Se realizó el diseño de la base de datos, abarcando tanto el modelo lógico como el físico, lo que aseguró una estructura robusta y eficiente para el almacenamiento de los datos. La manipulación de los datos se abordó de manera integral, desde la conexión con la base de datos hasta la persistencia de los resultados estadísticos. Se diseñó e implementó la comunicación con el pedal motorizado y la implementación de la interfaz gráfica para la representación de los datos EMG. Se definieron los escenarios de entrenamiento para las modalidades Ligero y Clínico, asegurando una cobertura completa de las necesidad de entrenamiento del usuario. Por último en el ámbito estadístico se desarrolló una serie de gráficos para el seguimiento de los resultados en las rutinas de entrenamiento.   
    
\end{thesischapter}

% DATA MANIPULATION
\begin{thesischapter}{2} {Diseño e implementación del Juego Serio}
En este capítulo se discuten los detalles de desarrollo de los aspectos citados en el capítulo anterior. Este comienza con una descripción y caracterización general del sistema, donde se  abordan cada uno de los componentes requeridos para su completo funcionamiento. Posteriormente se detalla la ingienría de software requerida en la etapa de conceptualización de la aplicación, se explican de forma detallada los aspectos teóricos y de implementación de la base de datos, el funcionamiento del protocolo de comunicación y por último los escenarios de juegos requeridos en las rutinas de entrenamiento ligero y clínico, y las estadísticas generadas por estos. Como herramienta de desarrollo se utilizó c\#.

% SYSTEM DESCRIPTION AND CHARACTERIZATION TO APPLY
\input{main/chapter2/section1/content.tex}
     
% SERIOUS GAME REQUIREMENTS
\input{main/chapter2/section2/content.tex}    

% USE CASE DEFINITION
\input{main/chapter2/section3/content.tex}

% USE CASE REALIZATION
\input{main/chapter2/section4/content.tex}

% DATABASE DESIGN 
\input{main/chapter2/section5/content.tex}

% DATA MANIPULATION
\input{main/chapter2/section6/content.tex}

% COMMUNICATION 
\input{main/chapter2/section7/content.tex}

% TRAINING SCENARIOS
\input{main/chapter2/section8/content.tex}

% EMG GRAPHIC
\input{main/chapter2/section9/content.tex}

% STATICAL REPORTS GENERATION
\input{main/chapter2/section10/content.tex}

\subthesischapter{Conclusiones del capítulo}
Se presentó una descripción del sistema de adquisición de datos para rehabilitación, sus componentes, características distintivas y su funcionamiento. Se identificaron y definieron los requisitos del juego  serio, tanto funcionales como no funcionales, así como los actores y casos de usos del sistema que establecieron las bases fundamentales para el desarrollo de la aplicación. Se realizó el diseño de la base de datos, abarcando tanto el modelo lógico como el físico, lo que aseguró una estructura robusta y eficiente para el almacenamiento de los datos. La manipulación de los datos se abordó de manera integral, desde la conexión con la base de datos hasta la persistencia de los resultados estadísticos. Se diseñó e implementó la comunicación con el pedal motorizado y la implementación de la interfaz gráfica para la representación de los datos EMG. Se definieron los escenarios de entrenamiento para las modalidades Ligero y Clínico, asegurando una cobertura completa de las necesidad de entrenamiento del usuario. Por último en el ámbito estadístico se desarrolló una serie de gráficos para el seguimiento de los resultados en las rutinas de entrenamiento.   
    
\end{thesischapter}

% COMMUNICATION 
\begin{thesischapter}{2} {Diseño e implementación del Juego Serio}
En este capítulo se discuten los detalles de desarrollo de los aspectos citados en el capítulo anterior. Este comienza con una descripción y caracterización general del sistema, donde se  abordan cada uno de los componentes requeridos para su completo funcionamiento. Posteriormente se detalla la ingienría de software requerida en la etapa de conceptualización de la aplicación, se explican de forma detallada los aspectos teóricos y de implementación de la base de datos, el funcionamiento del protocolo de comunicación y por último los escenarios de juegos requeridos en las rutinas de entrenamiento ligero y clínico, y las estadísticas generadas por estos. Como herramienta de desarrollo se utilizó c\#.

% SYSTEM DESCRIPTION AND CHARACTERIZATION TO APPLY
\input{main/chapter2/section1/content.tex}
     
% SERIOUS GAME REQUIREMENTS
\input{main/chapter2/section2/content.tex}    

% USE CASE DEFINITION
\input{main/chapter2/section3/content.tex}

% USE CASE REALIZATION
\input{main/chapter2/section4/content.tex}

% DATABASE DESIGN 
\input{main/chapter2/section5/content.tex}

% DATA MANIPULATION
\input{main/chapter2/section6/content.tex}

% COMMUNICATION 
\input{main/chapter2/section7/content.tex}

% TRAINING SCENARIOS
\input{main/chapter2/section8/content.tex}

% EMG GRAPHIC
\input{main/chapter2/section9/content.tex}

% STATICAL REPORTS GENERATION
\input{main/chapter2/section10/content.tex}

\subthesischapter{Conclusiones del capítulo}
Se presentó una descripción del sistema de adquisición de datos para rehabilitación, sus componentes, características distintivas y su funcionamiento. Se identificaron y definieron los requisitos del juego  serio, tanto funcionales como no funcionales, así como los actores y casos de usos del sistema que establecieron las bases fundamentales para el desarrollo de la aplicación. Se realizó el diseño de la base de datos, abarcando tanto el modelo lógico como el físico, lo que aseguró una estructura robusta y eficiente para el almacenamiento de los datos. La manipulación de los datos se abordó de manera integral, desde la conexión con la base de datos hasta la persistencia de los resultados estadísticos. Se diseñó e implementó la comunicación con el pedal motorizado y la implementación de la interfaz gráfica para la representación de los datos EMG. Se definieron los escenarios de entrenamiento para las modalidades Ligero y Clínico, asegurando una cobertura completa de las necesidad de entrenamiento del usuario. Por último en el ámbito estadístico se desarrolló una serie de gráficos para el seguimiento de los resultados en las rutinas de entrenamiento.   
    
\end{thesischapter}

% TRAINING SCENARIOS
\begin{thesischapter}{2} {Diseño e implementación del Juego Serio}
En este capítulo se discuten los detalles de desarrollo de los aspectos citados en el capítulo anterior. Este comienza con una descripción y caracterización general del sistema, donde se  abordan cada uno de los componentes requeridos para su completo funcionamiento. Posteriormente se detalla la ingienría de software requerida en la etapa de conceptualización de la aplicación, se explican de forma detallada los aspectos teóricos y de implementación de la base de datos, el funcionamiento del protocolo de comunicación y por último los escenarios de juegos requeridos en las rutinas de entrenamiento ligero y clínico, y las estadísticas generadas por estos. Como herramienta de desarrollo se utilizó c\#.

% SYSTEM DESCRIPTION AND CHARACTERIZATION TO APPLY
\input{main/chapter2/section1/content.tex}
     
% SERIOUS GAME REQUIREMENTS
\input{main/chapter2/section2/content.tex}    

% USE CASE DEFINITION
\input{main/chapter2/section3/content.tex}

% USE CASE REALIZATION
\input{main/chapter2/section4/content.tex}

% DATABASE DESIGN 
\input{main/chapter2/section5/content.tex}

% DATA MANIPULATION
\input{main/chapter2/section6/content.tex}

% COMMUNICATION 
\input{main/chapter2/section7/content.tex}

% TRAINING SCENARIOS
\input{main/chapter2/section8/content.tex}

% EMG GRAPHIC
\input{main/chapter2/section9/content.tex}

% STATICAL REPORTS GENERATION
\input{main/chapter2/section10/content.tex}

\subthesischapter{Conclusiones del capítulo}
Se presentó una descripción del sistema de adquisición de datos para rehabilitación, sus componentes, características distintivas y su funcionamiento. Se identificaron y definieron los requisitos del juego  serio, tanto funcionales como no funcionales, así como los actores y casos de usos del sistema que establecieron las bases fundamentales para el desarrollo de la aplicación. Se realizó el diseño de la base de datos, abarcando tanto el modelo lógico como el físico, lo que aseguró una estructura robusta y eficiente para el almacenamiento de los datos. La manipulación de los datos se abordó de manera integral, desde la conexión con la base de datos hasta la persistencia de los resultados estadísticos. Se diseñó e implementó la comunicación con el pedal motorizado y la implementación de la interfaz gráfica para la representación de los datos EMG. Se definieron los escenarios de entrenamiento para las modalidades Ligero y Clínico, asegurando una cobertura completa de las necesidad de entrenamiento del usuario. Por último en el ámbito estadístico se desarrolló una serie de gráficos para el seguimiento de los resultados en las rutinas de entrenamiento.   
    
\end{thesischapter}

% EMG GRAPHIC
\begin{thesischapter}{2} {Diseño e implementación del Juego Serio}
En este capítulo se discuten los detalles de desarrollo de los aspectos citados en el capítulo anterior. Este comienza con una descripción y caracterización general del sistema, donde se  abordan cada uno de los componentes requeridos para su completo funcionamiento. Posteriormente se detalla la ingienría de software requerida en la etapa de conceptualización de la aplicación, se explican de forma detallada los aspectos teóricos y de implementación de la base de datos, el funcionamiento del protocolo de comunicación y por último los escenarios de juegos requeridos en las rutinas de entrenamiento ligero y clínico, y las estadísticas generadas por estos. Como herramienta de desarrollo se utilizó c\#.

% SYSTEM DESCRIPTION AND CHARACTERIZATION TO APPLY
\input{main/chapter2/section1/content.tex}
     
% SERIOUS GAME REQUIREMENTS
\input{main/chapter2/section2/content.tex}    

% USE CASE DEFINITION
\input{main/chapter2/section3/content.tex}

% USE CASE REALIZATION
\input{main/chapter2/section4/content.tex}

% DATABASE DESIGN 
\input{main/chapter2/section5/content.tex}

% DATA MANIPULATION
\input{main/chapter2/section6/content.tex}

% COMMUNICATION 
\input{main/chapter2/section7/content.tex}

% TRAINING SCENARIOS
\input{main/chapter2/section8/content.tex}

% EMG GRAPHIC
\input{main/chapter2/section9/content.tex}

% STATICAL REPORTS GENERATION
\input{main/chapter2/section10/content.tex}

\subthesischapter{Conclusiones del capítulo}
Se presentó una descripción del sistema de adquisición de datos para rehabilitación, sus componentes, características distintivas y su funcionamiento. Se identificaron y definieron los requisitos del juego  serio, tanto funcionales como no funcionales, así como los actores y casos de usos del sistema que establecieron las bases fundamentales para el desarrollo de la aplicación. Se realizó el diseño de la base de datos, abarcando tanto el modelo lógico como el físico, lo que aseguró una estructura robusta y eficiente para el almacenamiento de los datos. La manipulación de los datos se abordó de manera integral, desde la conexión con la base de datos hasta la persistencia de los resultados estadísticos. Se diseñó e implementó la comunicación con el pedal motorizado y la implementación de la interfaz gráfica para la representación de los datos EMG. Se definieron los escenarios de entrenamiento para las modalidades Ligero y Clínico, asegurando una cobertura completa de las necesidad de entrenamiento del usuario. Por último en el ámbito estadístico se desarrolló una serie de gráficos para el seguimiento de los resultados en las rutinas de entrenamiento.   
    
\end{thesischapter}

% STATICAL REPORTS GENERATION
\begin{thesischapter}{2} {Diseño e implementación del Juego Serio}
En este capítulo se discuten los detalles de desarrollo de los aspectos citados en el capítulo anterior. Este comienza con una descripción y caracterización general del sistema, donde se  abordan cada uno de los componentes requeridos para su completo funcionamiento. Posteriormente se detalla la ingienría de software requerida en la etapa de conceptualización de la aplicación, se explican de forma detallada los aspectos teóricos y de implementación de la base de datos, el funcionamiento del protocolo de comunicación y por último los escenarios de juegos requeridos en las rutinas de entrenamiento ligero y clínico, y las estadísticas generadas por estos. Como herramienta de desarrollo se utilizó c\#.

% SYSTEM DESCRIPTION AND CHARACTERIZATION TO APPLY
\input{main/chapter2/section1/content.tex}
     
% SERIOUS GAME REQUIREMENTS
\input{main/chapter2/section2/content.tex}    

% USE CASE DEFINITION
\input{main/chapter2/section3/content.tex}

% USE CASE REALIZATION
\input{main/chapter2/section4/content.tex}

% DATABASE DESIGN 
\input{main/chapter2/section5/content.tex}

% DATA MANIPULATION
\input{main/chapter2/section6/content.tex}

% COMMUNICATION 
\input{main/chapter2/section7/content.tex}

% TRAINING SCENARIOS
\input{main/chapter2/section8/content.tex}

% EMG GRAPHIC
\input{main/chapter2/section9/content.tex}

% STATICAL REPORTS GENERATION
\input{main/chapter2/section10/content.tex}

\subthesischapter{Conclusiones del capítulo}
Se presentó una descripción del sistema de adquisición de datos para rehabilitación, sus componentes, características distintivas y su funcionamiento. Se identificaron y definieron los requisitos del juego  serio, tanto funcionales como no funcionales, así como los actores y casos de usos del sistema que establecieron las bases fundamentales para el desarrollo de la aplicación. Se realizó el diseño de la base de datos, abarcando tanto el modelo lógico como el físico, lo que aseguró una estructura robusta y eficiente para el almacenamiento de los datos. La manipulación de los datos se abordó de manera integral, desde la conexión con la base de datos hasta la persistencia de los resultados estadísticos. Se diseñó e implementó la comunicación con el pedal motorizado y la implementación de la interfaz gráfica para la representación de los datos EMG. Se definieron los escenarios de entrenamiento para las modalidades Ligero y Clínico, asegurando una cobertura completa de las necesidad de entrenamiento del usuario. Por último en el ámbito estadístico se desarrolló una serie de gráficos para el seguimiento de los resultados en las rutinas de entrenamiento.   
    
\end{thesischapter}

\subthesischapter{Conclusiones del capítulo}
Se presentó una descripción del sistema de adquisición de datos para rehabilitación, sus componentes, características distintivas y su funcionamiento. Se identificaron y definieron los requisitos del juego  serio, tanto funcionales como no funcionales, así como los actores y casos de usos del sistema que establecieron las bases fundamentales para el desarrollo de la aplicación. Se realizó el diseño de la base de datos, abarcando tanto el modelo lógico como el físico, lo que aseguró una estructura robusta y eficiente para el almacenamiento de los datos. La manipulación de los datos se abordó de manera integral, desde la conexión con la base de datos hasta la persistencia de los resultados estadísticos. Se diseñó e implementó la comunicación con el pedal motorizado y la implementación de la interfaz gráfica para la representación de los datos EMG. Se definieron los escenarios de entrenamiento para las modalidades Ligero y Clínico, asegurando una cobertura completa de las necesidad de entrenamiento del usuario. Por último en el ámbito estadístico se desarrolló una serie de gráficos para el seguimiento de los resultados en las rutinas de entrenamiento.   
    
\end{thesischapter}

% STATICAL REPORTS GENERATION
\begin{thesischapter}{2} {Diseño e implementación del Juego Serio}
En este capítulo se discuten los detalles de desarrollo de los aspectos citados en el capítulo anterior. Este comienza con una descripción y caracterización general del sistema, donde se  abordan cada uno de los componentes requeridos para su completo funcionamiento. Posteriormente se detalla la ingienría de software requerida en la etapa de conceptualización de la aplicación, se explican de forma detallada los aspectos teóricos y de implementación de la base de datos, el funcionamiento del protocolo de comunicación y por último los escenarios de juegos requeridos en las rutinas de entrenamiento ligero y clínico, y las estadísticas generadas por estos. Como herramienta de desarrollo se utilizó c\#.

% SYSTEM DESCRIPTION AND CHARACTERIZATION TO APPLY
\begin{thesischapter}{2} {Diseño e implementación del Juego Serio}
En este capítulo se discuten los detalles de desarrollo de los aspectos citados en el capítulo anterior. Este comienza con una descripción y caracterización general del sistema, donde se  abordan cada uno de los componentes requeridos para su completo funcionamiento. Posteriormente se detalla la ingienría de software requerida en la etapa de conceptualización de la aplicación, se explican de forma detallada los aspectos teóricos y de implementación de la base de datos, el funcionamiento del protocolo de comunicación y por último los escenarios de juegos requeridos en las rutinas de entrenamiento ligero y clínico, y las estadísticas generadas por estos. Como herramienta de desarrollo se utilizó c\#.

% SYSTEM DESCRIPTION AND CHARACTERIZATION TO APPLY
\input{main/chapter2/section1/content.tex}
     
% SERIOUS GAME REQUIREMENTS
\input{main/chapter2/section2/content.tex}    

% USE CASE DEFINITION
\input{main/chapter2/section3/content.tex}

% USE CASE REALIZATION
\input{main/chapter2/section4/content.tex}

% DATABASE DESIGN 
\input{main/chapter2/section5/content.tex}

% DATA MANIPULATION
\input{main/chapter2/section6/content.tex}

% COMMUNICATION 
\input{main/chapter2/section7/content.tex}

% TRAINING SCENARIOS
\input{main/chapter2/section8/content.tex}

% EMG GRAPHIC
\input{main/chapter2/section9/content.tex}

% STATICAL REPORTS GENERATION
\input{main/chapter2/section10/content.tex}

\subthesischapter{Conclusiones del capítulo}
Se presentó una descripción del sistema de adquisición de datos para rehabilitación, sus componentes, características distintivas y su funcionamiento. Se identificaron y definieron los requisitos del juego  serio, tanto funcionales como no funcionales, así como los actores y casos de usos del sistema que establecieron las bases fundamentales para el desarrollo de la aplicación. Se realizó el diseño de la base de datos, abarcando tanto el modelo lógico como el físico, lo que aseguró una estructura robusta y eficiente para el almacenamiento de los datos. La manipulación de los datos se abordó de manera integral, desde la conexión con la base de datos hasta la persistencia de los resultados estadísticos. Se diseñó e implementó la comunicación con el pedal motorizado y la implementación de la interfaz gráfica para la representación de los datos EMG. Se definieron los escenarios de entrenamiento para las modalidades Ligero y Clínico, asegurando una cobertura completa de las necesidad de entrenamiento del usuario. Por último en el ámbito estadístico se desarrolló una serie de gráficos para el seguimiento de los resultados en las rutinas de entrenamiento.   
    
\end{thesischapter}
     
% SERIOUS GAME REQUIREMENTS
\begin{thesischapter}{2} {Diseño e implementación del Juego Serio}
En este capítulo se discuten los detalles de desarrollo de los aspectos citados en el capítulo anterior. Este comienza con una descripción y caracterización general del sistema, donde se  abordan cada uno de los componentes requeridos para su completo funcionamiento. Posteriormente se detalla la ingienría de software requerida en la etapa de conceptualización de la aplicación, se explican de forma detallada los aspectos teóricos y de implementación de la base de datos, el funcionamiento del protocolo de comunicación y por último los escenarios de juegos requeridos en las rutinas de entrenamiento ligero y clínico, y las estadísticas generadas por estos. Como herramienta de desarrollo se utilizó c\#.

% SYSTEM DESCRIPTION AND CHARACTERIZATION TO APPLY
\input{main/chapter2/section1/content.tex}
     
% SERIOUS GAME REQUIREMENTS
\input{main/chapter2/section2/content.tex}    

% USE CASE DEFINITION
\input{main/chapter2/section3/content.tex}

% USE CASE REALIZATION
\input{main/chapter2/section4/content.tex}

% DATABASE DESIGN 
\input{main/chapter2/section5/content.tex}

% DATA MANIPULATION
\input{main/chapter2/section6/content.tex}

% COMMUNICATION 
\input{main/chapter2/section7/content.tex}

% TRAINING SCENARIOS
\input{main/chapter2/section8/content.tex}

% EMG GRAPHIC
\input{main/chapter2/section9/content.tex}

% STATICAL REPORTS GENERATION
\input{main/chapter2/section10/content.tex}

\subthesischapter{Conclusiones del capítulo}
Se presentó una descripción del sistema de adquisición de datos para rehabilitación, sus componentes, características distintivas y su funcionamiento. Se identificaron y definieron los requisitos del juego  serio, tanto funcionales como no funcionales, así como los actores y casos de usos del sistema que establecieron las bases fundamentales para el desarrollo de la aplicación. Se realizó el diseño de la base de datos, abarcando tanto el modelo lógico como el físico, lo que aseguró una estructura robusta y eficiente para el almacenamiento de los datos. La manipulación de los datos se abordó de manera integral, desde la conexión con la base de datos hasta la persistencia de los resultados estadísticos. Se diseñó e implementó la comunicación con el pedal motorizado y la implementación de la interfaz gráfica para la representación de los datos EMG. Se definieron los escenarios de entrenamiento para las modalidades Ligero y Clínico, asegurando una cobertura completa de las necesidad de entrenamiento del usuario. Por último en el ámbito estadístico se desarrolló una serie de gráficos para el seguimiento de los resultados en las rutinas de entrenamiento.   
    
\end{thesischapter}    

% USE CASE DEFINITION
\begin{thesischapter}{2} {Diseño e implementación del Juego Serio}
En este capítulo se discuten los detalles de desarrollo de los aspectos citados en el capítulo anterior. Este comienza con una descripción y caracterización general del sistema, donde se  abordan cada uno de los componentes requeridos para su completo funcionamiento. Posteriormente se detalla la ingienría de software requerida en la etapa de conceptualización de la aplicación, se explican de forma detallada los aspectos teóricos y de implementación de la base de datos, el funcionamiento del protocolo de comunicación y por último los escenarios de juegos requeridos en las rutinas de entrenamiento ligero y clínico, y las estadísticas generadas por estos. Como herramienta de desarrollo se utilizó c\#.

% SYSTEM DESCRIPTION AND CHARACTERIZATION TO APPLY
\input{main/chapter2/section1/content.tex}
     
% SERIOUS GAME REQUIREMENTS
\input{main/chapter2/section2/content.tex}    

% USE CASE DEFINITION
\input{main/chapter2/section3/content.tex}

% USE CASE REALIZATION
\input{main/chapter2/section4/content.tex}

% DATABASE DESIGN 
\input{main/chapter2/section5/content.tex}

% DATA MANIPULATION
\input{main/chapter2/section6/content.tex}

% COMMUNICATION 
\input{main/chapter2/section7/content.tex}

% TRAINING SCENARIOS
\input{main/chapter2/section8/content.tex}

% EMG GRAPHIC
\input{main/chapter2/section9/content.tex}

% STATICAL REPORTS GENERATION
\input{main/chapter2/section10/content.tex}

\subthesischapter{Conclusiones del capítulo}
Se presentó una descripción del sistema de adquisición de datos para rehabilitación, sus componentes, características distintivas y su funcionamiento. Se identificaron y definieron los requisitos del juego  serio, tanto funcionales como no funcionales, así como los actores y casos de usos del sistema que establecieron las bases fundamentales para el desarrollo de la aplicación. Se realizó el diseño de la base de datos, abarcando tanto el modelo lógico como el físico, lo que aseguró una estructura robusta y eficiente para el almacenamiento de los datos. La manipulación de los datos se abordó de manera integral, desde la conexión con la base de datos hasta la persistencia de los resultados estadísticos. Se diseñó e implementó la comunicación con el pedal motorizado y la implementación de la interfaz gráfica para la representación de los datos EMG. Se definieron los escenarios de entrenamiento para las modalidades Ligero y Clínico, asegurando una cobertura completa de las necesidad de entrenamiento del usuario. Por último en el ámbito estadístico se desarrolló una serie de gráficos para el seguimiento de los resultados en las rutinas de entrenamiento.   
    
\end{thesischapter}

% USE CASE REALIZATION
\begin{thesischapter}{2} {Diseño e implementación del Juego Serio}
En este capítulo se discuten los detalles de desarrollo de los aspectos citados en el capítulo anterior. Este comienza con una descripción y caracterización general del sistema, donde se  abordan cada uno de los componentes requeridos para su completo funcionamiento. Posteriormente se detalla la ingienría de software requerida en la etapa de conceptualización de la aplicación, se explican de forma detallada los aspectos teóricos y de implementación de la base de datos, el funcionamiento del protocolo de comunicación y por último los escenarios de juegos requeridos en las rutinas de entrenamiento ligero y clínico, y las estadísticas generadas por estos. Como herramienta de desarrollo se utilizó c\#.

% SYSTEM DESCRIPTION AND CHARACTERIZATION TO APPLY
\input{main/chapter2/section1/content.tex}
     
% SERIOUS GAME REQUIREMENTS
\input{main/chapter2/section2/content.tex}    

% USE CASE DEFINITION
\input{main/chapter2/section3/content.tex}

% USE CASE REALIZATION
\input{main/chapter2/section4/content.tex}

% DATABASE DESIGN 
\input{main/chapter2/section5/content.tex}

% DATA MANIPULATION
\input{main/chapter2/section6/content.tex}

% COMMUNICATION 
\input{main/chapter2/section7/content.tex}

% TRAINING SCENARIOS
\input{main/chapter2/section8/content.tex}

% EMG GRAPHIC
\input{main/chapter2/section9/content.tex}

% STATICAL REPORTS GENERATION
\input{main/chapter2/section10/content.tex}

\subthesischapter{Conclusiones del capítulo}
Se presentó una descripción del sistema de adquisición de datos para rehabilitación, sus componentes, características distintivas y su funcionamiento. Se identificaron y definieron los requisitos del juego  serio, tanto funcionales como no funcionales, así como los actores y casos de usos del sistema que establecieron las bases fundamentales para el desarrollo de la aplicación. Se realizó el diseño de la base de datos, abarcando tanto el modelo lógico como el físico, lo que aseguró una estructura robusta y eficiente para el almacenamiento de los datos. La manipulación de los datos se abordó de manera integral, desde la conexión con la base de datos hasta la persistencia de los resultados estadísticos. Se diseñó e implementó la comunicación con el pedal motorizado y la implementación de la interfaz gráfica para la representación de los datos EMG. Se definieron los escenarios de entrenamiento para las modalidades Ligero y Clínico, asegurando una cobertura completa de las necesidad de entrenamiento del usuario. Por último en el ámbito estadístico se desarrolló una serie de gráficos para el seguimiento de los resultados en las rutinas de entrenamiento.   
    
\end{thesischapter}

% DATABASE DESIGN 
\begin{thesischapter}{2} {Diseño e implementación del Juego Serio}
En este capítulo se discuten los detalles de desarrollo de los aspectos citados en el capítulo anterior. Este comienza con una descripción y caracterización general del sistema, donde se  abordan cada uno de los componentes requeridos para su completo funcionamiento. Posteriormente se detalla la ingienría de software requerida en la etapa de conceptualización de la aplicación, se explican de forma detallada los aspectos teóricos y de implementación de la base de datos, el funcionamiento del protocolo de comunicación y por último los escenarios de juegos requeridos en las rutinas de entrenamiento ligero y clínico, y las estadísticas generadas por estos. Como herramienta de desarrollo se utilizó c\#.

% SYSTEM DESCRIPTION AND CHARACTERIZATION TO APPLY
\input{main/chapter2/section1/content.tex}
     
% SERIOUS GAME REQUIREMENTS
\input{main/chapter2/section2/content.tex}    

% USE CASE DEFINITION
\input{main/chapter2/section3/content.tex}

% USE CASE REALIZATION
\input{main/chapter2/section4/content.tex}

% DATABASE DESIGN 
\input{main/chapter2/section5/content.tex}

% DATA MANIPULATION
\input{main/chapter2/section6/content.tex}

% COMMUNICATION 
\input{main/chapter2/section7/content.tex}

% TRAINING SCENARIOS
\input{main/chapter2/section8/content.tex}

% EMG GRAPHIC
\input{main/chapter2/section9/content.tex}

% STATICAL REPORTS GENERATION
\input{main/chapter2/section10/content.tex}

\subthesischapter{Conclusiones del capítulo}
Se presentó una descripción del sistema de adquisición de datos para rehabilitación, sus componentes, características distintivas y su funcionamiento. Se identificaron y definieron los requisitos del juego  serio, tanto funcionales como no funcionales, así como los actores y casos de usos del sistema que establecieron las bases fundamentales para el desarrollo de la aplicación. Se realizó el diseño de la base de datos, abarcando tanto el modelo lógico como el físico, lo que aseguró una estructura robusta y eficiente para el almacenamiento de los datos. La manipulación de los datos se abordó de manera integral, desde la conexión con la base de datos hasta la persistencia de los resultados estadísticos. Se diseñó e implementó la comunicación con el pedal motorizado y la implementación de la interfaz gráfica para la representación de los datos EMG. Se definieron los escenarios de entrenamiento para las modalidades Ligero y Clínico, asegurando una cobertura completa de las necesidad de entrenamiento del usuario. Por último en el ámbito estadístico se desarrolló una serie de gráficos para el seguimiento de los resultados en las rutinas de entrenamiento.   
    
\end{thesischapter}

% DATA MANIPULATION
\begin{thesischapter}{2} {Diseño e implementación del Juego Serio}
En este capítulo se discuten los detalles de desarrollo de los aspectos citados en el capítulo anterior. Este comienza con una descripción y caracterización general del sistema, donde se  abordan cada uno de los componentes requeridos para su completo funcionamiento. Posteriormente se detalla la ingienría de software requerida en la etapa de conceptualización de la aplicación, se explican de forma detallada los aspectos teóricos y de implementación de la base de datos, el funcionamiento del protocolo de comunicación y por último los escenarios de juegos requeridos en las rutinas de entrenamiento ligero y clínico, y las estadísticas generadas por estos. Como herramienta de desarrollo se utilizó c\#.

% SYSTEM DESCRIPTION AND CHARACTERIZATION TO APPLY
\input{main/chapter2/section1/content.tex}
     
% SERIOUS GAME REQUIREMENTS
\input{main/chapter2/section2/content.tex}    

% USE CASE DEFINITION
\input{main/chapter2/section3/content.tex}

% USE CASE REALIZATION
\input{main/chapter2/section4/content.tex}

% DATABASE DESIGN 
\input{main/chapter2/section5/content.tex}

% DATA MANIPULATION
\input{main/chapter2/section6/content.tex}

% COMMUNICATION 
\input{main/chapter2/section7/content.tex}

% TRAINING SCENARIOS
\input{main/chapter2/section8/content.tex}

% EMG GRAPHIC
\input{main/chapter2/section9/content.tex}

% STATICAL REPORTS GENERATION
\input{main/chapter2/section10/content.tex}

\subthesischapter{Conclusiones del capítulo}
Se presentó una descripción del sistema de adquisición de datos para rehabilitación, sus componentes, características distintivas y su funcionamiento. Se identificaron y definieron los requisitos del juego  serio, tanto funcionales como no funcionales, así como los actores y casos de usos del sistema que establecieron las bases fundamentales para el desarrollo de la aplicación. Se realizó el diseño de la base de datos, abarcando tanto el modelo lógico como el físico, lo que aseguró una estructura robusta y eficiente para el almacenamiento de los datos. La manipulación de los datos se abordó de manera integral, desde la conexión con la base de datos hasta la persistencia de los resultados estadísticos. Se diseñó e implementó la comunicación con el pedal motorizado y la implementación de la interfaz gráfica para la representación de los datos EMG. Se definieron los escenarios de entrenamiento para las modalidades Ligero y Clínico, asegurando una cobertura completa de las necesidad de entrenamiento del usuario. Por último en el ámbito estadístico se desarrolló una serie de gráficos para el seguimiento de los resultados en las rutinas de entrenamiento.   
    
\end{thesischapter}

% COMMUNICATION 
\begin{thesischapter}{2} {Diseño e implementación del Juego Serio}
En este capítulo se discuten los detalles de desarrollo de los aspectos citados en el capítulo anterior. Este comienza con una descripción y caracterización general del sistema, donde se  abordan cada uno de los componentes requeridos para su completo funcionamiento. Posteriormente se detalla la ingienría de software requerida en la etapa de conceptualización de la aplicación, se explican de forma detallada los aspectos teóricos y de implementación de la base de datos, el funcionamiento del protocolo de comunicación y por último los escenarios de juegos requeridos en las rutinas de entrenamiento ligero y clínico, y las estadísticas generadas por estos. Como herramienta de desarrollo se utilizó c\#.

% SYSTEM DESCRIPTION AND CHARACTERIZATION TO APPLY
\input{main/chapter2/section1/content.tex}
     
% SERIOUS GAME REQUIREMENTS
\input{main/chapter2/section2/content.tex}    

% USE CASE DEFINITION
\input{main/chapter2/section3/content.tex}

% USE CASE REALIZATION
\input{main/chapter2/section4/content.tex}

% DATABASE DESIGN 
\input{main/chapter2/section5/content.tex}

% DATA MANIPULATION
\input{main/chapter2/section6/content.tex}

% COMMUNICATION 
\input{main/chapter2/section7/content.tex}

% TRAINING SCENARIOS
\input{main/chapter2/section8/content.tex}

% EMG GRAPHIC
\input{main/chapter2/section9/content.tex}

% STATICAL REPORTS GENERATION
\input{main/chapter2/section10/content.tex}

\subthesischapter{Conclusiones del capítulo}
Se presentó una descripción del sistema de adquisición de datos para rehabilitación, sus componentes, características distintivas y su funcionamiento. Se identificaron y definieron los requisitos del juego  serio, tanto funcionales como no funcionales, así como los actores y casos de usos del sistema que establecieron las bases fundamentales para el desarrollo de la aplicación. Se realizó el diseño de la base de datos, abarcando tanto el modelo lógico como el físico, lo que aseguró una estructura robusta y eficiente para el almacenamiento de los datos. La manipulación de los datos se abordó de manera integral, desde la conexión con la base de datos hasta la persistencia de los resultados estadísticos. Se diseñó e implementó la comunicación con el pedal motorizado y la implementación de la interfaz gráfica para la representación de los datos EMG. Se definieron los escenarios de entrenamiento para las modalidades Ligero y Clínico, asegurando una cobertura completa de las necesidad de entrenamiento del usuario. Por último en el ámbito estadístico se desarrolló una serie de gráficos para el seguimiento de los resultados en las rutinas de entrenamiento.   
    
\end{thesischapter}

% TRAINING SCENARIOS
\begin{thesischapter}{2} {Diseño e implementación del Juego Serio}
En este capítulo se discuten los detalles de desarrollo de los aspectos citados en el capítulo anterior. Este comienza con una descripción y caracterización general del sistema, donde se  abordan cada uno de los componentes requeridos para su completo funcionamiento. Posteriormente se detalla la ingienría de software requerida en la etapa de conceptualización de la aplicación, se explican de forma detallada los aspectos teóricos y de implementación de la base de datos, el funcionamiento del protocolo de comunicación y por último los escenarios de juegos requeridos en las rutinas de entrenamiento ligero y clínico, y las estadísticas generadas por estos. Como herramienta de desarrollo se utilizó c\#.

% SYSTEM DESCRIPTION AND CHARACTERIZATION TO APPLY
\input{main/chapter2/section1/content.tex}
     
% SERIOUS GAME REQUIREMENTS
\input{main/chapter2/section2/content.tex}    

% USE CASE DEFINITION
\input{main/chapter2/section3/content.tex}

% USE CASE REALIZATION
\input{main/chapter2/section4/content.tex}

% DATABASE DESIGN 
\input{main/chapter2/section5/content.tex}

% DATA MANIPULATION
\input{main/chapter2/section6/content.tex}

% COMMUNICATION 
\input{main/chapter2/section7/content.tex}

% TRAINING SCENARIOS
\input{main/chapter2/section8/content.tex}

% EMG GRAPHIC
\input{main/chapter2/section9/content.tex}

% STATICAL REPORTS GENERATION
\input{main/chapter2/section10/content.tex}

\subthesischapter{Conclusiones del capítulo}
Se presentó una descripción del sistema de adquisición de datos para rehabilitación, sus componentes, características distintivas y su funcionamiento. Se identificaron y definieron los requisitos del juego  serio, tanto funcionales como no funcionales, así como los actores y casos de usos del sistema que establecieron las bases fundamentales para el desarrollo de la aplicación. Se realizó el diseño de la base de datos, abarcando tanto el modelo lógico como el físico, lo que aseguró una estructura robusta y eficiente para el almacenamiento de los datos. La manipulación de los datos se abordó de manera integral, desde la conexión con la base de datos hasta la persistencia de los resultados estadísticos. Se diseñó e implementó la comunicación con el pedal motorizado y la implementación de la interfaz gráfica para la representación de los datos EMG. Se definieron los escenarios de entrenamiento para las modalidades Ligero y Clínico, asegurando una cobertura completa de las necesidad de entrenamiento del usuario. Por último en el ámbito estadístico se desarrolló una serie de gráficos para el seguimiento de los resultados en las rutinas de entrenamiento.   
    
\end{thesischapter}

% EMG GRAPHIC
\begin{thesischapter}{2} {Diseño e implementación del Juego Serio}
En este capítulo se discuten los detalles de desarrollo de los aspectos citados en el capítulo anterior. Este comienza con una descripción y caracterización general del sistema, donde se  abordan cada uno de los componentes requeridos para su completo funcionamiento. Posteriormente se detalla la ingienría de software requerida en la etapa de conceptualización de la aplicación, se explican de forma detallada los aspectos teóricos y de implementación de la base de datos, el funcionamiento del protocolo de comunicación y por último los escenarios de juegos requeridos en las rutinas de entrenamiento ligero y clínico, y las estadísticas generadas por estos. Como herramienta de desarrollo se utilizó c\#.

% SYSTEM DESCRIPTION AND CHARACTERIZATION TO APPLY
\input{main/chapter2/section1/content.tex}
     
% SERIOUS GAME REQUIREMENTS
\input{main/chapter2/section2/content.tex}    

% USE CASE DEFINITION
\input{main/chapter2/section3/content.tex}

% USE CASE REALIZATION
\input{main/chapter2/section4/content.tex}

% DATABASE DESIGN 
\input{main/chapter2/section5/content.tex}

% DATA MANIPULATION
\input{main/chapter2/section6/content.tex}

% COMMUNICATION 
\input{main/chapter2/section7/content.tex}

% TRAINING SCENARIOS
\input{main/chapter2/section8/content.tex}

% EMG GRAPHIC
\input{main/chapter2/section9/content.tex}

% STATICAL REPORTS GENERATION
\input{main/chapter2/section10/content.tex}

\subthesischapter{Conclusiones del capítulo}
Se presentó una descripción del sistema de adquisición de datos para rehabilitación, sus componentes, características distintivas y su funcionamiento. Se identificaron y definieron los requisitos del juego  serio, tanto funcionales como no funcionales, así como los actores y casos de usos del sistema que establecieron las bases fundamentales para el desarrollo de la aplicación. Se realizó el diseño de la base de datos, abarcando tanto el modelo lógico como el físico, lo que aseguró una estructura robusta y eficiente para el almacenamiento de los datos. La manipulación de los datos se abordó de manera integral, desde la conexión con la base de datos hasta la persistencia de los resultados estadísticos. Se diseñó e implementó la comunicación con el pedal motorizado y la implementación de la interfaz gráfica para la representación de los datos EMG. Se definieron los escenarios de entrenamiento para las modalidades Ligero y Clínico, asegurando una cobertura completa de las necesidad de entrenamiento del usuario. Por último en el ámbito estadístico se desarrolló una serie de gráficos para el seguimiento de los resultados en las rutinas de entrenamiento.   
    
\end{thesischapter}

% STATICAL REPORTS GENERATION
\begin{thesischapter}{2} {Diseño e implementación del Juego Serio}
En este capítulo se discuten los detalles de desarrollo de los aspectos citados en el capítulo anterior. Este comienza con una descripción y caracterización general del sistema, donde se  abordan cada uno de los componentes requeridos para su completo funcionamiento. Posteriormente se detalla la ingienría de software requerida en la etapa de conceptualización de la aplicación, se explican de forma detallada los aspectos teóricos y de implementación de la base de datos, el funcionamiento del protocolo de comunicación y por último los escenarios de juegos requeridos en las rutinas de entrenamiento ligero y clínico, y las estadísticas generadas por estos. Como herramienta de desarrollo se utilizó c\#.

% SYSTEM DESCRIPTION AND CHARACTERIZATION TO APPLY
\input{main/chapter2/section1/content.tex}
     
% SERIOUS GAME REQUIREMENTS
\input{main/chapter2/section2/content.tex}    

% USE CASE DEFINITION
\input{main/chapter2/section3/content.tex}

% USE CASE REALIZATION
\input{main/chapter2/section4/content.tex}

% DATABASE DESIGN 
\input{main/chapter2/section5/content.tex}

% DATA MANIPULATION
\input{main/chapter2/section6/content.tex}

% COMMUNICATION 
\input{main/chapter2/section7/content.tex}

% TRAINING SCENARIOS
\input{main/chapter2/section8/content.tex}

% EMG GRAPHIC
\input{main/chapter2/section9/content.tex}

% STATICAL REPORTS GENERATION
\input{main/chapter2/section10/content.tex}

\subthesischapter{Conclusiones del capítulo}
Se presentó una descripción del sistema de adquisición de datos para rehabilitación, sus componentes, características distintivas y su funcionamiento. Se identificaron y definieron los requisitos del juego  serio, tanto funcionales como no funcionales, así como los actores y casos de usos del sistema que establecieron las bases fundamentales para el desarrollo de la aplicación. Se realizó el diseño de la base de datos, abarcando tanto el modelo lógico como el físico, lo que aseguró una estructura robusta y eficiente para el almacenamiento de los datos. La manipulación de los datos se abordó de manera integral, desde la conexión con la base de datos hasta la persistencia de los resultados estadísticos. Se diseñó e implementó la comunicación con el pedal motorizado y la implementación de la interfaz gráfica para la representación de los datos EMG. Se definieron los escenarios de entrenamiento para las modalidades Ligero y Clínico, asegurando una cobertura completa de las necesidad de entrenamiento del usuario. Por último en el ámbito estadístico se desarrolló una serie de gráficos para el seguimiento de los resultados en las rutinas de entrenamiento.   
    
\end{thesischapter}

\subthesischapter{Conclusiones del capítulo}
Se presentó una descripción del sistema de adquisición de datos para rehabilitación, sus componentes, características distintivas y su funcionamiento. Se identificaron y definieron los requisitos del juego  serio, tanto funcionales como no funcionales, así como los actores y casos de usos del sistema que establecieron las bases fundamentales para el desarrollo de la aplicación. Se realizó el diseño de la base de datos, abarcando tanto el modelo lógico como el físico, lo que aseguró una estructura robusta y eficiente para el almacenamiento de los datos. La manipulación de los datos se abordó de manera integral, desde la conexión con la base de datos hasta la persistencia de los resultados estadísticos. Se diseñó e implementó la comunicación con el pedal motorizado y la implementación de la interfaz gráfica para la representación de los datos EMG. Se definieron los escenarios de entrenamiento para las modalidades Ligero y Clínico, asegurando una cobertura completa de las necesidad de entrenamiento del usuario. Por último en el ámbito estadístico se desarrolló una serie de gráficos para el seguimiento de los resultados en las rutinas de entrenamiento.   
    
\end{thesischapter}

\subthesischapter{Conclusiones del capítulo}
Se presentó una descripción del sistema de adquisición de datos para rehabilitación, sus componentes, características distintivas y su funcionamiento. Se identificaron y definieron los requisitos del juego  serio, tanto funcionales como no funcionales, así como los actores y casos de usos del sistema que establecieron las bases fundamentales para el desarrollo de la aplicación. Se realizó el diseño de la base de datos, abarcando tanto el modelo lógico como el físico, lo que aseguró una estructura robusta y eficiente para el almacenamiento de los datos. La manipulación de los datos se abordó de manera integral, desde la conexión con la base de datos hasta la persistencia de los resultados estadísticos. Se diseñó e implementó la comunicación con el pedal motorizado y la implementación de la interfaz gráfica para la representación de los datos EMG. Se definieron los escenarios de entrenamiento para las modalidades Ligero y Clínico, asegurando una cobertura completa de las necesidad de entrenamiento del usuario. Por último en el ámbito estadístico se desarrolló una serie de gráficos para el seguimiento de los resultados en las rutinas de entrenamiento.   
    
\end{thesischapter}

% SECTION 2: SOCIAL IMPACT  
\begin{thesischapter}{2} {Diseño e implementación del Juego Serio}
En este capítulo se discuten los detalles de desarrollo de los aspectos citados en el capítulo anterior. Este comienza con una descripción y caracterización general del sistema, donde se  abordan cada uno de los componentes requeridos para su completo funcionamiento. Posteriormente se detalla la ingienría de software requerida en la etapa de conceptualización de la aplicación, se explican de forma detallada los aspectos teóricos y de implementación de la base de datos, el funcionamiento del protocolo de comunicación y por último los escenarios de juegos requeridos en las rutinas de entrenamiento ligero y clínico, y las estadísticas generadas por estos. Como herramienta de desarrollo se utilizó c\#.

% SYSTEM DESCRIPTION AND CHARACTERIZATION TO APPLY
\begin{thesischapter}{2} {Diseño e implementación del Juego Serio}
En este capítulo se discuten los detalles de desarrollo de los aspectos citados en el capítulo anterior. Este comienza con una descripción y caracterización general del sistema, donde se  abordan cada uno de los componentes requeridos para su completo funcionamiento. Posteriormente se detalla la ingienría de software requerida en la etapa de conceptualización de la aplicación, se explican de forma detallada los aspectos teóricos y de implementación de la base de datos, el funcionamiento del protocolo de comunicación y por último los escenarios de juegos requeridos en las rutinas de entrenamiento ligero y clínico, y las estadísticas generadas por estos. Como herramienta de desarrollo se utilizó c\#.

% SYSTEM DESCRIPTION AND CHARACTERIZATION TO APPLY
\begin{thesischapter}{2} {Diseño e implementación del Juego Serio}
En este capítulo se discuten los detalles de desarrollo de los aspectos citados en el capítulo anterior. Este comienza con una descripción y caracterización general del sistema, donde se  abordan cada uno de los componentes requeridos para su completo funcionamiento. Posteriormente se detalla la ingienría de software requerida en la etapa de conceptualización de la aplicación, se explican de forma detallada los aspectos teóricos y de implementación de la base de datos, el funcionamiento del protocolo de comunicación y por último los escenarios de juegos requeridos en las rutinas de entrenamiento ligero y clínico, y las estadísticas generadas por estos. Como herramienta de desarrollo se utilizó c\#.

% SYSTEM DESCRIPTION AND CHARACTERIZATION TO APPLY
\input{main/chapter2/section1/content.tex}
     
% SERIOUS GAME REQUIREMENTS
\input{main/chapter2/section2/content.tex}    

% USE CASE DEFINITION
\input{main/chapter2/section3/content.tex}

% USE CASE REALIZATION
\input{main/chapter2/section4/content.tex}

% DATABASE DESIGN 
\input{main/chapter2/section5/content.tex}

% DATA MANIPULATION
\input{main/chapter2/section6/content.tex}

% COMMUNICATION 
\input{main/chapter2/section7/content.tex}

% TRAINING SCENARIOS
\input{main/chapter2/section8/content.tex}

% EMG GRAPHIC
\input{main/chapter2/section9/content.tex}

% STATICAL REPORTS GENERATION
\input{main/chapter2/section10/content.tex}

\subthesischapter{Conclusiones del capítulo}
Se presentó una descripción del sistema de adquisición de datos para rehabilitación, sus componentes, características distintivas y su funcionamiento. Se identificaron y definieron los requisitos del juego  serio, tanto funcionales como no funcionales, así como los actores y casos de usos del sistema que establecieron las bases fundamentales para el desarrollo de la aplicación. Se realizó el diseño de la base de datos, abarcando tanto el modelo lógico como el físico, lo que aseguró una estructura robusta y eficiente para el almacenamiento de los datos. La manipulación de los datos se abordó de manera integral, desde la conexión con la base de datos hasta la persistencia de los resultados estadísticos. Se diseñó e implementó la comunicación con el pedal motorizado y la implementación de la interfaz gráfica para la representación de los datos EMG. Se definieron los escenarios de entrenamiento para las modalidades Ligero y Clínico, asegurando una cobertura completa de las necesidad de entrenamiento del usuario. Por último en el ámbito estadístico se desarrolló una serie de gráficos para el seguimiento de los resultados en las rutinas de entrenamiento.   
    
\end{thesischapter}
     
% SERIOUS GAME REQUIREMENTS
\begin{thesischapter}{2} {Diseño e implementación del Juego Serio}
En este capítulo se discuten los detalles de desarrollo de los aspectos citados en el capítulo anterior. Este comienza con una descripción y caracterización general del sistema, donde se  abordan cada uno de los componentes requeridos para su completo funcionamiento. Posteriormente se detalla la ingienría de software requerida en la etapa de conceptualización de la aplicación, se explican de forma detallada los aspectos teóricos y de implementación de la base de datos, el funcionamiento del protocolo de comunicación y por último los escenarios de juegos requeridos en las rutinas de entrenamiento ligero y clínico, y las estadísticas generadas por estos. Como herramienta de desarrollo se utilizó c\#.

% SYSTEM DESCRIPTION AND CHARACTERIZATION TO APPLY
\input{main/chapter2/section1/content.tex}
     
% SERIOUS GAME REQUIREMENTS
\input{main/chapter2/section2/content.tex}    

% USE CASE DEFINITION
\input{main/chapter2/section3/content.tex}

% USE CASE REALIZATION
\input{main/chapter2/section4/content.tex}

% DATABASE DESIGN 
\input{main/chapter2/section5/content.tex}

% DATA MANIPULATION
\input{main/chapter2/section6/content.tex}

% COMMUNICATION 
\input{main/chapter2/section7/content.tex}

% TRAINING SCENARIOS
\input{main/chapter2/section8/content.tex}

% EMG GRAPHIC
\input{main/chapter2/section9/content.tex}

% STATICAL REPORTS GENERATION
\input{main/chapter2/section10/content.tex}

\subthesischapter{Conclusiones del capítulo}
Se presentó una descripción del sistema de adquisición de datos para rehabilitación, sus componentes, características distintivas y su funcionamiento. Se identificaron y definieron los requisitos del juego  serio, tanto funcionales como no funcionales, así como los actores y casos de usos del sistema que establecieron las bases fundamentales para el desarrollo de la aplicación. Se realizó el diseño de la base de datos, abarcando tanto el modelo lógico como el físico, lo que aseguró una estructura robusta y eficiente para el almacenamiento de los datos. La manipulación de los datos se abordó de manera integral, desde la conexión con la base de datos hasta la persistencia de los resultados estadísticos. Se diseñó e implementó la comunicación con el pedal motorizado y la implementación de la interfaz gráfica para la representación de los datos EMG. Se definieron los escenarios de entrenamiento para las modalidades Ligero y Clínico, asegurando una cobertura completa de las necesidad de entrenamiento del usuario. Por último en el ámbito estadístico se desarrolló una serie de gráficos para el seguimiento de los resultados en las rutinas de entrenamiento.   
    
\end{thesischapter}    

% USE CASE DEFINITION
\begin{thesischapter}{2} {Diseño e implementación del Juego Serio}
En este capítulo se discuten los detalles de desarrollo de los aspectos citados en el capítulo anterior. Este comienza con una descripción y caracterización general del sistema, donde se  abordan cada uno de los componentes requeridos para su completo funcionamiento. Posteriormente se detalla la ingienría de software requerida en la etapa de conceptualización de la aplicación, se explican de forma detallada los aspectos teóricos y de implementación de la base de datos, el funcionamiento del protocolo de comunicación y por último los escenarios de juegos requeridos en las rutinas de entrenamiento ligero y clínico, y las estadísticas generadas por estos. Como herramienta de desarrollo se utilizó c\#.

% SYSTEM DESCRIPTION AND CHARACTERIZATION TO APPLY
\input{main/chapter2/section1/content.tex}
     
% SERIOUS GAME REQUIREMENTS
\input{main/chapter2/section2/content.tex}    

% USE CASE DEFINITION
\input{main/chapter2/section3/content.tex}

% USE CASE REALIZATION
\input{main/chapter2/section4/content.tex}

% DATABASE DESIGN 
\input{main/chapter2/section5/content.tex}

% DATA MANIPULATION
\input{main/chapter2/section6/content.tex}

% COMMUNICATION 
\input{main/chapter2/section7/content.tex}

% TRAINING SCENARIOS
\input{main/chapter2/section8/content.tex}

% EMG GRAPHIC
\input{main/chapter2/section9/content.tex}

% STATICAL REPORTS GENERATION
\input{main/chapter2/section10/content.tex}

\subthesischapter{Conclusiones del capítulo}
Se presentó una descripción del sistema de adquisición de datos para rehabilitación, sus componentes, características distintivas y su funcionamiento. Se identificaron y definieron los requisitos del juego  serio, tanto funcionales como no funcionales, así como los actores y casos de usos del sistema que establecieron las bases fundamentales para el desarrollo de la aplicación. Se realizó el diseño de la base de datos, abarcando tanto el modelo lógico como el físico, lo que aseguró una estructura robusta y eficiente para el almacenamiento de los datos. La manipulación de los datos se abordó de manera integral, desde la conexión con la base de datos hasta la persistencia de los resultados estadísticos. Se diseñó e implementó la comunicación con el pedal motorizado y la implementación de la interfaz gráfica para la representación de los datos EMG. Se definieron los escenarios de entrenamiento para las modalidades Ligero y Clínico, asegurando una cobertura completa de las necesidad de entrenamiento del usuario. Por último en el ámbito estadístico se desarrolló una serie de gráficos para el seguimiento de los resultados en las rutinas de entrenamiento.   
    
\end{thesischapter}

% USE CASE REALIZATION
\begin{thesischapter}{2} {Diseño e implementación del Juego Serio}
En este capítulo se discuten los detalles de desarrollo de los aspectos citados en el capítulo anterior. Este comienza con una descripción y caracterización general del sistema, donde se  abordan cada uno de los componentes requeridos para su completo funcionamiento. Posteriormente se detalla la ingienría de software requerida en la etapa de conceptualización de la aplicación, se explican de forma detallada los aspectos teóricos y de implementación de la base de datos, el funcionamiento del protocolo de comunicación y por último los escenarios de juegos requeridos en las rutinas de entrenamiento ligero y clínico, y las estadísticas generadas por estos. Como herramienta de desarrollo se utilizó c\#.

% SYSTEM DESCRIPTION AND CHARACTERIZATION TO APPLY
\input{main/chapter2/section1/content.tex}
     
% SERIOUS GAME REQUIREMENTS
\input{main/chapter2/section2/content.tex}    

% USE CASE DEFINITION
\input{main/chapter2/section3/content.tex}

% USE CASE REALIZATION
\input{main/chapter2/section4/content.tex}

% DATABASE DESIGN 
\input{main/chapter2/section5/content.tex}

% DATA MANIPULATION
\input{main/chapter2/section6/content.tex}

% COMMUNICATION 
\input{main/chapter2/section7/content.tex}

% TRAINING SCENARIOS
\input{main/chapter2/section8/content.tex}

% EMG GRAPHIC
\input{main/chapter2/section9/content.tex}

% STATICAL REPORTS GENERATION
\input{main/chapter2/section10/content.tex}

\subthesischapter{Conclusiones del capítulo}
Se presentó una descripción del sistema de adquisición de datos para rehabilitación, sus componentes, características distintivas y su funcionamiento. Se identificaron y definieron los requisitos del juego  serio, tanto funcionales como no funcionales, así como los actores y casos de usos del sistema que establecieron las bases fundamentales para el desarrollo de la aplicación. Se realizó el diseño de la base de datos, abarcando tanto el modelo lógico como el físico, lo que aseguró una estructura robusta y eficiente para el almacenamiento de los datos. La manipulación de los datos se abordó de manera integral, desde la conexión con la base de datos hasta la persistencia de los resultados estadísticos. Se diseñó e implementó la comunicación con el pedal motorizado y la implementación de la interfaz gráfica para la representación de los datos EMG. Se definieron los escenarios de entrenamiento para las modalidades Ligero y Clínico, asegurando una cobertura completa de las necesidad de entrenamiento del usuario. Por último en el ámbito estadístico se desarrolló una serie de gráficos para el seguimiento de los resultados en las rutinas de entrenamiento.   
    
\end{thesischapter}

% DATABASE DESIGN 
\begin{thesischapter}{2} {Diseño e implementación del Juego Serio}
En este capítulo se discuten los detalles de desarrollo de los aspectos citados en el capítulo anterior. Este comienza con una descripción y caracterización general del sistema, donde se  abordan cada uno de los componentes requeridos para su completo funcionamiento. Posteriormente se detalla la ingienría de software requerida en la etapa de conceptualización de la aplicación, se explican de forma detallada los aspectos teóricos y de implementación de la base de datos, el funcionamiento del protocolo de comunicación y por último los escenarios de juegos requeridos en las rutinas de entrenamiento ligero y clínico, y las estadísticas generadas por estos. Como herramienta de desarrollo se utilizó c\#.

% SYSTEM DESCRIPTION AND CHARACTERIZATION TO APPLY
\input{main/chapter2/section1/content.tex}
     
% SERIOUS GAME REQUIREMENTS
\input{main/chapter2/section2/content.tex}    

% USE CASE DEFINITION
\input{main/chapter2/section3/content.tex}

% USE CASE REALIZATION
\input{main/chapter2/section4/content.tex}

% DATABASE DESIGN 
\input{main/chapter2/section5/content.tex}

% DATA MANIPULATION
\input{main/chapter2/section6/content.tex}

% COMMUNICATION 
\input{main/chapter2/section7/content.tex}

% TRAINING SCENARIOS
\input{main/chapter2/section8/content.tex}

% EMG GRAPHIC
\input{main/chapter2/section9/content.tex}

% STATICAL REPORTS GENERATION
\input{main/chapter2/section10/content.tex}

\subthesischapter{Conclusiones del capítulo}
Se presentó una descripción del sistema de adquisición de datos para rehabilitación, sus componentes, características distintivas y su funcionamiento. Se identificaron y definieron los requisitos del juego  serio, tanto funcionales como no funcionales, así como los actores y casos de usos del sistema que establecieron las bases fundamentales para el desarrollo de la aplicación. Se realizó el diseño de la base de datos, abarcando tanto el modelo lógico como el físico, lo que aseguró una estructura robusta y eficiente para el almacenamiento de los datos. La manipulación de los datos se abordó de manera integral, desde la conexión con la base de datos hasta la persistencia de los resultados estadísticos. Se diseñó e implementó la comunicación con el pedal motorizado y la implementación de la interfaz gráfica para la representación de los datos EMG. Se definieron los escenarios de entrenamiento para las modalidades Ligero y Clínico, asegurando una cobertura completa de las necesidad de entrenamiento del usuario. Por último en el ámbito estadístico se desarrolló una serie de gráficos para el seguimiento de los resultados en las rutinas de entrenamiento.   
    
\end{thesischapter}

% DATA MANIPULATION
\begin{thesischapter}{2} {Diseño e implementación del Juego Serio}
En este capítulo se discuten los detalles de desarrollo de los aspectos citados en el capítulo anterior. Este comienza con una descripción y caracterización general del sistema, donde se  abordan cada uno de los componentes requeridos para su completo funcionamiento. Posteriormente se detalla la ingienría de software requerida en la etapa de conceptualización de la aplicación, se explican de forma detallada los aspectos teóricos y de implementación de la base de datos, el funcionamiento del protocolo de comunicación y por último los escenarios de juegos requeridos en las rutinas de entrenamiento ligero y clínico, y las estadísticas generadas por estos. Como herramienta de desarrollo se utilizó c\#.

% SYSTEM DESCRIPTION AND CHARACTERIZATION TO APPLY
\input{main/chapter2/section1/content.tex}
     
% SERIOUS GAME REQUIREMENTS
\input{main/chapter2/section2/content.tex}    

% USE CASE DEFINITION
\input{main/chapter2/section3/content.tex}

% USE CASE REALIZATION
\input{main/chapter2/section4/content.tex}

% DATABASE DESIGN 
\input{main/chapter2/section5/content.tex}

% DATA MANIPULATION
\input{main/chapter2/section6/content.tex}

% COMMUNICATION 
\input{main/chapter2/section7/content.tex}

% TRAINING SCENARIOS
\input{main/chapter2/section8/content.tex}

% EMG GRAPHIC
\input{main/chapter2/section9/content.tex}

% STATICAL REPORTS GENERATION
\input{main/chapter2/section10/content.tex}

\subthesischapter{Conclusiones del capítulo}
Se presentó una descripción del sistema de adquisición de datos para rehabilitación, sus componentes, características distintivas y su funcionamiento. Se identificaron y definieron los requisitos del juego  serio, tanto funcionales como no funcionales, así como los actores y casos de usos del sistema que establecieron las bases fundamentales para el desarrollo de la aplicación. Se realizó el diseño de la base de datos, abarcando tanto el modelo lógico como el físico, lo que aseguró una estructura robusta y eficiente para el almacenamiento de los datos. La manipulación de los datos se abordó de manera integral, desde la conexión con la base de datos hasta la persistencia de los resultados estadísticos. Se diseñó e implementó la comunicación con el pedal motorizado y la implementación de la interfaz gráfica para la representación de los datos EMG. Se definieron los escenarios de entrenamiento para las modalidades Ligero y Clínico, asegurando una cobertura completa de las necesidad de entrenamiento del usuario. Por último en el ámbito estadístico se desarrolló una serie de gráficos para el seguimiento de los resultados en las rutinas de entrenamiento.   
    
\end{thesischapter}

% COMMUNICATION 
\begin{thesischapter}{2} {Diseño e implementación del Juego Serio}
En este capítulo se discuten los detalles de desarrollo de los aspectos citados en el capítulo anterior. Este comienza con una descripción y caracterización general del sistema, donde se  abordan cada uno de los componentes requeridos para su completo funcionamiento. Posteriormente se detalla la ingienría de software requerida en la etapa de conceptualización de la aplicación, se explican de forma detallada los aspectos teóricos y de implementación de la base de datos, el funcionamiento del protocolo de comunicación y por último los escenarios de juegos requeridos en las rutinas de entrenamiento ligero y clínico, y las estadísticas generadas por estos. Como herramienta de desarrollo se utilizó c\#.

% SYSTEM DESCRIPTION AND CHARACTERIZATION TO APPLY
\input{main/chapter2/section1/content.tex}
     
% SERIOUS GAME REQUIREMENTS
\input{main/chapter2/section2/content.tex}    

% USE CASE DEFINITION
\input{main/chapter2/section3/content.tex}

% USE CASE REALIZATION
\input{main/chapter2/section4/content.tex}

% DATABASE DESIGN 
\input{main/chapter2/section5/content.tex}

% DATA MANIPULATION
\input{main/chapter2/section6/content.tex}

% COMMUNICATION 
\input{main/chapter2/section7/content.tex}

% TRAINING SCENARIOS
\input{main/chapter2/section8/content.tex}

% EMG GRAPHIC
\input{main/chapter2/section9/content.tex}

% STATICAL REPORTS GENERATION
\input{main/chapter2/section10/content.tex}

\subthesischapter{Conclusiones del capítulo}
Se presentó una descripción del sistema de adquisición de datos para rehabilitación, sus componentes, características distintivas y su funcionamiento. Se identificaron y definieron los requisitos del juego  serio, tanto funcionales como no funcionales, así como los actores y casos de usos del sistema que establecieron las bases fundamentales para el desarrollo de la aplicación. Se realizó el diseño de la base de datos, abarcando tanto el modelo lógico como el físico, lo que aseguró una estructura robusta y eficiente para el almacenamiento de los datos. La manipulación de los datos se abordó de manera integral, desde la conexión con la base de datos hasta la persistencia de los resultados estadísticos. Se diseñó e implementó la comunicación con el pedal motorizado y la implementación de la interfaz gráfica para la representación de los datos EMG. Se definieron los escenarios de entrenamiento para las modalidades Ligero y Clínico, asegurando una cobertura completa de las necesidad de entrenamiento del usuario. Por último en el ámbito estadístico se desarrolló una serie de gráficos para el seguimiento de los resultados en las rutinas de entrenamiento.   
    
\end{thesischapter}

% TRAINING SCENARIOS
\begin{thesischapter}{2} {Diseño e implementación del Juego Serio}
En este capítulo se discuten los detalles de desarrollo de los aspectos citados en el capítulo anterior. Este comienza con una descripción y caracterización general del sistema, donde se  abordan cada uno de los componentes requeridos para su completo funcionamiento. Posteriormente se detalla la ingienría de software requerida en la etapa de conceptualización de la aplicación, se explican de forma detallada los aspectos teóricos y de implementación de la base de datos, el funcionamiento del protocolo de comunicación y por último los escenarios de juegos requeridos en las rutinas de entrenamiento ligero y clínico, y las estadísticas generadas por estos. Como herramienta de desarrollo se utilizó c\#.

% SYSTEM DESCRIPTION AND CHARACTERIZATION TO APPLY
\input{main/chapter2/section1/content.tex}
     
% SERIOUS GAME REQUIREMENTS
\input{main/chapter2/section2/content.tex}    

% USE CASE DEFINITION
\input{main/chapter2/section3/content.tex}

% USE CASE REALIZATION
\input{main/chapter2/section4/content.tex}

% DATABASE DESIGN 
\input{main/chapter2/section5/content.tex}

% DATA MANIPULATION
\input{main/chapter2/section6/content.tex}

% COMMUNICATION 
\input{main/chapter2/section7/content.tex}

% TRAINING SCENARIOS
\input{main/chapter2/section8/content.tex}

% EMG GRAPHIC
\input{main/chapter2/section9/content.tex}

% STATICAL REPORTS GENERATION
\input{main/chapter2/section10/content.tex}

\subthesischapter{Conclusiones del capítulo}
Se presentó una descripción del sistema de adquisición de datos para rehabilitación, sus componentes, características distintivas y su funcionamiento. Se identificaron y definieron los requisitos del juego  serio, tanto funcionales como no funcionales, así como los actores y casos de usos del sistema que establecieron las bases fundamentales para el desarrollo de la aplicación. Se realizó el diseño de la base de datos, abarcando tanto el modelo lógico como el físico, lo que aseguró una estructura robusta y eficiente para el almacenamiento de los datos. La manipulación de los datos se abordó de manera integral, desde la conexión con la base de datos hasta la persistencia de los resultados estadísticos. Se diseñó e implementó la comunicación con el pedal motorizado y la implementación de la interfaz gráfica para la representación de los datos EMG. Se definieron los escenarios de entrenamiento para las modalidades Ligero y Clínico, asegurando una cobertura completa de las necesidad de entrenamiento del usuario. Por último en el ámbito estadístico se desarrolló una serie de gráficos para el seguimiento de los resultados en las rutinas de entrenamiento.   
    
\end{thesischapter}

% EMG GRAPHIC
\begin{thesischapter}{2} {Diseño e implementación del Juego Serio}
En este capítulo se discuten los detalles de desarrollo de los aspectos citados en el capítulo anterior. Este comienza con una descripción y caracterización general del sistema, donde se  abordan cada uno de los componentes requeridos para su completo funcionamiento. Posteriormente se detalla la ingienría de software requerida en la etapa de conceptualización de la aplicación, se explican de forma detallada los aspectos teóricos y de implementación de la base de datos, el funcionamiento del protocolo de comunicación y por último los escenarios de juegos requeridos en las rutinas de entrenamiento ligero y clínico, y las estadísticas generadas por estos. Como herramienta de desarrollo se utilizó c\#.

% SYSTEM DESCRIPTION AND CHARACTERIZATION TO APPLY
\input{main/chapter2/section1/content.tex}
     
% SERIOUS GAME REQUIREMENTS
\input{main/chapter2/section2/content.tex}    

% USE CASE DEFINITION
\input{main/chapter2/section3/content.tex}

% USE CASE REALIZATION
\input{main/chapter2/section4/content.tex}

% DATABASE DESIGN 
\input{main/chapter2/section5/content.tex}

% DATA MANIPULATION
\input{main/chapter2/section6/content.tex}

% COMMUNICATION 
\input{main/chapter2/section7/content.tex}

% TRAINING SCENARIOS
\input{main/chapter2/section8/content.tex}

% EMG GRAPHIC
\input{main/chapter2/section9/content.tex}

% STATICAL REPORTS GENERATION
\input{main/chapter2/section10/content.tex}

\subthesischapter{Conclusiones del capítulo}
Se presentó una descripción del sistema de adquisición de datos para rehabilitación, sus componentes, características distintivas y su funcionamiento. Se identificaron y definieron los requisitos del juego  serio, tanto funcionales como no funcionales, así como los actores y casos de usos del sistema que establecieron las bases fundamentales para el desarrollo de la aplicación. Se realizó el diseño de la base de datos, abarcando tanto el modelo lógico como el físico, lo que aseguró una estructura robusta y eficiente para el almacenamiento de los datos. La manipulación de los datos se abordó de manera integral, desde la conexión con la base de datos hasta la persistencia de los resultados estadísticos. Se diseñó e implementó la comunicación con el pedal motorizado y la implementación de la interfaz gráfica para la representación de los datos EMG. Se definieron los escenarios de entrenamiento para las modalidades Ligero y Clínico, asegurando una cobertura completa de las necesidad de entrenamiento del usuario. Por último en el ámbito estadístico se desarrolló una serie de gráficos para el seguimiento de los resultados en las rutinas de entrenamiento.   
    
\end{thesischapter}

% STATICAL REPORTS GENERATION
\begin{thesischapter}{2} {Diseño e implementación del Juego Serio}
En este capítulo se discuten los detalles de desarrollo de los aspectos citados en el capítulo anterior. Este comienza con una descripción y caracterización general del sistema, donde se  abordan cada uno de los componentes requeridos para su completo funcionamiento. Posteriormente se detalla la ingienría de software requerida en la etapa de conceptualización de la aplicación, se explican de forma detallada los aspectos teóricos y de implementación de la base de datos, el funcionamiento del protocolo de comunicación y por último los escenarios de juegos requeridos en las rutinas de entrenamiento ligero y clínico, y las estadísticas generadas por estos. Como herramienta de desarrollo se utilizó c\#.

% SYSTEM DESCRIPTION AND CHARACTERIZATION TO APPLY
\input{main/chapter2/section1/content.tex}
     
% SERIOUS GAME REQUIREMENTS
\input{main/chapter2/section2/content.tex}    

% USE CASE DEFINITION
\input{main/chapter2/section3/content.tex}

% USE CASE REALIZATION
\input{main/chapter2/section4/content.tex}

% DATABASE DESIGN 
\input{main/chapter2/section5/content.tex}

% DATA MANIPULATION
\input{main/chapter2/section6/content.tex}

% COMMUNICATION 
\input{main/chapter2/section7/content.tex}

% TRAINING SCENARIOS
\input{main/chapter2/section8/content.tex}

% EMG GRAPHIC
\input{main/chapter2/section9/content.tex}

% STATICAL REPORTS GENERATION
\input{main/chapter2/section10/content.tex}

\subthesischapter{Conclusiones del capítulo}
Se presentó una descripción del sistema de adquisición de datos para rehabilitación, sus componentes, características distintivas y su funcionamiento. Se identificaron y definieron los requisitos del juego  serio, tanto funcionales como no funcionales, así como los actores y casos de usos del sistema que establecieron las bases fundamentales para el desarrollo de la aplicación. Se realizó el diseño de la base de datos, abarcando tanto el modelo lógico como el físico, lo que aseguró una estructura robusta y eficiente para el almacenamiento de los datos. La manipulación de los datos se abordó de manera integral, desde la conexión con la base de datos hasta la persistencia de los resultados estadísticos. Se diseñó e implementó la comunicación con el pedal motorizado y la implementación de la interfaz gráfica para la representación de los datos EMG. Se definieron los escenarios de entrenamiento para las modalidades Ligero y Clínico, asegurando una cobertura completa de las necesidad de entrenamiento del usuario. Por último en el ámbito estadístico se desarrolló una serie de gráficos para el seguimiento de los resultados en las rutinas de entrenamiento.   
    
\end{thesischapter}

\subthesischapter{Conclusiones del capítulo}
Se presentó una descripción del sistema de adquisición de datos para rehabilitación, sus componentes, características distintivas y su funcionamiento. Se identificaron y definieron los requisitos del juego  serio, tanto funcionales como no funcionales, así como los actores y casos de usos del sistema que establecieron las bases fundamentales para el desarrollo de la aplicación. Se realizó el diseño de la base de datos, abarcando tanto el modelo lógico como el físico, lo que aseguró una estructura robusta y eficiente para el almacenamiento de los datos. La manipulación de los datos se abordó de manera integral, desde la conexión con la base de datos hasta la persistencia de los resultados estadísticos. Se diseñó e implementó la comunicación con el pedal motorizado y la implementación de la interfaz gráfica para la representación de los datos EMG. Se definieron los escenarios de entrenamiento para las modalidades Ligero y Clínico, asegurando una cobertura completa de las necesidad de entrenamiento del usuario. Por último en el ámbito estadístico se desarrolló una serie de gráficos para el seguimiento de los resultados en las rutinas de entrenamiento.   
    
\end{thesischapter}
     
% SERIOUS GAME REQUIREMENTS
\begin{thesischapter}{2} {Diseño e implementación del Juego Serio}
En este capítulo se discuten los detalles de desarrollo de los aspectos citados en el capítulo anterior. Este comienza con una descripción y caracterización general del sistema, donde se  abordan cada uno de los componentes requeridos para su completo funcionamiento. Posteriormente se detalla la ingienría de software requerida en la etapa de conceptualización de la aplicación, se explican de forma detallada los aspectos teóricos y de implementación de la base de datos, el funcionamiento del protocolo de comunicación y por último los escenarios de juegos requeridos en las rutinas de entrenamiento ligero y clínico, y las estadísticas generadas por estos. Como herramienta de desarrollo se utilizó c\#.

% SYSTEM DESCRIPTION AND CHARACTERIZATION TO APPLY
\begin{thesischapter}{2} {Diseño e implementación del Juego Serio}
En este capítulo se discuten los detalles de desarrollo de los aspectos citados en el capítulo anterior. Este comienza con una descripción y caracterización general del sistema, donde se  abordan cada uno de los componentes requeridos para su completo funcionamiento. Posteriormente se detalla la ingienría de software requerida en la etapa de conceptualización de la aplicación, se explican de forma detallada los aspectos teóricos y de implementación de la base de datos, el funcionamiento del protocolo de comunicación y por último los escenarios de juegos requeridos en las rutinas de entrenamiento ligero y clínico, y las estadísticas generadas por estos. Como herramienta de desarrollo se utilizó c\#.

% SYSTEM DESCRIPTION AND CHARACTERIZATION TO APPLY
\input{main/chapter2/section1/content.tex}
     
% SERIOUS GAME REQUIREMENTS
\input{main/chapter2/section2/content.tex}    

% USE CASE DEFINITION
\input{main/chapter2/section3/content.tex}

% USE CASE REALIZATION
\input{main/chapter2/section4/content.tex}

% DATABASE DESIGN 
\input{main/chapter2/section5/content.tex}

% DATA MANIPULATION
\input{main/chapter2/section6/content.tex}

% COMMUNICATION 
\input{main/chapter2/section7/content.tex}

% TRAINING SCENARIOS
\input{main/chapter2/section8/content.tex}

% EMG GRAPHIC
\input{main/chapter2/section9/content.tex}

% STATICAL REPORTS GENERATION
\input{main/chapter2/section10/content.tex}

\subthesischapter{Conclusiones del capítulo}
Se presentó una descripción del sistema de adquisición de datos para rehabilitación, sus componentes, características distintivas y su funcionamiento. Se identificaron y definieron los requisitos del juego  serio, tanto funcionales como no funcionales, así como los actores y casos de usos del sistema que establecieron las bases fundamentales para el desarrollo de la aplicación. Se realizó el diseño de la base de datos, abarcando tanto el modelo lógico como el físico, lo que aseguró una estructura robusta y eficiente para el almacenamiento de los datos. La manipulación de los datos se abordó de manera integral, desde la conexión con la base de datos hasta la persistencia de los resultados estadísticos. Se diseñó e implementó la comunicación con el pedal motorizado y la implementación de la interfaz gráfica para la representación de los datos EMG. Se definieron los escenarios de entrenamiento para las modalidades Ligero y Clínico, asegurando una cobertura completa de las necesidad de entrenamiento del usuario. Por último en el ámbito estadístico se desarrolló una serie de gráficos para el seguimiento de los resultados en las rutinas de entrenamiento.   
    
\end{thesischapter}
     
% SERIOUS GAME REQUIREMENTS
\begin{thesischapter}{2} {Diseño e implementación del Juego Serio}
En este capítulo se discuten los detalles de desarrollo de los aspectos citados en el capítulo anterior. Este comienza con una descripción y caracterización general del sistema, donde se  abordan cada uno de los componentes requeridos para su completo funcionamiento. Posteriormente se detalla la ingienría de software requerida en la etapa de conceptualización de la aplicación, se explican de forma detallada los aspectos teóricos y de implementación de la base de datos, el funcionamiento del protocolo de comunicación y por último los escenarios de juegos requeridos en las rutinas de entrenamiento ligero y clínico, y las estadísticas generadas por estos. Como herramienta de desarrollo se utilizó c\#.

% SYSTEM DESCRIPTION AND CHARACTERIZATION TO APPLY
\input{main/chapter2/section1/content.tex}
     
% SERIOUS GAME REQUIREMENTS
\input{main/chapter2/section2/content.tex}    

% USE CASE DEFINITION
\input{main/chapter2/section3/content.tex}

% USE CASE REALIZATION
\input{main/chapter2/section4/content.tex}

% DATABASE DESIGN 
\input{main/chapter2/section5/content.tex}

% DATA MANIPULATION
\input{main/chapter2/section6/content.tex}

% COMMUNICATION 
\input{main/chapter2/section7/content.tex}

% TRAINING SCENARIOS
\input{main/chapter2/section8/content.tex}

% EMG GRAPHIC
\input{main/chapter2/section9/content.tex}

% STATICAL REPORTS GENERATION
\input{main/chapter2/section10/content.tex}

\subthesischapter{Conclusiones del capítulo}
Se presentó una descripción del sistema de adquisición de datos para rehabilitación, sus componentes, características distintivas y su funcionamiento. Se identificaron y definieron los requisitos del juego  serio, tanto funcionales como no funcionales, así como los actores y casos de usos del sistema que establecieron las bases fundamentales para el desarrollo de la aplicación. Se realizó el diseño de la base de datos, abarcando tanto el modelo lógico como el físico, lo que aseguró una estructura robusta y eficiente para el almacenamiento de los datos. La manipulación de los datos se abordó de manera integral, desde la conexión con la base de datos hasta la persistencia de los resultados estadísticos. Se diseñó e implementó la comunicación con el pedal motorizado y la implementación de la interfaz gráfica para la representación de los datos EMG. Se definieron los escenarios de entrenamiento para las modalidades Ligero y Clínico, asegurando una cobertura completa de las necesidad de entrenamiento del usuario. Por último en el ámbito estadístico se desarrolló una serie de gráficos para el seguimiento de los resultados en las rutinas de entrenamiento.   
    
\end{thesischapter}    

% USE CASE DEFINITION
\begin{thesischapter}{2} {Diseño e implementación del Juego Serio}
En este capítulo se discuten los detalles de desarrollo de los aspectos citados en el capítulo anterior. Este comienza con una descripción y caracterización general del sistema, donde se  abordan cada uno de los componentes requeridos para su completo funcionamiento. Posteriormente se detalla la ingienría de software requerida en la etapa de conceptualización de la aplicación, se explican de forma detallada los aspectos teóricos y de implementación de la base de datos, el funcionamiento del protocolo de comunicación y por último los escenarios de juegos requeridos en las rutinas de entrenamiento ligero y clínico, y las estadísticas generadas por estos. Como herramienta de desarrollo se utilizó c\#.

% SYSTEM DESCRIPTION AND CHARACTERIZATION TO APPLY
\input{main/chapter2/section1/content.tex}
     
% SERIOUS GAME REQUIREMENTS
\input{main/chapter2/section2/content.tex}    

% USE CASE DEFINITION
\input{main/chapter2/section3/content.tex}

% USE CASE REALIZATION
\input{main/chapter2/section4/content.tex}

% DATABASE DESIGN 
\input{main/chapter2/section5/content.tex}

% DATA MANIPULATION
\input{main/chapter2/section6/content.tex}

% COMMUNICATION 
\input{main/chapter2/section7/content.tex}

% TRAINING SCENARIOS
\input{main/chapter2/section8/content.tex}

% EMG GRAPHIC
\input{main/chapter2/section9/content.tex}

% STATICAL REPORTS GENERATION
\input{main/chapter2/section10/content.tex}

\subthesischapter{Conclusiones del capítulo}
Se presentó una descripción del sistema de adquisición de datos para rehabilitación, sus componentes, características distintivas y su funcionamiento. Se identificaron y definieron los requisitos del juego  serio, tanto funcionales como no funcionales, así como los actores y casos de usos del sistema que establecieron las bases fundamentales para el desarrollo de la aplicación. Se realizó el diseño de la base de datos, abarcando tanto el modelo lógico como el físico, lo que aseguró una estructura robusta y eficiente para el almacenamiento de los datos. La manipulación de los datos se abordó de manera integral, desde la conexión con la base de datos hasta la persistencia de los resultados estadísticos. Se diseñó e implementó la comunicación con el pedal motorizado y la implementación de la interfaz gráfica para la representación de los datos EMG. Se definieron los escenarios de entrenamiento para las modalidades Ligero y Clínico, asegurando una cobertura completa de las necesidad de entrenamiento del usuario. Por último en el ámbito estadístico se desarrolló una serie de gráficos para el seguimiento de los resultados en las rutinas de entrenamiento.   
    
\end{thesischapter}

% USE CASE REALIZATION
\begin{thesischapter}{2} {Diseño e implementación del Juego Serio}
En este capítulo se discuten los detalles de desarrollo de los aspectos citados en el capítulo anterior. Este comienza con una descripción y caracterización general del sistema, donde se  abordan cada uno de los componentes requeridos para su completo funcionamiento. Posteriormente se detalla la ingienría de software requerida en la etapa de conceptualización de la aplicación, se explican de forma detallada los aspectos teóricos y de implementación de la base de datos, el funcionamiento del protocolo de comunicación y por último los escenarios de juegos requeridos en las rutinas de entrenamiento ligero y clínico, y las estadísticas generadas por estos. Como herramienta de desarrollo se utilizó c\#.

% SYSTEM DESCRIPTION AND CHARACTERIZATION TO APPLY
\input{main/chapter2/section1/content.tex}
     
% SERIOUS GAME REQUIREMENTS
\input{main/chapter2/section2/content.tex}    

% USE CASE DEFINITION
\input{main/chapter2/section3/content.tex}

% USE CASE REALIZATION
\input{main/chapter2/section4/content.tex}

% DATABASE DESIGN 
\input{main/chapter2/section5/content.tex}

% DATA MANIPULATION
\input{main/chapter2/section6/content.tex}

% COMMUNICATION 
\input{main/chapter2/section7/content.tex}

% TRAINING SCENARIOS
\input{main/chapter2/section8/content.tex}

% EMG GRAPHIC
\input{main/chapter2/section9/content.tex}

% STATICAL REPORTS GENERATION
\input{main/chapter2/section10/content.tex}

\subthesischapter{Conclusiones del capítulo}
Se presentó una descripción del sistema de adquisición de datos para rehabilitación, sus componentes, características distintivas y su funcionamiento. Se identificaron y definieron los requisitos del juego  serio, tanto funcionales como no funcionales, así como los actores y casos de usos del sistema que establecieron las bases fundamentales para el desarrollo de la aplicación. Se realizó el diseño de la base de datos, abarcando tanto el modelo lógico como el físico, lo que aseguró una estructura robusta y eficiente para el almacenamiento de los datos. La manipulación de los datos se abordó de manera integral, desde la conexión con la base de datos hasta la persistencia de los resultados estadísticos. Se diseñó e implementó la comunicación con el pedal motorizado y la implementación de la interfaz gráfica para la representación de los datos EMG. Se definieron los escenarios de entrenamiento para las modalidades Ligero y Clínico, asegurando una cobertura completa de las necesidad de entrenamiento del usuario. Por último en el ámbito estadístico se desarrolló una serie de gráficos para el seguimiento de los resultados en las rutinas de entrenamiento.   
    
\end{thesischapter}

% DATABASE DESIGN 
\begin{thesischapter}{2} {Diseño e implementación del Juego Serio}
En este capítulo se discuten los detalles de desarrollo de los aspectos citados en el capítulo anterior. Este comienza con una descripción y caracterización general del sistema, donde se  abordan cada uno de los componentes requeridos para su completo funcionamiento. Posteriormente se detalla la ingienría de software requerida en la etapa de conceptualización de la aplicación, se explican de forma detallada los aspectos teóricos y de implementación de la base de datos, el funcionamiento del protocolo de comunicación y por último los escenarios de juegos requeridos en las rutinas de entrenamiento ligero y clínico, y las estadísticas generadas por estos. Como herramienta de desarrollo se utilizó c\#.

% SYSTEM DESCRIPTION AND CHARACTERIZATION TO APPLY
\input{main/chapter2/section1/content.tex}
     
% SERIOUS GAME REQUIREMENTS
\input{main/chapter2/section2/content.tex}    

% USE CASE DEFINITION
\input{main/chapter2/section3/content.tex}

% USE CASE REALIZATION
\input{main/chapter2/section4/content.tex}

% DATABASE DESIGN 
\input{main/chapter2/section5/content.tex}

% DATA MANIPULATION
\input{main/chapter2/section6/content.tex}

% COMMUNICATION 
\input{main/chapter2/section7/content.tex}

% TRAINING SCENARIOS
\input{main/chapter2/section8/content.tex}

% EMG GRAPHIC
\input{main/chapter2/section9/content.tex}

% STATICAL REPORTS GENERATION
\input{main/chapter2/section10/content.tex}

\subthesischapter{Conclusiones del capítulo}
Se presentó una descripción del sistema de adquisición de datos para rehabilitación, sus componentes, características distintivas y su funcionamiento. Se identificaron y definieron los requisitos del juego  serio, tanto funcionales como no funcionales, así como los actores y casos de usos del sistema que establecieron las bases fundamentales para el desarrollo de la aplicación. Se realizó el diseño de la base de datos, abarcando tanto el modelo lógico como el físico, lo que aseguró una estructura robusta y eficiente para el almacenamiento de los datos. La manipulación de los datos se abordó de manera integral, desde la conexión con la base de datos hasta la persistencia de los resultados estadísticos. Se diseñó e implementó la comunicación con el pedal motorizado y la implementación de la interfaz gráfica para la representación de los datos EMG. Se definieron los escenarios de entrenamiento para las modalidades Ligero y Clínico, asegurando una cobertura completa de las necesidad de entrenamiento del usuario. Por último en el ámbito estadístico se desarrolló una serie de gráficos para el seguimiento de los resultados en las rutinas de entrenamiento.   
    
\end{thesischapter}

% DATA MANIPULATION
\begin{thesischapter}{2} {Diseño e implementación del Juego Serio}
En este capítulo se discuten los detalles de desarrollo de los aspectos citados en el capítulo anterior. Este comienza con una descripción y caracterización general del sistema, donde se  abordan cada uno de los componentes requeridos para su completo funcionamiento. Posteriormente se detalla la ingienría de software requerida en la etapa de conceptualización de la aplicación, se explican de forma detallada los aspectos teóricos y de implementación de la base de datos, el funcionamiento del protocolo de comunicación y por último los escenarios de juegos requeridos en las rutinas de entrenamiento ligero y clínico, y las estadísticas generadas por estos. Como herramienta de desarrollo se utilizó c\#.

% SYSTEM DESCRIPTION AND CHARACTERIZATION TO APPLY
\input{main/chapter2/section1/content.tex}
     
% SERIOUS GAME REQUIREMENTS
\input{main/chapter2/section2/content.tex}    

% USE CASE DEFINITION
\input{main/chapter2/section3/content.tex}

% USE CASE REALIZATION
\input{main/chapter2/section4/content.tex}

% DATABASE DESIGN 
\input{main/chapter2/section5/content.tex}

% DATA MANIPULATION
\input{main/chapter2/section6/content.tex}

% COMMUNICATION 
\input{main/chapter2/section7/content.tex}

% TRAINING SCENARIOS
\input{main/chapter2/section8/content.tex}

% EMG GRAPHIC
\input{main/chapter2/section9/content.tex}

% STATICAL REPORTS GENERATION
\input{main/chapter2/section10/content.tex}

\subthesischapter{Conclusiones del capítulo}
Se presentó una descripción del sistema de adquisición de datos para rehabilitación, sus componentes, características distintivas y su funcionamiento. Se identificaron y definieron los requisitos del juego  serio, tanto funcionales como no funcionales, así como los actores y casos de usos del sistema que establecieron las bases fundamentales para el desarrollo de la aplicación. Se realizó el diseño de la base de datos, abarcando tanto el modelo lógico como el físico, lo que aseguró una estructura robusta y eficiente para el almacenamiento de los datos. La manipulación de los datos se abordó de manera integral, desde la conexión con la base de datos hasta la persistencia de los resultados estadísticos. Se diseñó e implementó la comunicación con el pedal motorizado y la implementación de la interfaz gráfica para la representación de los datos EMG. Se definieron los escenarios de entrenamiento para las modalidades Ligero y Clínico, asegurando una cobertura completa de las necesidad de entrenamiento del usuario. Por último en el ámbito estadístico se desarrolló una serie de gráficos para el seguimiento de los resultados en las rutinas de entrenamiento.   
    
\end{thesischapter}

% COMMUNICATION 
\begin{thesischapter}{2} {Diseño e implementación del Juego Serio}
En este capítulo se discuten los detalles de desarrollo de los aspectos citados en el capítulo anterior. Este comienza con una descripción y caracterización general del sistema, donde se  abordan cada uno de los componentes requeridos para su completo funcionamiento. Posteriormente se detalla la ingienría de software requerida en la etapa de conceptualización de la aplicación, se explican de forma detallada los aspectos teóricos y de implementación de la base de datos, el funcionamiento del protocolo de comunicación y por último los escenarios de juegos requeridos en las rutinas de entrenamiento ligero y clínico, y las estadísticas generadas por estos. Como herramienta de desarrollo se utilizó c\#.

% SYSTEM DESCRIPTION AND CHARACTERIZATION TO APPLY
\input{main/chapter2/section1/content.tex}
     
% SERIOUS GAME REQUIREMENTS
\input{main/chapter2/section2/content.tex}    

% USE CASE DEFINITION
\input{main/chapter2/section3/content.tex}

% USE CASE REALIZATION
\input{main/chapter2/section4/content.tex}

% DATABASE DESIGN 
\input{main/chapter2/section5/content.tex}

% DATA MANIPULATION
\input{main/chapter2/section6/content.tex}

% COMMUNICATION 
\input{main/chapter2/section7/content.tex}

% TRAINING SCENARIOS
\input{main/chapter2/section8/content.tex}

% EMG GRAPHIC
\input{main/chapter2/section9/content.tex}

% STATICAL REPORTS GENERATION
\input{main/chapter2/section10/content.tex}

\subthesischapter{Conclusiones del capítulo}
Se presentó una descripción del sistema de adquisición de datos para rehabilitación, sus componentes, características distintivas y su funcionamiento. Se identificaron y definieron los requisitos del juego  serio, tanto funcionales como no funcionales, así como los actores y casos de usos del sistema que establecieron las bases fundamentales para el desarrollo de la aplicación. Se realizó el diseño de la base de datos, abarcando tanto el modelo lógico como el físico, lo que aseguró una estructura robusta y eficiente para el almacenamiento de los datos. La manipulación de los datos se abordó de manera integral, desde la conexión con la base de datos hasta la persistencia de los resultados estadísticos. Se diseñó e implementó la comunicación con el pedal motorizado y la implementación de la interfaz gráfica para la representación de los datos EMG. Se definieron los escenarios de entrenamiento para las modalidades Ligero y Clínico, asegurando una cobertura completa de las necesidad de entrenamiento del usuario. Por último en el ámbito estadístico se desarrolló una serie de gráficos para el seguimiento de los resultados en las rutinas de entrenamiento.   
    
\end{thesischapter}

% TRAINING SCENARIOS
\begin{thesischapter}{2} {Diseño e implementación del Juego Serio}
En este capítulo se discuten los detalles de desarrollo de los aspectos citados en el capítulo anterior. Este comienza con una descripción y caracterización general del sistema, donde se  abordan cada uno de los componentes requeridos para su completo funcionamiento. Posteriormente se detalla la ingienría de software requerida en la etapa de conceptualización de la aplicación, se explican de forma detallada los aspectos teóricos y de implementación de la base de datos, el funcionamiento del protocolo de comunicación y por último los escenarios de juegos requeridos en las rutinas de entrenamiento ligero y clínico, y las estadísticas generadas por estos. Como herramienta de desarrollo se utilizó c\#.

% SYSTEM DESCRIPTION AND CHARACTERIZATION TO APPLY
\input{main/chapter2/section1/content.tex}
     
% SERIOUS GAME REQUIREMENTS
\input{main/chapter2/section2/content.tex}    

% USE CASE DEFINITION
\input{main/chapter2/section3/content.tex}

% USE CASE REALIZATION
\input{main/chapter2/section4/content.tex}

% DATABASE DESIGN 
\input{main/chapter2/section5/content.tex}

% DATA MANIPULATION
\input{main/chapter2/section6/content.tex}

% COMMUNICATION 
\input{main/chapter2/section7/content.tex}

% TRAINING SCENARIOS
\input{main/chapter2/section8/content.tex}

% EMG GRAPHIC
\input{main/chapter2/section9/content.tex}

% STATICAL REPORTS GENERATION
\input{main/chapter2/section10/content.tex}

\subthesischapter{Conclusiones del capítulo}
Se presentó una descripción del sistema de adquisición de datos para rehabilitación, sus componentes, características distintivas y su funcionamiento. Se identificaron y definieron los requisitos del juego  serio, tanto funcionales como no funcionales, así como los actores y casos de usos del sistema que establecieron las bases fundamentales para el desarrollo de la aplicación. Se realizó el diseño de la base de datos, abarcando tanto el modelo lógico como el físico, lo que aseguró una estructura robusta y eficiente para el almacenamiento de los datos. La manipulación de los datos se abordó de manera integral, desde la conexión con la base de datos hasta la persistencia de los resultados estadísticos. Se diseñó e implementó la comunicación con el pedal motorizado y la implementación de la interfaz gráfica para la representación de los datos EMG. Se definieron los escenarios de entrenamiento para las modalidades Ligero y Clínico, asegurando una cobertura completa de las necesidad de entrenamiento del usuario. Por último en el ámbito estadístico se desarrolló una serie de gráficos para el seguimiento de los resultados en las rutinas de entrenamiento.   
    
\end{thesischapter}

% EMG GRAPHIC
\begin{thesischapter}{2} {Diseño e implementación del Juego Serio}
En este capítulo se discuten los detalles de desarrollo de los aspectos citados en el capítulo anterior. Este comienza con una descripción y caracterización general del sistema, donde se  abordan cada uno de los componentes requeridos para su completo funcionamiento. Posteriormente se detalla la ingienría de software requerida en la etapa de conceptualización de la aplicación, se explican de forma detallada los aspectos teóricos y de implementación de la base de datos, el funcionamiento del protocolo de comunicación y por último los escenarios de juegos requeridos en las rutinas de entrenamiento ligero y clínico, y las estadísticas generadas por estos. Como herramienta de desarrollo se utilizó c\#.

% SYSTEM DESCRIPTION AND CHARACTERIZATION TO APPLY
\input{main/chapter2/section1/content.tex}
     
% SERIOUS GAME REQUIREMENTS
\input{main/chapter2/section2/content.tex}    

% USE CASE DEFINITION
\input{main/chapter2/section3/content.tex}

% USE CASE REALIZATION
\input{main/chapter2/section4/content.tex}

% DATABASE DESIGN 
\input{main/chapter2/section5/content.tex}

% DATA MANIPULATION
\input{main/chapter2/section6/content.tex}

% COMMUNICATION 
\input{main/chapter2/section7/content.tex}

% TRAINING SCENARIOS
\input{main/chapter2/section8/content.tex}

% EMG GRAPHIC
\input{main/chapter2/section9/content.tex}

% STATICAL REPORTS GENERATION
\input{main/chapter2/section10/content.tex}

\subthesischapter{Conclusiones del capítulo}
Se presentó una descripción del sistema de adquisición de datos para rehabilitación, sus componentes, características distintivas y su funcionamiento. Se identificaron y definieron los requisitos del juego  serio, tanto funcionales como no funcionales, así como los actores y casos de usos del sistema que establecieron las bases fundamentales para el desarrollo de la aplicación. Se realizó el diseño de la base de datos, abarcando tanto el modelo lógico como el físico, lo que aseguró una estructura robusta y eficiente para el almacenamiento de los datos. La manipulación de los datos se abordó de manera integral, desde la conexión con la base de datos hasta la persistencia de los resultados estadísticos. Se diseñó e implementó la comunicación con el pedal motorizado y la implementación de la interfaz gráfica para la representación de los datos EMG. Se definieron los escenarios de entrenamiento para las modalidades Ligero y Clínico, asegurando una cobertura completa de las necesidad de entrenamiento del usuario. Por último en el ámbito estadístico se desarrolló una serie de gráficos para el seguimiento de los resultados en las rutinas de entrenamiento.   
    
\end{thesischapter}

% STATICAL REPORTS GENERATION
\begin{thesischapter}{2} {Diseño e implementación del Juego Serio}
En este capítulo se discuten los detalles de desarrollo de los aspectos citados en el capítulo anterior. Este comienza con una descripción y caracterización general del sistema, donde se  abordan cada uno de los componentes requeridos para su completo funcionamiento. Posteriormente se detalla la ingienría de software requerida en la etapa de conceptualización de la aplicación, se explican de forma detallada los aspectos teóricos y de implementación de la base de datos, el funcionamiento del protocolo de comunicación y por último los escenarios de juegos requeridos en las rutinas de entrenamiento ligero y clínico, y las estadísticas generadas por estos. Como herramienta de desarrollo se utilizó c\#.

% SYSTEM DESCRIPTION AND CHARACTERIZATION TO APPLY
\input{main/chapter2/section1/content.tex}
     
% SERIOUS GAME REQUIREMENTS
\input{main/chapter2/section2/content.tex}    

% USE CASE DEFINITION
\input{main/chapter2/section3/content.tex}

% USE CASE REALIZATION
\input{main/chapter2/section4/content.tex}

% DATABASE DESIGN 
\input{main/chapter2/section5/content.tex}

% DATA MANIPULATION
\input{main/chapter2/section6/content.tex}

% COMMUNICATION 
\input{main/chapter2/section7/content.tex}

% TRAINING SCENARIOS
\input{main/chapter2/section8/content.tex}

% EMG GRAPHIC
\input{main/chapter2/section9/content.tex}

% STATICAL REPORTS GENERATION
\input{main/chapter2/section10/content.tex}

\subthesischapter{Conclusiones del capítulo}
Se presentó una descripción del sistema de adquisición de datos para rehabilitación, sus componentes, características distintivas y su funcionamiento. Se identificaron y definieron los requisitos del juego  serio, tanto funcionales como no funcionales, así como los actores y casos de usos del sistema que establecieron las bases fundamentales para el desarrollo de la aplicación. Se realizó el diseño de la base de datos, abarcando tanto el modelo lógico como el físico, lo que aseguró una estructura robusta y eficiente para el almacenamiento de los datos. La manipulación de los datos se abordó de manera integral, desde la conexión con la base de datos hasta la persistencia de los resultados estadísticos. Se diseñó e implementó la comunicación con el pedal motorizado y la implementación de la interfaz gráfica para la representación de los datos EMG. Se definieron los escenarios de entrenamiento para las modalidades Ligero y Clínico, asegurando una cobertura completa de las necesidad de entrenamiento del usuario. Por último en el ámbito estadístico se desarrolló una serie de gráficos para el seguimiento de los resultados en las rutinas de entrenamiento.   
    
\end{thesischapter}

\subthesischapter{Conclusiones del capítulo}
Se presentó una descripción del sistema de adquisición de datos para rehabilitación, sus componentes, características distintivas y su funcionamiento. Se identificaron y definieron los requisitos del juego  serio, tanto funcionales como no funcionales, así como los actores y casos de usos del sistema que establecieron las bases fundamentales para el desarrollo de la aplicación. Se realizó el diseño de la base de datos, abarcando tanto el modelo lógico como el físico, lo que aseguró una estructura robusta y eficiente para el almacenamiento de los datos. La manipulación de los datos se abordó de manera integral, desde la conexión con la base de datos hasta la persistencia de los resultados estadísticos. Se diseñó e implementó la comunicación con el pedal motorizado y la implementación de la interfaz gráfica para la representación de los datos EMG. Se definieron los escenarios de entrenamiento para las modalidades Ligero y Clínico, asegurando una cobertura completa de las necesidad de entrenamiento del usuario. Por último en el ámbito estadístico se desarrolló una serie de gráficos para el seguimiento de los resultados en las rutinas de entrenamiento.   
    
\end{thesischapter}    

% USE CASE DEFINITION
\begin{thesischapter}{2} {Diseño e implementación del Juego Serio}
En este capítulo se discuten los detalles de desarrollo de los aspectos citados en el capítulo anterior. Este comienza con una descripción y caracterización general del sistema, donde se  abordan cada uno de los componentes requeridos para su completo funcionamiento. Posteriormente se detalla la ingienría de software requerida en la etapa de conceptualización de la aplicación, se explican de forma detallada los aspectos teóricos y de implementación de la base de datos, el funcionamiento del protocolo de comunicación y por último los escenarios de juegos requeridos en las rutinas de entrenamiento ligero y clínico, y las estadísticas generadas por estos. Como herramienta de desarrollo se utilizó c\#.

% SYSTEM DESCRIPTION AND CHARACTERIZATION TO APPLY
\begin{thesischapter}{2} {Diseño e implementación del Juego Serio}
En este capítulo se discuten los detalles de desarrollo de los aspectos citados en el capítulo anterior. Este comienza con una descripción y caracterización general del sistema, donde se  abordan cada uno de los componentes requeridos para su completo funcionamiento. Posteriormente se detalla la ingienría de software requerida en la etapa de conceptualización de la aplicación, se explican de forma detallada los aspectos teóricos y de implementación de la base de datos, el funcionamiento del protocolo de comunicación y por último los escenarios de juegos requeridos en las rutinas de entrenamiento ligero y clínico, y las estadísticas generadas por estos. Como herramienta de desarrollo se utilizó c\#.

% SYSTEM DESCRIPTION AND CHARACTERIZATION TO APPLY
\input{main/chapter2/section1/content.tex}
     
% SERIOUS GAME REQUIREMENTS
\input{main/chapter2/section2/content.tex}    

% USE CASE DEFINITION
\input{main/chapter2/section3/content.tex}

% USE CASE REALIZATION
\input{main/chapter2/section4/content.tex}

% DATABASE DESIGN 
\input{main/chapter2/section5/content.tex}

% DATA MANIPULATION
\input{main/chapter2/section6/content.tex}

% COMMUNICATION 
\input{main/chapter2/section7/content.tex}

% TRAINING SCENARIOS
\input{main/chapter2/section8/content.tex}

% EMG GRAPHIC
\input{main/chapter2/section9/content.tex}

% STATICAL REPORTS GENERATION
\input{main/chapter2/section10/content.tex}

\subthesischapter{Conclusiones del capítulo}
Se presentó una descripción del sistema de adquisición de datos para rehabilitación, sus componentes, características distintivas y su funcionamiento. Se identificaron y definieron los requisitos del juego  serio, tanto funcionales como no funcionales, así como los actores y casos de usos del sistema que establecieron las bases fundamentales para el desarrollo de la aplicación. Se realizó el diseño de la base de datos, abarcando tanto el modelo lógico como el físico, lo que aseguró una estructura robusta y eficiente para el almacenamiento de los datos. La manipulación de los datos se abordó de manera integral, desde la conexión con la base de datos hasta la persistencia de los resultados estadísticos. Se diseñó e implementó la comunicación con el pedal motorizado y la implementación de la interfaz gráfica para la representación de los datos EMG. Se definieron los escenarios de entrenamiento para las modalidades Ligero y Clínico, asegurando una cobertura completa de las necesidad de entrenamiento del usuario. Por último en el ámbito estadístico se desarrolló una serie de gráficos para el seguimiento de los resultados en las rutinas de entrenamiento.   
    
\end{thesischapter}
     
% SERIOUS GAME REQUIREMENTS
\begin{thesischapter}{2} {Diseño e implementación del Juego Serio}
En este capítulo se discuten los detalles de desarrollo de los aspectos citados en el capítulo anterior. Este comienza con una descripción y caracterización general del sistema, donde se  abordan cada uno de los componentes requeridos para su completo funcionamiento. Posteriormente se detalla la ingienría de software requerida en la etapa de conceptualización de la aplicación, se explican de forma detallada los aspectos teóricos y de implementación de la base de datos, el funcionamiento del protocolo de comunicación y por último los escenarios de juegos requeridos en las rutinas de entrenamiento ligero y clínico, y las estadísticas generadas por estos. Como herramienta de desarrollo se utilizó c\#.

% SYSTEM DESCRIPTION AND CHARACTERIZATION TO APPLY
\input{main/chapter2/section1/content.tex}
     
% SERIOUS GAME REQUIREMENTS
\input{main/chapter2/section2/content.tex}    

% USE CASE DEFINITION
\input{main/chapter2/section3/content.tex}

% USE CASE REALIZATION
\input{main/chapter2/section4/content.tex}

% DATABASE DESIGN 
\input{main/chapter2/section5/content.tex}

% DATA MANIPULATION
\input{main/chapter2/section6/content.tex}

% COMMUNICATION 
\input{main/chapter2/section7/content.tex}

% TRAINING SCENARIOS
\input{main/chapter2/section8/content.tex}

% EMG GRAPHIC
\input{main/chapter2/section9/content.tex}

% STATICAL REPORTS GENERATION
\input{main/chapter2/section10/content.tex}

\subthesischapter{Conclusiones del capítulo}
Se presentó una descripción del sistema de adquisición de datos para rehabilitación, sus componentes, características distintivas y su funcionamiento. Se identificaron y definieron los requisitos del juego  serio, tanto funcionales como no funcionales, así como los actores y casos de usos del sistema que establecieron las bases fundamentales para el desarrollo de la aplicación. Se realizó el diseño de la base de datos, abarcando tanto el modelo lógico como el físico, lo que aseguró una estructura robusta y eficiente para el almacenamiento de los datos. La manipulación de los datos se abordó de manera integral, desde la conexión con la base de datos hasta la persistencia de los resultados estadísticos. Se diseñó e implementó la comunicación con el pedal motorizado y la implementación de la interfaz gráfica para la representación de los datos EMG. Se definieron los escenarios de entrenamiento para las modalidades Ligero y Clínico, asegurando una cobertura completa de las necesidad de entrenamiento del usuario. Por último en el ámbito estadístico se desarrolló una serie de gráficos para el seguimiento de los resultados en las rutinas de entrenamiento.   
    
\end{thesischapter}    

% USE CASE DEFINITION
\begin{thesischapter}{2} {Diseño e implementación del Juego Serio}
En este capítulo se discuten los detalles de desarrollo de los aspectos citados en el capítulo anterior. Este comienza con una descripción y caracterización general del sistema, donde se  abordan cada uno de los componentes requeridos para su completo funcionamiento. Posteriormente se detalla la ingienría de software requerida en la etapa de conceptualización de la aplicación, se explican de forma detallada los aspectos teóricos y de implementación de la base de datos, el funcionamiento del protocolo de comunicación y por último los escenarios de juegos requeridos en las rutinas de entrenamiento ligero y clínico, y las estadísticas generadas por estos. Como herramienta de desarrollo se utilizó c\#.

% SYSTEM DESCRIPTION AND CHARACTERIZATION TO APPLY
\input{main/chapter2/section1/content.tex}
     
% SERIOUS GAME REQUIREMENTS
\input{main/chapter2/section2/content.tex}    

% USE CASE DEFINITION
\input{main/chapter2/section3/content.tex}

% USE CASE REALIZATION
\input{main/chapter2/section4/content.tex}

% DATABASE DESIGN 
\input{main/chapter2/section5/content.tex}

% DATA MANIPULATION
\input{main/chapter2/section6/content.tex}

% COMMUNICATION 
\input{main/chapter2/section7/content.tex}

% TRAINING SCENARIOS
\input{main/chapter2/section8/content.tex}

% EMG GRAPHIC
\input{main/chapter2/section9/content.tex}

% STATICAL REPORTS GENERATION
\input{main/chapter2/section10/content.tex}

\subthesischapter{Conclusiones del capítulo}
Se presentó una descripción del sistema de adquisición de datos para rehabilitación, sus componentes, características distintivas y su funcionamiento. Se identificaron y definieron los requisitos del juego  serio, tanto funcionales como no funcionales, así como los actores y casos de usos del sistema que establecieron las bases fundamentales para el desarrollo de la aplicación. Se realizó el diseño de la base de datos, abarcando tanto el modelo lógico como el físico, lo que aseguró una estructura robusta y eficiente para el almacenamiento de los datos. La manipulación de los datos se abordó de manera integral, desde la conexión con la base de datos hasta la persistencia de los resultados estadísticos. Se diseñó e implementó la comunicación con el pedal motorizado y la implementación de la interfaz gráfica para la representación de los datos EMG. Se definieron los escenarios de entrenamiento para las modalidades Ligero y Clínico, asegurando una cobertura completa de las necesidad de entrenamiento del usuario. Por último en el ámbito estadístico se desarrolló una serie de gráficos para el seguimiento de los resultados en las rutinas de entrenamiento.   
    
\end{thesischapter}

% USE CASE REALIZATION
\begin{thesischapter}{2} {Diseño e implementación del Juego Serio}
En este capítulo se discuten los detalles de desarrollo de los aspectos citados en el capítulo anterior. Este comienza con una descripción y caracterización general del sistema, donde se  abordan cada uno de los componentes requeridos para su completo funcionamiento. Posteriormente se detalla la ingienría de software requerida en la etapa de conceptualización de la aplicación, se explican de forma detallada los aspectos teóricos y de implementación de la base de datos, el funcionamiento del protocolo de comunicación y por último los escenarios de juegos requeridos en las rutinas de entrenamiento ligero y clínico, y las estadísticas generadas por estos. Como herramienta de desarrollo se utilizó c\#.

% SYSTEM DESCRIPTION AND CHARACTERIZATION TO APPLY
\input{main/chapter2/section1/content.tex}
     
% SERIOUS GAME REQUIREMENTS
\input{main/chapter2/section2/content.tex}    

% USE CASE DEFINITION
\input{main/chapter2/section3/content.tex}

% USE CASE REALIZATION
\input{main/chapter2/section4/content.tex}

% DATABASE DESIGN 
\input{main/chapter2/section5/content.tex}

% DATA MANIPULATION
\input{main/chapter2/section6/content.tex}

% COMMUNICATION 
\input{main/chapter2/section7/content.tex}

% TRAINING SCENARIOS
\input{main/chapter2/section8/content.tex}

% EMG GRAPHIC
\input{main/chapter2/section9/content.tex}

% STATICAL REPORTS GENERATION
\input{main/chapter2/section10/content.tex}

\subthesischapter{Conclusiones del capítulo}
Se presentó una descripción del sistema de adquisición de datos para rehabilitación, sus componentes, características distintivas y su funcionamiento. Se identificaron y definieron los requisitos del juego  serio, tanto funcionales como no funcionales, así como los actores y casos de usos del sistema que establecieron las bases fundamentales para el desarrollo de la aplicación. Se realizó el diseño de la base de datos, abarcando tanto el modelo lógico como el físico, lo que aseguró una estructura robusta y eficiente para el almacenamiento de los datos. La manipulación de los datos se abordó de manera integral, desde la conexión con la base de datos hasta la persistencia de los resultados estadísticos. Se diseñó e implementó la comunicación con el pedal motorizado y la implementación de la interfaz gráfica para la representación de los datos EMG. Se definieron los escenarios de entrenamiento para las modalidades Ligero y Clínico, asegurando una cobertura completa de las necesidad de entrenamiento del usuario. Por último en el ámbito estadístico se desarrolló una serie de gráficos para el seguimiento de los resultados en las rutinas de entrenamiento.   
    
\end{thesischapter}

% DATABASE DESIGN 
\begin{thesischapter}{2} {Diseño e implementación del Juego Serio}
En este capítulo se discuten los detalles de desarrollo de los aspectos citados en el capítulo anterior. Este comienza con una descripción y caracterización general del sistema, donde se  abordan cada uno de los componentes requeridos para su completo funcionamiento. Posteriormente se detalla la ingienría de software requerida en la etapa de conceptualización de la aplicación, se explican de forma detallada los aspectos teóricos y de implementación de la base de datos, el funcionamiento del protocolo de comunicación y por último los escenarios de juegos requeridos en las rutinas de entrenamiento ligero y clínico, y las estadísticas generadas por estos. Como herramienta de desarrollo se utilizó c\#.

% SYSTEM DESCRIPTION AND CHARACTERIZATION TO APPLY
\input{main/chapter2/section1/content.tex}
     
% SERIOUS GAME REQUIREMENTS
\input{main/chapter2/section2/content.tex}    

% USE CASE DEFINITION
\input{main/chapter2/section3/content.tex}

% USE CASE REALIZATION
\input{main/chapter2/section4/content.tex}

% DATABASE DESIGN 
\input{main/chapter2/section5/content.tex}

% DATA MANIPULATION
\input{main/chapter2/section6/content.tex}

% COMMUNICATION 
\input{main/chapter2/section7/content.tex}

% TRAINING SCENARIOS
\input{main/chapter2/section8/content.tex}

% EMG GRAPHIC
\input{main/chapter2/section9/content.tex}

% STATICAL REPORTS GENERATION
\input{main/chapter2/section10/content.tex}

\subthesischapter{Conclusiones del capítulo}
Se presentó una descripción del sistema de adquisición de datos para rehabilitación, sus componentes, características distintivas y su funcionamiento. Se identificaron y definieron los requisitos del juego  serio, tanto funcionales como no funcionales, así como los actores y casos de usos del sistema que establecieron las bases fundamentales para el desarrollo de la aplicación. Se realizó el diseño de la base de datos, abarcando tanto el modelo lógico como el físico, lo que aseguró una estructura robusta y eficiente para el almacenamiento de los datos. La manipulación de los datos se abordó de manera integral, desde la conexión con la base de datos hasta la persistencia de los resultados estadísticos. Se diseñó e implementó la comunicación con el pedal motorizado y la implementación de la interfaz gráfica para la representación de los datos EMG. Se definieron los escenarios de entrenamiento para las modalidades Ligero y Clínico, asegurando una cobertura completa de las necesidad de entrenamiento del usuario. Por último en el ámbito estadístico se desarrolló una serie de gráficos para el seguimiento de los resultados en las rutinas de entrenamiento.   
    
\end{thesischapter}

% DATA MANIPULATION
\begin{thesischapter}{2} {Diseño e implementación del Juego Serio}
En este capítulo se discuten los detalles de desarrollo de los aspectos citados en el capítulo anterior. Este comienza con una descripción y caracterización general del sistema, donde se  abordan cada uno de los componentes requeridos para su completo funcionamiento. Posteriormente se detalla la ingienría de software requerida en la etapa de conceptualización de la aplicación, se explican de forma detallada los aspectos teóricos y de implementación de la base de datos, el funcionamiento del protocolo de comunicación y por último los escenarios de juegos requeridos en las rutinas de entrenamiento ligero y clínico, y las estadísticas generadas por estos. Como herramienta de desarrollo se utilizó c\#.

% SYSTEM DESCRIPTION AND CHARACTERIZATION TO APPLY
\input{main/chapter2/section1/content.tex}
     
% SERIOUS GAME REQUIREMENTS
\input{main/chapter2/section2/content.tex}    

% USE CASE DEFINITION
\input{main/chapter2/section3/content.tex}

% USE CASE REALIZATION
\input{main/chapter2/section4/content.tex}

% DATABASE DESIGN 
\input{main/chapter2/section5/content.tex}

% DATA MANIPULATION
\input{main/chapter2/section6/content.tex}

% COMMUNICATION 
\input{main/chapter2/section7/content.tex}

% TRAINING SCENARIOS
\input{main/chapter2/section8/content.tex}

% EMG GRAPHIC
\input{main/chapter2/section9/content.tex}

% STATICAL REPORTS GENERATION
\input{main/chapter2/section10/content.tex}

\subthesischapter{Conclusiones del capítulo}
Se presentó una descripción del sistema de adquisición de datos para rehabilitación, sus componentes, características distintivas y su funcionamiento. Se identificaron y definieron los requisitos del juego  serio, tanto funcionales como no funcionales, así como los actores y casos de usos del sistema que establecieron las bases fundamentales para el desarrollo de la aplicación. Se realizó el diseño de la base de datos, abarcando tanto el modelo lógico como el físico, lo que aseguró una estructura robusta y eficiente para el almacenamiento de los datos. La manipulación de los datos se abordó de manera integral, desde la conexión con la base de datos hasta la persistencia de los resultados estadísticos. Se diseñó e implementó la comunicación con el pedal motorizado y la implementación de la interfaz gráfica para la representación de los datos EMG. Se definieron los escenarios de entrenamiento para las modalidades Ligero y Clínico, asegurando una cobertura completa de las necesidad de entrenamiento del usuario. Por último en el ámbito estadístico se desarrolló una serie de gráficos para el seguimiento de los resultados en las rutinas de entrenamiento.   
    
\end{thesischapter}

% COMMUNICATION 
\begin{thesischapter}{2} {Diseño e implementación del Juego Serio}
En este capítulo se discuten los detalles de desarrollo de los aspectos citados en el capítulo anterior. Este comienza con una descripción y caracterización general del sistema, donde se  abordan cada uno de los componentes requeridos para su completo funcionamiento. Posteriormente se detalla la ingienría de software requerida en la etapa de conceptualización de la aplicación, se explican de forma detallada los aspectos teóricos y de implementación de la base de datos, el funcionamiento del protocolo de comunicación y por último los escenarios de juegos requeridos en las rutinas de entrenamiento ligero y clínico, y las estadísticas generadas por estos. Como herramienta de desarrollo se utilizó c\#.

% SYSTEM DESCRIPTION AND CHARACTERIZATION TO APPLY
\input{main/chapter2/section1/content.tex}
     
% SERIOUS GAME REQUIREMENTS
\input{main/chapter2/section2/content.tex}    

% USE CASE DEFINITION
\input{main/chapter2/section3/content.tex}

% USE CASE REALIZATION
\input{main/chapter2/section4/content.tex}

% DATABASE DESIGN 
\input{main/chapter2/section5/content.tex}

% DATA MANIPULATION
\input{main/chapter2/section6/content.tex}

% COMMUNICATION 
\input{main/chapter2/section7/content.tex}

% TRAINING SCENARIOS
\input{main/chapter2/section8/content.tex}

% EMG GRAPHIC
\input{main/chapter2/section9/content.tex}

% STATICAL REPORTS GENERATION
\input{main/chapter2/section10/content.tex}

\subthesischapter{Conclusiones del capítulo}
Se presentó una descripción del sistema de adquisición de datos para rehabilitación, sus componentes, características distintivas y su funcionamiento. Se identificaron y definieron los requisitos del juego  serio, tanto funcionales como no funcionales, así como los actores y casos de usos del sistema que establecieron las bases fundamentales para el desarrollo de la aplicación. Se realizó el diseño de la base de datos, abarcando tanto el modelo lógico como el físico, lo que aseguró una estructura robusta y eficiente para el almacenamiento de los datos. La manipulación de los datos se abordó de manera integral, desde la conexión con la base de datos hasta la persistencia de los resultados estadísticos. Se diseñó e implementó la comunicación con el pedal motorizado y la implementación de la interfaz gráfica para la representación de los datos EMG. Se definieron los escenarios de entrenamiento para las modalidades Ligero y Clínico, asegurando una cobertura completa de las necesidad de entrenamiento del usuario. Por último en el ámbito estadístico se desarrolló una serie de gráficos para el seguimiento de los resultados en las rutinas de entrenamiento.   
    
\end{thesischapter}

% TRAINING SCENARIOS
\begin{thesischapter}{2} {Diseño e implementación del Juego Serio}
En este capítulo se discuten los detalles de desarrollo de los aspectos citados en el capítulo anterior. Este comienza con una descripción y caracterización general del sistema, donde se  abordan cada uno de los componentes requeridos para su completo funcionamiento. Posteriormente se detalla la ingienría de software requerida en la etapa de conceptualización de la aplicación, se explican de forma detallada los aspectos teóricos y de implementación de la base de datos, el funcionamiento del protocolo de comunicación y por último los escenarios de juegos requeridos en las rutinas de entrenamiento ligero y clínico, y las estadísticas generadas por estos. Como herramienta de desarrollo se utilizó c\#.

% SYSTEM DESCRIPTION AND CHARACTERIZATION TO APPLY
\input{main/chapter2/section1/content.tex}
     
% SERIOUS GAME REQUIREMENTS
\input{main/chapter2/section2/content.tex}    

% USE CASE DEFINITION
\input{main/chapter2/section3/content.tex}

% USE CASE REALIZATION
\input{main/chapter2/section4/content.tex}

% DATABASE DESIGN 
\input{main/chapter2/section5/content.tex}

% DATA MANIPULATION
\input{main/chapter2/section6/content.tex}

% COMMUNICATION 
\input{main/chapter2/section7/content.tex}

% TRAINING SCENARIOS
\input{main/chapter2/section8/content.tex}

% EMG GRAPHIC
\input{main/chapter2/section9/content.tex}

% STATICAL REPORTS GENERATION
\input{main/chapter2/section10/content.tex}

\subthesischapter{Conclusiones del capítulo}
Se presentó una descripción del sistema de adquisición de datos para rehabilitación, sus componentes, características distintivas y su funcionamiento. Se identificaron y definieron los requisitos del juego  serio, tanto funcionales como no funcionales, así como los actores y casos de usos del sistema que establecieron las bases fundamentales para el desarrollo de la aplicación. Se realizó el diseño de la base de datos, abarcando tanto el modelo lógico como el físico, lo que aseguró una estructura robusta y eficiente para el almacenamiento de los datos. La manipulación de los datos se abordó de manera integral, desde la conexión con la base de datos hasta la persistencia de los resultados estadísticos. Se diseñó e implementó la comunicación con el pedal motorizado y la implementación de la interfaz gráfica para la representación de los datos EMG. Se definieron los escenarios de entrenamiento para las modalidades Ligero y Clínico, asegurando una cobertura completa de las necesidad de entrenamiento del usuario. Por último en el ámbito estadístico se desarrolló una serie de gráficos para el seguimiento de los resultados en las rutinas de entrenamiento.   
    
\end{thesischapter}

% EMG GRAPHIC
\begin{thesischapter}{2} {Diseño e implementación del Juego Serio}
En este capítulo se discuten los detalles de desarrollo de los aspectos citados en el capítulo anterior. Este comienza con una descripción y caracterización general del sistema, donde se  abordan cada uno de los componentes requeridos para su completo funcionamiento. Posteriormente se detalla la ingienría de software requerida en la etapa de conceptualización de la aplicación, se explican de forma detallada los aspectos teóricos y de implementación de la base de datos, el funcionamiento del protocolo de comunicación y por último los escenarios de juegos requeridos en las rutinas de entrenamiento ligero y clínico, y las estadísticas generadas por estos. Como herramienta de desarrollo se utilizó c\#.

% SYSTEM DESCRIPTION AND CHARACTERIZATION TO APPLY
\input{main/chapter2/section1/content.tex}
     
% SERIOUS GAME REQUIREMENTS
\input{main/chapter2/section2/content.tex}    

% USE CASE DEFINITION
\input{main/chapter2/section3/content.tex}

% USE CASE REALIZATION
\input{main/chapter2/section4/content.tex}

% DATABASE DESIGN 
\input{main/chapter2/section5/content.tex}

% DATA MANIPULATION
\input{main/chapter2/section6/content.tex}

% COMMUNICATION 
\input{main/chapter2/section7/content.tex}

% TRAINING SCENARIOS
\input{main/chapter2/section8/content.tex}

% EMG GRAPHIC
\input{main/chapter2/section9/content.tex}

% STATICAL REPORTS GENERATION
\input{main/chapter2/section10/content.tex}

\subthesischapter{Conclusiones del capítulo}
Se presentó una descripción del sistema de adquisición de datos para rehabilitación, sus componentes, características distintivas y su funcionamiento. Se identificaron y definieron los requisitos del juego  serio, tanto funcionales como no funcionales, así como los actores y casos de usos del sistema que establecieron las bases fundamentales para el desarrollo de la aplicación. Se realizó el diseño de la base de datos, abarcando tanto el modelo lógico como el físico, lo que aseguró una estructura robusta y eficiente para el almacenamiento de los datos. La manipulación de los datos se abordó de manera integral, desde la conexión con la base de datos hasta la persistencia de los resultados estadísticos. Se diseñó e implementó la comunicación con el pedal motorizado y la implementación de la interfaz gráfica para la representación de los datos EMG. Se definieron los escenarios de entrenamiento para las modalidades Ligero y Clínico, asegurando una cobertura completa de las necesidad de entrenamiento del usuario. Por último en el ámbito estadístico se desarrolló una serie de gráficos para el seguimiento de los resultados en las rutinas de entrenamiento.   
    
\end{thesischapter}

% STATICAL REPORTS GENERATION
\begin{thesischapter}{2} {Diseño e implementación del Juego Serio}
En este capítulo se discuten los detalles de desarrollo de los aspectos citados en el capítulo anterior. Este comienza con una descripción y caracterización general del sistema, donde se  abordan cada uno de los componentes requeridos para su completo funcionamiento. Posteriormente se detalla la ingienría de software requerida en la etapa de conceptualización de la aplicación, se explican de forma detallada los aspectos teóricos y de implementación de la base de datos, el funcionamiento del protocolo de comunicación y por último los escenarios de juegos requeridos en las rutinas de entrenamiento ligero y clínico, y las estadísticas generadas por estos. Como herramienta de desarrollo se utilizó c\#.

% SYSTEM DESCRIPTION AND CHARACTERIZATION TO APPLY
\input{main/chapter2/section1/content.tex}
     
% SERIOUS GAME REQUIREMENTS
\input{main/chapter2/section2/content.tex}    

% USE CASE DEFINITION
\input{main/chapter2/section3/content.tex}

% USE CASE REALIZATION
\input{main/chapter2/section4/content.tex}

% DATABASE DESIGN 
\input{main/chapter2/section5/content.tex}

% DATA MANIPULATION
\input{main/chapter2/section6/content.tex}

% COMMUNICATION 
\input{main/chapter2/section7/content.tex}

% TRAINING SCENARIOS
\input{main/chapter2/section8/content.tex}

% EMG GRAPHIC
\input{main/chapter2/section9/content.tex}

% STATICAL REPORTS GENERATION
\input{main/chapter2/section10/content.tex}

\subthesischapter{Conclusiones del capítulo}
Se presentó una descripción del sistema de adquisición de datos para rehabilitación, sus componentes, características distintivas y su funcionamiento. Se identificaron y definieron los requisitos del juego  serio, tanto funcionales como no funcionales, así como los actores y casos de usos del sistema que establecieron las bases fundamentales para el desarrollo de la aplicación. Se realizó el diseño de la base de datos, abarcando tanto el modelo lógico como el físico, lo que aseguró una estructura robusta y eficiente para el almacenamiento de los datos. La manipulación de los datos se abordó de manera integral, desde la conexión con la base de datos hasta la persistencia de los resultados estadísticos. Se diseñó e implementó la comunicación con el pedal motorizado y la implementación de la interfaz gráfica para la representación de los datos EMG. Se definieron los escenarios de entrenamiento para las modalidades Ligero y Clínico, asegurando una cobertura completa de las necesidad de entrenamiento del usuario. Por último en el ámbito estadístico se desarrolló una serie de gráficos para el seguimiento de los resultados en las rutinas de entrenamiento.   
    
\end{thesischapter}

\subthesischapter{Conclusiones del capítulo}
Se presentó una descripción del sistema de adquisición de datos para rehabilitación, sus componentes, características distintivas y su funcionamiento. Se identificaron y definieron los requisitos del juego  serio, tanto funcionales como no funcionales, así como los actores y casos de usos del sistema que establecieron las bases fundamentales para el desarrollo de la aplicación. Se realizó el diseño de la base de datos, abarcando tanto el modelo lógico como el físico, lo que aseguró una estructura robusta y eficiente para el almacenamiento de los datos. La manipulación de los datos se abordó de manera integral, desde la conexión con la base de datos hasta la persistencia de los resultados estadísticos. Se diseñó e implementó la comunicación con el pedal motorizado y la implementación de la interfaz gráfica para la representación de los datos EMG. Se definieron los escenarios de entrenamiento para las modalidades Ligero y Clínico, asegurando una cobertura completa de las necesidad de entrenamiento del usuario. Por último en el ámbito estadístico se desarrolló una serie de gráficos para el seguimiento de los resultados en las rutinas de entrenamiento.   
    
\end{thesischapter}

% USE CASE REALIZATION
\begin{thesischapter}{2} {Diseño e implementación del Juego Serio}
En este capítulo se discuten los detalles de desarrollo de los aspectos citados en el capítulo anterior. Este comienza con una descripción y caracterización general del sistema, donde se  abordan cada uno de los componentes requeridos para su completo funcionamiento. Posteriormente se detalla la ingienría de software requerida en la etapa de conceptualización de la aplicación, se explican de forma detallada los aspectos teóricos y de implementación de la base de datos, el funcionamiento del protocolo de comunicación y por último los escenarios de juegos requeridos en las rutinas de entrenamiento ligero y clínico, y las estadísticas generadas por estos. Como herramienta de desarrollo se utilizó c\#.

% SYSTEM DESCRIPTION AND CHARACTERIZATION TO APPLY
\begin{thesischapter}{2} {Diseño e implementación del Juego Serio}
En este capítulo se discuten los detalles de desarrollo de los aspectos citados en el capítulo anterior. Este comienza con una descripción y caracterización general del sistema, donde se  abordan cada uno de los componentes requeridos para su completo funcionamiento. Posteriormente se detalla la ingienría de software requerida en la etapa de conceptualización de la aplicación, se explican de forma detallada los aspectos teóricos y de implementación de la base de datos, el funcionamiento del protocolo de comunicación y por último los escenarios de juegos requeridos en las rutinas de entrenamiento ligero y clínico, y las estadísticas generadas por estos. Como herramienta de desarrollo se utilizó c\#.

% SYSTEM DESCRIPTION AND CHARACTERIZATION TO APPLY
\input{main/chapter2/section1/content.tex}
     
% SERIOUS GAME REQUIREMENTS
\input{main/chapter2/section2/content.tex}    

% USE CASE DEFINITION
\input{main/chapter2/section3/content.tex}

% USE CASE REALIZATION
\input{main/chapter2/section4/content.tex}

% DATABASE DESIGN 
\input{main/chapter2/section5/content.tex}

% DATA MANIPULATION
\input{main/chapter2/section6/content.tex}

% COMMUNICATION 
\input{main/chapter2/section7/content.tex}

% TRAINING SCENARIOS
\input{main/chapter2/section8/content.tex}

% EMG GRAPHIC
\input{main/chapter2/section9/content.tex}

% STATICAL REPORTS GENERATION
\input{main/chapter2/section10/content.tex}

\subthesischapter{Conclusiones del capítulo}
Se presentó una descripción del sistema de adquisición de datos para rehabilitación, sus componentes, características distintivas y su funcionamiento. Se identificaron y definieron los requisitos del juego  serio, tanto funcionales como no funcionales, así como los actores y casos de usos del sistema que establecieron las bases fundamentales para el desarrollo de la aplicación. Se realizó el diseño de la base de datos, abarcando tanto el modelo lógico como el físico, lo que aseguró una estructura robusta y eficiente para el almacenamiento de los datos. La manipulación de los datos se abordó de manera integral, desde la conexión con la base de datos hasta la persistencia de los resultados estadísticos. Se diseñó e implementó la comunicación con el pedal motorizado y la implementación de la interfaz gráfica para la representación de los datos EMG. Se definieron los escenarios de entrenamiento para las modalidades Ligero y Clínico, asegurando una cobertura completa de las necesidad de entrenamiento del usuario. Por último en el ámbito estadístico se desarrolló una serie de gráficos para el seguimiento de los resultados en las rutinas de entrenamiento.   
    
\end{thesischapter}
     
% SERIOUS GAME REQUIREMENTS
\begin{thesischapter}{2} {Diseño e implementación del Juego Serio}
En este capítulo se discuten los detalles de desarrollo de los aspectos citados en el capítulo anterior. Este comienza con una descripción y caracterización general del sistema, donde se  abordan cada uno de los componentes requeridos para su completo funcionamiento. Posteriormente se detalla la ingienría de software requerida en la etapa de conceptualización de la aplicación, se explican de forma detallada los aspectos teóricos y de implementación de la base de datos, el funcionamiento del protocolo de comunicación y por último los escenarios de juegos requeridos en las rutinas de entrenamiento ligero y clínico, y las estadísticas generadas por estos. Como herramienta de desarrollo se utilizó c\#.

% SYSTEM DESCRIPTION AND CHARACTERIZATION TO APPLY
\input{main/chapter2/section1/content.tex}
     
% SERIOUS GAME REQUIREMENTS
\input{main/chapter2/section2/content.tex}    

% USE CASE DEFINITION
\input{main/chapter2/section3/content.tex}

% USE CASE REALIZATION
\input{main/chapter2/section4/content.tex}

% DATABASE DESIGN 
\input{main/chapter2/section5/content.tex}

% DATA MANIPULATION
\input{main/chapter2/section6/content.tex}

% COMMUNICATION 
\input{main/chapter2/section7/content.tex}

% TRAINING SCENARIOS
\input{main/chapter2/section8/content.tex}

% EMG GRAPHIC
\input{main/chapter2/section9/content.tex}

% STATICAL REPORTS GENERATION
\input{main/chapter2/section10/content.tex}

\subthesischapter{Conclusiones del capítulo}
Se presentó una descripción del sistema de adquisición de datos para rehabilitación, sus componentes, características distintivas y su funcionamiento. Se identificaron y definieron los requisitos del juego  serio, tanto funcionales como no funcionales, así como los actores y casos de usos del sistema que establecieron las bases fundamentales para el desarrollo de la aplicación. Se realizó el diseño de la base de datos, abarcando tanto el modelo lógico como el físico, lo que aseguró una estructura robusta y eficiente para el almacenamiento de los datos. La manipulación de los datos se abordó de manera integral, desde la conexión con la base de datos hasta la persistencia de los resultados estadísticos. Se diseñó e implementó la comunicación con el pedal motorizado y la implementación de la interfaz gráfica para la representación de los datos EMG. Se definieron los escenarios de entrenamiento para las modalidades Ligero y Clínico, asegurando una cobertura completa de las necesidad de entrenamiento del usuario. Por último en el ámbito estadístico se desarrolló una serie de gráficos para el seguimiento de los resultados en las rutinas de entrenamiento.   
    
\end{thesischapter}    

% USE CASE DEFINITION
\begin{thesischapter}{2} {Diseño e implementación del Juego Serio}
En este capítulo se discuten los detalles de desarrollo de los aspectos citados en el capítulo anterior. Este comienza con una descripción y caracterización general del sistema, donde se  abordan cada uno de los componentes requeridos para su completo funcionamiento. Posteriormente se detalla la ingienría de software requerida en la etapa de conceptualización de la aplicación, se explican de forma detallada los aspectos teóricos y de implementación de la base de datos, el funcionamiento del protocolo de comunicación y por último los escenarios de juegos requeridos en las rutinas de entrenamiento ligero y clínico, y las estadísticas generadas por estos. Como herramienta de desarrollo se utilizó c\#.

% SYSTEM DESCRIPTION AND CHARACTERIZATION TO APPLY
\input{main/chapter2/section1/content.tex}
     
% SERIOUS GAME REQUIREMENTS
\input{main/chapter2/section2/content.tex}    

% USE CASE DEFINITION
\input{main/chapter2/section3/content.tex}

% USE CASE REALIZATION
\input{main/chapter2/section4/content.tex}

% DATABASE DESIGN 
\input{main/chapter2/section5/content.tex}

% DATA MANIPULATION
\input{main/chapter2/section6/content.tex}

% COMMUNICATION 
\input{main/chapter2/section7/content.tex}

% TRAINING SCENARIOS
\input{main/chapter2/section8/content.tex}

% EMG GRAPHIC
\input{main/chapter2/section9/content.tex}

% STATICAL REPORTS GENERATION
\input{main/chapter2/section10/content.tex}

\subthesischapter{Conclusiones del capítulo}
Se presentó una descripción del sistema de adquisición de datos para rehabilitación, sus componentes, características distintivas y su funcionamiento. Se identificaron y definieron los requisitos del juego  serio, tanto funcionales como no funcionales, así como los actores y casos de usos del sistema que establecieron las bases fundamentales para el desarrollo de la aplicación. Se realizó el diseño de la base de datos, abarcando tanto el modelo lógico como el físico, lo que aseguró una estructura robusta y eficiente para el almacenamiento de los datos. La manipulación de los datos se abordó de manera integral, desde la conexión con la base de datos hasta la persistencia de los resultados estadísticos. Se diseñó e implementó la comunicación con el pedal motorizado y la implementación de la interfaz gráfica para la representación de los datos EMG. Se definieron los escenarios de entrenamiento para las modalidades Ligero y Clínico, asegurando una cobertura completa de las necesidad de entrenamiento del usuario. Por último en el ámbito estadístico se desarrolló una serie de gráficos para el seguimiento de los resultados en las rutinas de entrenamiento.   
    
\end{thesischapter}

% USE CASE REALIZATION
\begin{thesischapter}{2} {Diseño e implementación del Juego Serio}
En este capítulo se discuten los detalles de desarrollo de los aspectos citados en el capítulo anterior. Este comienza con una descripción y caracterización general del sistema, donde se  abordan cada uno de los componentes requeridos para su completo funcionamiento. Posteriormente se detalla la ingienría de software requerida en la etapa de conceptualización de la aplicación, se explican de forma detallada los aspectos teóricos y de implementación de la base de datos, el funcionamiento del protocolo de comunicación y por último los escenarios de juegos requeridos en las rutinas de entrenamiento ligero y clínico, y las estadísticas generadas por estos. Como herramienta de desarrollo se utilizó c\#.

% SYSTEM DESCRIPTION AND CHARACTERIZATION TO APPLY
\input{main/chapter2/section1/content.tex}
     
% SERIOUS GAME REQUIREMENTS
\input{main/chapter2/section2/content.tex}    

% USE CASE DEFINITION
\input{main/chapter2/section3/content.tex}

% USE CASE REALIZATION
\input{main/chapter2/section4/content.tex}

% DATABASE DESIGN 
\input{main/chapter2/section5/content.tex}

% DATA MANIPULATION
\input{main/chapter2/section6/content.tex}

% COMMUNICATION 
\input{main/chapter2/section7/content.tex}

% TRAINING SCENARIOS
\input{main/chapter2/section8/content.tex}

% EMG GRAPHIC
\input{main/chapter2/section9/content.tex}

% STATICAL REPORTS GENERATION
\input{main/chapter2/section10/content.tex}

\subthesischapter{Conclusiones del capítulo}
Se presentó una descripción del sistema de adquisición de datos para rehabilitación, sus componentes, características distintivas y su funcionamiento. Se identificaron y definieron los requisitos del juego  serio, tanto funcionales como no funcionales, así como los actores y casos de usos del sistema que establecieron las bases fundamentales para el desarrollo de la aplicación. Se realizó el diseño de la base de datos, abarcando tanto el modelo lógico como el físico, lo que aseguró una estructura robusta y eficiente para el almacenamiento de los datos. La manipulación de los datos se abordó de manera integral, desde la conexión con la base de datos hasta la persistencia de los resultados estadísticos. Se diseñó e implementó la comunicación con el pedal motorizado y la implementación de la interfaz gráfica para la representación de los datos EMG. Se definieron los escenarios de entrenamiento para las modalidades Ligero y Clínico, asegurando una cobertura completa de las necesidad de entrenamiento del usuario. Por último en el ámbito estadístico se desarrolló una serie de gráficos para el seguimiento de los resultados en las rutinas de entrenamiento.   
    
\end{thesischapter}

% DATABASE DESIGN 
\begin{thesischapter}{2} {Diseño e implementación del Juego Serio}
En este capítulo se discuten los detalles de desarrollo de los aspectos citados en el capítulo anterior. Este comienza con una descripción y caracterización general del sistema, donde se  abordan cada uno de los componentes requeridos para su completo funcionamiento. Posteriormente se detalla la ingienría de software requerida en la etapa de conceptualización de la aplicación, se explican de forma detallada los aspectos teóricos y de implementación de la base de datos, el funcionamiento del protocolo de comunicación y por último los escenarios de juegos requeridos en las rutinas de entrenamiento ligero y clínico, y las estadísticas generadas por estos. Como herramienta de desarrollo se utilizó c\#.

% SYSTEM DESCRIPTION AND CHARACTERIZATION TO APPLY
\input{main/chapter2/section1/content.tex}
     
% SERIOUS GAME REQUIREMENTS
\input{main/chapter2/section2/content.tex}    

% USE CASE DEFINITION
\input{main/chapter2/section3/content.tex}

% USE CASE REALIZATION
\input{main/chapter2/section4/content.tex}

% DATABASE DESIGN 
\input{main/chapter2/section5/content.tex}

% DATA MANIPULATION
\input{main/chapter2/section6/content.tex}

% COMMUNICATION 
\input{main/chapter2/section7/content.tex}

% TRAINING SCENARIOS
\input{main/chapter2/section8/content.tex}

% EMG GRAPHIC
\input{main/chapter2/section9/content.tex}

% STATICAL REPORTS GENERATION
\input{main/chapter2/section10/content.tex}

\subthesischapter{Conclusiones del capítulo}
Se presentó una descripción del sistema de adquisición de datos para rehabilitación, sus componentes, características distintivas y su funcionamiento. Se identificaron y definieron los requisitos del juego  serio, tanto funcionales como no funcionales, así como los actores y casos de usos del sistema que establecieron las bases fundamentales para el desarrollo de la aplicación. Se realizó el diseño de la base de datos, abarcando tanto el modelo lógico como el físico, lo que aseguró una estructura robusta y eficiente para el almacenamiento de los datos. La manipulación de los datos se abordó de manera integral, desde la conexión con la base de datos hasta la persistencia de los resultados estadísticos. Se diseñó e implementó la comunicación con el pedal motorizado y la implementación de la interfaz gráfica para la representación de los datos EMG. Se definieron los escenarios de entrenamiento para las modalidades Ligero y Clínico, asegurando una cobertura completa de las necesidad de entrenamiento del usuario. Por último en el ámbito estadístico se desarrolló una serie de gráficos para el seguimiento de los resultados en las rutinas de entrenamiento.   
    
\end{thesischapter}

% DATA MANIPULATION
\begin{thesischapter}{2} {Diseño e implementación del Juego Serio}
En este capítulo se discuten los detalles de desarrollo de los aspectos citados en el capítulo anterior. Este comienza con una descripción y caracterización general del sistema, donde se  abordan cada uno de los componentes requeridos para su completo funcionamiento. Posteriormente se detalla la ingienría de software requerida en la etapa de conceptualización de la aplicación, se explican de forma detallada los aspectos teóricos y de implementación de la base de datos, el funcionamiento del protocolo de comunicación y por último los escenarios de juegos requeridos en las rutinas de entrenamiento ligero y clínico, y las estadísticas generadas por estos. Como herramienta de desarrollo se utilizó c\#.

% SYSTEM DESCRIPTION AND CHARACTERIZATION TO APPLY
\input{main/chapter2/section1/content.tex}
     
% SERIOUS GAME REQUIREMENTS
\input{main/chapter2/section2/content.tex}    

% USE CASE DEFINITION
\input{main/chapter2/section3/content.tex}

% USE CASE REALIZATION
\input{main/chapter2/section4/content.tex}

% DATABASE DESIGN 
\input{main/chapter2/section5/content.tex}

% DATA MANIPULATION
\input{main/chapter2/section6/content.tex}

% COMMUNICATION 
\input{main/chapter2/section7/content.tex}

% TRAINING SCENARIOS
\input{main/chapter2/section8/content.tex}

% EMG GRAPHIC
\input{main/chapter2/section9/content.tex}

% STATICAL REPORTS GENERATION
\input{main/chapter2/section10/content.tex}

\subthesischapter{Conclusiones del capítulo}
Se presentó una descripción del sistema de adquisición de datos para rehabilitación, sus componentes, características distintivas y su funcionamiento. Se identificaron y definieron los requisitos del juego  serio, tanto funcionales como no funcionales, así como los actores y casos de usos del sistema que establecieron las bases fundamentales para el desarrollo de la aplicación. Se realizó el diseño de la base de datos, abarcando tanto el modelo lógico como el físico, lo que aseguró una estructura robusta y eficiente para el almacenamiento de los datos. La manipulación de los datos se abordó de manera integral, desde la conexión con la base de datos hasta la persistencia de los resultados estadísticos. Se diseñó e implementó la comunicación con el pedal motorizado y la implementación de la interfaz gráfica para la representación de los datos EMG. Se definieron los escenarios de entrenamiento para las modalidades Ligero y Clínico, asegurando una cobertura completa de las necesidad de entrenamiento del usuario. Por último en el ámbito estadístico se desarrolló una serie de gráficos para el seguimiento de los resultados en las rutinas de entrenamiento.   
    
\end{thesischapter}

% COMMUNICATION 
\begin{thesischapter}{2} {Diseño e implementación del Juego Serio}
En este capítulo se discuten los detalles de desarrollo de los aspectos citados en el capítulo anterior. Este comienza con una descripción y caracterización general del sistema, donde se  abordan cada uno de los componentes requeridos para su completo funcionamiento. Posteriormente se detalla la ingienría de software requerida en la etapa de conceptualización de la aplicación, se explican de forma detallada los aspectos teóricos y de implementación de la base de datos, el funcionamiento del protocolo de comunicación y por último los escenarios de juegos requeridos en las rutinas de entrenamiento ligero y clínico, y las estadísticas generadas por estos. Como herramienta de desarrollo se utilizó c\#.

% SYSTEM DESCRIPTION AND CHARACTERIZATION TO APPLY
\input{main/chapter2/section1/content.tex}
     
% SERIOUS GAME REQUIREMENTS
\input{main/chapter2/section2/content.tex}    

% USE CASE DEFINITION
\input{main/chapter2/section3/content.tex}

% USE CASE REALIZATION
\input{main/chapter2/section4/content.tex}

% DATABASE DESIGN 
\input{main/chapter2/section5/content.tex}

% DATA MANIPULATION
\input{main/chapter2/section6/content.tex}

% COMMUNICATION 
\input{main/chapter2/section7/content.tex}

% TRAINING SCENARIOS
\input{main/chapter2/section8/content.tex}

% EMG GRAPHIC
\input{main/chapter2/section9/content.tex}

% STATICAL REPORTS GENERATION
\input{main/chapter2/section10/content.tex}

\subthesischapter{Conclusiones del capítulo}
Se presentó una descripción del sistema de adquisición de datos para rehabilitación, sus componentes, características distintivas y su funcionamiento. Se identificaron y definieron los requisitos del juego  serio, tanto funcionales como no funcionales, así como los actores y casos de usos del sistema que establecieron las bases fundamentales para el desarrollo de la aplicación. Se realizó el diseño de la base de datos, abarcando tanto el modelo lógico como el físico, lo que aseguró una estructura robusta y eficiente para el almacenamiento de los datos. La manipulación de los datos se abordó de manera integral, desde la conexión con la base de datos hasta la persistencia de los resultados estadísticos. Se diseñó e implementó la comunicación con el pedal motorizado y la implementación de la interfaz gráfica para la representación de los datos EMG. Se definieron los escenarios de entrenamiento para las modalidades Ligero y Clínico, asegurando una cobertura completa de las necesidad de entrenamiento del usuario. Por último en el ámbito estadístico se desarrolló una serie de gráficos para el seguimiento de los resultados en las rutinas de entrenamiento.   
    
\end{thesischapter}

% TRAINING SCENARIOS
\begin{thesischapter}{2} {Diseño e implementación del Juego Serio}
En este capítulo se discuten los detalles de desarrollo de los aspectos citados en el capítulo anterior. Este comienza con una descripción y caracterización general del sistema, donde se  abordan cada uno de los componentes requeridos para su completo funcionamiento. Posteriormente se detalla la ingienría de software requerida en la etapa de conceptualización de la aplicación, se explican de forma detallada los aspectos teóricos y de implementación de la base de datos, el funcionamiento del protocolo de comunicación y por último los escenarios de juegos requeridos en las rutinas de entrenamiento ligero y clínico, y las estadísticas generadas por estos. Como herramienta de desarrollo se utilizó c\#.

% SYSTEM DESCRIPTION AND CHARACTERIZATION TO APPLY
\input{main/chapter2/section1/content.tex}
     
% SERIOUS GAME REQUIREMENTS
\input{main/chapter2/section2/content.tex}    

% USE CASE DEFINITION
\input{main/chapter2/section3/content.tex}

% USE CASE REALIZATION
\input{main/chapter2/section4/content.tex}

% DATABASE DESIGN 
\input{main/chapter2/section5/content.tex}

% DATA MANIPULATION
\input{main/chapter2/section6/content.tex}

% COMMUNICATION 
\input{main/chapter2/section7/content.tex}

% TRAINING SCENARIOS
\input{main/chapter2/section8/content.tex}

% EMG GRAPHIC
\input{main/chapter2/section9/content.tex}

% STATICAL REPORTS GENERATION
\input{main/chapter2/section10/content.tex}

\subthesischapter{Conclusiones del capítulo}
Se presentó una descripción del sistema de adquisición de datos para rehabilitación, sus componentes, características distintivas y su funcionamiento. Se identificaron y definieron los requisitos del juego  serio, tanto funcionales como no funcionales, así como los actores y casos de usos del sistema que establecieron las bases fundamentales para el desarrollo de la aplicación. Se realizó el diseño de la base de datos, abarcando tanto el modelo lógico como el físico, lo que aseguró una estructura robusta y eficiente para el almacenamiento de los datos. La manipulación de los datos se abordó de manera integral, desde la conexión con la base de datos hasta la persistencia de los resultados estadísticos. Se diseñó e implementó la comunicación con el pedal motorizado y la implementación de la interfaz gráfica para la representación de los datos EMG. Se definieron los escenarios de entrenamiento para las modalidades Ligero y Clínico, asegurando una cobertura completa de las necesidad de entrenamiento del usuario. Por último en el ámbito estadístico se desarrolló una serie de gráficos para el seguimiento de los resultados en las rutinas de entrenamiento.   
    
\end{thesischapter}

% EMG GRAPHIC
\begin{thesischapter}{2} {Diseño e implementación del Juego Serio}
En este capítulo se discuten los detalles de desarrollo de los aspectos citados en el capítulo anterior. Este comienza con una descripción y caracterización general del sistema, donde se  abordan cada uno de los componentes requeridos para su completo funcionamiento. Posteriormente se detalla la ingienría de software requerida en la etapa de conceptualización de la aplicación, se explican de forma detallada los aspectos teóricos y de implementación de la base de datos, el funcionamiento del protocolo de comunicación y por último los escenarios de juegos requeridos en las rutinas de entrenamiento ligero y clínico, y las estadísticas generadas por estos. Como herramienta de desarrollo se utilizó c\#.

% SYSTEM DESCRIPTION AND CHARACTERIZATION TO APPLY
\input{main/chapter2/section1/content.tex}
     
% SERIOUS GAME REQUIREMENTS
\input{main/chapter2/section2/content.tex}    

% USE CASE DEFINITION
\input{main/chapter2/section3/content.tex}

% USE CASE REALIZATION
\input{main/chapter2/section4/content.tex}

% DATABASE DESIGN 
\input{main/chapter2/section5/content.tex}

% DATA MANIPULATION
\input{main/chapter2/section6/content.tex}

% COMMUNICATION 
\input{main/chapter2/section7/content.tex}

% TRAINING SCENARIOS
\input{main/chapter2/section8/content.tex}

% EMG GRAPHIC
\input{main/chapter2/section9/content.tex}

% STATICAL REPORTS GENERATION
\input{main/chapter2/section10/content.tex}

\subthesischapter{Conclusiones del capítulo}
Se presentó una descripción del sistema de adquisición de datos para rehabilitación, sus componentes, características distintivas y su funcionamiento. Se identificaron y definieron los requisitos del juego  serio, tanto funcionales como no funcionales, así como los actores y casos de usos del sistema que establecieron las bases fundamentales para el desarrollo de la aplicación. Se realizó el diseño de la base de datos, abarcando tanto el modelo lógico como el físico, lo que aseguró una estructura robusta y eficiente para el almacenamiento de los datos. La manipulación de los datos se abordó de manera integral, desde la conexión con la base de datos hasta la persistencia de los resultados estadísticos. Se diseñó e implementó la comunicación con el pedal motorizado y la implementación de la interfaz gráfica para la representación de los datos EMG. Se definieron los escenarios de entrenamiento para las modalidades Ligero y Clínico, asegurando una cobertura completa de las necesidad de entrenamiento del usuario. Por último en el ámbito estadístico se desarrolló una serie de gráficos para el seguimiento de los resultados en las rutinas de entrenamiento.   
    
\end{thesischapter}

% STATICAL REPORTS GENERATION
\begin{thesischapter}{2} {Diseño e implementación del Juego Serio}
En este capítulo se discuten los detalles de desarrollo de los aspectos citados en el capítulo anterior. Este comienza con una descripción y caracterización general del sistema, donde se  abordan cada uno de los componentes requeridos para su completo funcionamiento. Posteriormente se detalla la ingienría de software requerida en la etapa de conceptualización de la aplicación, se explican de forma detallada los aspectos teóricos y de implementación de la base de datos, el funcionamiento del protocolo de comunicación y por último los escenarios de juegos requeridos en las rutinas de entrenamiento ligero y clínico, y las estadísticas generadas por estos. Como herramienta de desarrollo se utilizó c\#.

% SYSTEM DESCRIPTION AND CHARACTERIZATION TO APPLY
\input{main/chapter2/section1/content.tex}
     
% SERIOUS GAME REQUIREMENTS
\input{main/chapter2/section2/content.tex}    

% USE CASE DEFINITION
\input{main/chapter2/section3/content.tex}

% USE CASE REALIZATION
\input{main/chapter2/section4/content.tex}

% DATABASE DESIGN 
\input{main/chapter2/section5/content.tex}

% DATA MANIPULATION
\input{main/chapter2/section6/content.tex}

% COMMUNICATION 
\input{main/chapter2/section7/content.tex}

% TRAINING SCENARIOS
\input{main/chapter2/section8/content.tex}

% EMG GRAPHIC
\input{main/chapter2/section9/content.tex}

% STATICAL REPORTS GENERATION
\input{main/chapter2/section10/content.tex}

\subthesischapter{Conclusiones del capítulo}
Se presentó una descripción del sistema de adquisición de datos para rehabilitación, sus componentes, características distintivas y su funcionamiento. Se identificaron y definieron los requisitos del juego  serio, tanto funcionales como no funcionales, así como los actores y casos de usos del sistema que establecieron las bases fundamentales para el desarrollo de la aplicación. Se realizó el diseño de la base de datos, abarcando tanto el modelo lógico como el físico, lo que aseguró una estructura robusta y eficiente para el almacenamiento de los datos. La manipulación de los datos se abordó de manera integral, desde la conexión con la base de datos hasta la persistencia de los resultados estadísticos. Se diseñó e implementó la comunicación con el pedal motorizado y la implementación de la interfaz gráfica para la representación de los datos EMG. Se definieron los escenarios de entrenamiento para las modalidades Ligero y Clínico, asegurando una cobertura completa de las necesidad de entrenamiento del usuario. Por último en el ámbito estadístico se desarrolló una serie de gráficos para el seguimiento de los resultados en las rutinas de entrenamiento.   
    
\end{thesischapter}

\subthesischapter{Conclusiones del capítulo}
Se presentó una descripción del sistema de adquisición de datos para rehabilitación, sus componentes, características distintivas y su funcionamiento. Se identificaron y definieron los requisitos del juego  serio, tanto funcionales como no funcionales, así como los actores y casos de usos del sistema que establecieron las bases fundamentales para el desarrollo de la aplicación. Se realizó el diseño de la base de datos, abarcando tanto el modelo lógico como el físico, lo que aseguró una estructura robusta y eficiente para el almacenamiento de los datos. La manipulación de los datos se abordó de manera integral, desde la conexión con la base de datos hasta la persistencia de los resultados estadísticos. Se diseñó e implementó la comunicación con el pedal motorizado y la implementación de la interfaz gráfica para la representación de los datos EMG. Se definieron los escenarios de entrenamiento para las modalidades Ligero y Clínico, asegurando una cobertura completa de las necesidad de entrenamiento del usuario. Por último en el ámbito estadístico se desarrolló una serie de gráficos para el seguimiento de los resultados en las rutinas de entrenamiento.   
    
\end{thesischapter}

% DATABASE DESIGN 
\begin{thesischapter}{2} {Diseño e implementación del Juego Serio}
En este capítulo se discuten los detalles de desarrollo de los aspectos citados en el capítulo anterior. Este comienza con una descripción y caracterización general del sistema, donde se  abordan cada uno de los componentes requeridos para su completo funcionamiento. Posteriormente se detalla la ingienría de software requerida en la etapa de conceptualización de la aplicación, se explican de forma detallada los aspectos teóricos y de implementación de la base de datos, el funcionamiento del protocolo de comunicación y por último los escenarios de juegos requeridos en las rutinas de entrenamiento ligero y clínico, y las estadísticas generadas por estos. Como herramienta de desarrollo se utilizó c\#.

% SYSTEM DESCRIPTION AND CHARACTERIZATION TO APPLY
\begin{thesischapter}{2} {Diseño e implementación del Juego Serio}
En este capítulo se discuten los detalles de desarrollo de los aspectos citados en el capítulo anterior. Este comienza con una descripción y caracterización general del sistema, donde se  abordan cada uno de los componentes requeridos para su completo funcionamiento. Posteriormente se detalla la ingienría de software requerida en la etapa de conceptualización de la aplicación, se explican de forma detallada los aspectos teóricos y de implementación de la base de datos, el funcionamiento del protocolo de comunicación y por último los escenarios de juegos requeridos en las rutinas de entrenamiento ligero y clínico, y las estadísticas generadas por estos. Como herramienta de desarrollo se utilizó c\#.

% SYSTEM DESCRIPTION AND CHARACTERIZATION TO APPLY
\input{main/chapter2/section1/content.tex}
     
% SERIOUS GAME REQUIREMENTS
\input{main/chapter2/section2/content.tex}    

% USE CASE DEFINITION
\input{main/chapter2/section3/content.tex}

% USE CASE REALIZATION
\input{main/chapter2/section4/content.tex}

% DATABASE DESIGN 
\input{main/chapter2/section5/content.tex}

% DATA MANIPULATION
\input{main/chapter2/section6/content.tex}

% COMMUNICATION 
\input{main/chapter2/section7/content.tex}

% TRAINING SCENARIOS
\input{main/chapter2/section8/content.tex}

% EMG GRAPHIC
\input{main/chapter2/section9/content.tex}

% STATICAL REPORTS GENERATION
\input{main/chapter2/section10/content.tex}

\subthesischapter{Conclusiones del capítulo}
Se presentó una descripción del sistema de adquisición de datos para rehabilitación, sus componentes, características distintivas y su funcionamiento. Se identificaron y definieron los requisitos del juego  serio, tanto funcionales como no funcionales, así como los actores y casos de usos del sistema que establecieron las bases fundamentales para el desarrollo de la aplicación. Se realizó el diseño de la base de datos, abarcando tanto el modelo lógico como el físico, lo que aseguró una estructura robusta y eficiente para el almacenamiento de los datos. La manipulación de los datos se abordó de manera integral, desde la conexión con la base de datos hasta la persistencia de los resultados estadísticos. Se diseñó e implementó la comunicación con el pedal motorizado y la implementación de la interfaz gráfica para la representación de los datos EMG. Se definieron los escenarios de entrenamiento para las modalidades Ligero y Clínico, asegurando una cobertura completa de las necesidad de entrenamiento del usuario. Por último en el ámbito estadístico se desarrolló una serie de gráficos para el seguimiento de los resultados en las rutinas de entrenamiento.   
    
\end{thesischapter}
     
% SERIOUS GAME REQUIREMENTS
\begin{thesischapter}{2} {Diseño e implementación del Juego Serio}
En este capítulo se discuten los detalles de desarrollo de los aspectos citados en el capítulo anterior. Este comienza con una descripción y caracterización general del sistema, donde se  abordan cada uno de los componentes requeridos para su completo funcionamiento. Posteriormente se detalla la ingienría de software requerida en la etapa de conceptualización de la aplicación, se explican de forma detallada los aspectos teóricos y de implementación de la base de datos, el funcionamiento del protocolo de comunicación y por último los escenarios de juegos requeridos en las rutinas de entrenamiento ligero y clínico, y las estadísticas generadas por estos. Como herramienta de desarrollo se utilizó c\#.

% SYSTEM DESCRIPTION AND CHARACTERIZATION TO APPLY
\input{main/chapter2/section1/content.tex}
     
% SERIOUS GAME REQUIREMENTS
\input{main/chapter2/section2/content.tex}    

% USE CASE DEFINITION
\input{main/chapter2/section3/content.tex}

% USE CASE REALIZATION
\input{main/chapter2/section4/content.tex}

% DATABASE DESIGN 
\input{main/chapter2/section5/content.tex}

% DATA MANIPULATION
\input{main/chapter2/section6/content.tex}

% COMMUNICATION 
\input{main/chapter2/section7/content.tex}

% TRAINING SCENARIOS
\input{main/chapter2/section8/content.tex}

% EMG GRAPHIC
\input{main/chapter2/section9/content.tex}

% STATICAL REPORTS GENERATION
\input{main/chapter2/section10/content.tex}

\subthesischapter{Conclusiones del capítulo}
Se presentó una descripción del sistema de adquisición de datos para rehabilitación, sus componentes, características distintivas y su funcionamiento. Se identificaron y definieron los requisitos del juego  serio, tanto funcionales como no funcionales, así como los actores y casos de usos del sistema que establecieron las bases fundamentales para el desarrollo de la aplicación. Se realizó el diseño de la base de datos, abarcando tanto el modelo lógico como el físico, lo que aseguró una estructura robusta y eficiente para el almacenamiento de los datos. La manipulación de los datos se abordó de manera integral, desde la conexión con la base de datos hasta la persistencia de los resultados estadísticos. Se diseñó e implementó la comunicación con el pedal motorizado y la implementación de la interfaz gráfica para la representación de los datos EMG. Se definieron los escenarios de entrenamiento para las modalidades Ligero y Clínico, asegurando una cobertura completa de las necesidad de entrenamiento del usuario. Por último en el ámbito estadístico se desarrolló una serie de gráficos para el seguimiento de los resultados en las rutinas de entrenamiento.   
    
\end{thesischapter}    

% USE CASE DEFINITION
\begin{thesischapter}{2} {Diseño e implementación del Juego Serio}
En este capítulo se discuten los detalles de desarrollo de los aspectos citados en el capítulo anterior. Este comienza con una descripción y caracterización general del sistema, donde se  abordan cada uno de los componentes requeridos para su completo funcionamiento. Posteriormente se detalla la ingienría de software requerida en la etapa de conceptualización de la aplicación, se explican de forma detallada los aspectos teóricos y de implementación de la base de datos, el funcionamiento del protocolo de comunicación y por último los escenarios de juegos requeridos en las rutinas de entrenamiento ligero y clínico, y las estadísticas generadas por estos. Como herramienta de desarrollo se utilizó c\#.

% SYSTEM DESCRIPTION AND CHARACTERIZATION TO APPLY
\input{main/chapter2/section1/content.tex}
     
% SERIOUS GAME REQUIREMENTS
\input{main/chapter2/section2/content.tex}    

% USE CASE DEFINITION
\input{main/chapter2/section3/content.tex}

% USE CASE REALIZATION
\input{main/chapter2/section4/content.tex}

% DATABASE DESIGN 
\input{main/chapter2/section5/content.tex}

% DATA MANIPULATION
\input{main/chapter2/section6/content.tex}

% COMMUNICATION 
\input{main/chapter2/section7/content.tex}

% TRAINING SCENARIOS
\input{main/chapter2/section8/content.tex}

% EMG GRAPHIC
\input{main/chapter2/section9/content.tex}

% STATICAL REPORTS GENERATION
\input{main/chapter2/section10/content.tex}

\subthesischapter{Conclusiones del capítulo}
Se presentó una descripción del sistema de adquisición de datos para rehabilitación, sus componentes, características distintivas y su funcionamiento. Se identificaron y definieron los requisitos del juego  serio, tanto funcionales como no funcionales, así como los actores y casos de usos del sistema que establecieron las bases fundamentales para el desarrollo de la aplicación. Se realizó el diseño de la base de datos, abarcando tanto el modelo lógico como el físico, lo que aseguró una estructura robusta y eficiente para el almacenamiento de los datos. La manipulación de los datos se abordó de manera integral, desde la conexión con la base de datos hasta la persistencia de los resultados estadísticos. Se diseñó e implementó la comunicación con el pedal motorizado y la implementación de la interfaz gráfica para la representación de los datos EMG. Se definieron los escenarios de entrenamiento para las modalidades Ligero y Clínico, asegurando una cobertura completa de las necesidad de entrenamiento del usuario. Por último en el ámbito estadístico se desarrolló una serie de gráficos para el seguimiento de los resultados en las rutinas de entrenamiento.   
    
\end{thesischapter}

% USE CASE REALIZATION
\begin{thesischapter}{2} {Diseño e implementación del Juego Serio}
En este capítulo se discuten los detalles de desarrollo de los aspectos citados en el capítulo anterior. Este comienza con una descripción y caracterización general del sistema, donde se  abordan cada uno de los componentes requeridos para su completo funcionamiento. Posteriormente se detalla la ingienría de software requerida en la etapa de conceptualización de la aplicación, se explican de forma detallada los aspectos teóricos y de implementación de la base de datos, el funcionamiento del protocolo de comunicación y por último los escenarios de juegos requeridos en las rutinas de entrenamiento ligero y clínico, y las estadísticas generadas por estos. Como herramienta de desarrollo se utilizó c\#.

% SYSTEM DESCRIPTION AND CHARACTERIZATION TO APPLY
\input{main/chapter2/section1/content.tex}
     
% SERIOUS GAME REQUIREMENTS
\input{main/chapter2/section2/content.tex}    

% USE CASE DEFINITION
\input{main/chapter2/section3/content.tex}

% USE CASE REALIZATION
\input{main/chapter2/section4/content.tex}

% DATABASE DESIGN 
\input{main/chapter2/section5/content.tex}

% DATA MANIPULATION
\input{main/chapter2/section6/content.tex}

% COMMUNICATION 
\input{main/chapter2/section7/content.tex}

% TRAINING SCENARIOS
\input{main/chapter2/section8/content.tex}

% EMG GRAPHIC
\input{main/chapter2/section9/content.tex}

% STATICAL REPORTS GENERATION
\input{main/chapter2/section10/content.tex}

\subthesischapter{Conclusiones del capítulo}
Se presentó una descripción del sistema de adquisición de datos para rehabilitación, sus componentes, características distintivas y su funcionamiento. Se identificaron y definieron los requisitos del juego  serio, tanto funcionales como no funcionales, así como los actores y casos de usos del sistema que establecieron las bases fundamentales para el desarrollo de la aplicación. Se realizó el diseño de la base de datos, abarcando tanto el modelo lógico como el físico, lo que aseguró una estructura robusta y eficiente para el almacenamiento de los datos. La manipulación de los datos se abordó de manera integral, desde la conexión con la base de datos hasta la persistencia de los resultados estadísticos. Se diseñó e implementó la comunicación con el pedal motorizado y la implementación de la interfaz gráfica para la representación de los datos EMG. Se definieron los escenarios de entrenamiento para las modalidades Ligero y Clínico, asegurando una cobertura completa de las necesidad de entrenamiento del usuario. Por último en el ámbito estadístico se desarrolló una serie de gráficos para el seguimiento de los resultados en las rutinas de entrenamiento.   
    
\end{thesischapter}

% DATABASE DESIGN 
\begin{thesischapter}{2} {Diseño e implementación del Juego Serio}
En este capítulo se discuten los detalles de desarrollo de los aspectos citados en el capítulo anterior. Este comienza con una descripción y caracterización general del sistema, donde se  abordan cada uno de los componentes requeridos para su completo funcionamiento. Posteriormente se detalla la ingienría de software requerida en la etapa de conceptualización de la aplicación, se explican de forma detallada los aspectos teóricos y de implementación de la base de datos, el funcionamiento del protocolo de comunicación y por último los escenarios de juegos requeridos en las rutinas de entrenamiento ligero y clínico, y las estadísticas generadas por estos. Como herramienta de desarrollo se utilizó c\#.

% SYSTEM DESCRIPTION AND CHARACTERIZATION TO APPLY
\input{main/chapter2/section1/content.tex}
     
% SERIOUS GAME REQUIREMENTS
\input{main/chapter2/section2/content.tex}    

% USE CASE DEFINITION
\input{main/chapter2/section3/content.tex}

% USE CASE REALIZATION
\input{main/chapter2/section4/content.tex}

% DATABASE DESIGN 
\input{main/chapter2/section5/content.tex}

% DATA MANIPULATION
\input{main/chapter2/section6/content.tex}

% COMMUNICATION 
\input{main/chapter2/section7/content.tex}

% TRAINING SCENARIOS
\input{main/chapter2/section8/content.tex}

% EMG GRAPHIC
\input{main/chapter2/section9/content.tex}

% STATICAL REPORTS GENERATION
\input{main/chapter2/section10/content.tex}

\subthesischapter{Conclusiones del capítulo}
Se presentó una descripción del sistema de adquisición de datos para rehabilitación, sus componentes, características distintivas y su funcionamiento. Se identificaron y definieron los requisitos del juego  serio, tanto funcionales como no funcionales, así como los actores y casos de usos del sistema que establecieron las bases fundamentales para el desarrollo de la aplicación. Se realizó el diseño de la base de datos, abarcando tanto el modelo lógico como el físico, lo que aseguró una estructura robusta y eficiente para el almacenamiento de los datos. La manipulación de los datos se abordó de manera integral, desde la conexión con la base de datos hasta la persistencia de los resultados estadísticos. Se diseñó e implementó la comunicación con el pedal motorizado y la implementación de la interfaz gráfica para la representación de los datos EMG. Se definieron los escenarios de entrenamiento para las modalidades Ligero y Clínico, asegurando una cobertura completa de las necesidad de entrenamiento del usuario. Por último en el ámbito estadístico se desarrolló una serie de gráficos para el seguimiento de los resultados en las rutinas de entrenamiento.   
    
\end{thesischapter}

% DATA MANIPULATION
\begin{thesischapter}{2} {Diseño e implementación del Juego Serio}
En este capítulo se discuten los detalles de desarrollo de los aspectos citados en el capítulo anterior. Este comienza con una descripción y caracterización general del sistema, donde se  abordan cada uno de los componentes requeridos para su completo funcionamiento. Posteriormente se detalla la ingienría de software requerida en la etapa de conceptualización de la aplicación, se explican de forma detallada los aspectos teóricos y de implementación de la base de datos, el funcionamiento del protocolo de comunicación y por último los escenarios de juegos requeridos en las rutinas de entrenamiento ligero y clínico, y las estadísticas generadas por estos. Como herramienta de desarrollo se utilizó c\#.

% SYSTEM DESCRIPTION AND CHARACTERIZATION TO APPLY
\input{main/chapter2/section1/content.tex}
     
% SERIOUS GAME REQUIREMENTS
\input{main/chapter2/section2/content.tex}    

% USE CASE DEFINITION
\input{main/chapter2/section3/content.tex}

% USE CASE REALIZATION
\input{main/chapter2/section4/content.tex}

% DATABASE DESIGN 
\input{main/chapter2/section5/content.tex}

% DATA MANIPULATION
\input{main/chapter2/section6/content.tex}

% COMMUNICATION 
\input{main/chapter2/section7/content.tex}

% TRAINING SCENARIOS
\input{main/chapter2/section8/content.tex}

% EMG GRAPHIC
\input{main/chapter2/section9/content.tex}

% STATICAL REPORTS GENERATION
\input{main/chapter2/section10/content.tex}

\subthesischapter{Conclusiones del capítulo}
Se presentó una descripción del sistema de adquisición de datos para rehabilitación, sus componentes, características distintivas y su funcionamiento. Se identificaron y definieron los requisitos del juego  serio, tanto funcionales como no funcionales, así como los actores y casos de usos del sistema que establecieron las bases fundamentales para el desarrollo de la aplicación. Se realizó el diseño de la base de datos, abarcando tanto el modelo lógico como el físico, lo que aseguró una estructura robusta y eficiente para el almacenamiento de los datos. La manipulación de los datos se abordó de manera integral, desde la conexión con la base de datos hasta la persistencia de los resultados estadísticos. Se diseñó e implementó la comunicación con el pedal motorizado y la implementación de la interfaz gráfica para la representación de los datos EMG. Se definieron los escenarios de entrenamiento para las modalidades Ligero y Clínico, asegurando una cobertura completa de las necesidad de entrenamiento del usuario. Por último en el ámbito estadístico se desarrolló una serie de gráficos para el seguimiento de los resultados en las rutinas de entrenamiento.   
    
\end{thesischapter}

% COMMUNICATION 
\begin{thesischapter}{2} {Diseño e implementación del Juego Serio}
En este capítulo se discuten los detalles de desarrollo de los aspectos citados en el capítulo anterior. Este comienza con una descripción y caracterización general del sistema, donde se  abordan cada uno de los componentes requeridos para su completo funcionamiento. Posteriormente se detalla la ingienría de software requerida en la etapa de conceptualización de la aplicación, se explican de forma detallada los aspectos teóricos y de implementación de la base de datos, el funcionamiento del protocolo de comunicación y por último los escenarios de juegos requeridos en las rutinas de entrenamiento ligero y clínico, y las estadísticas generadas por estos. Como herramienta de desarrollo se utilizó c\#.

% SYSTEM DESCRIPTION AND CHARACTERIZATION TO APPLY
\input{main/chapter2/section1/content.tex}
     
% SERIOUS GAME REQUIREMENTS
\input{main/chapter2/section2/content.tex}    

% USE CASE DEFINITION
\input{main/chapter2/section3/content.tex}

% USE CASE REALIZATION
\input{main/chapter2/section4/content.tex}

% DATABASE DESIGN 
\input{main/chapter2/section5/content.tex}

% DATA MANIPULATION
\input{main/chapter2/section6/content.tex}

% COMMUNICATION 
\input{main/chapter2/section7/content.tex}

% TRAINING SCENARIOS
\input{main/chapter2/section8/content.tex}

% EMG GRAPHIC
\input{main/chapter2/section9/content.tex}

% STATICAL REPORTS GENERATION
\input{main/chapter2/section10/content.tex}

\subthesischapter{Conclusiones del capítulo}
Se presentó una descripción del sistema de adquisición de datos para rehabilitación, sus componentes, características distintivas y su funcionamiento. Se identificaron y definieron los requisitos del juego  serio, tanto funcionales como no funcionales, así como los actores y casos de usos del sistema que establecieron las bases fundamentales para el desarrollo de la aplicación. Se realizó el diseño de la base de datos, abarcando tanto el modelo lógico como el físico, lo que aseguró una estructura robusta y eficiente para el almacenamiento de los datos. La manipulación de los datos se abordó de manera integral, desde la conexión con la base de datos hasta la persistencia de los resultados estadísticos. Se diseñó e implementó la comunicación con el pedal motorizado y la implementación de la interfaz gráfica para la representación de los datos EMG. Se definieron los escenarios de entrenamiento para las modalidades Ligero y Clínico, asegurando una cobertura completa de las necesidad de entrenamiento del usuario. Por último en el ámbito estadístico se desarrolló una serie de gráficos para el seguimiento de los resultados en las rutinas de entrenamiento.   
    
\end{thesischapter}

% TRAINING SCENARIOS
\begin{thesischapter}{2} {Diseño e implementación del Juego Serio}
En este capítulo se discuten los detalles de desarrollo de los aspectos citados en el capítulo anterior. Este comienza con una descripción y caracterización general del sistema, donde se  abordan cada uno de los componentes requeridos para su completo funcionamiento. Posteriormente se detalla la ingienría de software requerida en la etapa de conceptualización de la aplicación, se explican de forma detallada los aspectos teóricos y de implementación de la base de datos, el funcionamiento del protocolo de comunicación y por último los escenarios de juegos requeridos en las rutinas de entrenamiento ligero y clínico, y las estadísticas generadas por estos. Como herramienta de desarrollo se utilizó c\#.

% SYSTEM DESCRIPTION AND CHARACTERIZATION TO APPLY
\input{main/chapter2/section1/content.tex}
     
% SERIOUS GAME REQUIREMENTS
\input{main/chapter2/section2/content.tex}    

% USE CASE DEFINITION
\input{main/chapter2/section3/content.tex}

% USE CASE REALIZATION
\input{main/chapter2/section4/content.tex}

% DATABASE DESIGN 
\input{main/chapter2/section5/content.tex}

% DATA MANIPULATION
\input{main/chapter2/section6/content.tex}

% COMMUNICATION 
\input{main/chapter2/section7/content.tex}

% TRAINING SCENARIOS
\input{main/chapter2/section8/content.tex}

% EMG GRAPHIC
\input{main/chapter2/section9/content.tex}

% STATICAL REPORTS GENERATION
\input{main/chapter2/section10/content.tex}

\subthesischapter{Conclusiones del capítulo}
Se presentó una descripción del sistema de adquisición de datos para rehabilitación, sus componentes, características distintivas y su funcionamiento. Se identificaron y definieron los requisitos del juego  serio, tanto funcionales como no funcionales, así como los actores y casos de usos del sistema que establecieron las bases fundamentales para el desarrollo de la aplicación. Se realizó el diseño de la base de datos, abarcando tanto el modelo lógico como el físico, lo que aseguró una estructura robusta y eficiente para el almacenamiento de los datos. La manipulación de los datos se abordó de manera integral, desde la conexión con la base de datos hasta la persistencia de los resultados estadísticos. Se diseñó e implementó la comunicación con el pedal motorizado y la implementación de la interfaz gráfica para la representación de los datos EMG. Se definieron los escenarios de entrenamiento para las modalidades Ligero y Clínico, asegurando una cobertura completa de las necesidad de entrenamiento del usuario. Por último en el ámbito estadístico se desarrolló una serie de gráficos para el seguimiento de los resultados en las rutinas de entrenamiento.   
    
\end{thesischapter}

% EMG GRAPHIC
\begin{thesischapter}{2} {Diseño e implementación del Juego Serio}
En este capítulo se discuten los detalles de desarrollo de los aspectos citados en el capítulo anterior. Este comienza con una descripción y caracterización general del sistema, donde se  abordan cada uno de los componentes requeridos para su completo funcionamiento. Posteriormente se detalla la ingienría de software requerida en la etapa de conceptualización de la aplicación, se explican de forma detallada los aspectos teóricos y de implementación de la base de datos, el funcionamiento del protocolo de comunicación y por último los escenarios de juegos requeridos en las rutinas de entrenamiento ligero y clínico, y las estadísticas generadas por estos. Como herramienta de desarrollo se utilizó c\#.

% SYSTEM DESCRIPTION AND CHARACTERIZATION TO APPLY
\input{main/chapter2/section1/content.tex}
     
% SERIOUS GAME REQUIREMENTS
\input{main/chapter2/section2/content.tex}    

% USE CASE DEFINITION
\input{main/chapter2/section3/content.tex}

% USE CASE REALIZATION
\input{main/chapter2/section4/content.tex}

% DATABASE DESIGN 
\input{main/chapter2/section5/content.tex}

% DATA MANIPULATION
\input{main/chapter2/section6/content.tex}

% COMMUNICATION 
\input{main/chapter2/section7/content.tex}

% TRAINING SCENARIOS
\input{main/chapter2/section8/content.tex}

% EMG GRAPHIC
\input{main/chapter2/section9/content.tex}

% STATICAL REPORTS GENERATION
\input{main/chapter2/section10/content.tex}

\subthesischapter{Conclusiones del capítulo}
Se presentó una descripción del sistema de adquisición de datos para rehabilitación, sus componentes, características distintivas y su funcionamiento. Se identificaron y definieron los requisitos del juego  serio, tanto funcionales como no funcionales, así como los actores y casos de usos del sistema que establecieron las bases fundamentales para el desarrollo de la aplicación. Se realizó el diseño de la base de datos, abarcando tanto el modelo lógico como el físico, lo que aseguró una estructura robusta y eficiente para el almacenamiento de los datos. La manipulación de los datos se abordó de manera integral, desde la conexión con la base de datos hasta la persistencia de los resultados estadísticos. Se diseñó e implementó la comunicación con el pedal motorizado y la implementación de la interfaz gráfica para la representación de los datos EMG. Se definieron los escenarios de entrenamiento para las modalidades Ligero y Clínico, asegurando una cobertura completa de las necesidad de entrenamiento del usuario. Por último en el ámbito estadístico se desarrolló una serie de gráficos para el seguimiento de los resultados en las rutinas de entrenamiento.   
    
\end{thesischapter}

% STATICAL REPORTS GENERATION
\begin{thesischapter}{2} {Diseño e implementación del Juego Serio}
En este capítulo se discuten los detalles de desarrollo de los aspectos citados en el capítulo anterior. Este comienza con una descripción y caracterización general del sistema, donde se  abordan cada uno de los componentes requeridos para su completo funcionamiento. Posteriormente se detalla la ingienría de software requerida en la etapa de conceptualización de la aplicación, se explican de forma detallada los aspectos teóricos y de implementación de la base de datos, el funcionamiento del protocolo de comunicación y por último los escenarios de juegos requeridos en las rutinas de entrenamiento ligero y clínico, y las estadísticas generadas por estos. Como herramienta de desarrollo se utilizó c\#.

% SYSTEM DESCRIPTION AND CHARACTERIZATION TO APPLY
\input{main/chapter2/section1/content.tex}
     
% SERIOUS GAME REQUIREMENTS
\input{main/chapter2/section2/content.tex}    

% USE CASE DEFINITION
\input{main/chapter2/section3/content.tex}

% USE CASE REALIZATION
\input{main/chapter2/section4/content.tex}

% DATABASE DESIGN 
\input{main/chapter2/section5/content.tex}

% DATA MANIPULATION
\input{main/chapter2/section6/content.tex}

% COMMUNICATION 
\input{main/chapter2/section7/content.tex}

% TRAINING SCENARIOS
\input{main/chapter2/section8/content.tex}

% EMG GRAPHIC
\input{main/chapter2/section9/content.tex}

% STATICAL REPORTS GENERATION
\input{main/chapter2/section10/content.tex}

\subthesischapter{Conclusiones del capítulo}
Se presentó una descripción del sistema de adquisición de datos para rehabilitación, sus componentes, características distintivas y su funcionamiento. Se identificaron y definieron los requisitos del juego  serio, tanto funcionales como no funcionales, así como los actores y casos de usos del sistema que establecieron las bases fundamentales para el desarrollo de la aplicación. Se realizó el diseño de la base de datos, abarcando tanto el modelo lógico como el físico, lo que aseguró una estructura robusta y eficiente para el almacenamiento de los datos. La manipulación de los datos se abordó de manera integral, desde la conexión con la base de datos hasta la persistencia de los resultados estadísticos. Se diseñó e implementó la comunicación con el pedal motorizado y la implementación de la interfaz gráfica para la representación de los datos EMG. Se definieron los escenarios de entrenamiento para las modalidades Ligero y Clínico, asegurando una cobertura completa de las necesidad de entrenamiento del usuario. Por último en el ámbito estadístico se desarrolló una serie de gráficos para el seguimiento de los resultados en las rutinas de entrenamiento.   
    
\end{thesischapter}

\subthesischapter{Conclusiones del capítulo}
Se presentó una descripción del sistema de adquisición de datos para rehabilitación, sus componentes, características distintivas y su funcionamiento. Se identificaron y definieron los requisitos del juego  serio, tanto funcionales como no funcionales, así como los actores y casos de usos del sistema que establecieron las bases fundamentales para el desarrollo de la aplicación. Se realizó el diseño de la base de datos, abarcando tanto el modelo lógico como el físico, lo que aseguró una estructura robusta y eficiente para el almacenamiento de los datos. La manipulación de los datos se abordó de manera integral, desde la conexión con la base de datos hasta la persistencia de los resultados estadísticos. Se diseñó e implementó la comunicación con el pedal motorizado y la implementación de la interfaz gráfica para la representación de los datos EMG. Se definieron los escenarios de entrenamiento para las modalidades Ligero y Clínico, asegurando una cobertura completa de las necesidad de entrenamiento del usuario. Por último en el ámbito estadístico se desarrolló una serie de gráficos para el seguimiento de los resultados en las rutinas de entrenamiento.   
    
\end{thesischapter}

% DATA MANIPULATION
\begin{thesischapter}{2} {Diseño e implementación del Juego Serio}
En este capítulo se discuten los detalles de desarrollo de los aspectos citados en el capítulo anterior. Este comienza con una descripción y caracterización general del sistema, donde se  abordan cada uno de los componentes requeridos para su completo funcionamiento. Posteriormente se detalla la ingienría de software requerida en la etapa de conceptualización de la aplicación, se explican de forma detallada los aspectos teóricos y de implementación de la base de datos, el funcionamiento del protocolo de comunicación y por último los escenarios de juegos requeridos en las rutinas de entrenamiento ligero y clínico, y las estadísticas generadas por estos. Como herramienta de desarrollo se utilizó c\#.

% SYSTEM DESCRIPTION AND CHARACTERIZATION TO APPLY
\begin{thesischapter}{2} {Diseño e implementación del Juego Serio}
En este capítulo se discuten los detalles de desarrollo de los aspectos citados en el capítulo anterior. Este comienza con una descripción y caracterización general del sistema, donde se  abordan cada uno de los componentes requeridos para su completo funcionamiento. Posteriormente se detalla la ingienría de software requerida en la etapa de conceptualización de la aplicación, se explican de forma detallada los aspectos teóricos y de implementación de la base de datos, el funcionamiento del protocolo de comunicación y por último los escenarios de juegos requeridos en las rutinas de entrenamiento ligero y clínico, y las estadísticas generadas por estos. Como herramienta de desarrollo se utilizó c\#.

% SYSTEM DESCRIPTION AND CHARACTERIZATION TO APPLY
\input{main/chapter2/section1/content.tex}
     
% SERIOUS GAME REQUIREMENTS
\input{main/chapter2/section2/content.tex}    

% USE CASE DEFINITION
\input{main/chapter2/section3/content.tex}

% USE CASE REALIZATION
\input{main/chapter2/section4/content.tex}

% DATABASE DESIGN 
\input{main/chapter2/section5/content.tex}

% DATA MANIPULATION
\input{main/chapter2/section6/content.tex}

% COMMUNICATION 
\input{main/chapter2/section7/content.tex}

% TRAINING SCENARIOS
\input{main/chapter2/section8/content.tex}

% EMG GRAPHIC
\input{main/chapter2/section9/content.tex}

% STATICAL REPORTS GENERATION
\input{main/chapter2/section10/content.tex}

\subthesischapter{Conclusiones del capítulo}
Se presentó una descripción del sistema de adquisición de datos para rehabilitación, sus componentes, características distintivas y su funcionamiento. Se identificaron y definieron los requisitos del juego  serio, tanto funcionales como no funcionales, así como los actores y casos de usos del sistema que establecieron las bases fundamentales para el desarrollo de la aplicación. Se realizó el diseño de la base de datos, abarcando tanto el modelo lógico como el físico, lo que aseguró una estructura robusta y eficiente para el almacenamiento de los datos. La manipulación de los datos se abordó de manera integral, desde la conexión con la base de datos hasta la persistencia de los resultados estadísticos. Se diseñó e implementó la comunicación con el pedal motorizado y la implementación de la interfaz gráfica para la representación de los datos EMG. Se definieron los escenarios de entrenamiento para las modalidades Ligero y Clínico, asegurando una cobertura completa de las necesidad de entrenamiento del usuario. Por último en el ámbito estadístico se desarrolló una serie de gráficos para el seguimiento de los resultados en las rutinas de entrenamiento.   
    
\end{thesischapter}
     
% SERIOUS GAME REQUIREMENTS
\begin{thesischapter}{2} {Diseño e implementación del Juego Serio}
En este capítulo se discuten los detalles de desarrollo de los aspectos citados en el capítulo anterior. Este comienza con una descripción y caracterización general del sistema, donde se  abordan cada uno de los componentes requeridos para su completo funcionamiento. Posteriormente se detalla la ingienría de software requerida en la etapa de conceptualización de la aplicación, se explican de forma detallada los aspectos teóricos y de implementación de la base de datos, el funcionamiento del protocolo de comunicación y por último los escenarios de juegos requeridos en las rutinas de entrenamiento ligero y clínico, y las estadísticas generadas por estos. Como herramienta de desarrollo se utilizó c\#.

% SYSTEM DESCRIPTION AND CHARACTERIZATION TO APPLY
\input{main/chapter2/section1/content.tex}
     
% SERIOUS GAME REQUIREMENTS
\input{main/chapter2/section2/content.tex}    

% USE CASE DEFINITION
\input{main/chapter2/section3/content.tex}

% USE CASE REALIZATION
\input{main/chapter2/section4/content.tex}

% DATABASE DESIGN 
\input{main/chapter2/section5/content.tex}

% DATA MANIPULATION
\input{main/chapter2/section6/content.tex}

% COMMUNICATION 
\input{main/chapter2/section7/content.tex}

% TRAINING SCENARIOS
\input{main/chapter2/section8/content.tex}

% EMG GRAPHIC
\input{main/chapter2/section9/content.tex}

% STATICAL REPORTS GENERATION
\input{main/chapter2/section10/content.tex}

\subthesischapter{Conclusiones del capítulo}
Se presentó una descripción del sistema de adquisición de datos para rehabilitación, sus componentes, características distintivas y su funcionamiento. Se identificaron y definieron los requisitos del juego  serio, tanto funcionales como no funcionales, así como los actores y casos de usos del sistema que establecieron las bases fundamentales para el desarrollo de la aplicación. Se realizó el diseño de la base de datos, abarcando tanto el modelo lógico como el físico, lo que aseguró una estructura robusta y eficiente para el almacenamiento de los datos. La manipulación de los datos se abordó de manera integral, desde la conexión con la base de datos hasta la persistencia de los resultados estadísticos. Se diseñó e implementó la comunicación con el pedal motorizado y la implementación de la interfaz gráfica para la representación de los datos EMG. Se definieron los escenarios de entrenamiento para las modalidades Ligero y Clínico, asegurando una cobertura completa de las necesidad de entrenamiento del usuario. Por último en el ámbito estadístico se desarrolló una serie de gráficos para el seguimiento de los resultados en las rutinas de entrenamiento.   
    
\end{thesischapter}    

% USE CASE DEFINITION
\begin{thesischapter}{2} {Diseño e implementación del Juego Serio}
En este capítulo se discuten los detalles de desarrollo de los aspectos citados en el capítulo anterior. Este comienza con una descripción y caracterización general del sistema, donde se  abordan cada uno de los componentes requeridos para su completo funcionamiento. Posteriormente se detalla la ingienría de software requerida en la etapa de conceptualización de la aplicación, se explican de forma detallada los aspectos teóricos y de implementación de la base de datos, el funcionamiento del protocolo de comunicación y por último los escenarios de juegos requeridos en las rutinas de entrenamiento ligero y clínico, y las estadísticas generadas por estos. Como herramienta de desarrollo se utilizó c\#.

% SYSTEM DESCRIPTION AND CHARACTERIZATION TO APPLY
\input{main/chapter2/section1/content.tex}
     
% SERIOUS GAME REQUIREMENTS
\input{main/chapter2/section2/content.tex}    

% USE CASE DEFINITION
\input{main/chapter2/section3/content.tex}

% USE CASE REALIZATION
\input{main/chapter2/section4/content.tex}

% DATABASE DESIGN 
\input{main/chapter2/section5/content.tex}

% DATA MANIPULATION
\input{main/chapter2/section6/content.tex}

% COMMUNICATION 
\input{main/chapter2/section7/content.tex}

% TRAINING SCENARIOS
\input{main/chapter2/section8/content.tex}

% EMG GRAPHIC
\input{main/chapter2/section9/content.tex}

% STATICAL REPORTS GENERATION
\input{main/chapter2/section10/content.tex}

\subthesischapter{Conclusiones del capítulo}
Se presentó una descripción del sistema de adquisición de datos para rehabilitación, sus componentes, características distintivas y su funcionamiento. Se identificaron y definieron los requisitos del juego  serio, tanto funcionales como no funcionales, así como los actores y casos de usos del sistema que establecieron las bases fundamentales para el desarrollo de la aplicación. Se realizó el diseño de la base de datos, abarcando tanto el modelo lógico como el físico, lo que aseguró una estructura robusta y eficiente para el almacenamiento de los datos. La manipulación de los datos se abordó de manera integral, desde la conexión con la base de datos hasta la persistencia de los resultados estadísticos. Se diseñó e implementó la comunicación con el pedal motorizado y la implementación de la interfaz gráfica para la representación de los datos EMG. Se definieron los escenarios de entrenamiento para las modalidades Ligero y Clínico, asegurando una cobertura completa de las necesidad de entrenamiento del usuario. Por último en el ámbito estadístico se desarrolló una serie de gráficos para el seguimiento de los resultados en las rutinas de entrenamiento.   
    
\end{thesischapter}

% USE CASE REALIZATION
\begin{thesischapter}{2} {Diseño e implementación del Juego Serio}
En este capítulo se discuten los detalles de desarrollo de los aspectos citados en el capítulo anterior. Este comienza con una descripción y caracterización general del sistema, donde se  abordan cada uno de los componentes requeridos para su completo funcionamiento. Posteriormente se detalla la ingienría de software requerida en la etapa de conceptualización de la aplicación, se explican de forma detallada los aspectos teóricos y de implementación de la base de datos, el funcionamiento del protocolo de comunicación y por último los escenarios de juegos requeridos en las rutinas de entrenamiento ligero y clínico, y las estadísticas generadas por estos. Como herramienta de desarrollo se utilizó c\#.

% SYSTEM DESCRIPTION AND CHARACTERIZATION TO APPLY
\input{main/chapter2/section1/content.tex}
     
% SERIOUS GAME REQUIREMENTS
\input{main/chapter2/section2/content.tex}    

% USE CASE DEFINITION
\input{main/chapter2/section3/content.tex}

% USE CASE REALIZATION
\input{main/chapter2/section4/content.tex}

% DATABASE DESIGN 
\input{main/chapter2/section5/content.tex}

% DATA MANIPULATION
\input{main/chapter2/section6/content.tex}

% COMMUNICATION 
\input{main/chapter2/section7/content.tex}

% TRAINING SCENARIOS
\input{main/chapter2/section8/content.tex}

% EMG GRAPHIC
\input{main/chapter2/section9/content.tex}

% STATICAL REPORTS GENERATION
\input{main/chapter2/section10/content.tex}

\subthesischapter{Conclusiones del capítulo}
Se presentó una descripción del sistema de adquisición de datos para rehabilitación, sus componentes, características distintivas y su funcionamiento. Se identificaron y definieron los requisitos del juego  serio, tanto funcionales como no funcionales, así como los actores y casos de usos del sistema que establecieron las bases fundamentales para el desarrollo de la aplicación. Se realizó el diseño de la base de datos, abarcando tanto el modelo lógico como el físico, lo que aseguró una estructura robusta y eficiente para el almacenamiento de los datos. La manipulación de los datos se abordó de manera integral, desde la conexión con la base de datos hasta la persistencia de los resultados estadísticos. Se diseñó e implementó la comunicación con el pedal motorizado y la implementación de la interfaz gráfica para la representación de los datos EMG. Se definieron los escenarios de entrenamiento para las modalidades Ligero y Clínico, asegurando una cobertura completa de las necesidad de entrenamiento del usuario. Por último en el ámbito estadístico se desarrolló una serie de gráficos para el seguimiento de los resultados en las rutinas de entrenamiento.   
    
\end{thesischapter}

% DATABASE DESIGN 
\begin{thesischapter}{2} {Diseño e implementación del Juego Serio}
En este capítulo se discuten los detalles de desarrollo de los aspectos citados en el capítulo anterior. Este comienza con una descripción y caracterización general del sistema, donde se  abordan cada uno de los componentes requeridos para su completo funcionamiento. Posteriormente se detalla la ingienría de software requerida en la etapa de conceptualización de la aplicación, se explican de forma detallada los aspectos teóricos y de implementación de la base de datos, el funcionamiento del protocolo de comunicación y por último los escenarios de juegos requeridos en las rutinas de entrenamiento ligero y clínico, y las estadísticas generadas por estos. Como herramienta de desarrollo se utilizó c\#.

% SYSTEM DESCRIPTION AND CHARACTERIZATION TO APPLY
\input{main/chapter2/section1/content.tex}
     
% SERIOUS GAME REQUIREMENTS
\input{main/chapter2/section2/content.tex}    

% USE CASE DEFINITION
\input{main/chapter2/section3/content.tex}

% USE CASE REALIZATION
\input{main/chapter2/section4/content.tex}

% DATABASE DESIGN 
\input{main/chapter2/section5/content.tex}

% DATA MANIPULATION
\input{main/chapter2/section6/content.tex}

% COMMUNICATION 
\input{main/chapter2/section7/content.tex}

% TRAINING SCENARIOS
\input{main/chapter2/section8/content.tex}

% EMG GRAPHIC
\input{main/chapter2/section9/content.tex}

% STATICAL REPORTS GENERATION
\input{main/chapter2/section10/content.tex}

\subthesischapter{Conclusiones del capítulo}
Se presentó una descripción del sistema de adquisición de datos para rehabilitación, sus componentes, características distintivas y su funcionamiento. Se identificaron y definieron los requisitos del juego  serio, tanto funcionales como no funcionales, así como los actores y casos de usos del sistema que establecieron las bases fundamentales para el desarrollo de la aplicación. Se realizó el diseño de la base de datos, abarcando tanto el modelo lógico como el físico, lo que aseguró una estructura robusta y eficiente para el almacenamiento de los datos. La manipulación de los datos se abordó de manera integral, desde la conexión con la base de datos hasta la persistencia de los resultados estadísticos. Se diseñó e implementó la comunicación con el pedal motorizado y la implementación de la interfaz gráfica para la representación de los datos EMG. Se definieron los escenarios de entrenamiento para las modalidades Ligero y Clínico, asegurando una cobertura completa de las necesidad de entrenamiento del usuario. Por último en el ámbito estadístico se desarrolló una serie de gráficos para el seguimiento de los resultados en las rutinas de entrenamiento.   
    
\end{thesischapter}

% DATA MANIPULATION
\begin{thesischapter}{2} {Diseño e implementación del Juego Serio}
En este capítulo se discuten los detalles de desarrollo de los aspectos citados en el capítulo anterior. Este comienza con una descripción y caracterización general del sistema, donde se  abordan cada uno de los componentes requeridos para su completo funcionamiento. Posteriormente se detalla la ingienría de software requerida en la etapa de conceptualización de la aplicación, se explican de forma detallada los aspectos teóricos y de implementación de la base de datos, el funcionamiento del protocolo de comunicación y por último los escenarios de juegos requeridos en las rutinas de entrenamiento ligero y clínico, y las estadísticas generadas por estos. Como herramienta de desarrollo se utilizó c\#.

% SYSTEM DESCRIPTION AND CHARACTERIZATION TO APPLY
\input{main/chapter2/section1/content.tex}
     
% SERIOUS GAME REQUIREMENTS
\input{main/chapter2/section2/content.tex}    

% USE CASE DEFINITION
\input{main/chapter2/section3/content.tex}

% USE CASE REALIZATION
\input{main/chapter2/section4/content.tex}

% DATABASE DESIGN 
\input{main/chapter2/section5/content.tex}

% DATA MANIPULATION
\input{main/chapter2/section6/content.tex}

% COMMUNICATION 
\input{main/chapter2/section7/content.tex}

% TRAINING SCENARIOS
\input{main/chapter2/section8/content.tex}

% EMG GRAPHIC
\input{main/chapter2/section9/content.tex}

% STATICAL REPORTS GENERATION
\input{main/chapter2/section10/content.tex}

\subthesischapter{Conclusiones del capítulo}
Se presentó una descripción del sistema de adquisición de datos para rehabilitación, sus componentes, características distintivas y su funcionamiento. Se identificaron y definieron los requisitos del juego  serio, tanto funcionales como no funcionales, así como los actores y casos de usos del sistema que establecieron las bases fundamentales para el desarrollo de la aplicación. Se realizó el diseño de la base de datos, abarcando tanto el modelo lógico como el físico, lo que aseguró una estructura robusta y eficiente para el almacenamiento de los datos. La manipulación de los datos se abordó de manera integral, desde la conexión con la base de datos hasta la persistencia de los resultados estadísticos. Se diseñó e implementó la comunicación con el pedal motorizado y la implementación de la interfaz gráfica para la representación de los datos EMG. Se definieron los escenarios de entrenamiento para las modalidades Ligero y Clínico, asegurando una cobertura completa de las necesidad de entrenamiento del usuario. Por último en el ámbito estadístico se desarrolló una serie de gráficos para el seguimiento de los resultados en las rutinas de entrenamiento.   
    
\end{thesischapter}

% COMMUNICATION 
\begin{thesischapter}{2} {Diseño e implementación del Juego Serio}
En este capítulo se discuten los detalles de desarrollo de los aspectos citados en el capítulo anterior. Este comienza con una descripción y caracterización general del sistema, donde se  abordan cada uno de los componentes requeridos para su completo funcionamiento. Posteriormente se detalla la ingienría de software requerida en la etapa de conceptualización de la aplicación, se explican de forma detallada los aspectos teóricos y de implementación de la base de datos, el funcionamiento del protocolo de comunicación y por último los escenarios de juegos requeridos en las rutinas de entrenamiento ligero y clínico, y las estadísticas generadas por estos. Como herramienta de desarrollo se utilizó c\#.

% SYSTEM DESCRIPTION AND CHARACTERIZATION TO APPLY
\input{main/chapter2/section1/content.tex}
     
% SERIOUS GAME REQUIREMENTS
\input{main/chapter2/section2/content.tex}    

% USE CASE DEFINITION
\input{main/chapter2/section3/content.tex}

% USE CASE REALIZATION
\input{main/chapter2/section4/content.tex}

% DATABASE DESIGN 
\input{main/chapter2/section5/content.tex}

% DATA MANIPULATION
\input{main/chapter2/section6/content.tex}

% COMMUNICATION 
\input{main/chapter2/section7/content.tex}

% TRAINING SCENARIOS
\input{main/chapter2/section8/content.tex}

% EMG GRAPHIC
\input{main/chapter2/section9/content.tex}

% STATICAL REPORTS GENERATION
\input{main/chapter2/section10/content.tex}

\subthesischapter{Conclusiones del capítulo}
Se presentó una descripción del sistema de adquisición de datos para rehabilitación, sus componentes, características distintivas y su funcionamiento. Se identificaron y definieron los requisitos del juego  serio, tanto funcionales como no funcionales, así como los actores y casos de usos del sistema que establecieron las bases fundamentales para el desarrollo de la aplicación. Se realizó el diseño de la base de datos, abarcando tanto el modelo lógico como el físico, lo que aseguró una estructura robusta y eficiente para el almacenamiento de los datos. La manipulación de los datos se abordó de manera integral, desde la conexión con la base de datos hasta la persistencia de los resultados estadísticos. Se diseñó e implementó la comunicación con el pedal motorizado y la implementación de la interfaz gráfica para la representación de los datos EMG. Se definieron los escenarios de entrenamiento para las modalidades Ligero y Clínico, asegurando una cobertura completa de las necesidad de entrenamiento del usuario. Por último en el ámbito estadístico se desarrolló una serie de gráficos para el seguimiento de los resultados en las rutinas de entrenamiento.   
    
\end{thesischapter}

% TRAINING SCENARIOS
\begin{thesischapter}{2} {Diseño e implementación del Juego Serio}
En este capítulo se discuten los detalles de desarrollo de los aspectos citados en el capítulo anterior. Este comienza con una descripción y caracterización general del sistema, donde se  abordan cada uno de los componentes requeridos para su completo funcionamiento. Posteriormente se detalla la ingienría de software requerida en la etapa de conceptualización de la aplicación, se explican de forma detallada los aspectos teóricos y de implementación de la base de datos, el funcionamiento del protocolo de comunicación y por último los escenarios de juegos requeridos en las rutinas de entrenamiento ligero y clínico, y las estadísticas generadas por estos. Como herramienta de desarrollo se utilizó c\#.

% SYSTEM DESCRIPTION AND CHARACTERIZATION TO APPLY
\input{main/chapter2/section1/content.tex}
     
% SERIOUS GAME REQUIREMENTS
\input{main/chapter2/section2/content.tex}    

% USE CASE DEFINITION
\input{main/chapter2/section3/content.tex}

% USE CASE REALIZATION
\input{main/chapter2/section4/content.tex}

% DATABASE DESIGN 
\input{main/chapter2/section5/content.tex}

% DATA MANIPULATION
\input{main/chapter2/section6/content.tex}

% COMMUNICATION 
\input{main/chapter2/section7/content.tex}

% TRAINING SCENARIOS
\input{main/chapter2/section8/content.tex}

% EMG GRAPHIC
\input{main/chapter2/section9/content.tex}

% STATICAL REPORTS GENERATION
\input{main/chapter2/section10/content.tex}

\subthesischapter{Conclusiones del capítulo}
Se presentó una descripción del sistema de adquisición de datos para rehabilitación, sus componentes, características distintivas y su funcionamiento. Se identificaron y definieron los requisitos del juego  serio, tanto funcionales como no funcionales, así como los actores y casos de usos del sistema que establecieron las bases fundamentales para el desarrollo de la aplicación. Se realizó el diseño de la base de datos, abarcando tanto el modelo lógico como el físico, lo que aseguró una estructura robusta y eficiente para el almacenamiento de los datos. La manipulación de los datos se abordó de manera integral, desde la conexión con la base de datos hasta la persistencia de los resultados estadísticos. Se diseñó e implementó la comunicación con el pedal motorizado y la implementación de la interfaz gráfica para la representación de los datos EMG. Se definieron los escenarios de entrenamiento para las modalidades Ligero y Clínico, asegurando una cobertura completa de las necesidad de entrenamiento del usuario. Por último en el ámbito estadístico se desarrolló una serie de gráficos para el seguimiento de los resultados en las rutinas de entrenamiento.   
    
\end{thesischapter}

% EMG GRAPHIC
\begin{thesischapter}{2} {Diseño e implementación del Juego Serio}
En este capítulo se discuten los detalles de desarrollo de los aspectos citados en el capítulo anterior. Este comienza con una descripción y caracterización general del sistema, donde se  abordan cada uno de los componentes requeridos para su completo funcionamiento. Posteriormente se detalla la ingienría de software requerida en la etapa de conceptualización de la aplicación, se explican de forma detallada los aspectos teóricos y de implementación de la base de datos, el funcionamiento del protocolo de comunicación y por último los escenarios de juegos requeridos en las rutinas de entrenamiento ligero y clínico, y las estadísticas generadas por estos. Como herramienta de desarrollo se utilizó c\#.

% SYSTEM DESCRIPTION AND CHARACTERIZATION TO APPLY
\input{main/chapter2/section1/content.tex}
     
% SERIOUS GAME REQUIREMENTS
\input{main/chapter2/section2/content.tex}    

% USE CASE DEFINITION
\input{main/chapter2/section3/content.tex}

% USE CASE REALIZATION
\input{main/chapter2/section4/content.tex}

% DATABASE DESIGN 
\input{main/chapter2/section5/content.tex}

% DATA MANIPULATION
\input{main/chapter2/section6/content.tex}

% COMMUNICATION 
\input{main/chapter2/section7/content.tex}

% TRAINING SCENARIOS
\input{main/chapter2/section8/content.tex}

% EMG GRAPHIC
\input{main/chapter2/section9/content.tex}

% STATICAL REPORTS GENERATION
\input{main/chapter2/section10/content.tex}

\subthesischapter{Conclusiones del capítulo}
Se presentó una descripción del sistema de adquisición de datos para rehabilitación, sus componentes, características distintivas y su funcionamiento. Se identificaron y definieron los requisitos del juego  serio, tanto funcionales como no funcionales, así como los actores y casos de usos del sistema que establecieron las bases fundamentales para el desarrollo de la aplicación. Se realizó el diseño de la base de datos, abarcando tanto el modelo lógico como el físico, lo que aseguró una estructura robusta y eficiente para el almacenamiento de los datos. La manipulación de los datos se abordó de manera integral, desde la conexión con la base de datos hasta la persistencia de los resultados estadísticos. Se diseñó e implementó la comunicación con el pedal motorizado y la implementación de la interfaz gráfica para la representación de los datos EMG. Se definieron los escenarios de entrenamiento para las modalidades Ligero y Clínico, asegurando una cobertura completa de las necesidad de entrenamiento del usuario. Por último en el ámbito estadístico se desarrolló una serie de gráficos para el seguimiento de los resultados en las rutinas de entrenamiento.   
    
\end{thesischapter}

% STATICAL REPORTS GENERATION
\begin{thesischapter}{2} {Diseño e implementación del Juego Serio}
En este capítulo se discuten los detalles de desarrollo de los aspectos citados en el capítulo anterior. Este comienza con una descripción y caracterización general del sistema, donde se  abordan cada uno de los componentes requeridos para su completo funcionamiento. Posteriormente se detalla la ingienría de software requerida en la etapa de conceptualización de la aplicación, se explican de forma detallada los aspectos teóricos y de implementación de la base de datos, el funcionamiento del protocolo de comunicación y por último los escenarios de juegos requeridos en las rutinas de entrenamiento ligero y clínico, y las estadísticas generadas por estos. Como herramienta de desarrollo se utilizó c\#.

% SYSTEM DESCRIPTION AND CHARACTERIZATION TO APPLY
\input{main/chapter2/section1/content.tex}
     
% SERIOUS GAME REQUIREMENTS
\input{main/chapter2/section2/content.tex}    

% USE CASE DEFINITION
\input{main/chapter2/section3/content.tex}

% USE CASE REALIZATION
\input{main/chapter2/section4/content.tex}

% DATABASE DESIGN 
\input{main/chapter2/section5/content.tex}

% DATA MANIPULATION
\input{main/chapter2/section6/content.tex}

% COMMUNICATION 
\input{main/chapter2/section7/content.tex}

% TRAINING SCENARIOS
\input{main/chapter2/section8/content.tex}

% EMG GRAPHIC
\input{main/chapter2/section9/content.tex}

% STATICAL REPORTS GENERATION
\input{main/chapter2/section10/content.tex}

\subthesischapter{Conclusiones del capítulo}
Se presentó una descripción del sistema de adquisición de datos para rehabilitación, sus componentes, características distintivas y su funcionamiento. Se identificaron y definieron los requisitos del juego  serio, tanto funcionales como no funcionales, así como los actores y casos de usos del sistema que establecieron las bases fundamentales para el desarrollo de la aplicación. Se realizó el diseño de la base de datos, abarcando tanto el modelo lógico como el físico, lo que aseguró una estructura robusta y eficiente para el almacenamiento de los datos. La manipulación de los datos se abordó de manera integral, desde la conexión con la base de datos hasta la persistencia de los resultados estadísticos. Se diseñó e implementó la comunicación con el pedal motorizado y la implementación de la interfaz gráfica para la representación de los datos EMG. Se definieron los escenarios de entrenamiento para las modalidades Ligero y Clínico, asegurando una cobertura completa de las necesidad de entrenamiento del usuario. Por último en el ámbito estadístico se desarrolló una serie de gráficos para el seguimiento de los resultados en las rutinas de entrenamiento.   
    
\end{thesischapter}

\subthesischapter{Conclusiones del capítulo}
Se presentó una descripción del sistema de adquisición de datos para rehabilitación, sus componentes, características distintivas y su funcionamiento. Se identificaron y definieron los requisitos del juego  serio, tanto funcionales como no funcionales, así como los actores y casos de usos del sistema que establecieron las bases fundamentales para el desarrollo de la aplicación. Se realizó el diseño de la base de datos, abarcando tanto el modelo lógico como el físico, lo que aseguró una estructura robusta y eficiente para el almacenamiento de los datos. La manipulación de los datos se abordó de manera integral, desde la conexión con la base de datos hasta la persistencia de los resultados estadísticos. Se diseñó e implementó la comunicación con el pedal motorizado y la implementación de la interfaz gráfica para la representación de los datos EMG. Se definieron los escenarios de entrenamiento para las modalidades Ligero y Clínico, asegurando una cobertura completa de las necesidad de entrenamiento del usuario. Por último en el ámbito estadístico se desarrolló una serie de gráficos para el seguimiento de los resultados en las rutinas de entrenamiento.   
    
\end{thesischapter}

% COMMUNICATION 
\begin{thesischapter}{2} {Diseño e implementación del Juego Serio}
En este capítulo se discuten los detalles de desarrollo de los aspectos citados en el capítulo anterior. Este comienza con una descripción y caracterización general del sistema, donde se  abordan cada uno de los componentes requeridos para su completo funcionamiento. Posteriormente se detalla la ingienría de software requerida en la etapa de conceptualización de la aplicación, se explican de forma detallada los aspectos teóricos y de implementación de la base de datos, el funcionamiento del protocolo de comunicación y por último los escenarios de juegos requeridos en las rutinas de entrenamiento ligero y clínico, y las estadísticas generadas por estos. Como herramienta de desarrollo se utilizó c\#.

% SYSTEM DESCRIPTION AND CHARACTERIZATION TO APPLY
\begin{thesischapter}{2} {Diseño e implementación del Juego Serio}
En este capítulo se discuten los detalles de desarrollo de los aspectos citados en el capítulo anterior. Este comienza con una descripción y caracterización general del sistema, donde se  abordan cada uno de los componentes requeridos para su completo funcionamiento. Posteriormente se detalla la ingienría de software requerida en la etapa de conceptualización de la aplicación, se explican de forma detallada los aspectos teóricos y de implementación de la base de datos, el funcionamiento del protocolo de comunicación y por último los escenarios de juegos requeridos en las rutinas de entrenamiento ligero y clínico, y las estadísticas generadas por estos. Como herramienta de desarrollo se utilizó c\#.

% SYSTEM DESCRIPTION AND CHARACTERIZATION TO APPLY
\input{main/chapter2/section1/content.tex}
     
% SERIOUS GAME REQUIREMENTS
\input{main/chapter2/section2/content.tex}    

% USE CASE DEFINITION
\input{main/chapter2/section3/content.tex}

% USE CASE REALIZATION
\input{main/chapter2/section4/content.tex}

% DATABASE DESIGN 
\input{main/chapter2/section5/content.tex}

% DATA MANIPULATION
\input{main/chapter2/section6/content.tex}

% COMMUNICATION 
\input{main/chapter2/section7/content.tex}

% TRAINING SCENARIOS
\input{main/chapter2/section8/content.tex}

% EMG GRAPHIC
\input{main/chapter2/section9/content.tex}

% STATICAL REPORTS GENERATION
\input{main/chapter2/section10/content.tex}

\subthesischapter{Conclusiones del capítulo}
Se presentó una descripción del sistema de adquisición de datos para rehabilitación, sus componentes, características distintivas y su funcionamiento. Se identificaron y definieron los requisitos del juego  serio, tanto funcionales como no funcionales, así como los actores y casos de usos del sistema que establecieron las bases fundamentales para el desarrollo de la aplicación. Se realizó el diseño de la base de datos, abarcando tanto el modelo lógico como el físico, lo que aseguró una estructura robusta y eficiente para el almacenamiento de los datos. La manipulación de los datos se abordó de manera integral, desde la conexión con la base de datos hasta la persistencia de los resultados estadísticos. Se diseñó e implementó la comunicación con el pedal motorizado y la implementación de la interfaz gráfica para la representación de los datos EMG. Se definieron los escenarios de entrenamiento para las modalidades Ligero y Clínico, asegurando una cobertura completa de las necesidad de entrenamiento del usuario. Por último en el ámbito estadístico se desarrolló una serie de gráficos para el seguimiento de los resultados en las rutinas de entrenamiento.   
    
\end{thesischapter}
     
% SERIOUS GAME REQUIREMENTS
\begin{thesischapter}{2} {Diseño e implementación del Juego Serio}
En este capítulo se discuten los detalles de desarrollo de los aspectos citados en el capítulo anterior. Este comienza con una descripción y caracterización general del sistema, donde se  abordan cada uno de los componentes requeridos para su completo funcionamiento. Posteriormente se detalla la ingienría de software requerida en la etapa de conceptualización de la aplicación, se explican de forma detallada los aspectos teóricos y de implementación de la base de datos, el funcionamiento del protocolo de comunicación y por último los escenarios de juegos requeridos en las rutinas de entrenamiento ligero y clínico, y las estadísticas generadas por estos. Como herramienta de desarrollo se utilizó c\#.

% SYSTEM DESCRIPTION AND CHARACTERIZATION TO APPLY
\input{main/chapter2/section1/content.tex}
     
% SERIOUS GAME REQUIREMENTS
\input{main/chapter2/section2/content.tex}    

% USE CASE DEFINITION
\input{main/chapter2/section3/content.tex}

% USE CASE REALIZATION
\input{main/chapter2/section4/content.tex}

% DATABASE DESIGN 
\input{main/chapter2/section5/content.tex}

% DATA MANIPULATION
\input{main/chapter2/section6/content.tex}

% COMMUNICATION 
\input{main/chapter2/section7/content.tex}

% TRAINING SCENARIOS
\input{main/chapter2/section8/content.tex}

% EMG GRAPHIC
\input{main/chapter2/section9/content.tex}

% STATICAL REPORTS GENERATION
\input{main/chapter2/section10/content.tex}

\subthesischapter{Conclusiones del capítulo}
Se presentó una descripción del sistema de adquisición de datos para rehabilitación, sus componentes, características distintivas y su funcionamiento. Se identificaron y definieron los requisitos del juego  serio, tanto funcionales como no funcionales, así como los actores y casos de usos del sistema que establecieron las bases fundamentales para el desarrollo de la aplicación. Se realizó el diseño de la base de datos, abarcando tanto el modelo lógico como el físico, lo que aseguró una estructura robusta y eficiente para el almacenamiento de los datos. La manipulación de los datos se abordó de manera integral, desde la conexión con la base de datos hasta la persistencia de los resultados estadísticos. Se diseñó e implementó la comunicación con el pedal motorizado y la implementación de la interfaz gráfica para la representación de los datos EMG. Se definieron los escenarios de entrenamiento para las modalidades Ligero y Clínico, asegurando una cobertura completa de las necesidad de entrenamiento del usuario. Por último en el ámbito estadístico se desarrolló una serie de gráficos para el seguimiento de los resultados en las rutinas de entrenamiento.   
    
\end{thesischapter}    

% USE CASE DEFINITION
\begin{thesischapter}{2} {Diseño e implementación del Juego Serio}
En este capítulo se discuten los detalles de desarrollo de los aspectos citados en el capítulo anterior. Este comienza con una descripción y caracterización general del sistema, donde se  abordan cada uno de los componentes requeridos para su completo funcionamiento. Posteriormente se detalla la ingienría de software requerida en la etapa de conceptualización de la aplicación, se explican de forma detallada los aspectos teóricos y de implementación de la base de datos, el funcionamiento del protocolo de comunicación y por último los escenarios de juegos requeridos en las rutinas de entrenamiento ligero y clínico, y las estadísticas generadas por estos. Como herramienta de desarrollo se utilizó c\#.

% SYSTEM DESCRIPTION AND CHARACTERIZATION TO APPLY
\input{main/chapter2/section1/content.tex}
     
% SERIOUS GAME REQUIREMENTS
\input{main/chapter2/section2/content.tex}    

% USE CASE DEFINITION
\input{main/chapter2/section3/content.tex}

% USE CASE REALIZATION
\input{main/chapter2/section4/content.tex}

% DATABASE DESIGN 
\input{main/chapter2/section5/content.tex}

% DATA MANIPULATION
\input{main/chapter2/section6/content.tex}

% COMMUNICATION 
\input{main/chapter2/section7/content.tex}

% TRAINING SCENARIOS
\input{main/chapter2/section8/content.tex}

% EMG GRAPHIC
\input{main/chapter2/section9/content.tex}

% STATICAL REPORTS GENERATION
\input{main/chapter2/section10/content.tex}

\subthesischapter{Conclusiones del capítulo}
Se presentó una descripción del sistema de adquisición de datos para rehabilitación, sus componentes, características distintivas y su funcionamiento. Se identificaron y definieron los requisitos del juego  serio, tanto funcionales como no funcionales, así como los actores y casos de usos del sistema que establecieron las bases fundamentales para el desarrollo de la aplicación. Se realizó el diseño de la base de datos, abarcando tanto el modelo lógico como el físico, lo que aseguró una estructura robusta y eficiente para el almacenamiento de los datos. La manipulación de los datos se abordó de manera integral, desde la conexión con la base de datos hasta la persistencia de los resultados estadísticos. Se diseñó e implementó la comunicación con el pedal motorizado y la implementación de la interfaz gráfica para la representación de los datos EMG. Se definieron los escenarios de entrenamiento para las modalidades Ligero y Clínico, asegurando una cobertura completa de las necesidad de entrenamiento del usuario. Por último en el ámbito estadístico se desarrolló una serie de gráficos para el seguimiento de los resultados en las rutinas de entrenamiento.   
    
\end{thesischapter}

% USE CASE REALIZATION
\begin{thesischapter}{2} {Diseño e implementación del Juego Serio}
En este capítulo se discuten los detalles de desarrollo de los aspectos citados en el capítulo anterior. Este comienza con una descripción y caracterización general del sistema, donde se  abordan cada uno de los componentes requeridos para su completo funcionamiento. Posteriormente se detalla la ingienría de software requerida en la etapa de conceptualización de la aplicación, se explican de forma detallada los aspectos teóricos y de implementación de la base de datos, el funcionamiento del protocolo de comunicación y por último los escenarios de juegos requeridos en las rutinas de entrenamiento ligero y clínico, y las estadísticas generadas por estos. Como herramienta de desarrollo se utilizó c\#.

% SYSTEM DESCRIPTION AND CHARACTERIZATION TO APPLY
\input{main/chapter2/section1/content.tex}
     
% SERIOUS GAME REQUIREMENTS
\input{main/chapter2/section2/content.tex}    

% USE CASE DEFINITION
\input{main/chapter2/section3/content.tex}

% USE CASE REALIZATION
\input{main/chapter2/section4/content.tex}

% DATABASE DESIGN 
\input{main/chapter2/section5/content.tex}

% DATA MANIPULATION
\input{main/chapter2/section6/content.tex}

% COMMUNICATION 
\input{main/chapter2/section7/content.tex}

% TRAINING SCENARIOS
\input{main/chapter2/section8/content.tex}

% EMG GRAPHIC
\input{main/chapter2/section9/content.tex}

% STATICAL REPORTS GENERATION
\input{main/chapter2/section10/content.tex}

\subthesischapter{Conclusiones del capítulo}
Se presentó una descripción del sistema de adquisición de datos para rehabilitación, sus componentes, características distintivas y su funcionamiento. Se identificaron y definieron los requisitos del juego  serio, tanto funcionales como no funcionales, así como los actores y casos de usos del sistema que establecieron las bases fundamentales para el desarrollo de la aplicación. Se realizó el diseño de la base de datos, abarcando tanto el modelo lógico como el físico, lo que aseguró una estructura robusta y eficiente para el almacenamiento de los datos. La manipulación de los datos se abordó de manera integral, desde la conexión con la base de datos hasta la persistencia de los resultados estadísticos. Se diseñó e implementó la comunicación con el pedal motorizado y la implementación de la interfaz gráfica para la representación de los datos EMG. Se definieron los escenarios de entrenamiento para las modalidades Ligero y Clínico, asegurando una cobertura completa de las necesidad de entrenamiento del usuario. Por último en el ámbito estadístico se desarrolló una serie de gráficos para el seguimiento de los resultados en las rutinas de entrenamiento.   
    
\end{thesischapter}

% DATABASE DESIGN 
\begin{thesischapter}{2} {Diseño e implementación del Juego Serio}
En este capítulo se discuten los detalles de desarrollo de los aspectos citados en el capítulo anterior. Este comienza con una descripción y caracterización general del sistema, donde se  abordan cada uno de los componentes requeridos para su completo funcionamiento. Posteriormente se detalla la ingienría de software requerida en la etapa de conceptualización de la aplicación, se explican de forma detallada los aspectos teóricos y de implementación de la base de datos, el funcionamiento del protocolo de comunicación y por último los escenarios de juegos requeridos en las rutinas de entrenamiento ligero y clínico, y las estadísticas generadas por estos. Como herramienta de desarrollo se utilizó c\#.

% SYSTEM DESCRIPTION AND CHARACTERIZATION TO APPLY
\input{main/chapter2/section1/content.tex}
     
% SERIOUS GAME REQUIREMENTS
\input{main/chapter2/section2/content.tex}    

% USE CASE DEFINITION
\input{main/chapter2/section3/content.tex}

% USE CASE REALIZATION
\input{main/chapter2/section4/content.tex}

% DATABASE DESIGN 
\input{main/chapter2/section5/content.tex}

% DATA MANIPULATION
\input{main/chapter2/section6/content.tex}

% COMMUNICATION 
\input{main/chapter2/section7/content.tex}

% TRAINING SCENARIOS
\input{main/chapter2/section8/content.tex}

% EMG GRAPHIC
\input{main/chapter2/section9/content.tex}

% STATICAL REPORTS GENERATION
\input{main/chapter2/section10/content.tex}

\subthesischapter{Conclusiones del capítulo}
Se presentó una descripción del sistema de adquisición de datos para rehabilitación, sus componentes, características distintivas y su funcionamiento. Se identificaron y definieron los requisitos del juego  serio, tanto funcionales como no funcionales, así como los actores y casos de usos del sistema que establecieron las bases fundamentales para el desarrollo de la aplicación. Se realizó el diseño de la base de datos, abarcando tanto el modelo lógico como el físico, lo que aseguró una estructura robusta y eficiente para el almacenamiento de los datos. La manipulación de los datos se abordó de manera integral, desde la conexión con la base de datos hasta la persistencia de los resultados estadísticos. Se diseñó e implementó la comunicación con el pedal motorizado y la implementación de la interfaz gráfica para la representación de los datos EMG. Se definieron los escenarios de entrenamiento para las modalidades Ligero y Clínico, asegurando una cobertura completa de las necesidad de entrenamiento del usuario. Por último en el ámbito estadístico se desarrolló una serie de gráficos para el seguimiento de los resultados en las rutinas de entrenamiento.   
    
\end{thesischapter}

% DATA MANIPULATION
\begin{thesischapter}{2} {Diseño e implementación del Juego Serio}
En este capítulo se discuten los detalles de desarrollo de los aspectos citados en el capítulo anterior. Este comienza con una descripción y caracterización general del sistema, donde se  abordan cada uno de los componentes requeridos para su completo funcionamiento. Posteriormente se detalla la ingienría de software requerida en la etapa de conceptualización de la aplicación, se explican de forma detallada los aspectos teóricos y de implementación de la base de datos, el funcionamiento del protocolo de comunicación y por último los escenarios de juegos requeridos en las rutinas de entrenamiento ligero y clínico, y las estadísticas generadas por estos. Como herramienta de desarrollo se utilizó c\#.

% SYSTEM DESCRIPTION AND CHARACTERIZATION TO APPLY
\input{main/chapter2/section1/content.tex}
     
% SERIOUS GAME REQUIREMENTS
\input{main/chapter2/section2/content.tex}    

% USE CASE DEFINITION
\input{main/chapter2/section3/content.tex}

% USE CASE REALIZATION
\input{main/chapter2/section4/content.tex}

% DATABASE DESIGN 
\input{main/chapter2/section5/content.tex}

% DATA MANIPULATION
\input{main/chapter2/section6/content.tex}

% COMMUNICATION 
\input{main/chapter2/section7/content.tex}

% TRAINING SCENARIOS
\input{main/chapter2/section8/content.tex}

% EMG GRAPHIC
\input{main/chapter2/section9/content.tex}

% STATICAL REPORTS GENERATION
\input{main/chapter2/section10/content.tex}

\subthesischapter{Conclusiones del capítulo}
Se presentó una descripción del sistema de adquisición de datos para rehabilitación, sus componentes, características distintivas y su funcionamiento. Se identificaron y definieron los requisitos del juego  serio, tanto funcionales como no funcionales, así como los actores y casos de usos del sistema que establecieron las bases fundamentales para el desarrollo de la aplicación. Se realizó el diseño de la base de datos, abarcando tanto el modelo lógico como el físico, lo que aseguró una estructura robusta y eficiente para el almacenamiento de los datos. La manipulación de los datos se abordó de manera integral, desde la conexión con la base de datos hasta la persistencia de los resultados estadísticos. Se diseñó e implementó la comunicación con el pedal motorizado y la implementación de la interfaz gráfica para la representación de los datos EMG. Se definieron los escenarios de entrenamiento para las modalidades Ligero y Clínico, asegurando una cobertura completa de las necesidad de entrenamiento del usuario. Por último en el ámbito estadístico se desarrolló una serie de gráficos para el seguimiento de los resultados en las rutinas de entrenamiento.   
    
\end{thesischapter}

% COMMUNICATION 
\begin{thesischapter}{2} {Diseño e implementación del Juego Serio}
En este capítulo se discuten los detalles de desarrollo de los aspectos citados en el capítulo anterior. Este comienza con una descripción y caracterización general del sistema, donde se  abordan cada uno de los componentes requeridos para su completo funcionamiento. Posteriormente se detalla la ingienría de software requerida en la etapa de conceptualización de la aplicación, se explican de forma detallada los aspectos teóricos y de implementación de la base de datos, el funcionamiento del protocolo de comunicación y por último los escenarios de juegos requeridos en las rutinas de entrenamiento ligero y clínico, y las estadísticas generadas por estos. Como herramienta de desarrollo se utilizó c\#.

% SYSTEM DESCRIPTION AND CHARACTERIZATION TO APPLY
\input{main/chapter2/section1/content.tex}
     
% SERIOUS GAME REQUIREMENTS
\input{main/chapter2/section2/content.tex}    

% USE CASE DEFINITION
\input{main/chapter2/section3/content.tex}

% USE CASE REALIZATION
\input{main/chapter2/section4/content.tex}

% DATABASE DESIGN 
\input{main/chapter2/section5/content.tex}

% DATA MANIPULATION
\input{main/chapter2/section6/content.tex}

% COMMUNICATION 
\input{main/chapter2/section7/content.tex}

% TRAINING SCENARIOS
\input{main/chapter2/section8/content.tex}

% EMG GRAPHIC
\input{main/chapter2/section9/content.tex}

% STATICAL REPORTS GENERATION
\input{main/chapter2/section10/content.tex}

\subthesischapter{Conclusiones del capítulo}
Se presentó una descripción del sistema de adquisición de datos para rehabilitación, sus componentes, características distintivas y su funcionamiento. Se identificaron y definieron los requisitos del juego  serio, tanto funcionales como no funcionales, así como los actores y casos de usos del sistema que establecieron las bases fundamentales para el desarrollo de la aplicación. Se realizó el diseño de la base de datos, abarcando tanto el modelo lógico como el físico, lo que aseguró una estructura robusta y eficiente para el almacenamiento de los datos. La manipulación de los datos se abordó de manera integral, desde la conexión con la base de datos hasta la persistencia de los resultados estadísticos. Se diseñó e implementó la comunicación con el pedal motorizado y la implementación de la interfaz gráfica para la representación de los datos EMG. Se definieron los escenarios de entrenamiento para las modalidades Ligero y Clínico, asegurando una cobertura completa de las necesidad de entrenamiento del usuario. Por último en el ámbito estadístico se desarrolló una serie de gráficos para el seguimiento de los resultados en las rutinas de entrenamiento.   
    
\end{thesischapter}

% TRAINING SCENARIOS
\begin{thesischapter}{2} {Diseño e implementación del Juego Serio}
En este capítulo se discuten los detalles de desarrollo de los aspectos citados en el capítulo anterior. Este comienza con una descripción y caracterización general del sistema, donde se  abordan cada uno de los componentes requeridos para su completo funcionamiento. Posteriormente se detalla la ingienría de software requerida en la etapa de conceptualización de la aplicación, se explican de forma detallada los aspectos teóricos y de implementación de la base de datos, el funcionamiento del protocolo de comunicación y por último los escenarios de juegos requeridos en las rutinas de entrenamiento ligero y clínico, y las estadísticas generadas por estos. Como herramienta de desarrollo se utilizó c\#.

% SYSTEM DESCRIPTION AND CHARACTERIZATION TO APPLY
\input{main/chapter2/section1/content.tex}
     
% SERIOUS GAME REQUIREMENTS
\input{main/chapter2/section2/content.tex}    

% USE CASE DEFINITION
\input{main/chapter2/section3/content.tex}

% USE CASE REALIZATION
\input{main/chapter2/section4/content.tex}

% DATABASE DESIGN 
\input{main/chapter2/section5/content.tex}

% DATA MANIPULATION
\input{main/chapter2/section6/content.tex}

% COMMUNICATION 
\input{main/chapter2/section7/content.tex}

% TRAINING SCENARIOS
\input{main/chapter2/section8/content.tex}

% EMG GRAPHIC
\input{main/chapter2/section9/content.tex}

% STATICAL REPORTS GENERATION
\input{main/chapter2/section10/content.tex}

\subthesischapter{Conclusiones del capítulo}
Se presentó una descripción del sistema de adquisición de datos para rehabilitación, sus componentes, características distintivas y su funcionamiento. Se identificaron y definieron los requisitos del juego  serio, tanto funcionales como no funcionales, así como los actores y casos de usos del sistema que establecieron las bases fundamentales para el desarrollo de la aplicación. Se realizó el diseño de la base de datos, abarcando tanto el modelo lógico como el físico, lo que aseguró una estructura robusta y eficiente para el almacenamiento de los datos. La manipulación de los datos se abordó de manera integral, desde la conexión con la base de datos hasta la persistencia de los resultados estadísticos. Se diseñó e implementó la comunicación con el pedal motorizado y la implementación de la interfaz gráfica para la representación de los datos EMG. Se definieron los escenarios de entrenamiento para las modalidades Ligero y Clínico, asegurando una cobertura completa de las necesidad de entrenamiento del usuario. Por último en el ámbito estadístico se desarrolló una serie de gráficos para el seguimiento de los resultados en las rutinas de entrenamiento.   
    
\end{thesischapter}

% EMG GRAPHIC
\begin{thesischapter}{2} {Diseño e implementación del Juego Serio}
En este capítulo se discuten los detalles de desarrollo de los aspectos citados en el capítulo anterior. Este comienza con una descripción y caracterización general del sistema, donde se  abordan cada uno de los componentes requeridos para su completo funcionamiento. Posteriormente se detalla la ingienría de software requerida en la etapa de conceptualización de la aplicación, se explican de forma detallada los aspectos teóricos y de implementación de la base de datos, el funcionamiento del protocolo de comunicación y por último los escenarios de juegos requeridos en las rutinas de entrenamiento ligero y clínico, y las estadísticas generadas por estos. Como herramienta de desarrollo se utilizó c\#.

% SYSTEM DESCRIPTION AND CHARACTERIZATION TO APPLY
\input{main/chapter2/section1/content.tex}
     
% SERIOUS GAME REQUIREMENTS
\input{main/chapter2/section2/content.tex}    

% USE CASE DEFINITION
\input{main/chapter2/section3/content.tex}

% USE CASE REALIZATION
\input{main/chapter2/section4/content.tex}

% DATABASE DESIGN 
\input{main/chapter2/section5/content.tex}

% DATA MANIPULATION
\input{main/chapter2/section6/content.tex}

% COMMUNICATION 
\input{main/chapter2/section7/content.tex}

% TRAINING SCENARIOS
\input{main/chapter2/section8/content.tex}

% EMG GRAPHIC
\input{main/chapter2/section9/content.tex}

% STATICAL REPORTS GENERATION
\input{main/chapter2/section10/content.tex}

\subthesischapter{Conclusiones del capítulo}
Se presentó una descripción del sistema de adquisición de datos para rehabilitación, sus componentes, características distintivas y su funcionamiento. Se identificaron y definieron los requisitos del juego  serio, tanto funcionales como no funcionales, así como los actores y casos de usos del sistema que establecieron las bases fundamentales para el desarrollo de la aplicación. Se realizó el diseño de la base de datos, abarcando tanto el modelo lógico como el físico, lo que aseguró una estructura robusta y eficiente para el almacenamiento de los datos. La manipulación de los datos se abordó de manera integral, desde la conexión con la base de datos hasta la persistencia de los resultados estadísticos. Se diseñó e implementó la comunicación con el pedal motorizado y la implementación de la interfaz gráfica para la representación de los datos EMG. Se definieron los escenarios de entrenamiento para las modalidades Ligero y Clínico, asegurando una cobertura completa de las necesidad de entrenamiento del usuario. Por último en el ámbito estadístico se desarrolló una serie de gráficos para el seguimiento de los resultados en las rutinas de entrenamiento.   
    
\end{thesischapter}

% STATICAL REPORTS GENERATION
\begin{thesischapter}{2} {Diseño e implementación del Juego Serio}
En este capítulo se discuten los detalles de desarrollo de los aspectos citados en el capítulo anterior. Este comienza con una descripción y caracterización general del sistema, donde se  abordan cada uno de los componentes requeridos para su completo funcionamiento. Posteriormente se detalla la ingienría de software requerida en la etapa de conceptualización de la aplicación, se explican de forma detallada los aspectos teóricos y de implementación de la base de datos, el funcionamiento del protocolo de comunicación y por último los escenarios de juegos requeridos en las rutinas de entrenamiento ligero y clínico, y las estadísticas generadas por estos. Como herramienta de desarrollo se utilizó c\#.

% SYSTEM DESCRIPTION AND CHARACTERIZATION TO APPLY
\input{main/chapter2/section1/content.tex}
     
% SERIOUS GAME REQUIREMENTS
\input{main/chapter2/section2/content.tex}    

% USE CASE DEFINITION
\input{main/chapter2/section3/content.tex}

% USE CASE REALIZATION
\input{main/chapter2/section4/content.tex}

% DATABASE DESIGN 
\input{main/chapter2/section5/content.tex}

% DATA MANIPULATION
\input{main/chapter2/section6/content.tex}

% COMMUNICATION 
\input{main/chapter2/section7/content.tex}

% TRAINING SCENARIOS
\input{main/chapter2/section8/content.tex}

% EMG GRAPHIC
\input{main/chapter2/section9/content.tex}

% STATICAL REPORTS GENERATION
\input{main/chapter2/section10/content.tex}

\subthesischapter{Conclusiones del capítulo}
Se presentó una descripción del sistema de adquisición de datos para rehabilitación, sus componentes, características distintivas y su funcionamiento. Se identificaron y definieron los requisitos del juego  serio, tanto funcionales como no funcionales, así como los actores y casos de usos del sistema que establecieron las bases fundamentales para el desarrollo de la aplicación. Se realizó el diseño de la base de datos, abarcando tanto el modelo lógico como el físico, lo que aseguró una estructura robusta y eficiente para el almacenamiento de los datos. La manipulación de los datos se abordó de manera integral, desde la conexión con la base de datos hasta la persistencia de los resultados estadísticos. Se diseñó e implementó la comunicación con el pedal motorizado y la implementación de la interfaz gráfica para la representación de los datos EMG. Se definieron los escenarios de entrenamiento para las modalidades Ligero y Clínico, asegurando una cobertura completa de las necesidad de entrenamiento del usuario. Por último en el ámbito estadístico se desarrolló una serie de gráficos para el seguimiento de los resultados en las rutinas de entrenamiento.   
    
\end{thesischapter}

\subthesischapter{Conclusiones del capítulo}
Se presentó una descripción del sistema de adquisición de datos para rehabilitación, sus componentes, características distintivas y su funcionamiento. Se identificaron y definieron los requisitos del juego  serio, tanto funcionales como no funcionales, así como los actores y casos de usos del sistema que establecieron las bases fundamentales para el desarrollo de la aplicación. Se realizó el diseño de la base de datos, abarcando tanto el modelo lógico como el físico, lo que aseguró una estructura robusta y eficiente para el almacenamiento de los datos. La manipulación de los datos se abordó de manera integral, desde la conexión con la base de datos hasta la persistencia de los resultados estadísticos. Se diseñó e implementó la comunicación con el pedal motorizado y la implementación de la interfaz gráfica para la representación de los datos EMG. Se definieron los escenarios de entrenamiento para las modalidades Ligero y Clínico, asegurando una cobertura completa de las necesidad de entrenamiento del usuario. Por último en el ámbito estadístico se desarrolló una serie de gráficos para el seguimiento de los resultados en las rutinas de entrenamiento.   
    
\end{thesischapter}

% TRAINING SCENARIOS
\begin{thesischapter}{2} {Diseño e implementación del Juego Serio}
En este capítulo se discuten los detalles de desarrollo de los aspectos citados en el capítulo anterior. Este comienza con una descripción y caracterización general del sistema, donde se  abordan cada uno de los componentes requeridos para su completo funcionamiento. Posteriormente se detalla la ingienría de software requerida en la etapa de conceptualización de la aplicación, se explican de forma detallada los aspectos teóricos y de implementación de la base de datos, el funcionamiento del protocolo de comunicación y por último los escenarios de juegos requeridos en las rutinas de entrenamiento ligero y clínico, y las estadísticas generadas por estos. Como herramienta de desarrollo se utilizó c\#.

% SYSTEM DESCRIPTION AND CHARACTERIZATION TO APPLY
\begin{thesischapter}{2} {Diseño e implementación del Juego Serio}
En este capítulo se discuten los detalles de desarrollo de los aspectos citados en el capítulo anterior. Este comienza con una descripción y caracterización general del sistema, donde se  abordan cada uno de los componentes requeridos para su completo funcionamiento. Posteriormente se detalla la ingienría de software requerida en la etapa de conceptualización de la aplicación, se explican de forma detallada los aspectos teóricos y de implementación de la base de datos, el funcionamiento del protocolo de comunicación y por último los escenarios de juegos requeridos en las rutinas de entrenamiento ligero y clínico, y las estadísticas generadas por estos. Como herramienta de desarrollo se utilizó c\#.

% SYSTEM DESCRIPTION AND CHARACTERIZATION TO APPLY
\input{main/chapter2/section1/content.tex}
     
% SERIOUS GAME REQUIREMENTS
\input{main/chapter2/section2/content.tex}    

% USE CASE DEFINITION
\input{main/chapter2/section3/content.tex}

% USE CASE REALIZATION
\input{main/chapter2/section4/content.tex}

% DATABASE DESIGN 
\input{main/chapter2/section5/content.tex}

% DATA MANIPULATION
\input{main/chapter2/section6/content.tex}

% COMMUNICATION 
\input{main/chapter2/section7/content.tex}

% TRAINING SCENARIOS
\input{main/chapter2/section8/content.tex}

% EMG GRAPHIC
\input{main/chapter2/section9/content.tex}

% STATICAL REPORTS GENERATION
\input{main/chapter2/section10/content.tex}

\subthesischapter{Conclusiones del capítulo}
Se presentó una descripción del sistema de adquisición de datos para rehabilitación, sus componentes, características distintivas y su funcionamiento. Se identificaron y definieron los requisitos del juego  serio, tanto funcionales como no funcionales, así como los actores y casos de usos del sistema que establecieron las bases fundamentales para el desarrollo de la aplicación. Se realizó el diseño de la base de datos, abarcando tanto el modelo lógico como el físico, lo que aseguró una estructura robusta y eficiente para el almacenamiento de los datos. La manipulación de los datos se abordó de manera integral, desde la conexión con la base de datos hasta la persistencia de los resultados estadísticos. Se diseñó e implementó la comunicación con el pedal motorizado y la implementación de la interfaz gráfica para la representación de los datos EMG. Se definieron los escenarios de entrenamiento para las modalidades Ligero y Clínico, asegurando una cobertura completa de las necesidad de entrenamiento del usuario. Por último en el ámbito estadístico se desarrolló una serie de gráficos para el seguimiento de los resultados en las rutinas de entrenamiento.   
    
\end{thesischapter}
     
% SERIOUS GAME REQUIREMENTS
\begin{thesischapter}{2} {Diseño e implementación del Juego Serio}
En este capítulo se discuten los detalles de desarrollo de los aspectos citados en el capítulo anterior. Este comienza con una descripción y caracterización general del sistema, donde se  abordan cada uno de los componentes requeridos para su completo funcionamiento. Posteriormente se detalla la ingienría de software requerida en la etapa de conceptualización de la aplicación, se explican de forma detallada los aspectos teóricos y de implementación de la base de datos, el funcionamiento del protocolo de comunicación y por último los escenarios de juegos requeridos en las rutinas de entrenamiento ligero y clínico, y las estadísticas generadas por estos. Como herramienta de desarrollo se utilizó c\#.

% SYSTEM DESCRIPTION AND CHARACTERIZATION TO APPLY
\input{main/chapter2/section1/content.tex}
     
% SERIOUS GAME REQUIREMENTS
\input{main/chapter2/section2/content.tex}    

% USE CASE DEFINITION
\input{main/chapter2/section3/content.tex}

% USE CASE REALIZATION
\input{main/chapter2/section4/content.tex}

% DATABASE DESIGN 
\input{main/chapter2/section5/content.tex}

% DATA MANIPULATION
\input{main/chapter2/section6/content.tex}

% COMMUNICATION 
\input{main/chapter2/section7/content.tex}

% TRAINING SCENARIOS
\input{main/chapter2/section8/content.tex}

% EMG GRAPHIC
\input{main/chapter2/section9/content.tex}

% STATICAL REPORTS GENERATION
\input{main/chapter2/section10/content.tex}

\subthesischapter{Conclusiones del capítulo}
Se presentó una descripción del sistema de adquisición de datos para rehabilitación, sus componentes, características distintivas y su funcionamiento. Se identificaron y definieron los requisitos del juego  serio, tanto funcionales como no funcionales, así como los actores y casos de usos del sistema que establecieron las bases fundamentales para el desarrollo de la aplicación. Se realizó el diseño de la base de datos, abarcando tanto el modelo lógico como el físico, lo que aseguró una estructura robusta y eficiente para el almacenamiento de los datos. La manipulación de los datos se abordó de manera integral, desde la conexión con la base de datos hasta la persistencia de los resultados estadísticos. Se diseñó e implementó la comunicación con el pedal motorizado y la implementación de la interfaz gráfica para la representación de los datos EMG. Se definieron los escenarios de entrenamiento para las modalidades Ligero y Clínico, asegurando una cobertura completa de las necesidad de entrenamiento del usuario. Por último en el ámbito estadístico se desarrolló una serie de gráficos para el seguimiento de los resultados en las rutinas de entrenamiento.   
    
\end{thesischapter}    

% USE CASE DEFINITION
\begin{thesischapter}{2} {Diseño e implementación del Juego Serio}
En este capítulo se discuten los detalles de desarrollo de los aspectos citados en el capítulo anterior. Este comienza con una descripción y caracterización general del sistema, donde se  abordan cada uno de los componentes requeridos para su completo funcionamiento. Posteriormente se detalla la ingienría de software requerida en la etapa de conceptualización de la aplicación, se explican de forma detallada los aspectos teóricos y de implementación de la base de datos, el funcionamiento del protocolo de comunicación y por último los escenarios de juegos requeridos en las rutinas de entrenamiento ligero y clínico, y las estadísticas generadas por estos. Como herramienta de desarrollo se utilizó c\#.

% SYSTEM DESCRIPTION AND CHARACTERIZATION TO APPLY
\input{main/chapter2/section1/content.tex}
     
% SERIOUS GAME REQUIREMENTS
\input{main/chapter2/section2/content.tex}    

% USE CASE DEFINITION
\input{main/chapter2/section3/content.tex}

% USE CASE REALIZATION
\input{main/chapter2/section4/content.tex}

% DATABASE DESIGN 
\input{main/chapter2/section5/content.tex}

% DATA MANIPULATION
\input{main/chapter2/section6/content.tex}

% COMMUNICATION 
\input{main/chapter2/section7/content.tex}

% TRAINING SCENARIOS
\input{main/chapter2/section8/content.tex}

% EMG GRAPHIC
\input{main/chapter2/section9/content.tex}

% STATICAL REPORTS GENERATION
\input{main/chapter2/section10/content.tex}

\subthesischapter{Conclusiones del capítulo}
Se presentó una descripción del sistema de adquisición de datos para rehabilitación, sus componentes, características distintivas y su funcionamiento. Se identificaron y definieron los requisitos del juego  serio, tanto funcionales como no funcionales, así como los actores y casos de usos del sistema que establecieron las bases fundamentales para el desarrollo de la aplicación. Se realizó el diseño de la base de datos, abarcando tanto el modelo lógico como el físico, lo que aseguró una estructura robusta y eficiente para el almacenamiento de los datos. La manipulación de los datos se abordó de manera integral, desde la conexión con la base de datos hasta la persistencia de los resultados estadísticos. Se diseñó e implementó la comunicación con el pedal motorizado y la implementación de la interfaz gráfica para la representación de los datos EMG. Se definieron los escenarios de entrenamiento para las modalidades Ligero y Clínico, asegurando una cobertura completa de las necesidad de entrenamiento del usuario. Por último en el ámbito estadístico se desarrolló una serie de gráficos para el seguimiento de los resultados en las rutinas de entrenamiento.   
    
\end{thesischapter}

% USE CASE REALIZATION
\begin{thesischapter}{2} {Diseño e implementación del Juego Serio}
En este capítulo se discuten los detalles de desarrollo de los aspectos citados en el capítulo anterior. Este comienza con una descripción y caracterización general del sistema, donde se  abordan cada uno de los componentes requeridos para su completo funcionamiento. Posteriormente se detalla la ingienría de software requerida en la etapa de conceptualización de la aplicación, se explican de forma detallada los aspectos teóricos y de implementación de la base de datos, el funcionamiento del protocolo de comunicación y por último los escenarios de juegos requeridos en las rutinas de entrenamiento ligero y clínico, y las estadísticas generadas por estos. Como herramienta de desarrollo se utilizó c\#.

% SYSTEM DESCRIPTION AND CHARACTERIZATION TO APPLY
\input{main/chapter2/section1/content.tex}
     
% SERIOUS GAME REQUIREMENTS
\input{main/chapter2/section2/content.tex}    

% USE CASE DEFINITION
\input{main/chapter2/section3/content.tex}

% USE CASE REALIZATION
\input{main/chapter2/section4/content.tex}

% DATABASE DESIGN 
\input{main/chapter2/section5/content.tex}

% DATA MANIPULATION
\input{main/chapter2/section6/content.tex}

% COMMUNICATION 
\input{main/chapter2/section7/content.tex}

% TRAINING SCENARIOS
\input{main/chapter2/section8/content.tex}

% EMG GRAPHIC
\input{main/chapter2/section9/content.tex}

% STATICAL REPORTS GENERATION
\input{main/chapter2/section10/content.tex}

\subthesischapter{Conclusiones del capítulo}
Se presentó una descripción del sistema de adquisición de datos para rehabilitación, sus componentes, características distintivas y su funcionamiento. Se identificaron y definieron los requisitos del juego  serio, tanto funcionales como no funcionales, así como los actores y casos de usos del sistema que establecieron las bases fundamentales para el desarrollo de la aplicación. Se realizó el diseño de la base de datos, abarcando tanto el modelo lógico como el físico, lo que aseguró una estructura robusta y eficiente para el almacenamiento de los datos. La manipulación de los datos se abordó de manera integral, desde la conexión con la base de datos hasta la persistencia de los resultados estadísticos. Se diseñó e implementó la comunicación con el pedal motorizado y la implementación de la interfaz gráfica para la representación de los datos EMG. Se definieron los escenarios de entrenamiento para las modalidades Ligero y Clínico, asegurando una cobertura completa de las necesidad de entrenamiento del usuario. Por último en el ámbito estadístico se desarrolló una serie de gráficos para el seguimiento de los resultados en las rutinas de entrenamiento.   
    
\end{thesischapter}

% DATABASE DESIGN 
\begin{thesischapter}{2} {Diseño e implementación del Juego Serio}
En este capítulo se discuten los detalles de desarrollo de los aspectos citados en el capítulo anterior. Este comienza con una descripción y caracterización general del sistema, donde se  abordan cada uno de los componentes requeridos para su completo funcionamiento. Posteriormente se detalla la ingienría de software requerida en la etapa de conceptualización de la aplicación, se explican de forma detallada los aspectos teóricos y de implementación de la base de datos, el funcionamiento del protocolo de comunicación y por último los escenarios de juegos requeridos en las rutinas de entrenamiento ligero y clínico, y las estadísticas generadas por estos. Como herramienta de desarrollo se utilizó c\#.

% SYSTEM DESCRIPTION AND CHARACTERIZATION TO APPLY
\input{main/chapter2/section1/content.tex}
     
% SERIOUS GAME REQUIREMENTS
\input{main/chapter2/section2/content.tex}    

% USE CASE DEFINITION
\input{main/chapter2/section3/content.tex}

% USE CASE REALIZATION
\input{main/chapter2/section4/content.tex}

% DATABASE DESIGN 
\input{main/chapter2/section5/content.tex}

% DATA MANIPULATION
\input{main/chapter2/section6/content.tex}

% COMMUNICATION 
\input{main/chapter2/section7/content.tex}

% TRAINING SCENARIOS
\input{main/chapter2/section8/content.tex}

% EMG GRAPHIC
\input{main/chapter2/section9/content.tex}

% STATICAL REPORTS GENERATION
\input{main/chapter2/section10/content.tex}

\subthesischapter{Conclusiones del capítulo}
Se presentó una descripción del sistema de adquisición de datos para rehabilitación, sus componentes, características distintivas y su funcionamiento. Se identificaron y definieron los requisitos del juego  serio, tanto funcionales como no funcionales, así como los actores y casos de usos del sistema que establecieron las bases fundamentales para el desarrollo de la aplicación. Se realizó el diseño de la base de datos, abarcando tanto el modelo lógico como el físico, lo que aseguró una estructura robusta y eficiente para el almacenamiento de los datos. La manipulación de los datos se abordó de manera integral, desde la conexión con la base de datos hasta la persistencia de los resultados estadísticos. Se diseñó e implementó la comunicación con el pedal motorizado y la implementación de la interfaz gráfica para la representación de los datos EMG. Se definieron los escenarios de entrenamiento para las modalidades Ligero y Clínico, asegurando una cobertura completa de las necesidad de entrenamiento del usuario. Por último en el ámbito estadístico se desarrolló una serie de gráficos para el seguimiento de los resultados en las rutinas de entrenamiento.   
    
\end{thesischapter}

% DATA MANIPULATION
\begin{thesischapter}{2} {Diseño e implementación del Juego Serio}
En este capítulo se discuten los detalles de desarrollo de los aspectos citados en el capítulo anterior. Este comienza con una descripción y caracterización general del sistema, donde se  abordan cada uno de los componentes requeridos para su completo funcionamiento. Posteriormente se detalla la ingienría de software requerida en la etapa de conceptualización de la aplicación, se explican de forma detallada los aspectos teóricos y de implementación de la base de datos, el funcionamiento del protocolo de comunicación y por último los escenarios de juegos requeridos en las rutinas de entrenamiento ligero y clínico, y las estadísticas generadas por estos. Como herramienta de desarrollo se utilizó c\#.

% SYSTEM DESCRIPTION AND CHARACTERIZATION TO APPLY
\input{main/chapter2/section1/content.tex}
     
% SERIOUS GAME REQUIREMENTS
\input{main/chapter2/section2/content.tex}    

% USE CASE DEFINITION
\input{main/chapter2/section3/content.tex}

% USE CASE REALIZATION
\input{main/chapter2/section4/content.tex}

% DATABASE DESIGN 
\input{main/chapter2/section5/content.tex}

% DATA MANIPULATION
\input{main/chapter2/section6/content.tex}

% COMMUNICATION 
\input{main/chapter2/section7/content.tex}

% TRAINING SCENARIOS
\input{main/chapter2/section8/content.tex}

% EMG GRAPHIC
\input{main/chapter2/section9/content.tex}

% STATICAL REPORTS GENERATION
\input{main/chapter2/section10/content.tex}

\subthesischapter{Conclusiones del capítulo}
Se presentó una descripción del sistema de adquisición de datos para rehabilitación, sus componentes, características distintivas y su funcionamiento. Se identificaron y definieron los requisitos del juego  serio, tanto funcionales como no funcionales, así como los actores y casos de usos del sistema que establecieron las bases fundamentales para el desarrollo de la aplicación. Se realizó el diseño de la base de datos, abarcando tanto el modelo lógico como el físico, lo que aseguró una estructura robusta y eficiente para el almacenamiento de los datos. La manipulación de los datos se abordó de manera integral, desde la conexión con la base de datos hasta la persistencia de los resultados estadísticos. Se diseñó e implementó la comunicación con el pedal motorizado y la implementación de la interfaz gráfica para la representación de los datos EMG. Se definieron los escenarios de entrenamiento para las modalidades Ligero y Clínico, asegurando una cobertura completa de las necesidad de entrenamiento del usuario. Por último en el ámbito estadístico se desarrolló una serie de gráficos para el seguimiento de los resultados en las rutinas de entrenamiento.   
    
\end{thesischapter}

% COMMUNICATION 
\begin{thesischapter}{2} {Diseño e implementación del Juego Serio}
En este capítulo se discuten los detalles de desarrollo de los aspectos citados en el capítulo anterior. Este comienza con una descripción y caracterización general del sistema, donde se  abordan cada uno de los componentes requeridos para su completo funcionamiento. Posteriormente se detalla la ingienría de software requerida en la etapa de conceptualización de la aplicación, se explican de forma detallada los aspectos teóricos y de implementación de la base de datos, el funcionamiento del protocolo de comunicación y por último los escenarios de juegos requeridos en las rutinas de entrenamiento ligero y clínico, y las estadísticas generadas por estos. Como herramienta de desarrollo se utilizó c\#.

% SYSTEM DESCRIPTION AND CHARACTERIZATION TO APPLY
\input{main/chapter2/section1/content.tex}
     
% SERIOUS GAME REQUIREMENTS
\input{main/chapter2/section2/content.tex}    

% USE CASE DEFINITION
\input{main/chapter2/section3/content.tex}

% USE CASE REALIZATION
\input{main/chapter2/section4/content.tex}

% DATABASE DESIGN 
\input{main/chapter2/section5/content.tex}

% DATA MANIPULATION
\input{main/chapter2/section6/content.tex}

% COMMUNICATION 
\input{main/chapter2/section7/content.tex}

% TRAINING SCENARIOS
\input{main/chapter2/section8/content.tex}

% EMG GRAPHIC
\input{main/chapter2/section9/content.tex}

% STATICAL REPORTS GENERATION
\input{main/chapter2/section10/content.tex}

\subthesischapter{Conclusiones del capítulo}
Se presentó una descripción del sistema de adquisición de datos para rehabilitación, sus componentes, características distintivas y su funcionamiento. Se identificaron y definieron los requisitos del juego  serio, tanto funcionales como no funcionales, así como los actores y casos de usos del sistema que establecieron las bases fundamentales para el desarrollo de la aplicación. Se realizó el diseño de la base de datos, abarcando tanto el modelo lógico como el físico, lo que aseguró una estructura robusta y eficiente para el almacenamiento de los datos. La manipulación de los datos se abordó de manera integral, desde la conexión con la base de datos hasta la persistencia de los resultados estadísticos. Se diseñó e implementó la comunicación con el pedal motorizado y la implementación de la interfaz gráfica para la representación de los datos EMG. Se definieron los escenarios de entrenamiento para las modalidades Ligero y Clínico, asegurando una cobertura completa de las necesidad de entrenamiento del usuario. Por último en el ámbito estadístico se desarrolló una serie de gráficos para el seguimiento de los resultados en las rutinas de entrenamiento.   
    
\end{thesischapter}

% TRAINING SCENARIOS
\begin{thesischapter}{2} {Diseño e implementación del Juego Serio}
En este capítulo se discuten los detalles de desarrollo de los aspectos citados en el capítulo anterior. Este comienza con una descripción y caracterización general del sistema, donde se  abordan cada uno de los componentes requeridos para su completo funcionamiento. Posteriormente se detalla la ingienría de software requerida en la etapa de conceptualización de la aplicación, se explican de forma detallada los aspectos teóricos y de implementación de la base de datos, el funcionamiento del protocolo de comunicación y por último los escenarios de juegos requeridos en las rutinas de entrenamiento ligero y clínico, y las estadísticas generadas por estos. Como herramienta de desarrollo se utilizó c\#.

% SYSTEM DESCRIPTION AND CHARACTERIZATION TO APPLY
\input{main/chapter2/section1/content.tex}
     
% SERIOUS GAME REQUIREMENTS
\input{main/chapter2/section2/content.tex}    

% USE CASE DEFINITION
\input{main/chapter2/section3/content.tex}

% USE CASE REALIZATION
\input{main/chapter2/section4/content.tex}

% DATABASE DESIGN 
\input{main/chapter2/section5/content.tex}

% DATA MANIPULATION
\input{main/chapter2/section6/content.tex}

% COMMUNICATION 
\input{main/chapter2/section7/content.tex}

% TRAINING SCENARIOS
\input{main/chapter2/section8/content.tex}

% EMG GRAPHIC
\input{main/chapter2/section9/content.tex}

% STATICAL REPORTS GENERATION
\input{main/chapter2/section10/content.tex}

\subthesischapter{Conclusiones del capítulo}
Se presentó una descripción del sistema de adquisición de datos para rehabilitación, sus componentes, características distintivas y su funcionamiento. Se identificaron y definieron los requisitos del juego  serio, tanto funcionales como no funcionales, así como los actores y casos de usos del sistema que establecieron las bases fundamentales para el desarrollo de la aplicación. Se realizó el diseño de la base de datos, abarcando tanto el modelo lógico como el físico, lo que aseguró una estructura robusta y eficiente para el almacenamiento de los datos. La manipulación de los datos se abordó de manera integral, desde la conexión con la base de datos hasta la persistencia de los resultados estadísticos. Se diseñó e implementó la comunicación con el pedal motorizado y la implementación de la interfaz gráfica para la representación de los datos EMG. Se definieron los escenarios de entrenamiento para las modalidades Ligero y Clínico, asegurando una cobertura completa de las necesidad de entrenamiento del usuario. Por último en el ámbito estadístico se desarrolló una serie de gráficos para el seguimiento de los resultados en las rutinas de entrenamiento.   
    
\end{thesischapter}

% EMG GRAPHIC
\begin{thesischapter}{2} {Diseño e implementación del Juego Serio}
En este capítulo se discuten los detalles de desarrollo de los aspectos citados en el capítulo anterior. Este comienza con una descripción y caracterización general del sistema, donde se  abordan cada uno de los componentes requeridos para su completo funcionamiento. Posteriormente se detalla la ingienría de software requerida en la etapa de conceptualización de la aplicación, se explican de forma detallada los aspectos teóricos y de implementación de la base de datos, el funcionamiento del protocolo de comunicación y por último los escenarios de juegos requeridos en las rutinas de entrenamiento ligero y clínico, y las estadísticas generadas por estos. Como herramienta de desarrollo se utilizó c\#.

% SYSTEM DESCRIPTION AND CHARACTERIZATION TO APPLY
\input{main/chapter2/section1/content.tex}
     
% SERIOUS GAME REQUIREMENTS
\input{main/chapter2/section2/content.tex}    

% USE CASE DEFINITION
\input{main/chapter2/section3/content.tex}

% USE CASE REALIZATION
\input{main/chapter2/section4/content.tex}

% DATABASE DESIGN 
\input{main/chapter2/section5/content.tex}

% DATA MANIPULATION
\input{main/chapter2/section6/content.tex}

% COMMUNICATION 
\input{main/chapter2/section7/content.tex}

% TRAINING SCENARIOS
\input{main/chapter2/section8/content.tex}

% EMG GRAPHIC
\input{main/chapter2/section9/content.tex}

% STATICAL REPORTS GENERATION
\input{main/chapter2/section10/content.tex}

\subthesischapter{Conclusiones del capítulo}
Se presentó una descripción del sistema de adquisición de datos para rehabilitación, sus componentes, características distintivas y su funcionamiento. Se identificaron y definieron los requisitos del juego  serio, tanto funcionales como no funcionales, así como los actores y casos de usos del sistema que establecieron las bases fundamentales para el desarrollo de la aplicación. Se realizó el diseño de la base de datos, abarcando tanto el modelo lógico como el físico, lo que aseguró una estructura robusta y eficiente para el almacenamiento de los datos. La manipulación de los datos se abordó de manera integral, desde la conexión con la base de datos hasta la persistencia de los resultados estadísticos. Se diseñó e implementó la comunicación con el pedal motorizado y la implementación de la interfaz gráfica para la representación de los datos EMG. Se definieron los escenarios de entrenamiento para las modalidades Ligero y Clínico, asegurando una cobertura completa de las necesidad de entrenamiento del usuario. Por último en el ámbito estadístico se desarrolló una serie de gráficos para el seguimiento de los resultados en las rutinas de entrenamiento.   
    
\end{thesischapter}

% STATICAL REPORTS GENERATION
\begin{thesischapter}{2} {Diseño e implementación del Juego Serio}
En este capítulo se discuten los detalles de desarrollo de los aspectos citados en el capítulo anterior. Este comienza con una descripción y caracterización general del sistema, donde se  abordan cada uno de los componentes requeridos para su completo funcionamiento. Posteriormente se detalla la ingienría de software requerida en la etapa de conceptualización de la aplicación, se explican de forma detallada los aspectos teóricos y de implementación de la base de datos, el funcionamiento del protocolo de comunicación y por último los escenarios de juegos requeridos en las rutinas de entrenamiento ligero y clínico, y las estadísticas generadas por estos. Como herramienta de desarrollo se utilizó c\#.

% SYSTEM DESCRIPTION AND CHARACTERIZATION TO APPLY
\input{main/chapter2/section1/content.tex}
     
% SERIOUS GAME REQUIREMENTS
\input{main/chapter2/section2/content.tex}    

% USE CASE DEFINITION
\input{main/chapter2/section3/content.tex}

% USE CASE REALIZATION
\input{main/chapter2/section4/content.tex}

% DATABASE DESIGN 
\input{main/chapter2/section5/content.tex}

% DATA MANIPULATION
\input{main/chapter2/section6/content.tex}

% COMMUNICATION 
\input{main/chapter2/section7/content.tex}

% TRAINING SCENARIOS
\input{main/chapter2/section8/content.tex}

% EMG GRAPHIC
\input{main/chapter2/section9/content.tex}

% STATICAL REPORTS GENERATION
\input{main/chapter2/section10/content.tex}

\subthesischapter{Conclusiones del capítulo}
Se presentó una descripción del sistema de adquisición de datos para rehabilitación, sus componentes, características distintivas y su funcionamiento. Se identificaron y definieron los requisitos del juego  serio, tanto funcionales como no funcionales, así como los actores y casos de usos del sistema que establecieron las bases fundamentales para el desarrollo de la aplicación. Se realizó el diseño de la base de datos, abarcando tanto el modelo lógico como el físico, lo que aseguró una estructura robusta y eficiente para el almacenamiento de los datos. La manipulación de los datos se abordó de manera integral, desde la conexión con la base de datos hasta la persistencia de los resultados estadísticos. Se diseñó e implementó la comunicación con el pedal motorizado y la implementación de la interfaz gráfica para la representación de los datos EMG. Se definieron los escenarios de entrenamiento para las modalidades Ligero y Clínico, asegurando una cobertura completa de las necesidad de entrenamiento del usuario. Por último en el ámbito estadístico se desarrolló una serie de gráficos para el seguimiento de los resultados en las rutinas de entrenamiento.   
    
\end{thesischapter}

\subthesischapter{Conclusiones del capítulo}
Se presentó una descripción del sistema de adquisición de datos para rehabilitación, sus componentes, características distintivas y su funcionamiento. Se identificaron y definieron los requisitos del juego  serio, tanto funcionales como no funcionales, así como los actores y casos de usos del sistema que establecieron las bases fundamentales para el desarrollo de la aplicación. Se realizó el diseño de la base de datos, abarcando tanto el modelo lógico como el físico, lo que aseguró una estructura robusta y eficiente para el almacenamiento de los datos. La manipulación de los datos se abordó de manera integral, desde la conexión con la base de datos hasta la persistencia de los resultados estadísticos. Se diseñó e implementó la comunicación con el pedal motorizado y la implementación de la interfaz gráfica para la representación de los datos EMG. Se definieron los escenarios de entrenamiento para las modalidades Ligero y Clínico, asegurando una cobertura completa de las necesidad de entrenamiento del usuario. Por último en el ámbito estadístico se desarrolló una serie de gráficos para el seguimiento de los resultados en las rutinas de entrenamiento.   
    
\end{thesischapter}

% EMG GRAPHIC
\begin{thesischapter}{2} {Diseño e implementación del Juego Serio}
En este capítulo se discuten los detalles de desarrollo de los aspectos citados en el capítulo anterior. Este comienza con una descripción y caracterización general del sistema, donde se  abordan cada uno de los componentes requeridos para su completo funcionamiento. Posteriormente se detalla la ingienría de software requerida en la etapa de conceptualización de la aplicación, se explican de forma detallada los aspectos teóricos y de implementación de la base de datos, el funcionamiento del protocolo de comunicación y por último los escenarios de juegos requeridos en las rutinas de entrenamiento ligero y clínico, y las estadísticas generadas por estos. Como herramienta de desarrollo se utilizó c\#.

% SYSTEM DESCRIPTION AND CHARACTERIZATION TO APPLY
\begin{thesischapter}{2} {Diseño e implementación del Juego Serio}
En este capítulo se discuten los detalles de desarrollo de los aspectos citados en el capítulo anterior. Este comienza con una descripción y caracterización general del sistema, donde se  abordan cada uno de los componentes requeridos para su completo funcionamiento. Posteriormente se detalla la ingienría de software requerida en la etapa de conceptualización de la aplicación, se explican de forma detallada los aspectos teóricos y de implementación de la base de datos, el funcionamiento del protocolo de comunicación y por último los escenarios de juegos requeridos en las rutinas de entrenamiento ligero y clínico, y las estadísticas generadas por estos. Como herramienta de desarrollo se utilizó c\#.

% SYSTEM DESCRIPTION AND CHARACTERIZATION TO APPLY
\input{main/chapter2/section1/content.tex}
     
% SERIOUS GAME REQUIREMENTS
\input{main/chapter2/section2/content.tex}    

% USE CASE DEFINITION
\input{main/chapter2/section3/content.tex}

% USE CASE REALIZATION
\input{main/chapter2/section4/content.tex}

% DATABASE DESIGN 
\input{main/chapter2/section5/content.tex}

% DATA MANIPULATION
\input{main/chapter2/section6/content.tex}

% COMMUNICATION 
\input{main/chapter2/section7/content.tex}

% TRAINING SCENARIOS
\input{main/chapter2/section8/content.tex}

% EMG GRAPHIC
\input{main/chapter2/section9/content.tex}

% STATICAL REPORTS GENERATION
\input{main/chapter2/section10/content.tex}

\subthesischapter{Conclusiones del capítulo}
Se presentó una descripción del sistema de adquisición de datos para rehabilitación, sus componentes, características distintivas y su funcionamiento. Se identificaron y definieron los requisitos del juego  serio, tanto funcionales como no funcionales, así como los actores y casos de usos del sistema que establecieron las bases fundamentales para el desarrollo de la aplicación. Se realizó el diseño de la base de datos, abarcando tanto el modelo lógico como el físico, lo que aseguró una estructura robusta y eficiente para el almacenamiento de los datos. La manipulación de los datos se abordó de manera integral, desde la conexión con la base de datos hasta la persistencia de los resultados estadísticos. Se diseñó e implementó la comunicación con el pedal motorizado y la implementación de la interfaz gráfica para la representación de los datos EMG. Se definieron los escenarios de entrenamiento para las modalidades Ligero y Clínico, asegurando una cobertura completa de las necesidad de entrenamiento del usuario. Por último en el ámbito estadístico se desarrolló una serie de gráficos para el seguimiento de los resultados en las rutinas de entrenamiento.   
    
\end{thesischapter}
     
% SERIOUS GAME REQUIREMENTS
\begin{thesischapter}{2} {Diseño e implementación del Juego Serio}
En este capítulo se discuten los detalles de desarrollo de los aspectos citados en el capítulo anterior. Este comienza con una descripción y caracterización general del sistema, donde se  abordan cada uno de los componentes requeridos para su completo funcionamiento. Posteriormente se detalla la ingienría de software requerida en la etapa de conceptualización de la aplicación, se explican de forma detallada los aspectos teóricos y de implementación de la base de datos, el funcionamiento del protocolo de comunicación y por último los escenarios de juegos requeridos en las rutinas de entrenamiento ligero y clínico, y las estadísticas generadas por estos. Como herramienta de desarrollo se utilizó c\#.

% SYSTEM DESCRIPTION AND CHARACTERIZATION TO APPLY
\input{main/chapter2/section1/content.tex}
     
% SERIOUS GAME REQUIREMENTS
\input{main/chapter2/section2/content.tex}    

% USE CASE DEFINITION
\input{main/chapter2/section3/content.tex}

% USE CASE REALIZATION
\input{main/chapter2/section4/content.tex}

% DATABASE DESIGN 
\input{main/chapter2/section5/content.tex}

% DATA MANIPULATION
\input{main/chapter2/section6/content.tex}

% COMMUNICATION 
\input{main/chapter2/section7/content.tex}

% TRAINING SCENARIOS
\input{main/chapter2/section8/content.tex}

% EMG GRAPHIC
\input{main/chapter2/section9/content.tex}

% STATICAL REPORTS GENERATION
\input{main/chapter2/section10/content.tex}

\subthesischapter{Conclusiones del capítulo}
Se presentó una descripción del sistema de adquisición de datos para rehabilitación, sus componentes, características distintivas y su funcionamiento. Se identificaron y definieron los requisitos del juego  serio, tanto funcionales como no funcionales, así como los actores y casos de usos del sistema que establecieron las bases fundamentales para el desarrollo de la aplicación. Se realizó el diseño de la base de datos, abarcando tanto el modelo lógico como el físico, lo que aseguró una estructura robusta y eficiente para el almacenamiento de los datos. La manipulación de los datos se abordó de manera integral, desde la conexión con la base de datos hasta la persistencia de los resultados estadísticos. Se diseñó e implementó la comunicación con el pedal motorizado y la implementación de la interfaz gráfica para la representación de los datos EMG. Se definieron los escenarios de entrenamiento para las modalidades Ligero y Clínico, asegurando una cobertura completa de las necesidad de entrenamiento del usuario. Por último en el ámbito estadístico se desarrolló una serie de gráficos para el seguimiento de los resultados en las rutinas de entrenamiento.   
    
\end{thesischapter}    

% USE CASE DEFINITION
\begin{thesischapter}{2} {Diseño e implementación del Juego Serio}
En este capítulo se discuten los detalles de desarrollo de los aspectos citados en el capítulo anterior. Este comienza con una descripción y caracterización general del sistema, donde se  abordan cada uno de los componentes requeridos para su completo funcionamiento. Posteriormente se detalla la ingienría de software requerida en la etapa de conceptualización de la aplicación, se explican de forma detallada los aspectos teóricos y de implementación de la base de datos, el funcionamiento del protocolo de comunicación y por último los escenarios de juegos requeridos en las rutinas de entrenamiento ligero y clínico, y las estadísticas generadas por estos. Como herramienta de desarrollo se utilizó c\#.

% SYSTEM DESCRIPTION AND CHARACTERIZATION TO APPLY
\input{main/chapter2/section1/content.tex}
     
% SERIOUS GAME REQUIREMENTS
\input{main/chapter2/section2/content.tex}    

% USE CASE DEFINITION
\input{main/chapter2/section3/content.tex}

% USE CASE REALIZATION
\input{main/chapter2/section4/content.tex}

% DATABASE DESIGN 
\input{main/chapter2/section5/content.tex}

% DATA MANIPULATION
\input{main/chapter2/section6/content.tex}

% COMMUNICATION 
\input{main/chapter2/section7/content.tex}

% TRAINING SCENARIOS
\input{main/chapter2/section8/content.tex}

% EMG GRAPHIC
\input{main/chapter2/section9/content.tex}

% STATICAL REPORTS GENERATION
\input{main/chapter2/section10/content.tex}

\subthesischapter{Conclusiones del capítulo}
Se presentó una descripción del sistema de adquisición de datos para rehabilitación, sus componentes, características distintivas y su funcionamiento. Se identificaron y definieron los requisitos del juego  serio, tanto funcionales como no funcionales, así como los actores y casos de usos del sistema que establecieron las bases fundamentales para el desarrollo de la aplicación. Se realizó el diseño de la base de datos, abarcando tanto el modelo lógico como el físico, lo que aseguró una estructura robusta y eficiente para el almacenamiento de los datos. La manipulación de los datos se abordó de manera integral, desde la conexión con la base de datos hasta la persistencia de los resultados estadísticos. Se diseñó e implementó la comunicación con el pedal motorizado y la implementación de la interfaz gráfica para la representación de los datos EMG. Se definieron los escenarios de entrenamiento para las modalidades Ligero y Clínico, asegurando una cobertura completa de las necesidad de entrenamiento del usuario. Por último en el ámbito estadístico se desarrolló una serie de gráficos para el seguimiento de los resultados en las rutinas de entrenamiento.   
    
\end{thesischapter}

% USE CASE REALIZATION
\begin{thesischapter}{2} {Diseño e implementación del Juego Serio}
En este capítulo se discuten los detalles de desarrollo de los aspectos citados en el capítulo anterior. Este comienza con una descripción y caracterización general del sistema, donde se  abordan cada uno de los componentes requeridos para su completo funcionamiento. Posteriormente se detalla la ingienría de software requerida en la etapa de conceptualización de la aplicación, se explican de forma detallada los aspectos teóricos y de implementación de la base de datos, el funcionamiento del protocolo de comunicación y por último los escenarios de juegos requeridos en las rutinas de entrenamiento ligero y clínico, y las estadísticas generadas por estos. Como herramienta de desarrollo se utilizó c\#.

% SYSTEM DESCRIPTION AND CHARACTERIZATION TO APPLY
\input{main/chapter2/section1/content.tex}
     
% SERIOUS GAME REQUIREMENTS
\input{main/chapter2/section2/content.tex}    

% USE CASE DEFINITION
\input{main/chapter2/section3/content.tex}

% USE CASE REALIZATION
\input{main/chapter2/section4/content.tex}

% DATABASE DESIGN 
\input{main/chapter2/section5/content.tex}

% DATA MANIPULATION
\input{main/chapter2/section6/content.tex}

% COMMUNICATION 
\input{main/chapter2/section7/content.tex}

% TRAINING SCENARIOS
\input{main/chapter2/section8/content.tex}

% EMG GRAPHIC
\input{main/chapter2/section9/content.tex}

% STATICAL REPORTS GENERATION
\input{main/chapter2/section10/content.tex}

\subthesischapter{Conclusiones del capítulo}
Se presentó una descripción del sistema de adquisición de datos para rehabilitación, sus componentes, características distintivas y su funcionamiento. Se identificaron y definieron los requisitos del juego  serio, tanto funcionales como no funcionales, así como los actores y casos de usos del sistema que establecieron las bases fundamentales para el desarrollo de la aplicación. Se realizó el diseño de la base de datos, abarcando tanto el modelo lógico como el físico, lo que aseguró una estructura robusta y eficiente para el almacenamiento de los datos. La manipulación de los datos se abordó de manera integral, desde la conexión con la base de datos hasta la persistencia de los resultados estadísticos. Se diseñó e implementó la comunicación con el pedal motorizado y la implementación de la interfaz gráfica para la representación de los datos EMG. Se definieron los escenarios de entrenamiento para las modalidades Ligero y Clínico, asegurando una cobertura completa de las necesidad de entrenamiento del usuario. Por último en el ámbito estadístico se desarrolló una serie de gráficos para el seguimiento de los resultados en las rutinas de entrenamiento.   
    
\end{thesischapter}

% DATABASE DESIGN 
\begin{thesischapter}{2} {Diseño e implementación del Juego Serio}
En este capítulo se discuten los detalles de desarrollo de los aspectos citados en el capítulo anterior. Este comienza con una descripción y caracterización general del sistema, donde se  abordan cada uno de los componentes requeridos para su completo funcionamiento. Posteriormente se detalla la ingienría de software requerida en la etapa de conceptualización de la aplicación, se explican de forma detallada los aspectos teóricos y de implementación de la base de datos, el funcionamiento del protocolo de comunicación y por último los escenarios de juegos requeridos en las rutinas de entrenamiento ligero y clínico, y las estadísticas generadas por estos. Como herramienta de desarrollo se utilizó c\#.

% SYSTEM DESCRIPTION AND CHARACTERIZATION TO APPLY
\input{main/chapter2/section1/content.tex}
     
% SERIOUS GAME REQUIREMENTS
\input{main/chapter2/section2/content.tex}    

% USE CASE DEFINITION
\input{main/chapter2/section3/content.tex}

% USE CASE REALIZATION
\input{main/chapter2/section4/content.tex}

% DATABASE DESIGN 
\input{main/chapter2/section5/content.tex}

% DATA MANIPULATION
\input{main/chapter2/section6/content.tex}

% COMMUNICATION 
\input{main/chapter2/section7/content.tex}

% TRAINING SCENARIOS
\input{main/chapter2/section8/content.tex}

% EMG GRAPHIC
\input{main/chapter2/section9/content.tex}

% STATICAL REPORTS GENERATION
\input{main/chapter2/section10/content.tex}

\subthesischapter{Conclusiones del capítulo}
Se presentó una descripción del sistema de adquisición de datos para rehabilitación, sus componentes, características distintivas y su funcionamiento. Se identificaron y definieron los requisitos del juego  serio, tanto funcionales como no funcionales, así como los actores y casos de usos del sistema que establecieron las bases fundamentales para el desarrollo de la aplicación. Se realizó el diseño de la base de datos, abarcando tanto el modelo lógico como el físico, lo que aseguró una estructura robusta y eficiente para el almacenamiento de los datos. La manipulación de los datos se abordó de manera integral, desde la conexión con la base de datos hasta la persistencia de los resultados estadísticos. Se diseñó e implementó la comunicación con el pedal motorizado y la implementación de la interfaz gráfica para la representación de los datos EMG. Se definieron los escenarios de entrenamiento para las modalidades Ligero y Clínico, asegurando una cobertura completa de las necesidad de entrenamiento del usuario. Por último en el ámbito estadístico se desarrolló una serie de gráficos para el seguimiento de los resultados en las rutinas de entrenamiento.   
    
\end{thesischapter}

% DATA MANIPULATION
\begin{thesischapter}{2} {Diseño e implementación del Juego Serio}
En este capítulo se discuten los detalles de desarrollo de los aspectos citados en el capítulo anterior. Este comienza con una descripción y caracterización general del sistema, donde se  abordan cada uno de los componentes requeridos para su completo funcionamiento. Posteriormente se detalla la ingienría de software requerida en la etapa de conceptualización de la aplicación, se explican de forma detallada los aspectos teóricos y de implementación de la base de datos, el funcionamiento del protocolo de comunicación y por último los escenarios de juegos requeridos en las rutinas de entrenamiento ligero y clínico, y las estadísticas generadas por estos. Como herramienta de desarrollo se utilizó c\#.

% SYSTEM DESCRIPTION AND CHARACTERIZATION TO APPLY
\input{main/chapter2/section1/content.tex}
     
% SERIOUS GAME REQUIREMENTS
\input{main/chapter2/section2/content.tex}    

% USE CASE DEFINITION
\input{main/chapter2/section3/content.tex}

% USE CASE REALIZATION
\input{main/chapter2/section4/content.tex}

% DATABASE DESIGN 
\input{main/chapter2/section5/content.tex}

% DATA MANIPULATION
\input{main/chapter2/section6/content.tex}

% COMMUNICATION 
\input{main/chapter2/section7/content.tex}

% TRAINING SCENARIOS
\input{main/chapter2/section8/content.tex}

% EMG GRAPHIC
\input{main/chapter2/section9/content.tex}

% STATICAL REPORTS GENERATION
\input{main/chapter2/section10/content.tex}

\subthesischapter{Conclusiones del capítulo}
Se presentó una descripción del sistema de adquisición de datos para rehabilitación, sus componentes, características distintivas y su funcionamiento. Se identificaron y definieron los requisitos del juego  serio, tanto funcionales como no funcionales, así como los actores y casos de usos del sistema que establecieron las bases fundamentales para el desarrollo de la aplicación. Se realizó el diseño de la base de datos, abarcando tanto el modelo lógico como el físico, lo que aseguró una estructura robusta y eficiente para el almacenamiento de los datos. La manipulación de los datos se abordó de manera integral, desde la conexión con la base de datos hasta la persistencia de los resultados estadísticos. Se diseñó e implementó la comunicación con el pedal motorizado y la implementación de la interfaz gráfica para la representación de los datos EMG. Se definieron los escenarios de entrenamiento para las modalidades Ligero y Clínico, asegurando una cobertura completa de las necesidad de entrenamiento del usuario. Por último en el ámbito estadístico se desarrolló una serie de gráficos para el seguimiento de los resultados en las rutinas de entrenamiento.   
    
\end{thesischapter}

% COMMUNICATION 
\begin{thesischapter}{2} {Diseño e implementación del Juego Serio}
En este capítulo se discuten los detalles de desarrollo de los aspectos citados en el capítulo anterior. Este comienza con una descripción y caracterización general del sistema, donde se  abordan cada uno de los componentes requeridos para su completo funcionamiento. Posteriormente se detalla la ingienría de software requerida en la etapa de conceptualización de la aplicación, se explican de forma detallada los aspectos teóricos y de implementación de la base de datos, el funcionamiento del protocolo de comunicación y por último los escenarios de juegos requeridos en las rutinas de entrenamiento ligero y clínico, y las estadísticas generadas por estos. Como herramienta de desarrollo se utilizó c\#.

% SYSTEM DESCRIPTION AND CHARACTERIZATION TO APPLY
\input{main/chapter2/section1/content.tex}
     
% SERIOUS GAME REQUIREMENTS
\input{main/chapter2/section2/content.tex}    

% USE CASE DEFINITION
\input{main/chapter2/section3/content.tex}

% USE CASE REALIZATION
\input{main/chapter2/section4/content.tex}

% DATABASE DESIGN 
\input{main/chapter2/section5/content.tex}

% DATA MANIPULATION
\input{main/chapter2/section6/content.tex}

% COMMUNICATION 
\input{main/chapter2/section7/content.tex}

% TRAINING SCENARIOS
\input{main/chapter2/section8/content.tex}

% EMG GRAPHIC
\input{main/chapter2/section9/content.tex}

% STATICAL REPORTS GENERATION
\input{main/chapter2/section10/content.tex}

\subthesischapter{Conclusiones del capítulo}
Se presentó una descripción del sistema de adquisición de datos para rehabilitación, sus componentes, características distintivas y su funcionamiento. Se identificaron y definieron los requisitos del juego  serio, tanto funcionales como no funcionales, así como los actores y casos de usos del sistema que establecieron las bases fundamentales para el desarrollo de la aplicación. Se realizó el diseño de la base de datos, abarcando tanto el modelo lógico como el físico, lo que aseguró una estructura robusta y eficiente para el almacenamiento de los datos. La manipulación de los datos se abordó de manera integral, desde la conexión con la base de datos hasta la persistencia de los resultados estadísticos. Se diseñó e implementó la comunicación con el pedal motorizado y la implementación de la interfaz gráfica para la representación de los datos EMG. Se definieron los escenarios de entrenamiento para las modalidades Ligero y Clínico, asegurando una cobertura completa de las necesidad de entrenamiento del usuario. Por último en el ámbito estadístico se desarrolló una serie de gráficos para el seguimiento de los resultados en las rutinas de entrenamiento.   
    
\end{thesischapter}

% TRAINING SCENARIOS
\begin{thesischapter}{2} {Diseño e implementación del Juego Serio}
En este capítulo se discuten los detalles de desarrollo de los aspectos citados en el capítulo anterior. Este comienza con una descripción y caracterización general del sistema, donde se  abordan cada uno de los componentes requeridos para su completo funcionamiento. Posteriormente se detalla la ingienría de software requerida en la etapa de conceptualización de la aplicación, se explican de forma detallada los aspectos teóricos y de implementación de la base de datos, el funcionamiento del protocolo de comunicación y por último los escenarios de juegos requeridos en las rutinas de entrenamiento ligero y clínico, y las estadísticas generadas por estos. Como herramienta de desarrollo se utilizó c\#.

% SYSTEM DESCRIPTION AND CHARACTERIZATION TO APPLY
\input{main/chapter2/section1/content.tex}
     
% SERIOUS GAME REQUIREMENTS
\input{main/chapter2/section2/content.tex}    

% USE CASE DEFINITION
\input{main/chapter2/section3/content.tex}

% USE CASE REALIZATION
\input{main/chapter2/section4/content.tex}

% DATABASE DESIGN 
\input{main/chapter2/section5/content.tex}

% DATA MANIPULATION
\input{main/chapter2/section6/content.tex}

% COMMUNICATION 
\input{main/chapter2/section7/content.tex}

% TRAINING SCENARIOS
\input{main/chapter2/section8/content.tex}

% EMG GRAPHIC
\input{main/chapter2/section9/content.tex}

% STATICAL REPORTS GENERATION
\input{main/chapter2/section10/content.tex}

\subthesischapter{Conclusiones del capítulo}
Se presentó una descripción del sistema de adquisición de datos para rehabilitación, sus componentes, características distintivas y su funcionamiento. Se identificaron y definieron los requisitos del juego  serio, tanto funcionales como no funcionales, así como los actores y casos de usos del sistema que establecieron las bases fundamentales para el desarrollo de la aplicación. Se realizó el diseño de la base de datos, abarcando tanto el modelo lógico como el físico, lo que aseguró una estructura robusta y eficiente para el almacenamiento de los datos. La manipulación de los datos se abordó de manera integral, desde la conexión con la base de datos hasta la persistencia de los resultados estadísticos. Se diseñó e implementó la comunicación con el pedal motorizado y la implementación de la interfaz gráfica para la representación de los datos EMG. Se definieron los escenarios de entrenamiento para las modalidades Ligero y Clínico, asegurando una cobertura completa de las necesidad de entrenamiento del usuario. Por último en el ámbito estadístico se desarrolló una serie de gráficos para el seguimiento de los resultados en las rutinas de entrenamiento.   
    
\end{thesischapter}

% EMG GRAPHIC
\begin{thesischapter}{2} {Diseño e implementación del Juego Serio}
En este capítulo se discuten los detalles de desarrollo de los aspectos citados en el capítulo anterior. Este comienza con una descripción y caracterización general del sistema, donde se  abordan cada uno de los componentes requeridos para su completo funcionamiento. Posteriormente se detalla la ingienría de software requerida en la etapa de conceptualización de la aplicación, se explican de forma detallada los aspectos teóricos y de implementación de la base de datos, el funcionamiento del protocolo de comunicación y por último los escenarios de juegos requeridos en las rutinas de entrenamiento ligero y clínico, y las estadísticas generadas por estos. Como herramienta de desarrollo se utilizó c\#.

% SYSTEM DESCRIPTION AND CHARACTERIZATION TO APPLY
\input{main/chapter2/section1/content.tex}
     
% SERIOUS GAME REQUIREMENTS
\input{main/chapter2/section2/content.tex}    

% USE CASE DEFINITION
\input{main/chapter2/section3/content.tex}

% USE CASE REALIZATION
\input{main/chapter2/section4/content.tex}

% DATABASE DESIGN 
\input{main/chapter2/section5/content.tex}

% DATA MANIPULATION
\input{main/chapter2/section6/content.tex}

% COMMUNICATION 
\input{main/chapter2/section7/content.tex}

% TRAINING SCENARIOS
\input{main/chapter2/section8/content.tex}

% EMG GRAPHIC
\input{main/chapter2/section9/content.tex}

% STATICAL REPORTS GENERATION
\input{main/chapter2/section10/content.tex}

\subthesischapter{Conclusiones del capítulo}
Se presentó una descripción del sistema de adquisición de datos para rehabilitación, sus componentes, características distintivas y su funcionamiento. Se identificaron y definieron los requisitos del juego  serio, tanto funcionales como no funcionales, así como los actores y casos de usos del sistema que establecieron las bases fundamentales para el desarrollo de la aplicación. Se realizó el diseño de la base de datos, abarcando tanto el modelo lógico como el físico, lo que aseguró una estructura robusta y eficiente para el almacenamiento de los datos. La manipulación de los datos se abordó de manera integral, desde la conexión con la base de datos hasta la persistencia de los resultados estadísticos. Se diseñó e implementó la comunicación con el pedal motorizado y la implementación de la interfaz gráfica para la representación de los datos EMG. Se definieron los escenarios de entrenamiento para las modalidades Ligero y Clínico, asegurando una cobertura completa de las necesidad de entrenamiento del usuario. Por último en el ámbito estadístico se desarrolló una serie de gráficos para el seguimiento de los resultados en las rutinas de entrenamiento.   
    
\end{thesischapter}

% STATICAL REPORTS GENERATION
\begin{thesischapter}{2} {Diseño e implementación del Juego Serio}
En este capítulo se discuten los detalles de desarrollo de los aspectos citados en el capítulo anterior. Este comienza con una descripción y caracterización general del sistema, donde se  abordan cada uno de los componentes requeridos para su completo funcionamiento. Posteriormente se detalla la ingienría de software requerida en la etapa de conceptualización de la aplicación, se explican de forma detallada los aspectos teóricos y de implementación de la base de datos, el funcionamiento del protocolo de comunicación y por último los escenarios de juegos requeridos en las rutinas de entrenamiento ligero y clínico, y las estadísticas generadas por estos. Como herramienta de desarrollo se utilizó c\#.

% SYSTEM DESCRIPTION AND CHARACTERIZATION TO APPLY
\input{main/chapter2/section1/content.tex}
     
% SERIOUS GAME REQUIREMENTS
\input{main/chapter2/section2/content.tex}    

% USE CASE DEFINITION
\input{main/chapter2/section3/content.tex}

% USE CASE REALIZATION
\input{main/chapter2/section4/content.tex}

% DATABASE DESIGN 
\input{main/chapter2/section5/content.tex}

% DATA MANIPULATION
\input{main/chapter2/section6/content.tex}

% COMMUNICATION 
\input{main/chapter2/section7/content.tex}

% TRAINING SCENARIOS
\input{main/chapter2/section8/content.tex}

% EMG GRAPHIC
\input{main/chapter2/section9/content.tex}

% STATICAL REPORTS GENERATION
\input{main/chapter2/section10/content.tex}

\subthesischapter{Conclusiones del capítulo}
Se presentó una descripción del sistema de adquisición de datos para rehabilitación, sus componentes, características distintivas y su funcionamiento. Se identificaron y definieron los requisitos del juego  serio, tanto funcionales como no funcionales, así como los actores y casos de usos del sistema que establecieron las bases fundamentales para el desarrollo de la aplicación. Se realizó el diseño de la base de datos, abarcando tanto el modelo lógico como el físico, lo que aseguró una estructura robusta y eficiente para el almacenamiento de los datos. La manipulación de los datos se abordó de manera integral, desde la conexión con la base de datos hasta la persistencia de los resultados estadísticos. Se diseñó e implementó la comunicación con el pedal motorizado y la implementación de la interfaz gráfica para la representación de los datos EMG. Se definieron los escenarios de entrenamiento para las modalidades Ligero y Clínico, asegurando una cobertura completa de las necesidad de entrenamiento del usuario. Por último en el ámbito estadístico se desarrolló una serie de gráficos para el seguimiento de los resultados en las rutinas de entrenamiento.   
    
\end{thesischapter}

\subthesischapter{Conclusiones del capítulo}
Se presentó una descripción del sistema de adquisición de datos para rehabilitación, sus componentes, características distintivas y su funcionamiento. Se identificaron y definieron los requisitos del juego  serio, tanto funcionales como no funcionales, así como los actores y casos de usos del sistema que establecieron las bases fundamentales para el desarrollo de la aplicación. Se realizó el diseño de la base de datos, abarcando tanto el modelo lógico como el físico, lo que aseguró una estructura robusta y eficiente para el almacenamiento de los datos. La manipulación de los datos se abordó de manera integral, desde la conexión con la base de datos hasta la persistencia de los resultados estadísticos. Se diseñó e implementó la comunicación con el pedal motorizado y la implementación de la interfaz gráfica para la representación de los datos EMG. Se definieron los escenarios de entrenamiento para las modalidades Ligero y Clínico, asegurando una cobertura completa de las necesidad de entrenamiento del usuario. Por último en el ámbito estadístico se desarrolló una serie de gráficos para el seguimiento de los resultados en las rutinas de entrenamiento.   
    
\end{thesischapter}

% STATICAL REPORTS GENERATION
\begin{thesischapter}{2} {Diseño e implementación del Juego Serio}
En este capítulo se discuten los detalles de desarrollo de los aspectos citados en el capítulo anterior. Este comienza con una descripción y caracterización general del sistema, donde se  abordan cada uno de los componentes requeridos para su completo funcionamiento. Posteriormente se detalla la ingienría de software requerida en la etapa de conceptualización de la aplicación, se explican de forma detallada los aspectos teóricos y de implementación de la base de datos, el funcionamiento del protocolo de comunicación y por último los escenarios de juegos requeridos en las rutinas de entrenamiento ligero y clínico, y las estadísticas generadas por estos. Como herramienta de desarrollo se utilizó c\#.

% SYSTEM DESCRIPTION AND CHARACTERIZATION TO APPLY
\begin{thesischapter}{2} {Diseño e implementación del Juego Serio}
En este capítulo se discuten los detalles de desarrollo de los aspectos citados en el capítulo anterior. Este comienza con una descripción y caracterización general del sistema, donde se  abordan cada uno de los componentes requeridos para su completo funcionamiento. Posteriormente se detalla la ingienría de software requerida en la etapa de conceptualización de la aplicación, se explican de forma detallada los aspectos teóricos y de implementación de la base de datos, el funcionamiento del protocolo de comunicación y por último los escenarios de juegos requeridos en las rutinas de entrenamiento ligero y clínico, y las estadísticas generadas por estos. Como herramienta de desarrollo se utilizó c\#.

% SYSTEM DESCRIPTION AND CHARACTERIZATION TO APPLY
\input{main/chapter2/section1/content.tex}
     
% SERIOUS GAME REQUIREMENTS
\input{main/chapter2/section2/content.tex}    

% USE CASE DEFINITION
\input{main/chapter2/section3/content.tex}

% USE CASE REALIZATION
\input{main/chapter2/section4/content.tex}

% DATABASE DESIGN 
\input{main/chapter2/section5/content.tex}

% DATA MANIPULATION
\input{main/chapter2/section6/content.tex}

% COMMUNICATION 
\input{main/chapter2/section7/content.tex}

% TRAINING SCENARIOS
\input{main/chapter2/section8/content.tex}

% EMG GRAPHIC
\input{main/chapter2/section9/content.tex}

% STATICAL REPORTS GENERATION
\input{main/chapter2/section10/content.tex}

\subthesischapter{Conclusiones del capítulo}
Se presentó una descripción del sistema de adquisición de datos para rehabilitación, sus componentes, características distintivas y su funcionamiento. Se identificaron y definieron los requisitos del juego  serio, tanto funcionales como no funcionales, así como los actores y casos de usos del sistema que establecieron las bases fundamentales para el desarrollo de la aplicación. Se realizó el diseño de la base de datos, abarcando tanto el modelo lógico como el físico, lo que aseguró una estructura robusta y eficiente para el almacenamiento de los datos. La manipulación de los datos se abordó de manera integral, desde la conexión con la base de datos hasta la persistencia de los resultados estadísticos. Se diseñó e implementó la comunicación con el pedal motorizado y la implementación de la interfaz gráfica para la representación de los datos EMG. Se definieron los escenarios de entrenamiento para las modalidades Ligero y Clínico, asegurando una cobertura completa de las necesidad de entrenamiento del usuario. Por último en el ámbito estadístico se desarrolló una serie de gráficos para el seguimiento de los resultados en las rutinas de entrenamiento.   
    
\end{thesischapter}
     
% SERIOUS GAME REQUIREMENTS
\begin{thesischapter}{2} {Diseño e implementación del Juego Serio}
En este capítulo se discuten los detalles de desarrollo de los aspectos citados en el capítulo anterior. Este comienza con una descripción y caracterización general del sistema, donde se  abordan cada uno de los componentes requeridos para su completo funcionamiento. Posteriormente se detalla la ingienría de software requerida en la etapa de conceptualización de la aplicación, se explican de forma detallada los aspectos teóricos y de implementación de la base de datos, el funcionamiento del protocolo de comunicación y por último los escenarios de juegos requeridos en las rutinas de entrenamiento ligero y clínico, y las estadísticas generadas por estos. Como herramienta de desarrollo se utilizó c\#.

% SYSTEM DESCRIPTION AND CHARACTERIZATION TO APPLY
\input{main/chapter2/section1/content.tex}
     
% SERIOUS GAME REQUIREMENTS
\input{main/chapter2/section2/content.tex}    

% USE CASE DEFINITION
\input{main/chapter2/section3/content.tex}

% USE CASE REALIZATION
\input{main/chapter2/section4/content.tex}

% DATABASE DESIGN 
\input{main/chapter2/section5/content.tex}

% DATA MANIPULATION
\input{main/chapter2/section6/content.tex}

% COMMUNICATION 
\input{main/chapter2/section7/content.tex}

% TRAINING SCENARIOS
\input{main/chapter2/section8/content.tex}

% EMG GRAPHIC
\input{main/chapter2/section9/content.tex}

% STATICAL REPORTS GENERATION
\input{main/chapter2/section10/content.tex}

\subthesischapter{Conclusiones del capítulo}
Se presentó una descripción del sistema de adquisición de datos para rehabilitación, sus componentes, características distintivas y su funcionamiento. Se identificaron y definieron los requisitos del juego  serio, tanto funcionales como no funcionales, así como los actores y casos de usos del sistema que establecieron las bases fundamentales para el desarrollo de la aplicación. Se realizó el diseño de la base de datos, abarcando tanto el modelo lógico como el físico, lo que aseguró una estructura robusta y eficiente para el almacenamiento de los datos. La manipulación de los datos se abordó de manera integral, desde la conexión con la base de datos hasta la persistencia de los resultados estadísticos. Se diseñó e implementó la comunicación con el pedal motorizado y la implementación de la interfaz gráfica para la representación de los datos EMG. Se definieron los escenarios de entrenamiento para las modalidades Ligero y Clínico, asegurando una cobertura completa de las necesidad de entrenamiento del usuario. Por último en el ámbito estadístico se desarrolló una serie de gráficos para el seguimiento de los resultados en las rutinas de entrenamiento.   
    
\end{thesischapter}    

% USE CASE DEFINITION
\begin{thesischapter}{2} {Diseño e implementación del Juego Serio}
En este capítulo se discuten los detalles de desarrollo de los aspectos citados en el capítulo anterior. Este comienza con una descripción y caracterización general del sistema, donde se  abordan cada uno de los componentes requeridos para su completo funcionamiento. Posteriormente se detalla la ingienría de software requerida en la etapa de conceptualización de la aplicación, se explican de forma detallada los aspectos teóricos y de implementación de la base de datos, el funcionamiento del protocolo de comunicación y por último los escenarios de juegos requeridos en las rutinas de entrenamiento ligero y clínico, y las estadísticas generadas por estos. Como herramienta de desarrollo se utilizó c\#.

% SYSTEM DESCRIPTION AND CHARACTERIZATION TO APPLY
\input{main/chapter2/section1/content.tex}
     
% SERIOUS GAME REQUIREMENTS
\input{main/chapter2/section2/content.tex}    

% USE CASE DEFINITION
\input{main/chapter2/section3/content.tex}

% USE CASE REALIZATION
\input{main/chapter2/section4/content.tex}

% DATABASE DESIGN 
\input{main/chapter2/section5/content.tex}

% DATA MANIPULATION
\input{main/chapter2/section6/content.tex}

% COMMUNICATION 
\input{main/chapter2/section7/content.tex}

% TRAINING SCENARIOS
\input{main/chapter2/section8/content.tex}

% EMG GRAPHIC
\input{main/chapter2/section9/content.tex}

% STATICAL REPORTS GENERATION
\input{main/chapter2/section10/content.tex}

\subthesischapter{Conclusiones del capítulo}
Se presentó una descripción del sistema de adquisición de datos para rehabilitación, sus componentes, características distintivas y su funcionamiento. Se identificaron y definieron los requisitos del juego  serio, tanto funcionales como no funcionales, así como los actores y casos de usos del sistema que establecieron las bases fundamentales para el desarrollo de la aplicación. Se realizó el diseño de la base de datos, abarcando tanto el modelo lógico como el físico, lo que aseguró una estructura robusta y eficiente para el almacenamiento de los datos. La manipulación de los datos se abordó de manera integral, desde la conexión con la base de datos hasta la persistencia de los resultados estadísticos. Se diseñó e implementó la comunicación con el pedal motorizado y la implementación de la interfaz gráfica para la representación de los datos EMG. Se definieron los escenarios de entrenamiento para las modalidades Ligero y Clínico, asegurando una cobertura completa de las necesidad de entrenamiento del usuario. Por último en el ámbito estadístico se desarrolló una serie de gráficos para el seguimiento de los resultados en las rutinas de entrenamiento.   
    
\end{thesischapter}

% USE CASE REALIZATION
\begin{thesischapter}{2} {Diseño e implementación del Juego Serio}
En este capítulo se discuten los detalles de desarrollo de los aspectos citados en el capítulo anterior. Este comienza con una descripción y caracterización general del sistema, donde se  abordan cada uno de los componentes requeridos para su completo funcionamiento. Posteriormente se detalla la ingienría de software requerida en la etapa de conceptualización de la aplicación, se explican de forma detallada los aspectos teóricos y de implementación de la base de datos, el funcionamiento del protocolo de comunicación y por último los escenarios de juegos requeridos en las rutinas de entrenamiento ligero y clínico, y las estadísticas generadas por estos. Como herramienta de desarrollo se utilizó c\#.

% SYSTEM DESCRIPTION AND CHARACTERIZATION TO APPLY
\input{main/chapter2/section1/content.tex}
     
% SERIOUS GAME REQUIREMENTS
\input{main/chapter2/section2/content.tex}    

% USE CASE DEFINITION
\input{main/chapter2/section3/content.tex}

% USE CASE REALIZATION
\input{main/chapter2/section4/content.tex}

% DATABASE DESIGN 
\input{main/chapter2/section5/content.tex}

% DATA MANIPULATION
\input{main/chapter2/section6/content.tex}

% COMMUNICATION 
\input{main/chapter2/section7/content.tex}

% TRAINING SCENARIOS
\input{main/chapter2/section8/content.tex}

% EMG GRAPHIC
\input{main/chapter2/section9/content.tex}

% STATICAL REPORTS GENERATION
\input{main/chapter2/section10/content.tex}

\subthesischapter{Conclusiones del capítulo}
Se presentó una descripción del sistema de adquisición de datos para rehabilitación, sus componentes, características distintivas y su funcionamiento. Se identificaron y definieron los requisitos del juego  serio, tanto funcionales como no funcionales, así como los actores y casos de usos del sistema que establecieron las bases fundamentales para el desarrollo de la aplicación. Se realizó el diseño de la base de datos, abarcando tanto el modelo lógico como el físico, lo que aseguró una estructura robusta y eficiente para el almacenamiento de los datos. La manipulación de los datos se abordó de manera integral, desde la conexión con la base de datos hasta la persistencia de los resultados estadísticos. Se diseñó e implementó la comunicación con el pedal motorizado y la implementación de la interfaz gráfica para la representación de los datos EMG. Se definieron los escenarios de entrenamiento para las modalidades Ligero y Clínico, asegurando una cobertura completa de las necesidad de entrenamiento del usuario. Por último en el ámbito estadístico se desarrolló una serie de gráficos para el seguimiento de los resultados en las rutinas de entrenamiento.   
    
\end{thesischapter}

% DATABASE DESIGN 
\begin{thesischapter}{2} {Diseño e implementación del Juego Serio}
En este capítulo se discuten los detalles de desarrollo de los aspectos citados en el capítulo anterior. Este comienza con una descripción y caracterización general del sistema, donde se  abordan cada uno de los componentes requeridos para su completo funcionamiento. Posteriormente se detalla la ingienría de software requerida en la etapa de conceptualización de la aplicación, se explican de forma detallada los aspectos teóricos y de implementación de la base de datos, el funcionamiento del protocolo de comunicación y por último los escenarios de juegos requeridos en las rutinas de entrenamiento ligero y clínico, y las estadísticas generadas por estos. Como herramienta de desarrollo se utilizó c\#.

% SYSTEM DESCRIPTION AND CHARACTERIZATION TO APPLY
\input{main/chapter2/section1/content.tex}
     
% SERIOUS GAME REQUIREMENTS
\input{main/chapter2/section2/content.tex}    

% USE CASE DEFINITION
\input{main/chapter2/section3/content.tex}

% USE CASE REALIZATION
\input{main/chapter2/section4/content.tex}

% DATABASE DESIGN 
\input{main/chapter2/section5/content.tex}

% DATA MANIPULATION
\input{main/chapter2/section6/content.tex}

% COMMUNICATION 
\input{main/chapter2/section7/content.tex}

% TRAINING SCENARIOS
\input{main/chapter2/section8/content.tex}

% EMG GRAPHIC
\input{main/chapter2/section9/content.tex}

% STATICAL REPORTS GENERATION
\input{main/chapter2/section10/content.tex}

\subthesischapter{Conclusiones del capítulo}
Se presentó una descripción del sistema de adquisición de datos para rehabilitación, sus componentes, características distintivas y su funcionamiento. Se identificaron y definieron los requisitos del juego  serio, tanto funcionales como no funcionales, así como los actores y casos de usos del sistema que establecieron las bases fundamentales para el desarrollo de la aplicación. Se realizó el diseño de la base de datos, abarcando tanto el modelo lógico como el físico, lo que aseguró una estructura robusta y eficiente para el almacenamiento de los datos. La manipulación de los datos se abordó de manera integral, desde la conexión con la base de datos hasta la persistencia de los resultados estadísticos. Se diseñó e implementó la comunicación con el pedal motorizado y la implementación de la interfaz gráfica para la representación de los datos EMG. Se definieron los escenarios de entrenamiento para las modalidades Ligero y Clínico, asegurando una cobertura completa de las necesidad de entrenamiento del usuario. Por último en el ámbito estadístico se desarrolló una serie de gráficos para el seguimiento de los resultados en las rutinas de entrenamiento.   
    
\end{thesischapter}

% DATA MANIPULATION
\begin{thesischapter}{2} {Diseño e implementación del Juego Serio}
En este capítulo se discuten los detalles de desarrollo de los aspectos citados en el capítulo anterior. Este comienza con una descripción y caracterización general del sistema, donde se  abordan cada uno de los componentes requeridos para su completo funcionamiento. Posteriormente se detalla la ingienría de software requerida en la etapa de conceptualización de la aplicación, se explican de forma detallada los aspectos teóricos y de implementación de la base de datos, el funcionamiento del protocolo de comunicación y por último los escenarios de juegos requeridos en las rutinas de entrenamiento ligero y clínico, y las estadísticas generadas por estos. Como herramienta de desarrollo se utilizó c\#.

% SYSTEM DESCRIPTION AND CHARACTERIZATION TO APPLY
\input{main/chapter2/section1/content.tex}
     
% SERIOUS GAME REQUIREMENTS
\input{main/chapter2/section2/content.tex}    

% USE CASE DEFINITION
\input{main/chapter2/section3/content.tex}

% USE CASE REALIZATION
\input{main/chapter2/section4/content.tex}

% DATABASE DESIGN 
\input{main/chapter2/section5/content.tex}

% DATA MANIPULATION
\input{main/chapter2/section6/content.tex}

% COMMUNICATION 
\input{main/chapter2/section7/content.tex}

% TRAINING SCENARIOS
\input{main/chapter2/section8/content.tex}

% EMG GRAPHIC
\input{main/chapter2/section9/content.tex}

% STATICAL REPORTS GENERATION
\input{main/chapter2/section10/content.tex}

\subthesischapter{Conclusiones del capítulo}
Se presentó una descripción del sistema de adquisición de datos para rehabilitación, sus componentes, características distintivas y su funcionamiento. Se identificaron y definieron los requisitos del juego  serio, tanto funcionales como no funcionales, así como los actores y casos de usos del sistema que establecieron las bases fundamentales para el desarrollo de la aplicación. Se realizó el diseño de la base de datos, abarcando tanto el modelo lógico como el físico, lo que aseguró una estructura robusta y eficiente para el almacenamiento de los datos. La manipulación de los datos se abordó de manera integral, desde la conexión con la base de datos hasta la persistencia de los resultados estadísticos. Se diseñó e implementó la comunicación con el pedal motorizado y la implementación de la interfaz gráfica para la representación de los datos EMG. Se definieron los escenarios de entrenamiento para las modalidades Ligero y Clínico, asegurando una cobertura completa de las necesidad de entrenamiento del usuario. Por último en el ámbito estadístico se desarrolló una serie de gráficos para el seguimiento de los resultados en las rutinas de entrenamiento.   
    
\end{thesischapter}

% COMMUNICATION 
\begin{thesischapter}{2} {Diseño e implementación del Juego Serio}
En este capítulo se discuten los detalles de desarrollo de los aspectos citados en el capítulo anterior. Este comienza con una descripción y caracterización general del sistema, donde se  abordan cada uno de los componentes requeridos para su completo funcionamiento. Posteriormente se detalla la ingienría de software requerida en la etapa de conceptualización de la aplicación, se explican de forma detallada los aspectos teóricos y de implementación de la base de datos, el funcionamiento del protocolo de comunicación y por último los escenarios de juegos requeridos en las rutinas de entrenamiento ligero y clínico, y las estadísticas generadas por estos. Como herramienta de desarrollo se utilizó c\#.

% SYSTEM DESCRIPTION AND CHARACTERIZATION TO APPLY
\input{main/chapter2/section1/content.tex}
     
% SERIOUS GAME REQUIREMENTS
\input{main/chapter2/section2/content.tex}    

% USE CASE DEFINITION
\input{main/chapter2/section3/content.tex}

% USE CASE REALIZATION
\input{main/chapter2/section4/content.tex}

% DATABASE DESIGN 
\input{main/chapter2/section5/content.tex}

% DATA MANIPULATION
\input{main/chapter2/section6/content.tex}

% COMMUNICATION 
\input{main/chapter2/section7/content.tex}

% TRAINING SCENARIOS
\input{main/chapter2/section8/content.tex}

% EMG GRAPHIC
\input{main/chapter2/section9/content.tex}

% STATICAL REPORTS GENERATION
\input{main/chapter2/section10/content.tex}

\subthesischapter{Conclusiones del capítulo}
Se presentó una descripción del sistema de adquisición de datos para rehabilitación, sus componentes, características distintivas y su funcionamiento. Se identificaron y definieron los requisitos del juego  serio, tanto funcionales como no funcionales, así como los actores y casos de usos del sistema que establecieron las bases fundamentales para el desarrollo de la aplicación. Se realizó el diseño de la base de datos, abarcando tanto el modelo lógico como el físico, lo que aseguró una estructura robusta y eficiente para el almacenamiento de los datos. La manipulación de los datos se abordó de manera integral, desde la conexión con la base de datos hasta la persistencia de los resultados estadísticos. Se diseñó e implementó la comunicación con el pedal motorizado y la implementación de la interfaz gráfica para la representación de los datos EMG. Se definieron los escenarios de entrenamiento para las modalidades Ligero y Clínico, asegurando una cobertura completa de las necesidad de entrenamiento del usuario. Por último en el ámbito estadístico se desarrolló una serie de gráficos para el seguimiento de los resultados en las rutinas de entrenamiento.   
    
\end{thesischapter}

% TRAINING SCENARIOS
\begin{thesischapter}{2} {Diseño e implementación del Juego Serio}
En este capítulo se discuten los detalles de desarrollo de los aspectos citados en el capítulo anterior. Este comienza con una descripción y caracterización general del sistema, donde se  abordan cada uno de los componentes requeridos para su completo funcionamiento. Posteriormente se detalla la ingienría de software requerida en la etapa de conceptualización de la aplicación, se explican de forma detallada los aspectos teóricos y de implementación de la base de datos, el funcionamiento del protocolo de comunicación y por último los escenarios de juegos requeridos en las rutinas de entrenamiento ligero y clínico, y las estadísticas generadas por estos. Como herramienta de desarrollo se utilizó c\#.

% SYSTEM DESCRIPTION AND CHARACTERIZATION TO APPLY
\input{main/chapter2/section1/content.tex}
     
% SERIOUS GAME REQUIREMENTS
\input{main/chapter2/section2/content.tex}    

% USE CASE DEFINITION
\input{main/chapter2/section3/content.tex}

% USE CASE REALIZATION
\input{main/chapter2/section4/content.tex}

% DATABASE DESIGN 
\input{main/chapter2/section5/content.tex}

% DATA MANIPULATION
\input{main/chapter2/section6/content.tex}

% COMMUNICATION 
\input{main/chapter2/section7/content.tex}

% TRAINING SCENARIOS
\input{main/chapter2/section8/content.tex}

% EMG GRAPHIC
\input{main/chapter2/section9/content.tex}

% STATICAL REPORTS GENERATION
\input{main/chapter2/section10/content.tex}

\subthesischapter{Conclusiones del capítulo}
Se presentó una descripción del sistema de adquisición de datos para rehabilitación, sus componentes, características distintivas y su funcionamiento. Se identificaron y definieron los requisitos del juego  serio, tanto funcionales como no funcionales, así como los actores y casos de usos del sistema que establecieron las bases fundamentales para el desarrollo de la aplicación. Se realizó el diseño de la base de datos, abarcando tanto el modelo lógico como el físico, lo que aseguró una estructura robusta y eficiente para el almacenamiento de los datos. La manipulación de los datos se abordó de manera integral, desde la conexión con la base de datos hasta la persistencia de los resultados estadísticos. Se diseñó e implementó la comunicación con el pedal motorizado y la implementación de la interfaz gráfica para la representación de los datos EMG. Se definieron los escenarios de entrenamiento para las modalidades Ligero y Clínico, asegurando una cobertura completa de las necesidad de entrenamiento del usuario. Por último en el ámbito estadístico se desarrolló una serie de gráficos para el seguimiento de los resultados en las rutinas de entrenamiento.   
    
\end{thesischapter}

% EMG GRAPHIC
\begin{thesischapter}{2} {Diseño e implementación del Juego Serio}
En este capítulo se discuten los detalles de desarrollo de los aspectos citados en el capítulo anterior. Este comienza con una descripción y caracterización general del sistema, donde se  abordan cada uno de los componentes requeridos para su completo funcionamiento. Posteriormente se detalla la ingienría de software requerida en la etapa de conceptualización de la aplicación, se explican de forma detallada los aspectos teóricos y de implementación de la base de datos, el funcionamiento del protocolo de comunicación y por último los escenarios de juegos requeridos en las rutinas de entrenamiento ligero y clínico, y las estadísticas generadas por estos. Como herramienta de desarrollo se utilizó c\#.

% SYSTEM DESCRIPTION AND CHARACTERIZATION TO APPLY
\input{main/chapter2/section1/content.tex}
     
% SERIOUS GAME REQUIREMENTS
\input{main/chapter2/section2/content.tex}    

% USE CASE DEFINITION
\input{main/chapter2/section3/content.tex}

% USE CASE REALIZATION
\input{main/chapter2/section4/content.tex}

% DATABASE DESIGN 
\input{main/chapter2/section5/content.tex}

% DATA MANIPULATION
\input{main/chapter2/section6/content.tex}

% COMMUNICATION 
\input{main/chapter2/section7/content.tex}

% TRAINING SCENARIOS
\input{main/chapter2/section8/content.tex}

% EMG GRAPHIC
\input{main/chapter2/section9/content.tex}

% STATICAL REPORTS GENERATION
\input{main/chapter2/section10/content.tex}

\subthesischapter{Conclusiones del capítulo}
Se presentó una descripción del sistema de adquisición de datos para rehabilitación, sus componentes, características distintivas y su funcionamiento. Se identificaron y definieron los requisitos del juego  serio, tanto funcionales como no funcionales, así como los actores y casos de usos del sistema que establecieron las bases fundamentales para el desarrollo de la aplicación. Se realizó el diseño de la base de datos, abarcando tanto el modelo lógico como el físico, lo que aseguró una estructura robusta y eficiente para el almacenamiento de los datos. La manipulación de los datos se abordó de manera integral, desde la conexión con la base de datos hasta la persistencia de los resultados estadísticos. Se diseñó e implementó la comunicación con el pedal motorizado y la implementación de la interfaz gráfica para la representación de los datos EMG. Se definieron los escenarios de entrenamiento para las modalidades Ligero y Clínico, asegurando una cobertura completa de las necesidad de entrenamiento del usuario. Por último en el ámbito estadístico se desarrolló una serie de gráficos para el seguimiento de los resultados en las rutinas de entrenamiento.   
    
\end{thesischapter}

% STATICAL REPORTS GENERATION
\begin{thesischapter}{2} {Diseño e implementación del Juego Serio}
En este capítulo se discuten los detalles de desarrollo de los aspectos citados en el capítulo anterior. Este comienza con una descripción y caracterización general del sistema, donde se  abordan cada uno de los componentes requeridos para su completo funcionamiento. Posteriormente se detalla la ingienría de software requerida en la etapa de conceptualización de la aplicación, se explican de forma detallada los aspectos teóricos y de implementación de la base de datos, el funcionamiento del protocolo de comunicación y por último los escenarios de juegos requeridos en las rutinas de entrenamiento ligero y clínico, y las estadísticas generadas por estos. Como herramienta de desarrollo se utilizó c\#.

% SYSTEM DESCRIPTION AND CHARACTERIZATION TO APPLY
\input{main/chapter2/section1/content.tex}
     
% SERIOUS GAME REQUIREMENTS
\input{main/chapter2/section2/content.tex}    

% USE CASE DEFINITION
\input{main/chapter2/section3/content.tex}

% USE CASE REALIZATION
\input{main/chapter2/section4/content.tex}

% DATABASE DESIGN 
\input{main/chapter2/section5/content.tex}

% DATA MANIPULATION
\input{main/chapter2/section6/content.tex}

% COMMUNICATION 
\input{main/chapter2/section7/content.tex}

% TRAINING SCENARIOS
\input{main/chapter2/section8/content.tex}

% EMG GRAPHIC
\input{main/chapter2/section9/content.tex}

% STATICAL REPORTS GENERATION
\input{main/chapter2/section10/content.tex}

\subthesischapter{Conclusiones del capítulo}
Se presentó una descripción del sistema de adquisición de datos para rehabilitación, sus componentes, características distintivas y su funcionamiento. Se identificaron y definieron los requisitos del juego  serio, tanto funcionales como no funcionales, así como los actores y casos de usos del sistema que establecieron las bases fundamentales para el desarrollo de la aplicación. Se realizó el diseño de la base de datos, abarcando tanto el modelo lógico como el físico, lo que aseguró una estructura robusta y eficiente para el almacenamiento de los datos. La manipulación de los datos se abordó de manera integral, desde la conexión con la base de datos hasta la persistencia de los resultados estadísticos. Se diseñó e implementó la comunicación con el pedal motorizado y la implementación de la interfaz gráfica para la representación de los datos EMG. Se definieron los escenarios de entrenamiento para las modalidades Ligero y Clínico, asegurando una cobertura completa de las necesidad de entrenamiento del usuario. Por último en el ámbito estadístico se desarrolló una serie de gráficos para el seguimiento de los resultados en las rutinas de entrenamiento.   
    
\end{thesischapter}

\subthesischapter{Conclusiones del capítulo}
Se presentó una descripción del sistema de adquisición de datos para rehabilitación, sus componentes, características distintivas y su funcionamiento. Se identificaron y definieron los requisitos del juego  serio, tanto funcionales como no funcionales, así como los actores y casos de usos del sistema que establecieron las bases fundamentales para el desarrollo de la aplicación. Se realizó el diseño de la base de datos, abarcando tanto el modelo lógico como el físico, lo que aseguró una estructura robusta y eficiente para el almacenamiento de los datos. La manipulación de los datos se abordó de manera integral, desde la conexión con la base de datos hasta la persistencia de los resultados estadísticos. Se diseñó e implementó la comunicación con el pedal motorizado y la implementación de la interfaz gráfica para la representación de los datos EMG. Se definieron los escenarios de entrenamiento para las modalidades Ligero y Clínico, asegurando una cobertura completa de las necesidad de entrenamiento del usuario. Por último en el ámbito estadístico se desarrolló una serie de gráficos para el seguimiento de los resultados en las rutinas de entrenamiento.   
    
\end{thesischapter}

\subthesischapter{Conclusiones del capítulo}
Se presentó una descripción del sistema de adquisición de datos para rehabilitación, sus componentes, características distintivas y su funcionamiento. Se identificaron y definieron los requisitos del juego  serio, tanto funcionales como no funcionales, así como los actores y casos de usos del sistema que establecieron las bases fundamentales para el desarrollo de la aplicación. Se realizó el diseño de la base de datos, abarcando tanto el modelo lógico como el físico, lo que aseguró una estructura robusta y eficiente para el almacenamiento de los datos. La manipulación de los datos se abordó de manera integral, desde la conexión con la base de datos hasta la persistencia de los resultados estadísticos. Se diseñó e implementó la comunicación con el pedal motorizado y la implementación de la interfaz gráfica para la representación de los datos EMG. Se definieron los escenarios de entrenamiento para las modalidades Ligero y Clínico, asegurando una cobertura completa de las necesidad de entrenamiento del usuario. Por último en el ámbito estadístico se desarrolló una serie de gráficos para el seguimiento de los resultados en las rutinas de entrenamiento.   
    
\end{thesischapter}

% SECTION 3: SYSTEM TESTING  
\begin{thesischapter}{2} {Diseño e implementación del Juego Serio}
En este capítulo se discuten los detalles de desarrollo de los aspectos citados en el capítulo anterior. Este comienza con una descripción y caracterización general del sistema, donde se  abordan cada uno de los componentes requeridos para su completo funcionamiento. Posteriormente se detalla la ingienría de software requerida en la etapa de conceptualización de la aplicación, se explican de forma detallada los aspectos teóricos y de implementación de la base de datos, el funcionamiento del protocolo de comunicación y por último los escenarios de juegos requeridos en las rutinas de entrenamiento ligero y clínico, y las estadísticas generadas por estos. Como herramienta de desarrollo se utilizó c\#.

% SYSTEM DESCRIPTION AND CHARACTERIZATION TO APPLY
\begin{thesischapter}{2} {Diseño e implementación del Juego Serio}
En este capítulo se discuten los detalles de desarrollo de los aspectos citados en el capítulo anterior. Este comienza con una descripción y caracterización general del sistema, donde se  abordan cada uno de los componentes requeridos para su completo funcionamiento. Posteriormente se detalla la ingienría de software requerida en la etapa de conceptualización de la aplicación, se explican de forma detallada los aspectos teóricos y de implementación de la base de datos, el funcionamiento del protocolo de comunicación y por último los escenarios de juegos requeridos en las rutinas de entrenamiento ligero y clínico, y las estadísticas generadas por estos. Como herramienta de desarrollo se utilizó c\#.

% SYSTEM DESCRIPTION AND CHARACTERIZATION TO APPLY
\begin{thesischapter}{2} {Diseño e implementación del Juego Serio}
En este capítulo se discuten los detalles de desarrollo de los aspectos citados en el capítulo anterior. Este comienza con una descripción y caracterización general del sistema, donde se  abordan cada uno de los componentes requeridos para su completo funcionamiento. Posteriormente se detalla la ingienría de software requerida en la etapa de conceptualización de la aplicación, se explican de forma detallada los aspectos teóricos y de implementación de la base de datos, el funcionamiento del protocolo de comunicación y por último los escenarios de juegos requeridos en las rutinas de entrenamiento ligero y clínico, y las estadísticas generadas por estos. Como herramienta de desarrollo se utilizó c\#.

% SYSTEM DESCRIPTION AND CHARACTERIZATION TO APPLY
\input{main/chapter2/section1/content.tex}
     
% SERIOUS GAME REQUIREMENTS
\input{main/chapter2/section2/content.tex}    

% USE CASE DEFINITION
\input{main/chapter2/section3/content.tex}

% USE CASE REALIZATION
\input{main/chapter2/section4/content.tex}

% DATABASE DESIGN 
\input{main/chapter2/section5/content.tex}

% DATA MANIPULATION
\input{main/chapter2/section6/content.tex}

% COMMUNICATION 
\input{main/chapter2/section7/content.tex}

% TRAINING SCENARIOS
\input{main/chapter2/section8/content.tex}

% EMG GRAPHIC
\input{main/chapter2/section9/content.tex}

% STATICAL REPORTS GENERATION
\input{main/chapter2/section10/content.tex}

\subthesischapter{Conclusiones del capítulo}
Se presentó una descripción del sistema de adquisición de datos para rehabilitación, sus componentes, características distintivas y su funcionamiento. Se identificaron y definieron los requisitos del juego  serio, tanto funcionales como no funcionales, así como los actores y casos de usos del sistema que establecieron las bases fundamentales para el desarrollo de la aplicación. Se realizó el diseño de la base de datos, abarcando tanto el modelo lógico como el físico, lo que aseguró una estructura robusta y eficiente para el almacenamiento de los datos. La manipulación de los datos se abordó de manera integral, desde la conexión con la base de datos hasta la persistencia de los resultados estadísticos. Se diseñó e implementó la comunicación con el pedal motorizado y la implementación de la interfaz gráfica para la representación de los datos EMG. Se definieron los escenarios de entrenamiento para las modalidades Ligero y Clínico, asegurando una cobertura completa de las necesidad de entrenamiento del usuario. Por último en el ámbito estadístico se desarrolló una serie de gráficos para el seguimiento de los resultados en las rutinas de entrenamiento.   
    
\end{thesischapter}
     
% SERIOUS GAME REQUIREMENTS
\begin{thesischapter}{2} {Diseño e implementación del Juego Serio}
En este capítulo se discuten los detalles de desarrollo de los aspectos citados en el capítulo anterior. Este comienza con una descripción y caracterización general del sistema, donde se  abordan cada uno de los componentes requeridos para su completo funcionamiento. Posteriormente se detalla la ingienría de software requerida en la etapa de conceptualización de la aplicación, se explican de forma detallada los aspectos teóricos y de implementación de la base de datos, el funcionamiento del protocolo de comunicación y por último los escenarios de juegos requeridos en las rutinas de entrenamiento ligero y clínico, y las estadísticas generadas por estos. Como herramienta de desarrollo se utilizó c\#.

% SYSTEM DESCRIPTION AND CHARACTERIZATION TO APPLY
\input{main/chapter2/section1/content.tex}
     
% SERIOUS GAME REQUIREMENTS
\input{main/chapter2/section2/content.tex}    

% USE CASE DEFINITION
\input{main/chapter2/section3/content.tex}

% USE CASE REALIZATION
\input{main/chapter2/section4/content.tex}

% DATABASE DESIGN 
\input{main/chapter2/section5/content.tex}

% DATA MANIPULATION
\input{main/chapter2/section6/content.tex}

% COMMUNICATION 
\input{main/chapter2/section7/content.tex}

% TRAINING SCENARIOS
\input{main/chapter2/section8/content.tex}

% EMG GRAPHIC
\input{main/chapter2/section9/content.tex}

% STATICAL REPORTS GENERATION
\input{main/chapter2/section10/content.tex}

\subthesischapter{Conclusiones del capítulo}
Se presentó una descripción del sistema de adquisición de datos para rehabilitación, sus componentes, características distintivas y su funcionamiento. Se identificaron y definieron los requisitos del juego  serio, tanto funcionales como no funcionales, así como los actores y casos de usos del sistema que establecieron las bases fundamentales para el desarrollo de la aplicación. Se realizó el diseño de la base de datos, abarcando tanto el modelo lógico como el físico, lo que aseguró una estructura robusta y eficiente para el almacenamiento de los datos. La manipulación de los datos se abordó de manera integral, desde la conexión con la base de datos hasta la persistencia de los resultados estadísticos. Se diseñó e implementó la comunicación con el pedal motorizado y la implementación de la interfaz gráfica para la representación de los datos EMG. Se definieron los escenarios de entrenamiento para las modalidades Ligero y Clínico, asegurando una cobertura completa de las necesidad de entrenamiento del usuario. Por último en el ámbito estadístico se desarrolló una serie de gráficos para el seguimiento de los resultados en las rutinas de entrenamiento.   
    
\end{thesischapter}    

% USE CASE DEFINITION
\begin{thesischapter}{2} {Diseño e implementación del Juego Serio}
En este capítulo se discuten los detalles de desarrollo de los aspectos citados en el capítulo anterior. Este comienza con una descripción y caracterización general del sistema, donde se  abordan cada uno de los componentes requeridos para su completo funcionamiento. Posteriormente se detalla la ingienría de software requerida en la etapa de conceptualización de la aplicación, se explican de forma detallada los aspectos teóricos y de implementación de la base de datos, el funcionamiento del protocolo de comunicación y por último los escenarios de juegos requeridos en las rutinas de entrenamiento ligero y clínico, y las estadísticas generadas por estos. Como herramienta de desarrollo se utilizó c\#.

% SYSTEM DESCRIPTION AND CHARACTERIZATION TO APPLY
\input{main/chapter2/section1/content.tex}
     
% SERIOUS GAME REQUIREMENTS
\input{main/chapter2/section2/content.tex}    

% USE CASE DEFINITION
\input{main/chapter2/section3/content.tex}

% USE CASE REALIZATION
\input{main/chapter2/section4/content.tex}

% DATABASE DESIGN 
\input{main/chapter2/section5/content.tex}

% DATA MANIPULATION
\input{main/chapter2/section6/content.tex}

% COMMUNICATION 
\input{main/chapter2/section7/content.tex}

% TRAINING SCENARIOS
\input{main/chapter2/section8/content.tex}

% EMG GRAPHIC
\input{main/chapter2/section9/content.tex}

% STATICAL REPORTS GENERATION
\input{main/chapter2/section10/content.tex}

\subthesischapter{Conclusiones del capítulo}
Se presentó una descripción del sistema de adquisición de datos para rehabilitación, sus componentes, características distintivas y su funcionamiento. Se identificaron y definieron los requisitos del juego  serio, tanto funcionales como no funcionales, así como los actores y casos de usos del sistema que establecieron las bases fundamentales para el desarrollo de la aplicación. Se realizó el diseño de la base de datos, abarcando tanto el modelo lógico como el físico, lo que aseguró una estructura robusta y eficiente para el almacenamiento de los datos. La manipulación de los datos se abordó de manera integral, desde la conexión con la base de datos hasta la persistencia de los resultados estadísticos. Se diseñó e implementó la comunicación con el pedal motorizado y la implementación de la interfaz gráfica para la representación de los datos EMG. Se definieron los escenarios de entrenamiento para las modalidades Ligero y Clínico, asegurando una cobertura completa de las necesidad de entrenamiento del usuario. Por último en el ámbito estadístico se desarrolló una serie de gráficos para el seguimiento de los resultados en las rutinas de entrenamiento.   
    
\end{thesischapter}

% USE CASE REALIZATION
\begin{thesischapter}{2} {Diseño e implementación del Juego Serio}
En este capítulo se discuten los detalles de desarrollo de los aspectos citados en el capítulo anterior. Este comienza con una descripción y caracterización general del sistema, donde se  abordan cada uno de los componentes requeridos para su completo funcionamiento. Posteriormente se detalla la ingienría de software requerida en la etapa de conceptualización de la aplicación, se explican de forma detallada los aspectos teóricos y de implementación de la base de datos, el funcionamiento del protocolo de comunicación y por último los escenarios de juegos requeridos en las rutinas de entrenamiento ligero y clínico, y las estadísticas generadas por estos. Como herramienta de desarrollo se utilizó c\#.

% SYSTEM DESCRIPTION AND CHARACTERIZATION TO APPLY
\input{main/chapter2/section1/content.tex}
     
% SERIOUS GAME REQUIREMENTS
\input{main/chapter2/section2/content.tex}    

% USE CASE DEFINITION
\input{main/chapter2/section3/content.tex}

% USE CASE REALIZATION
\input{main/chapter2/section4/content.tex}

% DATABASE DESIGN 
\input{main/chapter2/section5/content.tex}

% DATA MANIPULATION
\input{main/chapter2/section6/content.tex}

% COMMUNICATION 
\input{main/chapter2/section7/content.tex}

% TRAINING SCENARIOS
\input{main/chapter2/section8/content.tex}

% EMG GRAPHIC
\input{main/chapter2/section9/content.tex}

% STATICAL REPORTS GENERATION
\input{main/chapter2/section10/content.tex}

\subthesischapter{Conclusiones del capítulo}
Se presentó una descripción del sistema de adquisición de datos para rehabilitación, sus componentes, características distintivas y su funcionamiento. Se identificaron y definieron los requisitos del juego  serio, tanto funcionales como no funcionales, así como los actores y casos de usos del sistema que establecieron las bases fundamentales para el desarrollo de la aplicación. Se realizó el diseño de la base de datos, abarcando tanto el modelo lógico como el físico, lo que aseguró una estructura robusta y eficiente para el almacenamiento de los datos. La manipulación de los datos se abordó de manera integral, desde la conexión con la base de datos hasta la persistencia de los resultados estadísticos. Se diseñó e implementó la comunicación con el pedal motorizado y la implementación de la interfaz gráfica para la representación de los datos EMG. Se definieron los escenarios de entrenamiento para las modalidades Ligero y Clínico, asegurando una cobertura completa de las necesidad de entrenamiento del usuario. Por último en el ámbito estadístico se desarrolló una serie de gráficos para el seguimiento de los resultados en las rutinas de entrenamiento.   
    
\end{thesischapter}

% DATABASE DESIGN 
\begin{thesischapter}{2} {Diseño e implementación del Juego Serio}
En este capítulo se discuten los detalles de desarrollo de los aspectos citados en el capítulo anterior. Este comienza con una descripción y caracterización general del sistema, donde se  abordan cada uno de los componentes requeridos para su completo funcionamiento. Posteriormente se detalla la ingienría de software requerida en la etapa de conceptualización de la aplicación, se explican de forma detallada los aspectos teóricos y de implementación de la base de datos, el funcionamiento del protocolo de comunicación y por último los escenarios de juegos requeridos en las rutinas de entrenamiento ligero y clínico, y las estadísticas generadas por estos. Como herramienta de desarrollo se utilizó c\#.

% SYSTEM DESCRIPTION AND CHARACTERIZATION TO APPLY
\input{main/chapter2/section1/content.tex}
     
% SERIOUS GAME REQUIREMENTS
\input{main/chapter2/section2/content.tex}    

% USE CASE DEFINITION
\input{main/chapter2/section3/content.tex}

% USE CASE REALIZATION
\input{main/chapter2/section4/content.tex}

% DATABASE DESIGN 
\input{main/chapter2/section5/content.tex}

% DATA MANIPULATION
\input{main/chapter2/section6/content.tex}

% COMMUNICATION 
\input{main/chapter2/section7/content.tex}

% TRAINING SCENARIOS
\input{main/chapter2/section8/content.tex}

% EMG GRAPHIC
\input{main/chapter2/section9/content.tex}

% STATICAL REPORTS GENERATION
\input{main/chapter2/section10/content.tex}

\subthesischapter{Conclusiones del capítulo}
Se presentó una descripción del sistema de adquisición de datos para rehabilitación, sus componentes, características distintivas y su funcionamiento. Se identificaron y definieron los requisitos del juego  serio, tanto funcionales como no funcionales, así como los actores y casos de usos del sistema que establecieron las bases fundamentales para el desarrollo de la aplicación. Se realizó el diseño de la base de datos, abarcando tanto el modelo lógico como el físico, lo que aseguró una estructura robusta y eficiente para el almacenamiento de los datos. La manipulación de los datos se abordó de manera integral, desde la conexión con la base de datos hasta la persistencia de los resultados estadísticos. Se diseñó e implementó la comunicación con el pedal motorizado y la implementación de la interfaz gráfica para la representación de los datos EMG. Se definieron los escenarios de entrenamiento para las modalidades Ligero y Clínico, asegurando una cobertura completa de las necesidad de entrenamiento del usuario. Por último en el ámbito estadístico se desarrolló una serie de gráficos para el seguimiento de los resultados en las rutinas de entrenamiento.   
    
\end{thesischapter}

% DATA MANIPULATION
\begin{thesischapter}{2} {Diseño e implementación del Juego Serio}
En este capítulo se discuten los detalles de desarrollo de los aspectos citados en el capítulo anterior. Este comienza con una descripción y caracterización general del sistema, donde se  abordan cada uno de los componentes requeridos para su completo funcionamiento. Posteriormente se detalla la ingienría de software requerida en la etapa de conceptualización de la aplicación, se explican de forma detallada los aspectos teóricos y de implementación de la base de datos, el funcionamiento del protocolo de comunicación y por último los escenarios de juegos requeridos en las rutinas de entrenamiento ligero y clínico, y las estadísticas generadas por estos. Como herramienta de desarrollo se utilizó c\#.

% SYSTEM DESCRIPTION AND CHARACTERIZATION TO APPLY
\input{main/chapter2/section1/content.tex}
     
% SERIOUS GAME REQUIREMENTS
\input{main/chapter2/section2/content.tex}    

% USE CASE DEFINITION
\input{main/chapter2/section3/content.tex}

% USE CASE REALIZATION
\input{main/chapter2/section4/content.tex}

% DATABASE DESIGN 
\input{main/chapter2/section5/content.tex}

% DATA MANIPULATION
\input{main/chapter2/section6/content.tex}

% COMMUNICATION 
\input{main/chapter2/section7/content.tex}

% TRAINING SCENARIOS
\input{main/chapter2/section8/content.tex}

% EMG GRAPHIC
\input{main/chapter2/section9/content.tex}

% STATICAL REPORTS GENERATION
\input{main/chapter2/section10/content.tex}

\subthesischapter{Conclusiones del capítulo}
Se presentó una descripción del sistema de adquisición de datos para rehabilitación, sus componentes, características distintivas y su funcionamiento. Se identificaron y definieron los requisitos del juego  serio, tanto funcionales como no funcionales, así como los actores y casos de usos del sistema que establecieron las bases fundamentales para el desarrollo de la aplicación. Se realizó el diseño de la base de datos, abarcando tanto el modelo lógico como el físico, lo que aseguró una estructura robusta y eficiente para el almacenamiento de los datos. La manipulación de los datos se abordó de manera integral, desde la conexión con la base de datos hasta la persistencia de los resultados estadísticos. Se diseñó e implementó la comunicación con el pedal motorizado y la implementación de la interfaz gráfica para la representación de los datos EMG. Se definieron los escenarios de entrenamiento para las modalidades Ligero y Clínico, asegurando una cobertura completa de las necesidad de entrenamiento del usuario. Por último en el ámbito estadístico se desarrolló una serie de gráficos para el seguimiento de los resultados en las rutinas de entrenamiento.   
    
\end{thesischapter}

% COMMUNICATION 
\begin{thesischapter}{2} {Diseño e implementación del Juego Serio}
En este capítulo se discuten los detalles de desarrollo de los aspectos citados en el capítulo anterior. Este comienza con una descripción y caracterización general del sistema, donde se  abordan cada uno de los componentes requeridos para su completo funcionamiento. Posteriormente se detalla la ingienría de software requerida en la etapa de conceptualización de la aplicación, se explican de forma detallada los aspectos teóricos y de implementación de la base de datos, el funcionamiento del protocolo de comunicación y por último los escenarios de juegos requeridos en las rutinas de entrenamiento ligero y clínico, y las estadísticas generadas por estos. Como herramienta de desarrollo se utilizó c\#.

% SYSTEM DESCRIPTION AND CHARACTERIZATION TO APPLY
\input{main/chapter2/section1/content.tex}
     
% SERIOUS GAME REQUIREMENTS
\input{main/chapter2/section2/content.tex}    

% USE CASE DEFINITION
\input{main/chapter2/section3/content.tex}

% USE CASE REALIZATION
\input{main/chapter2/section4/content.tex}

% DATABASE DESIGN 
\input{main/chapter2/section5/content.tex}

% DATA MANIPULATION
\input{main/chapter2/section6/content.tex}

% COMMUNICATION 
\input{main/chapter2/section7/content.tex}

% TRAINING SCENARIOS
\input{main/chapter2/section8/content.tex}

% EMG GRAPHIC
\input{main/chapter2/section9/content.tex}

% STATICAL REPORTS GENERATION
\input{main/chapter2/section10/content.tex}

\subthesischapter{Conclusiones del capítulo}
Se presentó una descripción del sistema de adquisición de datos para rehabilitación, sus componentes, características distintivas y su funcionamiento. Se identificaron y definieron los requisitos del juego  serio, tanto funcionales como no funcionales, así como los actores y casos de usos del sistema que establecieron las bases fundamentales para el desarrollo de la aplicación. Se realizó el diseño de la base de datos, abarcando tanto el modelo lógico como el físico, lo que aseguró una estructura robusta y eficiente para el almacenamiento de los datos. La manipulación de los datos se abordó de manera integral, desde la conexión con la base de datos hasta la persistencia de los resultados estadísticos. Se diseñó e implementó la comunicación con el pedal motorizado y la implementación de la interfaz gráfica para la representación de los datos EMG. Se definieron los escenarios de entrenamiento para las modalidades Ligero y Clínico, asegurando una cobertura completa de las necesidad de entrenamiento del usuario. Por último en el ámbito estadístico se desarrolló una serie de gráficos para el seguimiento de los resultados en las rutinas de entrenamiento.   
    
\end{thesischapter}

% TRAINING SCENARIOS
\begin{thesischapter}{2} {Diseño e implementación del Juego Serio}
En este capítulo se discuten los detalles de desarrollo de los aspectos citados en el capítulo anterior. Este comienza con una descripción y caracterización general del sistema, donde se  abordan cada uno de los componentes requeridos para su completo funcionamiento. Posteriormente se detalla la ingienría de software requerida en la etapa de conceptualización de la aplicación, se explican de forma detallada los aspectos teóricos y de implementación de la base de datos, el funcionamiento del protocolo de comunicación y por último los escenarios de juegos requeridos en las rutinas de entrenamiento ligero y clínico, y las estadísticas generadas por estos. Como herramienta de desarrollo se utilizó c\#.

% SYSTEM DESCRIPTION AND CHARACTERIZATION TO APPLY
\input{main/chapter2/section1/content.tex}
     
% SERIOUS GAME REQUIREMENTS
\input{main/chapter2/section2/content.tex}    

% USE CASE DEFINITION
\input{main/chapter2/section3/content.tex}

% USE CASE REALIZATION
\input{main/chapter2/section4/content.tex}

% DATABASE DESIGN 
\input{main/chapter2/section5/content.tex}

% DATA MANIPULATION
\input{main/chapter2/section6/content.tex}

% COMMUNICATION 
\input{main/chapter2/section7/content.tex}

% TRAINING SCENARIOS
\input{main/chapter2/section8/content.tex}

% EMG GRAPHIC
\input{main/chapter2/section9/content.tex}

% STATICAL REPORTS GENERATION
\input{main/chapter2/section10/content.tex}

\subthesischapter{Conclusiones del capítulo}
Se presentó una descripción del sistema de adquisición de datos para rehabilitación, sus componentes, características distintivas y su funcionamiento. Se identificaron y definieron los requisitos del juego  serio, tanto funcionales como no funcionales, así como los actores y casos de usos del sistema que establecieron las bases fundamentales para el desarrollo de la aplicación. Se realizó el diseño de la base de datos, abarcando tanto el modelo lógico como el físico, lo que aseguró una estructura robusta y eficiente para el almacenamiento de los datos. La manipulación de los datos se abordó de manera integral, desde la conexión con la base de datos hasta la persistencia de los resultados estadísticos. Se diseñó e implementó la comunicación con el pedal motorizado y la implementación de la interfaz gráfica para la representación de los datos EMG. Se definieron los escenarios de entrenamiento para las modalidades Ligero y Clínico, asegurando una cobertura completa de las necesidad de entrenamiento del usuario. Por último en el ámbito estadístico se desarrolló una serie de gráficos para el seguimiento de los resultados en las rutinas de entrenamiento.   
    
\end{thesischapter}

% EMG GRAPHIC
\begin{thesischapter}{2} {Diseño e implementación del Juego Serio}
En este capítulo se discuten los detalles de desarrollo de los aspectos citados en el capítulo anterior. Este comienza con una descripción y caracterización general del sistema, donde se  abordan cada uno de los componentes requeridos para su completo funcionamiento. Posteriormente se detalla la ingienría de software requerida en la etapa de conceptualización de la aplicación, se explican de forma detallada los aspectos teóricos y de implementación de la base de datos, el funcionamiento del protocolo de comunicación y por último los escenarios de juegos requeridos en las rutinas de entrenamiento ligero y clínico, y las estadísticas generadas por estos. Como herramienta de desarrollo se utilizó c\#.

% SYSTEM DESCRIPTION AND CHARACTERIZATION TO APPLY
\input{main/chapter2/section1/content.tex}
     
% SERIOUS GAME REQUIREMENTS
\input{main/chapter2/section2/content.tex}    

% USE CASE DEFINITION
\input{main/chapter2/section3/content.tex}

% USE CASE REALIZATION
\input{main/chapter2/section4/content.tex}

% DATABASE DESIGN 
\input{main/chapter2/section5/content.tex}

% DATA MANIPULATION
\input{main/chapter2/section6/content.tex}

% COMMUNICATION 
\input{main/chapter2/section7/content.tex}

% TRAINING SCENARIOS
\input{main/chapter2/section8/content.tex}

% EMG GRAPHIC
\input{main/chapter2/section9/content.tex}

% STATICAL REPORTS GENERATION
\input{main/chapter2/section10/content.tex}

\subthesischapter{Conclusiones del capítulo}
Se presentó una descripción del sistema de adquisición de datos para rehabilitación, sus componentes, características distintivas y su funcionamiento. Se identificaron y definieron los requisitos del juego  serio, tanto funcionales como no funcionales, así como los actores y casos de usos del sistema que establecieron las bases fundamentales para el desarrollo de la aplicación. Se realizó el diseño de la base de datos, abarcando tanto el modelo lógico como el físico, lo que aseguró una estructura robusta y eficiente para el almacenamiento de los datos. La manipulación de los datos se abordó de manera integral, desde la conexión con la base de datos hasta la persistencia de los resultados estadísticos. Se diseñó e implementó la comunicación con el pedal motorizado y la implementación de la interfaz gráfica para la representación de los datos EMG. Se definieron los escenarios de entrenamiento para las modalidades Ligero y Clínico, asegurando una cobertura completa de las necesidad de entrenamiento del usuario. Por último en el ámbito estadístico se desarrolló una serie de gráficos para el seguimiento de los resultados en las rutinas de entrenamiento.   
    
\end{thesischapter}

% STATICAL REPORTS GENERATION
\begin{thesischapter}{2} {Diseño e implementación del Juego Serio}
En este capítulo se discuten los detalles de desarrollo de los aspectos citados en el capítulo anterior. Este comienza con una descripción y caracterización general del sistema, donde se  abordan cada uno de los componentes requeridos para su completo funcionamiento. Posteriormente se detalla la ingienría de software requerida en la etapa de conceptualización de la aplicación, se explican de forma detallada los aspectos teóricos y de implementación de la base de datos, el funcionamiento del protocolo de comunicación y por último los escenarios de juegos requeridos en las rutinas de entrenamiento ligero y clínico, y las estadísticas generadas por estos. Como herramienta de desarrollo se utilizó c\#.

% SYSTEM DESCRIPTION AND CHARACTERIZATION TO APPLY
\input{main/chapter2/section1/content.tex}
     
% SERIOUS GAME REQUIREMENTS
\input{main/chapter2/section2/content.tex}    

% USE CASE DEFINITION
\input{main/chapter2/section3/content.tex}

% USE CASE REALIZATION
\input{main/chapter2/section4/content.tex}

% DATABASE DESIGN 
\input{main/chapter2/section5/content.tex}

% DATA MANIPULATION
\input{main/chapter2/section6/content.tex}

% COMMUNICATION 
\input{main/chapter2/section7/content.tex}

% TRAINING SCENARIOS
\input{main/chapter2/section8/content.tex}

% EMG GRAPHIC
\input{main/chapter2/section9/content.tex}

% STATICAL REPORTS GENERATION
\input{main/chapter2/section10/content.tex}

\subthesischapter{Conclusiones del capítulo}
Se presentó una descripción del sistema de adquisición de datos para rehabilitación, sus componentes, características distintivas y su funcionamiento. Se identificaron y definieron los requisitos del juego  serio, tanto funcionales como no funcionales, así como los actores y casos de usos del sistema que establecieron las bases fundamentales para el desarrollo de la aplicación. Se realizó el diseño de la base de datos, abarcando tanto el modelo lógico como el físico, lo que aseguró una estructura robusta y eficiente para el almacenamiento de los datos. La manipulación de los datos se abordó de manera integral, desde la conexión con la base de datos hasta la persistencia de los resultados estadísticos. Se diseñó e implementó la comunicación con el pedal motorizado y la implementación de la interfaz gráfica para la representación de los datos EMG. Se definieron los escenarios de entrenamiento para las modalidades Ligero y Clínico, asegurando una cobertura completa de las necesidad de entrenamiento del usuario. Por último en el ámbito estadístico se desarrolló una serie de gráficos para el seguimiento de los resultados en las rutinas de entrenamiento.   
    
\end{thesischapter}

\subthesischapter{Conclusiones del capítulo}
Se presentó una descripción del sistema de adquisición de datos para rehabilitación, sus componentes, características distintivas y su funcionamiento. Se identificaron y definieron los requisitos del juego  serio, tanto funcionales como no funcionales, así como los actores y casos de usos del sistema que establecieron las bases fundamentales para el desarrollo de la aplicación. Se realizó el diseño de la base de datos, abarcando tanto el modelo lógico como el físico, lo que aseguró una estructura robusta y eficiente para el almacenamiento de los datos. La manipulación de los datos se abordó de manera integral, desde la conexión con la base de datos hasta la persistencia de los resultados estadísticos. Se diseñó e implementó la comunicación con el pedal motorizado y la implementación de la interfaz gráfica para la representación de los datos EMG. Se definieron los escenarios de entrenamiento para las modalidades Ligero y Clínico, asegurando una cobertura completa de las necesidad de entrenamiento del usuario. Por último en el ámbito estadístico se desarrolló una serie de gráficos para el seguimiento de los resultados en las rutinas de entrenamiento.   
    
\end{thesischapter}
     
% SERIOUS GAME REQUIREMENTS
\begin{thesischapter}{2} {Diseño e implementación del Juego Serio}
En este capítulo se discuten los detalles de desarrollo de los aspectos citados en el capítulo anterior. Este comienza con una descripción y caracterización general del sistema, donde se  abordan cada uno de los componentes requeridos para su completo funcionamiento. Posteriormente se detalla la ingienría de software requerida en la etapa de conceptualización de la aplicación, se explican de forma detallada los aspectos teóricos y de implementación de la base de datos, el funcionamiento del protocolo de comunicación y por último los escenarios de juegos requeridos en las rutinas de entrenamiento ligero y clínico, y las estadísticas generadas por estos. Como herramienta de desarrollo se utilizó c\#.

% SYSTEM DESCRIPTION AND CHARACTERIZATION TO APPLY
\begin{thesischapter}{2} {Diseño e implementación del Juego Serio}
En este capítulo se discuten los detalles de desarrollo de los aspectos citados en el capítulo anterior. Este comienza con una descripción y caracterización general del sistema, donde se  abordan cada uno de los componentes requeridos para su completo funcionamiento. Posteriormente se detalla la ingienría de software requerida en la etapa de conceptualización de la aplicación, se explican de forma detallada los aspectos teóricos y de implementación de la base de datos, el funcionamiento del protocolo de comunicación y por último los escenarios de juegos requeridos en las rutinas de entrenamiento ligero y clínico, y las estadísticas generadas por estos. Como herramienta de desarrollo se utilizó c\#.

% SYSTEM DESCRIPTION AND CHARACTERIZATION TO APPLY
\input{main/chapter2/section1/content.tex}
     
% SERIOUS GAME REQUIREMENTS
\input{main/chapter2/section2/content.tex}    

% USE CASE DEFINITION
\input{main/chapter2/section3/content.tex}

% USE CASE REALIZATION
\input{main/chapter2/section4/content.tex}

% DATABASE DESIGN 
\input{main/chapter2/section5/content.tex}

% DATA MANIPULATION
\input{main/chapter2/section6/content.tex}

% COMMUNICATION 
\input{main/chapter2/section7/content.tex}

% TRAINING SCENARIOS
\input{main/chapter2/section8/content.tex}

% EMG GRAPHIC
\input{main/chapter2/section9/content.tex}

% STATICAL REPORTS GENERATION
\input{main/chapter2/section10/content.tex}

\subthesischapter{Conclusiones del capítulo}
Se presentó una descripción del sistema de adquisición de datos para rehabilitación, sus componentes, características distintivas y su funcionamiento. Se identificaron y definieron los requisitos del juego  serio, tanto funcionales como no funcionales, así como los actores y casos de usos del sistema que establecieron las bases fundamentales para el desarrollo de la aplicación. Se realizó el diseño de la base de datos, abarcando tanto el modelo lógico como el físico, lo que aseguró una estructura robusta y eficiente para el almacenamiento de los datos. La manipulación de los datos se abordó de manera integral, desde la conexión con la base de datos hasta la persistencia de los resultados estadísticos. Se diseñó e implementó la comunicación con el pedal motorizado y la implementación de la interfaz gráfica para la representación de los datos EMG. Se definieron los escenarios de entrenamiento para las modalidades Ligero y Clínico, asegurando una cobertura completa de las necesidad de entrenamiento del usuario. Por último en el ámbito estadístico se desarrolló una serie de gráficos para el seguimiento de los resultados en las rutinas de entrenamiento.   
    
\end{thesischapter}
     
% SERIOUS GAME REQUIREMENTS
\begin{thesischapter}{2} {Diseño e implementación del Juego Serio}
En este capítulo se discuten los detalles de desarrollo de los aspectos citados en el capítulo anterior. Este comienza con una descripción y caracterización general del sistema, donde se  abordan cada uno de los componentes requeridos para su completo funcionamiento. Posteriormente se detalla la ingienría de software requerida en la etapa de conceptualización de la aplicación, se explican de forma detallada los aspectos teóricos y de implementación de la base de datos, el funcionamiento del protocolo de comunicación y por último los escenarios de juegos requeridos en las rutinas de entrenamiento ligero y clínico, y las estadísticas generadas por estos. Como herramienta de desarrollo se utilizó c\#.

% SYSTEM DESCRIPTION AND CHARACTERIZATION TO APPLY
\input{main/chapter2/section1/content.tex}
     
% SERIOUS GAME REQUIREMENTS
\input{main/chapter2/section2/content.tex}    

% USE CASE DEFINITION
\input{main/chapter2/section3/content.tex}

% USE CASE REALIZATION
\input{main/chapter2/section4/content.tex}

% DATABASE DESIGN 
\input{main/chapter2/section5/content.tex}

% DATA MANIPULATION
\input{main/chapter2/section6/content.tex}

% COMMUNICATION 
\input{main/chapter2/section7/content.tex}

% TRAINING SCENARIOS
\input{main/chapter2/section8/content.tex}

% EMG GRAPHIC
\input{main/chapter2/section9/content.tex}

% STATICAL REPORTS GENERATION
\input{main/chapter2/section10/content.tex}

\subthesischapter{Conclusiones del capítulo}
Se presentó una descripción del sistema de adquisición de datos para rehabilitación, sus componentes, características distintivas y su funcionamiento. Se identificaron y definieron los requisitos del juego  serio, tanto funcionales como no funcionales, así como los actores y casos de usos del sistema que establecieron las bases fundamentales para el desarrollo de la aplicación. Se realizó el diseño de la base de datos, abarcando tanto el modelo lógico como el físico, lo que aseguró una estructura robusta y eficiente para el almacenamiento de los datos. La manipulación de los datos se abordó de manera integral, desde la conexión con la base de datos hasta la persistencia de los resultados estadísticos. Se diseñó e implementó la comunicación con el pedal motorizado y la implementación de la interfaz gráfica para la representación de los datos EMG. Se definieron los escenarios de entrenamiento para las modalidades Ligero y Clínico, asegurando una cobertura completa de las necesidad de entrenamiento del usuario. Por último en el ámbito estadístico se desarrolló una serie de gráficos para el seguimiento de los resultados en las rutinas de entrenamiento.   
    
\end{thesischapter}    

% USE CASE DEFINITION
\begin{thesischapter}{2} {Diseño e implementación del Juego Serio}
En este capítulo se discuten los detalles de desarrollo de los aspectos citados en el capítulo anterior. Este comienza con una descripción y caracterización general del sistema, donde se  abordan cada uno de los componentes requeridos para su completo funcionamiento. Posteriormente se detalla la ingienría de software requerida en la etapa de conceptualización de la aplicación, se explican de forma detallada los aspectos teóricos y de implementación de la base de datos, el funcionamiento del protocolo de comunicación y por último los escenarios de juegos requeridos en las rutinas de entrenamiento ligero y clínico, y las estadísticas generadas por estos. Como herramienta de desarrollo se utilizó c\#.

% SYSTEM DESCRIPTION AND CHARACTERIZATION TO APPLY
\input{main/chapter2/section1/content.tex}
     
% SERIOUS GAME REQUIREMENTS
\input{main/chapter2/section2/content.tex}    

% USE CASE DEFINITION
\input{main/chapter2/section3/content.tex}

% USE CASE REALIZATION
\input{main/chapter2/section4/content.tex}

% DATABASE DESIGN 
\input{main/chapter2/section5/content.tex}

% DATA MANIPULATION
\input{main/chapter2/section6/content.tex}

% COMMUNICATION 
\input{main/chapter2/section7/content.tex}

% TRAINING SCENARIOS
\input{main/chapter2/section8/content.tex}

% EMG GRAPHIC
\input{main/chapter2/section9/content.tex}

% STATICAL REPORTS GENERATION
\input{main/chapter2/section10/content.tex}

\subthesischapter{Conclusiones del capítulo}
Se presentó una descripción del sistema de adquisición de datos para rehabilitación, sus componentes, características distintivas y su funcionamiento. Se identificaron y definieron los requisitos del juego  serio, tanto funcionales como no funcionales, así como los actores y casos de usos del sistema que establecieron las bases fundamentales para el desarrollo de la aplicación. Se realizó el diseño de la base de datos, abarcando tanto el modelo lógico como el físico, lo que aseguró una estructura robusta y eficiente para el almacenamiento de los datos. La manipulación de los datos se abordó de manera integral, desde la conexión con la base de datos hasta la persistencia de los resultados estadísticos. Se diseñó e implementó la comunicación con el pedal motorizado y la implementación de la interfaz gráfica para la representación de los datos EMG. Se definieron los escenarios de entrenamiento para las modalidades Ligero y Clínico, asegurando una cobertura completa de las necesidad de entrenamiento del usuario. Por último en el ámbito estadístico se desarrolló una serie de gráficos para el seguimiento de los resultados en las rutinas de entrenamiento.   
    
\end{thesischapter}

% USE CASE REALIZATION
\begin{thesischapter}{2} {Diseño e implementación del Juego Serio}
En este capítulo se discuten los detalles de desarrollo de los aspectos citados en el capítulo anterior. Este comienza con una descripción y caracterización general del sistema, donde se  abordan cada uno de los componentes requeridos para su completo funcionamiento. Posteriormente se detalla la ingienría de software requerida en la etapa de conceptualización de la aplicación, se explican de forma detallada los aspectos teóricos y de implementación de la base de datos, el funcionamiento del protocolo de comunicación y por último los escenarios de juegos requeridos en las rutinas de entrenamiento ligero y clínico, y las estadísticas generadas por estos. Como herramienta de desarrollo se utilizó c\#.

% SYSTEM DESCRIPTION AND CHARACTERIZATION TO APPLY
\input{main/chapter2/section1/content.tex}
     
% SERIOUS GAME REQUIREMENTS
\input{main/chapter2/section2/content.tex}    

% USE CASE DEFINITION
\input{main/chapter2/section3/content.tex}

% USE CASE REALIZATION
\input{main/chapter2/section4/content.tex}

% DATABASE DESIGN 
\input{main/chapter2/section5/content.tex}

% DATA MANIPULATION
\input{main/chapter2/section6/content.tex}

% COMMUNICATION 
\input{main/chapter2/section7/content.tex}

% TRAINING SCENARIOS
\input{main/chapter2/section8/content.tex}

% EMG GRAPHIC
\input{main/chapter2/section9/content.tex}

% STATICAL REPORTS GENERATION
\input{main/chapter2/section10/content.tex}

\subthesischapter{Conclusiones del capítulo}
Se presentó una descripción del sistema de adquisición de datos para rehabilitación, sus componentes, características distintivas y su funcionamiento. Se identificaron y definieron los requisitos del juego  serio, tanto funcionales como no funcionales, así como los actores y casos de usos del sistema que establecieron las bases fundamentales para el desarrollo de la aplicación. Se realizó el diseño de la base de datos, abarcando tanto el modelo lógico como el físico, lo que aseguró una estructura robusta y eficiente para el almacenamiento de los datos. La manipulación de los datos se abordó de manera integral, desde la conexión con la base de datos hasta la persistencia de los resultados estadísticos. Se diseñó e implementó la comunicación con el pedal motorizado y la implementación de la interfaz gráfica para la representación de los datos EMG. Se definieron los escenarios de entrenamiento para las modalidades Ligero y Clínico, asegurando una cobertura completa de las necesidad de entrenamiento del usuario. Por último en el ámbito estadístico se desarrolló una serie de gráficos para el seguimiento de los resultados en las rutinas de entrenamiento.   
    
\end{thesischapter}

% DATABASE DESIGN 
\begin{thesischapter}{2} {Diseño e implementación del Juego Serio}
En este capítulo se discuten los detalles de desarrollo de los aspectos citados en el capítulo anterior. Este comienza con una descripción y caracterización general del sistema, donde se  abordan cada uno de los componentes requeridos para su completo funcionamiento. Posteriormente se detalla la ingienría de software requerida en la etapa de conceptualización de la aplicación, se explican de forma detallada los aspectos teóricos y de implementación de la base de datos, el funcionamiento del protocolo de comunicación y por último los escenarios de juegos requeridos en las rutinas de entrenamiento ligero y clínico, y las estadísticas generadas por estos. Como herramienta de desarrollo se utilizó c\#.

% SYSTEM DESCRIPTION AND CHARACTERIZATION TO APPLY
\input{main/chapter2/section1/content.tex}
     
% SERIOUS GAME REQUIREMENTS
\input{main/chapter2/section2/content.tex}    

% USE CASE DEFINITION
\input{main/chapter2/section3/content.tex}

% USE CASE REALIZATION
\input{main/chapter2/section4/content.tex}

% DATABASE DESIGN 
\input{main/chapter2/section5/content.tex}

% DATA MANIPULATION
\input{main/chapter2/section6/content.tex}

% COMMUNICATION 
\input{main/chapter2/section7/content.tex}

% TRAINING SCENARIOS
\input{main/chapter2/section8/content.tex}

% EMG GRAPHIC
\input{main/chapter2/section9/content.tex}

% STATICAL REPORTS GENERATION
\input{main/chapter2/section10/content.tex}

\subthesischapter{Conclusiones del capítulo}
Se presentó una descripción del sistema de adquisición de datos para rehabilitación, sus componentes, características distintivas y su funcionamiento. Se identificaron y definieron los requisitos del juego  serio, tanto funcionales como no funcionales, así como los actores y casos de usos del sistema que establecieron las bases fundamentales para el desarrollo de la aplicación. Se realizó el diseño de la base de datos, abarcando tanto el modelo lógico como el físico, lo que aseguró una estructura robusta y eficiente para el almacenamiento de los datos. La manipulación de los datos se abordó de manera integral, desde la conexión con la base de datos hasta la persistencia de los resultados estadísticos. Se diseñó e implementó la comunicación con el pedal motorizado y la implementación de la interfaz gráfica para la representación de los datos EMG. Se definieron los escenarios de entrenamiento para las modalidades Ligero y Clínico, asegurando una cobertura completa de las necesidad de entrenamiento del usuario. Por último en el ámbito estadístico se desarrolló una serie de gráficos para el seguimiento de los resultados en las rutinas de entrenamiento.   
    
\end{thesischapter}

% DATA MANIPULATION
\begin{thesischapter}{2} {Diseño e implementación del Juego Serio}
En este capítulo se discuten los detalles de desarrollo de los aspectos citados en el capítulo anterior. Este comienza con una descripción y caracterización general del sistema, donde se  abordan cada uno de los componentes requeridos para su completo funcionamiento. Posteriormente se detalla la ingienría de software requerida en la etapa de conceptualización de la aplicación, se explican de forma detallada los aspectos teóricos y de implementación de la base de datos, el funcionamiento del protocolo de comunicación y por último los escenarios de juegos requeridos en las rutinas de entrenamiento ligero y clínico, y las estadísticas generadas por estos. Como herramienta de desarrollo se utilizó c\#.

% SYSTEM DESCRIPTION AND CHARACTERIZATION TO APPLY
\input{main/chapter2/section1/content.tex}
     
% SERIOUS GAME REQUIREMENTS
\input{main/chapter2/section2/content.tex}    

% USE CASE DEFINITION
\input{main/chapter2/section3/content.tex}

% USE CASE REALIZATION
\input{main/chapter2/section4/content.tex}

% DATABASE DESIGN 
\input{main/chapter2/section5/content.tex}

% DATA MANIPULATION
\input{main/chapter2/section6/content.tex}

% COMMUNICATION 
\input{main/chapter2/section7/content.tex}

% TRAINING SCENARIOS
\input{main/chapter2/section8/content.tex}

% EMG GRAPHIC
\input{main/chapter2/section9/content.tex}

% STATICAL REPORTS GENERATION
\input{main/chapter2/section10/content.tex}

\subthesischapter{Conclusiones del capítulo}
Se presentó una descripción del sistema de adquisición de datos para rehabilitación, sus componentes, características distintivas y su funcionamiento. Se identificaron y definieron los requisitos del juego  serio, tanto funcionales como no funcionales, así como los actores y casos de usos del sistema que establecieron las bases fundamentales para el desarrollo de la aplicación. Se realizó el diseño de la base de datos, abarcando tanto el modelo lógico como el físico, lo que aseguró una estructura robusta y eficiente para el almacenamiento de los datos. La manipulación de los datos se abordó de manera integral, desde la conexión con la base de datos hasta la persistencia de los resultados estadísticos. Se diseñó e implementó la comunicación con el pedal motorizado y la implementación de la interfaz gráfica para la representación de los datos EMG. Se definieron los escenarios de entrenamiento para las modalidades Ligero y Clínico, asegurando una cobertura completa de las necesidad de entrenamiento del usuario. Por último en el ámbito estadístico se desarrolló una serie de gráficos para el seguimiento de los resultados en las rutinas de entrenamiento.   
    
\end{thesischapter}

% COMMUNICATION 
\begin{thesischapter}{2} {Diseño e implementación del Juego Serio}
En este capítulo se discuten los detalles de desarrollo de los aspectos citados en el capítulo anterior. Este comienza con una descripción y caracterización general del sistema, donde se  abordan cada uno de los componentes requeridos para su completo funcionamiento. Posteriormente se detalla la ingienría de software requerida en la etapa de conceptualización de la aplicación, se explican de forma detallada los aspectos teóricos y de implementación de la base de datos, el funcionamiento del protocolo de comunicación y por último los escenarios de juegos requeridos en las rutinas de entrenamiento ligero y clínico, y las estadísticas generadas por estos. Como herramienta de desarrollo se utilizó c\#.

% SYSTEM DESCRIPTION AND CHARACTERIZATION TO APPLY
\input{main/chapter2/section1/content.tex}
     
% SERIOUS GAME REQUIREMENTS
\input{main/chapter2/section2/content.tex}    

% USE CASE DEFINITION
\input{main/chapter2/section3/content.tex}

% USE CASE REALIZATION
\input{main/chapter2/section4/content.tex}

% DATABASE DESIGN 
\input{main/chapter2/section5/content.tex}

% DATA MANIPULATION
\input{main/chapter2/section6/content.tex}

% COMMUNICATION 
\input{main/chapter2/section7/content.tex}

% TRAINING SCENARIOS
\input{main/chapter2/section8/content.tex}

% EMG GRAPHIC
\input{main/chapter2/section9/content.tex}

% STATICAL REPORTS GENERATION
\input{main/chapter2/section10/content.tex}

\subthesischapter{Conclusiones del capítulo}
Se presentó una descripción del sistema de adquisición de datos para rehabilitación, sus componentes, características distintivas y su funcionamiento. Se identificaron y definieron los requisitos del juego  serio, tanto funcionales como no funcionales, así como los actores y casos de usos del sistema que establecieron las bases fundamentales para el desarrollo de la aplicación. Se realizó el diseño de la base de datos, abarcando tanto el modelo lógico como el físico, lo que aseguró una estructura robusta y eficiente para el almacenamiento de los datos. La manipulación de los datos se abordó de manera integral, desde la conexión con la base de datos hasta la persistencia de los resultados estadísticos. Se diseñó e implementó la comunicación con el pedal motorizado y la implementación de la interfaz gráfica para la representación de los datos EMG. Se definieron los escenarios de entrenamiento para las modalidades Ligero y Clínico, asegurando una cobertura completa de las necesidad de entrenamiento del usuario. Por último en el ámbito estadístico se desarrolló una serie de gráficos para el seguimiento de los resultados en las rutinas de entrenamiento.   
    
\end{thesischapter}

% TRAINING SCENARIOS
\begin{thesischapter}{2} {Diseño e implementación del Juego Serio}
En este capítulo se discuten los detalles de desarrollo de los aspectos citados en el capítulo anterior. Este comienza con una descripción y caracterización general del sistema, donde se  abordan cada uno de los componentes requeridos para su completo funcionamiento. Posteriormente se detalla la ingienría de software requerida en la etapa de conceptualización de la aplicación, se explican de forma detallada los aspectos teóricos y de implementación de la base de datos, el funcionamiento del protocolo de comunicación y por último los escenarios de juegos requeridos en las rutinas de entrenamiento ligero y clínico, y las estadísticas generadas por estos. Como herramienta de desarrollo se utilizó c\#.

% SYSTEM DESCRIPTION AND CHARACTERIZATION TO APPLY
\input{main/chapter2/section1/content.tex}
     
% SERIOUS GAME REQUIREMENTS
\input{main/chapter2/section2/content.tex}    

% USE CASE DEFINITION
\input{main/chapter2/section3/content.tex}

% USE CASE REALIZATION
\input{main/chapter2/section4/content.tex}

% DATABASE DESIGN 
\input{main/chapter2/section5/content.tex}

% DATA MANIPULATION
\input{main/chapter2/section6/content.tex}

% COMMUNICATION 
\input{main/chapter2/section7/content.tex}

% TRAINING SCENARIOS
\input{main/chapter2/section8/content.tex}

% EMG GRAPHIC
\input{main/chapter2/section9/content.tex}

% STATICAL REPORTS GENERATION
\input{main/chapter2/section10/content.tex}

\subthesischapter{Conclusiones del capítulo}
Se presentó una descripción del sistema de adquisición de datos para rehabilitación, sus componentes, características distintivas y su funcionamiento. Se identificaron y definieron los requisitos del juego  serio, tanto funcionales como no funcionales, así como los actores y casos de usos del sistema que establecieron las bases fundamentales para el desarrollo de la aplicación. Se realizó el diseño de la base de datos, abarcando tanto el modelo lógico como el físico, lo que aseguró una estructura robusta y eficiente para el almacenamiento de los datos. La manipulación de los datos se abordó de manera integral, desde la conexión con la base de datos hasta la persistencia de los resultados estadísticos. Se diseñó e implementó la comunicación con el pedal motorizado y la implementación de la interfaz gráfica para la representación de los datos EMG. Se definieron los escenarios de entrenamiento para las modalidades Ligero y Clínico, asegurando una cobertura completa de las necesidad de entrenamiento del usuario. Por último en el ámbito estadístico se desarrolló una serie de gráficos para el seguimiento de los resultados en las rutinas de entrenamiento.   
    
\end{thesischapter}

% EMG GRAPHIC
\begin{thesischapter}{2} {Diseño e implementación del Juego Serio}
En este capítulo se discuten los detalles de desarrollo de los aspectos citados en el capítulo anterior. Este comienza con una descripción y caracterización general del sistema, donde se  abordan cada uno de los componentes requeridos para su completo funcionamiento. Posteriormente se detalla la ingienría de software requerida en la etapa de conceptualización de la aplicación, se explican de forma detallada los aspectos teóricos y de implementación de la base de datos, el funcionamiento del protocolo de comunicación y por último los escenarios de juegos requeridos en las rutinas de entrenamiento ligero y clínico, y las estadísticas generadas por estos. Como herramienta de desarrollo se utilizó c\#.

% SYSTEM DESCRIPTION AND CHARACTERIZATION TO APPLY
\input{main/chapter2/section1/content.tex}
     
% SERIOUS GAME REQUIREMENTS
\input{main/chapter2/section2/content.tex}    

% USE CASE DEFINITION
\input{main/chapter2/section3/content.tex}

% USE CASE REALIZATION
\input{main/chapter2/section4/content.tex}

% DATABASE DESIGN 
\input{main/chapter2/section5/content.tex}

% DATA MANIPULATION
\input{main/chapter2/section6/content.tex}

% COMMUNICATION 
\input{main/chapter2/section7/content.tex}

% TRAINING SCENARIOS
\input{main/chapter2/section8/content.tex}

% EMG GRAPHIC
\input{main/chapter2/section9/content.tex}

% STATICAL REPORTS GENERATION
\input{main/chapter2/section10/content.tex}

\subthesischapter{Conclusiones del capítulo}
Se presentó una descripción del sistema de adquisición de datos para rehabilitación, sus componentes, características distintivas y su funcionamiento. Se identificaron y definieron los requisitos del juego  serio, tanto funcionales como no funcionales, así como los actores y casos de usos del sistema que establecieron las bases fundamentales para el desarrollo de la aplicación. Se realizó el diseño de la base de datos, abarcando tanto el modelo lógico como el físico, lo que aseguró una estructura robusta y eficiente para el almacenamiento de los datos. La manipulación de los datos se abordó de manera integral, desde la conexión con la base de datos hasta la persistencia de los resultados estadísticos. Se diseñó e implementó la comunicación con el pedal motorizado y la implementación de la interfaz gráfica para la representación de los datos EMG. Se definieron los escenarios de entrenamiento para las modalidades Ligero y Clínico, asegurando una cobertura completa de las necesidad de entrenamiento del usuario. Por último en el ámbito estadístico se desarrolló una serie de gráficos para el seguimiento de los resultados en las rutinas de entrenamiento.   
    
\end{thesischapter}

% STATICAL REPORTS GENERATION
\begin{thesischapter}{2} {Diseño e implementación del Juego Serio}
En este capítulo se discuten los detalles de desarrollo de los aspectos citados en el capítulo anterior. Este comienza con una descripción y caracterización general del sistema, donde se  abordan cada uno de los componentes requeridos para su completo funcionamiento. Posteriormente se detalla la ingienría de software requerida en la etapa de conceptualización de la aplicación, se explican de forma detallada los aspectos teóricos y de implementación de la base de datos, el funcionamiento del protocolo de comunicación y por último los escenarios de juegos requeridos en las rutinas de entrenamiento ligero y clínico, y las estadísticas generadas por estos. Como herramienta de desarrollo se utilizó c\#.

% SYSTEM DESCRIPTION AND CHARACTERIZATION TO APPLY
\input{main/chapter2/section1/content.tex}
     
% SERIOUS GAME REQUIREMENTS
\input{main/chapter2/section2/content.tex}    

% USE CASE DEFINITION
\input{main/chapter2/section3/content.tex}

% USE CASE REALIZATION
\input{main/chapter2/section4/content.tex}

% DATABASE DESIGN 
\input{main/chapter2/section5/content.tex}

% DATA MANIPULATION
\input{main/chapter2/section6/content.tex}

% COMMUNICATION 
\input{main/chapter2/section7/content.tex}

% TRAINING SCENARIOS
\input{main/chapter2/section8/content.tex}

% EMG GRAPHIC
\input{main/chapter2/section9/content.tex}

% STATICAL REPORTS GENERATION
\input{main/chapter2/section10/content.tex}

\subthesischapter{Conclusiones del capítulo}
Se presentó una descripción del sistema de adquisición de datos para rehabilitación, sus componentes, características distintivas y su funcionamiento. Se identificaron y definieron los requisitos del juego  serio, tanto funcionales como no funcionales, así como los actores y casos de usos del sistema que establecieron las bases fundamentales para el desarrollo de la aplicación. Se realizó el diseño de la base de datos, abarcando tanto el modelo lógico como el físico, lo que aseguró una estructura robusta y eficiente para el almacenamiento de los datos. La manipulación de los datos se abordó de manera integral, desde la conexión con la base de datos hasta la persistencia de los resultados estadísticos. Se diseñó e implementó la comunicación con el pedal motorizado y la implementación de la interfaz gráfica para la representación de los datos EMG. Se definieron los escenarios de entrenamiento para las modalidades Ligero y Clínico, asegurando una cobertura completa de las necesidad de entrenamiento del usuario. Por último en el ámbito estadístico se desarrolló una serie de gráficos para el seguimiento de los resultados en las rutinas de entrenamiento.   
    
\end{thesischapter}

\subthesischapter{Conclusiones del capítulo}
Se presentó una descripción del sistema de adquisición de datos para rehabilitación, sus componentes, características distintivas y su funcionamiento. Se identificaron y definieron los requisitos del juego  serio, tanto funcionales como no funcionales, así como los actores y casos de usos del sistema que establecieron las bases fundamentales para el desarrollo de la aplicación. Se realizó el diseño de la base de datos, abarcando tanto el modelo lógico como el físico, lo que aseguró una estructura robusta y eficiente para el almacenamiento de los datos. La manipulación de los datos se abordó de manera integral, desde la conexión con la base de datos hasta la persistencia de los resultados estadísticos. Se diseñó e implementó la comunicación con el pedal motorizado y la implementación de la interfaz gráfica para la representación de los datos EMG. Se definieron los escenarios de entrenamiento para las modalidades Ligero y Clínico, asegurando una cobertura completa de las necesidad de entrenamiento del usuario. Por último en el ámbito estadístico se desarrolló una serie de gráficos para el seguimiento de los resultados en las rutinas de entrenamiento.   
    
\end{thesischapter}    

% USE CASE DEFINITION
\begin{thesischapter}{2} {Diseño e implementación del Juego Serio}
En este capítulo se discuten los detalles de desarrollo de los aspectos citados en el capítulo anterior. Este comienza con una descripción y caracterización general del sistema, donde se  abordan cada uno de los componentes requeridos para su completo funcionamiento. Posteriormente se detalla la ingienría de software requerida en la etapa de conceptualización de la aplicación, se explican de forma detallada los aspectos teóricos y de implementación de la base de datos, el funcionamiento del protocolo de comunicación y por último los escenarios de juegos requeridos en las rutinas de entrenamiento ligero y clínico, y las estadísticas generadas por estos. Como herramienta de desarrollo se utilizó c\#.

% SYSTEM DESCRIPTION AND CHARACTERIZATION TO APPLY
\begin{thesischapter}{2} {Diseño e implementación del Juego Serio}
En este capítulo se discuten los detalles de desarrollo de los aspectos citados en el capítulo anterior. Este comienza con una descripción y caracterización general del sistema, donde se  abordan cada uno de los componentes requeridos para su completo funcionamiento. Posteriormente se detalla la ingienría de software requerida en la etapa de conceptualización de la aplicación, se explican de forma detallada los aspectos teóricos y de implementación de la base de datos, el funcionamiento del protocolo de comunicación y por último los escenarios de juegos requeridos en las rutinas de entrenamiento ligero y clínico, y las estadísticas generadas por estos. Como herramienta de desarrollo se utilizó c\#.

% SYSTEM DESCRIPTION AND CHARACTERIZATION TO APPLY
\input{main/chapter2/section1/content.tex}
     
% SERIOUS GAME REQUIREMENTS
\input{main/chapter2/section2/content.tex}    

% USE CASE DEFINITION
\input{main/chapter2/section3/content.tex}

% USE CASE REALIZATION
\input{main/chapter2/section4/content.tex}

% DATABASE DESIGN 
\input{main/chapter2/section5/content.tex}

% DATA MANIPULATION
\input{main/chapter2/section6/content.tex}

% COMMUNICATION 
\input{main/chapter2/section7/content.tex}

% TRAINING SCENARIOS
\input{main/chapter2/section8/content.tex}

% EMG GRAPHIC
\input{main/chapter2/section9/content.tex}

% STATICAL REPORTS GENERATION
\input{main/chapter2/section10/content.tex}

\subthesischapter{Conclusiones del capítulo}
Se presentó una descripción del sistema de adquisición de datos para rehabilitación, sus componentes, características distintivas y su funcionamiento. Se identificaron y definieron los requisitos del juego  serio, tanto funcionales como no funcionales, así como los actores y casos de usos del sistema que establecieron las bases fundamentales para el desarrollo de la aplicación. Se realizó el diseño de la base de datos, abarcando tanto el modelo lógico como el físico, lo que aseguró una estructura robusta y eficiente para el almacenamiento de los datos. La manipulación de los datos se abordó de manera integral, desde la conexión con la base de datos hasta la persistencia de los resultados estadísticos. Se diseñó e implementó la comunicación con el pedal motorizado y la implementación de la interfaz gráfica para la representación de los datos EMG. Se definieron los escenarios de entrenamiento para las modalidades Ligero y Clínico, asegurando una cobertura completa de las necesidad de entrenamiento del usuario. Por último en el ámbito estadístico se desarrolló una serie de gráficos para el seguimiento de los resultados en las rutinas de entrenamiento.   
    
\end{thesischapter}
     
% SERIOUS GAME REQUIREMENTS
\begin{thesischapter}{2} {Diseño e implementación del Juego Serio}
En este capítulo se discuten los detalles de desarrollo de los aspectos citados en el capítulo anterior. Este comienza con una descripción y caracterización general del sistema, donde se  abordan cada uno de los componentes requeridos para su completo funcionamiento. Posteriormente se detalla la ingienría de software requerida en la etapa de conceptualización de la aplicación, se explican de forma detallada los aspectos teóricos y de implementación de la base de datos, el funcionamiento del protocolo de comunicación y por último los escenarios de juegos requeridos en las rutinas de entrenamiento ligero y clínico, y las estadísticas generadas por estos. Como herramienta de desarrollo se utilizó c\#.

% SYSTEM DESCRIPTION AND CHARACTERIZATION TO APPLY
\input{main/chapter2/section1/content.tex}
     
% SERIOUS GAME REQUIREMENTS
\input{main/chapter2/section2/content.tex}    

% USE CASE DEFINITION
\input{main/chapter2/section3/content.tex}

% USE CASE REALIZATION
\input{main/chapter2/section4/content.tex}

% DATABASE DESIGN 
\input{main/chapter2/section5/content.tex}

% DATA MANIPULATION
\input{main/chapter2/section6/content.tex}

% COMMUNICATION 
\input{main/chapter2/section7/content.tex}

% TRAINING SCENARIOS
\input{main/chapter2/section8/content.tex}

% EMG GRAPHIC
\input{main/chapter2/section9/content.tex}

% STATICAL REPORTS GENERATION
\input{main/chapter2/section10/content.tex}

\subthesischapter{Conclusiones del capítulo}
Se presentó una descripción del sistema de adquisición de datos para rehabilitación, sus componentes, características distintivas y su funcionamiento. Se identificaron y definieron los requisitos del juego  serio, tanto funcionales como no funcionales, así como los actores y casos de usos del sistema que establecieron las bases fundamentales para el desarrollo de la aplicación. Se realizó el diseño de la base de datos, abarcando tanto el modelo lógico como el físico, lo que aseguró una estructura robusta y eficiente para el almacenamiento de los datos. La manipulación de los datos se abordó de manera integral, desde la conexión con la base de datos hasta la persistencia de los resultados estadísticos. Se diseñó e implementó la comunicación con el pedal motorizado y la implementación de la interfaz gráfica para la representación de los datos EMG. Se definieron los escenarios de entrenamiento para las modalidades Ligero y Clínico, asegurando una cobertura completa de las necesidad de entrenamiento del usuario. Por último en el ámbito estadístico se desarrolló una serie de gráficos para el seguimiento de los resultados en las rutinas de entrenamiento.   
    
\end{thesischapter}    

% USE CASE DEFINITION
\begin{thesischapter}{2} {Diseño e implementación del Juego Serio}
En este capítulo se discuten los detalles de desarrollo de los aspectos citados en el capítulo anterior. Este comienza con una descripción y caracterización general del sistema, donde se  abordan cada uno de los componentes requeridos para su completo funcionamiento. Posteriormente se detalla la ingienría de software requerida en la etapa de conceptualización de la aplicación, se explican de forma detallada los aspectos teóricos y de implementación de la base de datos, el funcionamiento del protocolo de comunicación y por último los escenarios de juegos requeridos en las rutinas de entrenamiento ligero y clínico, y las estadísticas generadas por estos. Como herramienta de desarrollo se utilizó c\#.

% SYSTEM DESCRIPTION AND CHARACTERIZATION TO APPLY
\input{main/chapter2/section1/content.tex}
     
% SERIOUS GAME REQUIREMENTS
\input{main/chapter2/section2/content.tex}    

% USE CASE DEFINITION
\input{main/chapter2/section3/content.tex}

% USE CASE REALIZATION
\input{main/chapter2/section4/content.tex}

% DATABASE DESIGN 
\input{main/chapter2/section5/content.tex}

% DATA MANIPULATION
\input{main/chapter2/section6/content.tex}

% COMMUNICATION 
\input{main/chapter2/section7/content.tex}

% TRAINING SCENARIOS
\input{main/chapter2/section8/content.tex}

% EMG GRAPHIC
\input{main/chapter2/section9/content.tex}

% STATICAL REPORTS GENERATION
\input{main/chapter2/section10/content.tex}

\subthesischapter{Conclusiones del capítulo}
Se presentó una descripción del sistema de adquisición de datos para rehabilitación, sus componentes, características distintivas y su funcionamiento. Se identificaron y definieron los requisitos del juego  serio, tanto funcionales como no funcionales, así como los actores y casos de usos del sistema que establecieron las bases fundamentales para el desarrollo de la aplicación. Se realizó el diseño de la base de datos, abarcando tanto el modelo lógico como el físico, lo que aseguró una estructura robusta y eficiente para el almacenamiento de los datos. La manipulación de los datos se abordó de manera integral, desde la conexión con la base de datos hasta la persistencia de los resultados estadísticos. Se diseñó e implementó la comunicación con el pedal motorizado y la implementación de la interfaz gráfica para la representación de los datos EMG. Se definieron los escenarios de entrenamiento para las modalidades Ligero y Clínico, asegurando una cobertura completa de las necesidad de entrenamiento del usuario. Por último en el ámbito estadístico se desarrolló una serie de gráficos para el seguimiento de los resultados en las rutinas de entrenamiento.   
    
\end{thesischapter}

% USE CASE REALIZATION
\begin{thesischapter}{2} {Diseño e implementación del Juego Serio}
En este capítulo se discuten los detalles de desarrollo de los aspectos citados en el capítulo anterior. Este comienza con una descripción y caracterización general del sistema, donde se  abordan cada uno de los componentes requeridos para su completo funcionamiento. Posteriormente se detalla la ingienría de software requerida en la etapa de conceptualización de la aplicación, se explican de forma detallada los aspectos teóricos y de implementación de la base de datos, el funcionamiento del protocolo de comunicación y por último los escenarios de juegos requeridos en las rutinas de entrenamiento ligero y clínico, y las estadísticas generadas por estos. Como herramienta de desarrollo se utilizó c\#.

% SYSTEM DESCRIPTION AND CHARACTERIZATION TO APPLY
\input{main/chapter2/section1/content.tex}
     
% SERIOUS GAME REQUIREMENTS
\input{main/chapter2/section2/content.tex}    

% USE CASE DEFINITION
\input{main/chapter2/section3/content.tex}

% USE CASE REALIZATION
\input{main/chapter2/section4/content.tex}

% DATABASE DESIGN 
\input{main/chapter2/section5/content.tex}

% DATA MANIPULATION
\input{main/chapter2/section6/content.tex}

% COMMUNICATION 
\input{main/chapter2/section7/content.tex}

% TRAINING SCENARIOS
\input{main/chapter2/section8/content.tex}

% EMG GRAPHIC
\input{main/chapter2/section9/content.tex}

% STATICAL REPORTS GENERATION
\input{main/chapter2/section10/content.tex}

\subthesischapter{Conclusiones del capítulo}
Se presentó una descripción del sistema de adquisición de datos para rehabilitación, sus componentes, características distintivas y su funcionamiento. Se identificaron y definieron los requisitos del juego  serio, tanto funcionales como no funcionales, así como los actores y casos de usos del sistema que establecieron las bases fundamentales para el desarrollo de la aplicación. Se realizó el diseño de la base de datos, abarcando tanto el modelo lógico como el físico, lo que aseguró una estructura robusta y eficiente para el almacenamiento de los datos. La manipulación de los datos se abordó de manera integral, desde la conexión con la base de datos hasta la persistencia de los resultados estadísticos. Se diseñó e implementó la comunicación con el pedal motorizado y la implementación de la interfaz gráfica para la representación de los datos EMG. Se definieron los escenarios de entrenamiento para las modalidades Ligero y Clínico, asegurando una cobertura completa de las necesidad de entrenamiento del usuario. Por último en el ámbito estadístico se desarrolló una serie de gráficos para el seguimiento de los resultados en las rutinas de entrenamiento.   
    
\end{thesischapter}

% DATABASE DESIGN 
\begin{thesischapter}{2} {Diseño e implementación del Juego Serio}
En este capítulo se discuten los detalles de desarrollo de los aspectos citados en el capítulo anterior. Este comienza con una descripción y caracterización general del sistema, donde se  abordan cada uno de los componentes requeridos para su completo funcionamiento. Posteriormente se detalla la ingienría de software requerida en la etapa de conceptualización de la aplicación, se explican de forma detallada los aspectos teóricos y de implementación de la base de datos, el funcionamiento del protocolo de comunicación y por último los escenarios de juegos requeridos en las rutinas de entrenamiento ligero y clínico, y las estadísticas generadas por estos. Como herramienta de desarrollo se utilizó c\#.

% SYSTEM DESCRIPTION AND CHARACTERIZATION TO APPLY
\input{main/chapter2/section1/content.tex}
     
% SERIOUS GAME REQUIREMENTS
\input{main/chapter2/section2/content.tex}    

% USE CASE DEFINITION
\input{main/chapter2/section3/content.tex}

% USE CASE REALIZATION
\input{main/chapter2/section4/content.tex}

% DATABASE DESIGN 
\input{main/chapter2/section5/content.tex}

% DATA MANIPULATION
\input{main/chapter2/section6/content.tex}

% COMMUNICATION 
\input{main/chapter2/section7/content.tex}

% TRAINING SCENARIOS
\input{main/chapter2/section8/content.tex}

% EMG GRAPHIC
\input{main/chapter2/section9/content.tex}

% STATICAL REPORTS GENERATION
\input{main/chapter2/section10/content.tex}

\subthesischapter{Conclusiones del capítulo}
Se presentó una descripción del sistema de adquisición de datos para rehabilitación, sus componentes, características distintivas y su funcionamiento. Se identificaron y definieron los requisitos del juego  serio, tanto funcionales como no funcionales, así como los actores y casos de usos del sistema que establecieron las bases fundamentales para el desarrollo de la aplicación. Se realizó el diseño de la base de datos, abarcando tanto el modelo lógico como el físico, lo que aseguró una estructura robusta y eficiente para el almacenamiento de los datos. La manipulación de los datos se abordó de manera integral, desde la conexión con la base de datos hasta la persistencia de los resultados estadísticos. Se diseñó e implementó la comunicación con el pedal motorizado y la implementación de la interfaz gráfica para la representación de los datos EMG. Se definieron los escenarios de entrenamiento para las modalidades Ligero y Clínico, asegurando una cobertura completa de las necesidad de entrenamiento del usuario. Por último en el ámbito estadístico se desarrolló una serie de gráficos para el seguimiento de los resultados en las rutinas de entrenamiento.   
    
\end{thesischapter}

% DATA MANIPULATION
\begin{thesischapter}{2} {Diseño e implementación del Juego Serio}
En este capítulo se discuten los detalles de desarrollo de los aspectos citados en el capítulo anterior. Este comienza con una descripción y caracterización general del sistema, donde se  abordan cada uno de los componentes requeridos para su completo funcionamiento. Posteriormente se detalla la ingienría de software requerida en la etapa de conceptualización de la aplicación, se explican de forma detallada los aspectos teóricos y de implementación de la base de datos, el funcionamiento del protocolo de comunicación y por último los escenarios de juegos requeridos en las rutinas de entrenamiento ligero y clínico, y las estadísticas generadas por estos. Como herramienta de desarrollo se utilizó c\#.

% SYSTEM DESCRIPTION AND CHARACTERIZATION TO APPLY
\input{main/chapter2/section1/content.tex}
     
% SERIOUS GAME REQUIREMENTS
\input{main/chapter2/section2/content.tex}    

% USE CASE DEFINITION
\input{main/chapter2/section3/content.tex}

% USE CASE REALIZATION
\input{main/chapter2/section4/content.tex}

% DATABASE DESIGN 
\input{main/chapter2/section5/content.tex}

% DATA MANIPULATION
\input{main/chapter2/section6/content.tex}

% COMMUNICATION 
\input{main/chapter2/section7/content.tex}

% TRAINING SCENARIOS
\input{main/chapter2/section8/content.tex}

% EMG GRAPHIC
\input{main/chapter2/section9/content.tex}

% STATICAL REPORTS GENERATION
\input{main/chapter2/section10/content.tex}

\subthesischapter{Conclusiones del capítulo}
Se presentó una descripción del sistema de adquisición de datos para rehabilitación, sus componentes, características distintivas y su funcionamiento. Se identificaron y definieron los requisitos del juego  serio, tanto funcionales como no funcionales, así como los actores y casos de usos del sistema que establecieron las bases fundamentales para el desarrollo de la aplicación. Se realizó el diseño de la base de datos, abarcando tanto el modelo lógico como el físico, lo que aseguró una estructura robusta y eficiente para el almacenamiento de los datos. La manipulación de los datos se abordó de manera integral, desde la conexión con la base de datos hasta la persistencia de los resultados estadísticos. Se diseñó e implementó la comunicación con el pedal motorizado y la implementación de la interfaz gráfica para la representación de los datos EMG. Se definieron los escenarios de entrenamiento para las modalidades Ligero y Clínico, asegurando una cobertura completa de las necesidad de entrenamiento del usuario. Por último en el ámbito estadístico se desarrolló una serie de gráficos para el seguimiento de los resultados en las rutinas de entrenamiento.   
    
\end{thesischapter}

% COMMUNICATION 
\begin{thesischapter}{2} {Diseño e implementación del Juego Serio}
En este capítulo se discuten los detalles de desarrollo de los aspectos citados en el capítulo anterior. Este comienza con una descripción y caracterización general del sistema, donde se  abordan cada uno de los componentes requeridos para su completo funcionamiento. Posteriormente se detalla la ingienría de software requerida en la etapa de conceptualización de la aplicación, se explican de forma detallada los aspectos teóricos y de implementación de la base de datos, el funcionamiento del protocolo de comunicación y por último los escenarios de juegos requeridos en las rutinas de entrenamiento ligero y clínico, y las estadísticas generadas por estos. Como herramienta de desarrollo se utilizó c\#.

% SYSTEM DESCRIPTION AND CHARACTERIZATION TO APPLY
\input{main/chapter2/section1/content.tex}
     
% SERIOUS GAME REQUIREMENTS
\input{main/chapter2/section2/content.tex}    

% USE CASE DEFINITION
\input{main/chapter2/section3/content.tex}

% USE CASE REALIZATION
\input{main/chapter2/section4/content.tex}

% DATABASE DESIGN 
\input{main/chapter2/section5/content.tex}

% DATA MANIPULATION
\input{main/chapter2/section6/content.tex}

% COMMUNICATION 
\input{main/chapter2/section7/content.tex}

% TRAINING SCENARIOS
\input{main/chapter2/section8/content.tex}

% EMG GRAPHIC
\input{main/chapter2/section9/content.tex}

% STATICAL REPORTS GENERATION
\input{main/chapter2/section10/content.tex}

\subthesischapter{Conclusiones del capítulo}
Se presentó una descripción del sistema de adquisición de datos para rehabilitación, sus componentes, características distintivas y su funcionamiento. Se identificaron y definieron los requisitos del juego  serio, tanto funcionales como no funcionales, así como los actores y casos de usos del sistema que establecieron las bases fundamentales para el desarrollo de la aplicación. Se realizó el diseño de la base de datos, abarcando tanto el modelo lógico como el físico, lo que aseguró una estructura robusta y eficiente para el almacenamiento de los datos. La manipulación de los datos se abordó de manera integral, desde la conexión con la base de datos hasta la persistencia de los resultados estadísticos. Se diseñó e implementó la comunicación con el pedal motorizado y la implementación de la interfaz gráfica para la representación de los datos EMG. Se definieron los escenarios de entrenamiento para las modalidades Ligero y Clínico, asegurando una cobertura completa de las necesidad de entrenamiento del usuario. Por último en el ámbito estadístico se desarrolló una serie de gráficos para el seguimiento de los resultados en las rutinas de entrenamiento.   
    
\end{thesischapter}

% TRAINING SCENARIOS
\begin{thesischapter}{2} {Diseño e implementación del Juego Serio}
En este capítulo se discuten los detalles de desarrollo de los aspectos citados en el capítulo anterior. Este comienza con una descripción y caracterización general del sistema, donde se  abordan cada uno de los componentes requeridos para su completo funcionamiento. Posteriormente se detalla la ingienría de software requerida en la etapa de conceptualización de la aplicación, se explican de forma detallada los aspectos teóricos y de implementación de la base de datos, el funcionamiento del protocolo de comunicación y por último los escenarios de juegos requeridos en las rutinas de entrenamiento ligero y clínico, y las estadísticas generadas por estos. Como herramienta de desarrollo se utilizó c\#.

% SYSTEM DESCRIPTION AND CHARACTERIZATION TO APPLY
\input{main/chapter2/section1/content.tex}
     
% SERIOUS GAME REQUIREMENTS
\input{main/chapter2/section2/content.tex}    

% USE CASE DEFINITION
\input{main/chapter2/section3/content.tex}

% USE CASE REALIZATION
\input{main/chapter2/section4/content.tex}

% DATABASE DESIGN 
\input{main/chapter2/section5/content.tex}

% DATA MANIPULATION
\input{main/chapter2/section6/content.tex}

% COMMUNICATION 
\input{main/chapter2/section7/content.tex}

% TRAINING SCENARIOS
\input{main/chapter2/section8/content.tex}

% EMG GRAPHIC
\input{main/chapter2/section9/content.tex}

% STATICAL REPORTS GENERATION
\input{main/chapter2/section10/content.tex}

\subthesischapter{Conclusiones del capítulo}
Se presentó una descripción del sistema de adquisición de datos para rehabilitación, sus componentes, características distintivas y su funcionamiento. Se identificaron y definieron los requisitos del juego  serio, tanto funcionales como no funcionales, así como los actores y casos de usos del sistema que establecieron las bases fundamentales para el desarrollo de la aplicación. Se realizó el diseño de la base de datos, abarcando tanto el modelo lógico como el físico, lo que aseguró una estructura robusta y eficiente para el almacenamiento de los datos. La manipulación de los datos se abordó de manera integral, desde la conexión con la base de datos hasta la persistencia de los resultados estadísticos. Se diseñó e implementó la comunicación con el pedal motorizado y la implementación de la interfaz gráfica para la representación de los datos EMG. Se definieron los escenarios de entrenamiento para las modalidades Ligero y Clínico, asegurando una cobertura completa de las necesidad de entrenamiento del usuario. Por último en el ámbito estadístico se desarrolló una serie de gráficos para el seguimiento de los resultados en las rutinas de entrenamiento.   
    
\end{thesischapter}

% EMG GRAPHIC
\begin{thesischapter}{2} {Diseño e implementación del Juego Serio}
En este capítulo se discuten los detalles de desarrollo de los aspectos citados en el capítulo anterior. Este comienza con una descripción y caracterización general del sistema, donde se  abordan cada uno de los componentes requeridos para su completo funcionamiento. Posteriormente se detalla la ingienría de software requerida en la etapa de conceptualización de la aplicación, se explican de forma detallada los aspectos teóricos y de implementación de la base de datos, el funcionamiento del protocolo de comunicación y por último los escenarios de juegos requeridos en las rutinas de entrenamiento ligero y clínico, y las estadísticas generadas por estos. Como herramienta de desarrollo se utilizó c\#.

% SYSTEM DESCRIPTION AND CHARACTERIZATION TO APPLY
\input{main/chapter2/section1/content.tex}
     
% SERIOUS GAME REQUIREMENTS
\input{main/chapter2/section2/content.tex}    

% USE CASE DEFINITION
\input{main/chapter2/section3/content.tex}

% USE CASE REALIZATION
\input{main/chapter2/section4/content.tex}

% DATABASE DESIGN 
\input{main/chapter2/section5/content.tex}

% DATA MANIPULATION
\input{main/chapter2/section6/content.tex}

% COMMUNICATION 
\input{main/chapter2/section7/content.tex}

% TRAINING SCENARIOS
\input{main/chapter2/section8/content.tex}

% EMG GRAPHIC
\input{main/chapter2/section9/content.tex}

% STATICAL REPORTS GENERATION
\input{main/chapter2/section10/content.tex}

\subthesischapter{Conclusiones del capítulo}
Se presentó una descripción del sistema de adquisición de datos para rehabilitación, sus componentes, características distintivas y su funcionamiento. Se identificaron y definieron los requisitos del juego  serio, tanto funcionales como no funcionales, así como los actores y casos de usos del sistema que establecieron las bases fundamentales para el desarrollo de la aplicación. Se realizó el diseño de la base de datos, abarcando tanto el modelo lógico como el físico, lo que aseguró una estructura robusta y eficiente para el almacenamiento de los datos. La manipulación de los datos se abordó de manera integral, desde la conexión con la base de datos hasta la persistencia de los resultados estadísticos. Se diseñó e implementó la comunicación con el pedal motorizado y la implementación de la interfaz gráfica para la representación de los datos EMG. Se definieron los escenarios de entrenamiento para las modalidades Ligero y Clínico, asegurando una cobertura completa de las necesidad de entrenamiento del usuario. Por último en el ámbito estadístico se desarrolló una serie de gráficos para el seguimiento de los resultados en las rutinas de entrenamiento.   
    
\end{thesischapter}

% STATICAL REPORTS GENERATION
\begin{thesischapter}{2} {Diseño e implementación del Juego Serio}
En este capítulo se discuten los detalles de desarrollo de los aspectos citados en el capítulo anterior. Este comienza con una descripción y caracterización general del sistema, donde se  abordan cada uno de los componentes requeridos para su completo funcionamiento. Posteriormente se detalla la ingienría de software requerida en la etapa de conceptualización de la aplicación, se explican de forma detallada los aspectos teóricos y de implementación de la base de datos, el funcionamiento del protocolo de comunicación y por último los escenarios de juegos requeridos en las rutinas de entrenamiento ligero y clínico, y las estadísticas generadas por estos. Como herramienta de desarrollo se utilizó c\#.

% SYSTEM DESCRIPTION AND CHARACTERIZATION TO APPLY
\input{main/chapter2/section1/content.tex}
     
% SERIOUS GAME REQUIREMENTS
\input{main/chapter2/section2/content.tex}    

% USE CASE DEFINITION
\input{main/chapter2/section3/content.tex}

% USE CASE REALIZATION
\input{main/chapter2/section4/content.tex}

% DATABASE DESIGN 
\input{main/chapter2/section5/content.tex}

% DATA MANIPULATION
\input{main/chapter2/section6/content.tex}

% COMMUNICATION 
\input{main/chapter2/section7/content.tex}

% TRAINING SCENARIOS
\input{main/chapter2/section8/content.tex}

% EMG GRAPHIC
\input{main/chapter2/section9/content.tex}

% STATICAL REPORTS GENERATION
\input{main/chapter2/section10/content.tex}

\subthesischapter{Conclusiones del capítulo}
Se presentó una descripción del sistema de adquisición de datos para rehabilitación, sus componentes, características distintivas y su funcionamiento. Se identificaron y definieron los requisitos del juego  serio, tanto funcionales como no funcionales, así como los actores y casos de usos del sistema que establecieron las bases fundamentales para el desarrollo de la aplicación. Se realizó el diseño de la base de datos, abarcando tanto el modelo lógico como el físico, lo que aseguró una estructura robusta y eficiente para el almacenamiento de los datos. La manipulación de los datos se abordó de manera integral, desde la conexión con la base de datos hasta la persistencia de los resultados estadísticos. Se diseñó e implementó la comunicación con el pedal motorizado y la implementación de la interfaz gráfica para la representación de los datos EMG. Se definieron los escenarios de entrenamiento para las modalidades Ligero y Clínico, asegurando una cobertura completa de las necesidad de entrenamiento del usuario. Por último en el ámbito estadístico se desarrolló una serie de gráficos para el seguimiento de los resultados en las rutinas de entrenamiento.   
    
\end{thesischapter}

\subthesischapter{Conclusiones del capítulo}
Se presentó una descripción del sistema de adquisición de datos para rehabilitación, sus componentes, características distintivas y su funcionamiento. Se identificaron y definieron los requisitos del juego  serio, tanto funcionales como no funcionales, así como los actores y casos de usos del sistema que establecieron las bases fundamentales para el desarrollo de la aplicación. Se realizó el diseño de la base de datos, abarcando tanto el modelo lógico como el físico, lo que aseguró una estructura robusta y eficiente para el almacenamiento de los datos. La manipulación de los datos se abordó de manera integral, desde la conexión con la base de datos hasta la persistencia de los resultados estadísticos. Se diseñó e implementó la comunicación con el pedal motorizado y la implementación de la interfaz gráfica para la representación de los datos EMG. Se definieron los escenarios de entrenamiento para las modalidades Ligero y Clínico, asegurando una cobertura completa de las necesidad de entrenamiento del usuario. Por último en el ámbito estadístico se desarrolló una serie de gráficos para el seguimiento de los resultados en las rutinas de entrenamiento.   
    
\end{thesischapter}

% USE CASE REALIZATION
\begin{thesischapter}{2} {Diseño e implementación del Juego Serio}
En este capítulo se discuten los detalles de desarrollo de los aspectos citados en el capítulo anterior. Este comienza con una descripción y caracterización general del sistema, donde se  abordan cada uno de los componentes requeridos para su completo funcionamiento. Posteriormente se detalla la ingienría de software requerida en la etapa de conceptualización de la aplicación, se explican de forma detallada los aspectos teóricos y de implementación de la base de datos, el funcionamiento del protocolo de comunicación y por último los escenarios de juegos requeridos en las rutinas de entrenamiento ligero y clínico, y las estadísticas generadas por estos. Como herramienta de desarrollo se utilizó c\#.

% SYSTEM DESCRIPTION AND CHARACTERIZATION TO APPLY
\begin{thesischapter}{2} {Diseño e implementación del Juego Serio}
En este capítulo se discuten los detalles de desarrollo de los aspectos citados en el capítulo anterior. Este comienza con una descripción y caracterización general del sistema, donde se  abordan cada uno de los componentes requeridos para su completo funcionamiento. Posteriormente se detalla la ingienría de software requerida en la etapa de conceptualización de la aplicación, se explican de forma detallada los aspectos teóricos y de implementación de la base de datos, el funcionamiento del protocolo de comunicación y por último los escenarios de juegos requeridos en las rutinas de entrenamiento ligero y clínico, y las estadísticas generadas por estos. Como herramienta de desarrollo se utilizó c\#.

% SYSTEM DESCRIPTION AND CHARACTERIZATION TO APPLY
\input{main/chapter2/section1/content.tex}
     
% SERIOUS GAME REQUIREMENTS
\input{main/chapter2/section2/content.tex}    

% USE CASE DEFINITION
\input{main/chapter2/section3/content.tex}

% USE CASE REALIZATION
\input{main/chapter2/section4/content.tex}

% DATABASE DESIGN 
\input{main/chapter2/section5/content.tex}

% DATA MANIPULATION
\input{main/chapter2/section6/content.tex}

% COMMUNICATION 
\input{main/chapter2/section7/content.tex}

% TRAINING SCENARIOS
\input{main/chapter2/section8/content.tex}

% EMG GRAPHIC
\input{main/chapter2/section9/content.tex}

% STATICAL REPORTS GENERATION
\input{main/chapter2/section10/content.tex}

\subthesischapter{Conclusiones del capítulo}
Se presentó una descripción del sistema de adquisición de datos para rehabilitación, sus componentes, características distintivas y su funcionamiento. Se identificaron y definieron los requisitos del juego  serio, tanto funcionales como no funcionales, así como los actores y casos de usos del sistema que establecieron las bases fundamentales para el desarrollo de la aplicación. Se realizó el diseño de la base de datos, abarcando tanto el modelo lógico como el físico, lo que aseguró una estructura robusta y eficiente para el almacenamiento de los datos. La manipulación de los datos se abordó de manera integral, desde la conexión con la base de datos hasta la persistencia de los resultados estadísticos. Se diseñó e implementó la comunicación con el pedal motorizado y la implementación de la interfaz gráfica para la representación de los datos EMG. Se definieron los escenarios de entrenamiento para las modalidades Ligero y Clínico, asegurando una cobertura completa de las necesidad de entrenamiento del usuario. Por último en el ámbito estadístico se desarrolló una serie de gráficos para el seguimiento de los resultados en las rutinas de entrenamiento.   
    
\end{thesischapter}
     
% SERIOUS GAME REQUIREMENTS
\begin{thesischapter}{2} {Diseño e implementación del Juego Serio}
En este capítulo se discuten los detalles de desarrollo de los aspectos citados en el capítulo anterior. Este comienza con una descripción y caracterización general del sistema, donde se  abordan cada uno de los componentes requeridos para su completo funcionamiento. Posteriormente se detalla la ingienría de software requerida en la etapa de conceptualización de la aplicación, se explican de forma detallada los aspectos teóricos y de implementación de la base de datos, el funcionamiento del protocolo de comunicación y por último los escenarios de juegos requeridos en las rutinas de entrenamiento ligero y clínico, y las estadísticas generadas por estos. Como herramienta de desarrollo se utilizó c\#.

% SYSTEM DESCRIPTION AND CHARACTERIZATION TO APPLY
\input{main/chapter2/section1/content.tex}
     
% SERIOUS GAME REQUIREMENTS
\input{main/chapter2/section2/content.tex}    

% USE CASE DEFINITION
\input{main/chapter2/section3/content.tex}

% USE CASE REALIZATION
\input{main/chapter2/section4/content.tex}

% DATABASE DESIGN 
\input{main/chapter2/section5/content.tex}

% DATA MANIPULATION
\input{main/chapter2/section6/content.tex}

% COMMUNICATION 
\input{main/chapter2/section7/content.tex}

% TRAINING SCENARIOS
\input{main/chapter2/section8/content.tex}

% EMG GRAPHIC
\input{main/chapter2/section9/content.tex}

% STATICAL REPORTS GENERATION
\input{main/chapter2/section10/content.tex}

\subthesischapter{Conclusiones del capítulo}
Se presentó una descripción del sistema de adquisición de datos para rehabilitación, sus componentes, características distintivas y su funcionamiento. Se identificaron y definieron los requisitos del juego  serio, tanto funcionales como no funcionales, así como los actores y casos de usos del sistema que establecieron las bases fundamentales para el desarrollo de la aplicación. Se realizó el diseño de la base de datos, abarcando tanto el modelo lógico como el físico, lo que aseguró una estructura robusta y eficiente para el almacenamiento de los datos. La manipulación de los datos se abordó de manera integral, desde la conexión con la base de datos hasta la persistencia de los resultados estadísticos. Se diseñó e implementó la comunicación con el pedal motorizado y la implementación de la interfaz gráfica para la representación de los datos EMG. Se definieron los escenarios de entrenamiento para las modalidades Ligero y Clínico, asegurando una cobertura completa de las necesidad de entrenamiento del usuario. Por último en el ámbito estadístico se desarrolló una serie de gráficos para el seguimiento de los resultados en las rutinas de entrenamiento.   
    
\end{thesischapter}    

% USE CASE DEFINITION
\begin{thesischapter}{2} {Diseño e implementación del Juego Serio}
En este capítulo se discuten los detalles de desarrollo de los aspectos citados en el capítulo anterior. Este comienza con una descripción y caracterización general del sistema, donde se  abordan cada uno de los componentes requeridos para su completo funcionamiento. Posteriormente se detalla la ingienría de software requerida en la etapa de conceptualización de la aplicación, se explican de forma detallada los aspectos teóricos y de implementación de la base de datos, el funcionamiento del protocolo de comunicación y por último los escenarios de juegos requeridos en las rutinas de entrenamiento ligero y clínico, y las estadísticas generadas por estos. Como herramienta de desarrollo se utilizó c\#.

% SYSTEM DESCRIPTION AND CHARACTERIZATION TO APPLY
\input{main/chapter2/section1/content.tex}
     
% SERIOUS GAME REQUIREMENTS
\input{main/chapter2/section2/content.tex}    

% USE CASE DEFINITION
\input{main/chapter2/section3/content.tex}

% USE CASE REALIZATION
\input{main/chapter2/section4/content.tex}

% DATABASE DESIGN 
\input{main/chapter2/section5/content.tex}

% DATA MANIPULATION
\input{main/chapter2/section6/content.tex}

% COMMUNICATION 
\input{main/chapter2/section7/content.tex}

% TRAINING SCENARIOS
\input{main/chapter2/section8/content.tex}

% EMG GRAPHIC
\input{main/chapter2/section9/content.tex}

% STATICAL REPORTS GENERATION
\input{main/chapter2/section10/content.tex}

\subthesischapter{Conclusiones del capítulo}
Se presentó una descripción del sistema de adquisición de datos para rehabilitación, sus componentes, características distintivas y su funcionamiento. Se identificaron y definieron los requisitos del juego  serio, tanto funcionales como no funcionales, así como los actores y casos de usos del sistema que establecieron las bases fundamentales para el desarrollo de la aplicación. Se realizó el diseño de la base de datos, abarcando tanto el modelo lógico como el físico, lo que aseguró una estructura robusta y eficiente para el almacenamiento de los datos. La manipulación de los datos se abordó de manera integral, desde la conexión con la base de datos hasta la persistencia de los resultados estadísticos. Se diseñó e implementó la comunicación con el pedal motorizado y la implementación de la interfaz gráfica para la representación de los datos EMG. Se definieron los escenarios de entrenamiento para las modalidades Ligero y Clínico, asegurando una cobertura completa de las necesidad de entrenamiento del usuario. Por último en el ámbito estadístico se desarrolló una serie de gráficos para el seguimiento de los resultados en las rutinas de entrenamiento.   
    
\end{thesischapter}

% USE CASE REALIZATION
\begin{thesischapter}{2} {Diseño e implementación del Juego Serio}
En este capítulo se discuten los detalles de desarrollo de los aspectos citados en el capítulo anterior. Este comienza con una descripción y caracterización general del sistema, donde se  abordan cada uno de los componentes requeridos para su completo funcionamiento. Posteriormente se detalla la ingienría de software requerida en la etapa de conceptualización de la aplicación, se explican de forma detallada los aspectos teóricos y de implementación de la base de datos, el funcionamiento del protocolo de comunicación y por último los escenarios de juegos requeridos en las rutinas de entrenamiento ligero y clínico, y las estadísticas generadas por estos. Como herramienta de desarrollo se utilizó c\#.

% SYSTEM DESCRIPTION AND CHARACTERIZATION TO APPLY
\input{main/chapter2/section1/content.tex}
     
% SERIOUS GAME REQUIREMENTS
\input{main/chapter2/section2/content.tex}    

% USE CASE DEFINITION
\input{main/chapter2/section3/content.tex}

% USE CASE REALIZATION
\input{main/chapter2/section4/content.tex}

% DATABASE DESIGN 
\input{main/chapter2/section5/content.tex}

% DATA MANIPULATION
\input{main/chapter2/section6/content.tex}

% COMMUNICATION 
\input{main/chapter2/section7/content.tex}

% TRAINING SCENARIOS
\input{main/chapter2/section8/content.tex}

% EMG GRAPHIC
\input{main/chapter2/section9/content.tex}

% STATICAL REPORTS GENERATION
\input{main/chapter2/section10/content.tex}

\subthesischapter{Conclusiones del capítulo}
Se presentó una descripción del sistema de adquisición de datos para rehabilitación, sus componentes, características distintivas y su funcionamiento. Se identificaron y definieron los requisitos del juego  serio, tanto funcionales como no funcionales, así como los actores y casos de usos del sistema que establecieron las bases fundamentales para el desarrollo de la aplicación. Se realizó el diseño de la base de datos, abarcando tanto el modelo lógico como el físico, lo que aseguró una estructura robusta y eficiente para el almacenamiento de los datos. La manipulación de los datos se abordó de manera integral, desde la conexión con la base de datos hasta la persistencia de los resultados estadísticos. Se diseñó e implementó la comunicación con el pedal motorizado y la implementación de la interfaz gráfica para la representación de los datos EMG. Se definieron los escenarios de entrenamiento para las modalidades Ligero y Clínico, asegurando una cobertura completa de las necesidad de entrenamiento del usuario. Por último en el ámbito estadístico se desarrolló una serie de gráficos para el seguimiento de los resultados en las rutinas de entrenamiento.   
    
\end{thesischapter}

% DATABASE DESIGN 
\begin{thesischapter}{2} {Diseño e implementación del Juego Serio}
En este capítulo se discuten los detalles de desarrollo de los aspectos citados en el capítulo anterior. Este comienza con una descripción y caracterización general del sistema, donde se  abordan cada uno de los componentes requeridos para su completo funcionamiento. Posteriormente se detalla la ingienría de software requerida en la etapa de conceptualización de la aplicación, se explican de forma detallada los aspectos teóricos y de implementación de la base de datos, el funcionamiento del protocolo de comunicación y por último los escenarios de juegos requeridos en las rutinas de entrenamiento ligero y clínico, y las estadísticas generadas por estos. Como herramienta de desarrollo se utilizó c\#.

% SYSTEM DESCRIPTION AND CHARACTERIZATION TO APPLY
\input{main/chapter2/section1/content.tex}
     
% SERIOUS GAME REQUIREMENTS
\input{main/chapter2/section2/content.tex}    

% USE CASE DEFINITION
\input{main/chapter2/section3/content.tex}

% USE CASE REALIZATION
\input{main/chapter2/section4/content.tex}

% DATABASE DESIGN 
\input{main/chapter2/section5/content.tex}

% DATA MANIPULATION
\input{main/chapter2/section6/content.tex}

% COMMUNICATION 
\input{main/chapter2/section7/content.tex}

% TRAINING SCENARIOS
\input{main/chapter2/section8/content.tex}

% EMG GRAPHIC
\input{main/chapter2/section9/content.tex}

% STATICAL REPORTS GENERATION
\input{main/chapter2/section10/content.tex}

\subthesischapter{Conclusiones del capítulo}
Se presentó una descripción del sistema de adquisición de datos para rehabilitación, sus componentes, características distintivas y su funcionamiento. Se identificaron y definieron los requisitos del juego  serio, tanto funcionales como no funcionales, así como los actores y casos de usos del sistema que establecieron las bases fundamentales para el desarrollo de la aplicación. Se realizó el diseño de la base de datos, abarcando tanto el modelo lógico como el físico, lo que aseguró una estructura robusta y eficiente para el almacenamiento de los datos. La manipulación de los datos se abordó de manera integral, desde la conexión con la base de datos hasta la persistencia de los resultados estadísticos. Se diseñó e implementó la comunicación con el pedal motorizado y la implementación de la interfaz gráfica para la representación de los datos EMG. Se definieron los escenarios de entrenamiento para las modalidades Ligero y Clínico, asegurando una cobertura completa de las necesidad de entrenamiento del usuario. Por último en el ámbito estadístico se desarrolló una serie de gráficos para el seguimiento de los resultados en las rutinas de entrenamiento.   
    
\end{thesischapter}

% DATA MANIPULATION
\begin{thesischapter}{2} {Diseño e implementación del Juego Serio}
En este capítulo se discuten los detalles de desarrollo de los aspectos citados en el capítulo anterior. Este comienza con una descripción y caracterización general del sistema, donde se  abordan cada uno de los componentes requeridos para su completo funcionamiento. Posteriormente se detalla la ingienría de software requerida en la etapa de conceptualización de la aplicación, se explican de forma detallada los aspectos teóricos y de implementación de la base de datos, el funcionamiento del protocolo de comunicación y por último los escenarios de juegos requeridos en las rutinas de entrenamiento ligero y clínico, y las estadísticas generadas por estos. Como herramienta de desarrollo se utilizó c\#.

% SYSTEM DESCRIPTION AND CHARACTERIZATION TO APPLY
\input{main/chapter2/section1/content.tex}
     
% SERIOUS GAME REQUIREMENTS
\input{main/chapter2/section2/content.tex}    

% USE CASE DEFINITION
\input{main/chapter2/section3/content.tex}

% USE CASE REALIZATION
\input{main/chapter2/section4/content.tex}

% DATABASE DESIGN 
\input{main/chapter2/section5/content.tex}

% DATA MANIPULATION
\input{main/chapter2/section6/content.tex}

% COMMUNICATION 
\input{main/chapter2/section7/content.tex}

% TRAINING SCENARIOS
\input{main/chapter2/section8/content.tex}

% EMG GRAPHIC
\input{main/chapter2/section9/content.tex}

% STATICAL REPORTS GENERATION
\input{main/chapter2/section10/content.tex}

\subthesischapter{Conclusiones del capítulo}
Se presentó una descripción del sistema de adquisición de datos para rehabilitación, sus componentes, características distintivas y su funcionamiento. Se identificaron y definieron los requisitos del juego  serio, tanto funcionales como no funcionales, así como los actores y casos de usos del sistema que establecieron las bases fundamentales para el desarrollo de la aplicación. Se realizó el diseño de la base de datos, abarcando tanto el modelo lógico como el físico, lo que aseguró una estructura robusta y eficiente para el almacenamiento de los datos. La manipulación de los datos se abordó de manera integral, desde la conexión con la base de datos hasta la persistencia de los resultados estadísticos. Se diseñó e implementó la comunicación con el pedal motorizado y la implementación de la interfaz gráfica para la representación de los datos EMG. Se definieron los escenarios de entrenamiento para las modalidades Ligero y Clínico, asegurando una cobertura completa de las necesidad de entrenamiento del usuario. Por último en el ámbito estadístico se desarrolló una serie de gráficos para el seguimiento de los resultados en las rutinas de entrenamiento.   
    
\end{thesischapter}

% COMMUNICATION 
\begin{thesischapter}{2} {Diseño e implementación del Juego Serio}
En este capítulo se discuten los detalles de desarrollo de los aspectos citados en el capítulo anterior. Este comienza con una descripción y caracterización general del sistema, donde se  abordan cada uno de los componentes requeridos para su completo funcionamiento. Posteriormente se detalla la ingienría de software requerida en la etapa de conceptualización de la aplicación, se explican de forma detallada los aspectos teóricos y de implementación de la base de datos, el funcionamiento del protocolo de comunicación y por último los escenarios de juegos requeridos en las rutinas de entrenamiento ligero y clínico, y las estadísticas generadas por estos. Como herramienta de desarrollo se utilizó c\#.

% SYSTEM DESCRIPTION AND CHARACTERIZATION TO APPLY
\input{main/chapter2/section1/content.tex}
     
% SERIOUS GAME REQUIREMENTS
\input{main/chapter2/section2/content.tex}    

% USE CASE DEFINITION
\input{main/chapter2/section3/content.tex}

% USE CASE REALIZATION
\input{main/chapter2/section4/content.tex}

% DATABASE DESIGN 
\input{main/chapter2/section5/content.tex}

% DATA MANIPULATION
\input{main/chapter2/section6/content.tex}

% COMMUNICATION 
\input{main/chapter2/section7/content.tex}

% TRAINING SCENARIOS
\input{main/chapter2/section8/content.tex}

% EMG GRAPHIC
\input{main/chapter2/section9/content.tex}

% STATICAL REPORTS GENERATION
\input{main/chapter2/section10/content.tex}

\subthesischapter{Conclusiones del capítulo}
Se presentó una descripción del sistema de adquisición de datos para rehabilitación, sus componentes, características distintivas y su funcionamiento. Se identificaron y definieron los requisitos del juego  serio, tanto funcionales como no funcionales, así como los actores y casos de usos del sistema que establecieron las bases fundamentales para el desarrollo de la aplicación. Se realizó el diseño de la base de datos, abarcando tanto el modelo lógico como el físico, lo que aseguró una estructura robusta y eficiente para el almacenamiento de los datos. La manipulación de los datos se abordó de manera integral, desde la conexión con la base de datos hasta la persistencia de los resultados estadísticos. Se diseñó e implementó la comunicación con el pedal motorizado y la implementación de la interfaz gráfica para la representación de los datos EMG. Se definieron los escenarios de entrenamiento para las modalidades Ligero y Clínico, asegurando una cobertura completa de las necesidad de entrenamiento del usuario. Por último en el ámbito estadístico se desarrolló una serie de gráficos para el seguimiento de los resultados en las rutinas de entrenamiento.   
    
\end{thesischapter}

% TRAINING SCENARIOS
\begin{thesischapter}{2} {Diseño e implementación del Juego Serio}
En este capítulo se discuten los detalles de desarrollo de los aspectos citados en el capítulo anterior. Este comienza con una descripción y caracterización general del sistema, donde se  abordan cada uno de los componentes requeridos para su completo funcionamiento. Posteriormente se detalla la ingienría de software requerida en la etapa de conceptualización de la aplicación, se explican de forma detallada los aspectos teóricos y de implementación de la base de datos, el funcionamiento del protocolo de comunicación y por último los escenarios de juegos requeridos en las rutinas de entrenamiento ligero y clínico, y las estadísticas generadas por estos. Como herramienta de desarrollo se utilizó c\#.

% SYSTEM DESCRIPTION AND CHARACTERIZATION TO APPLY
\input{main/chapter2/section1/content.tex}
     
% SERIOUS GAME REQUIREMENTS
\input{main/chapter2/section2/content.tex}    

% USE CASE DEFINITION
\input{main/chapter2/section3/content.tex}

% USE CASE REALIZATION
\input{main/chapter2/section4/content.tex}

% DATABASE DESIGN 
\input{main/chapter2/section5/content.tex}

% DATA MANIPULATION
\input{main/chapter2/section6/content.tex}

% COMMUNICATION 
\input{main/chapter2/section7/content.tex}

% TRAINING SCENARIOS
\input{main/chapter2/section8/content.tex}

% EMG GRAPHIC
\input{main/chapter2/section9/content.tex}

% STATICAL REPORTS GENERATION
\input{main/chapter2/section10/content.tex}

\subthesischapter{Conclusiones del capítulo}
Se presentó una descripción del sistema de adquisición de datos para rehabilitación, sus componentes, características distintivas y su funcionamiento. Se identificaron y definieron los requisitos del juego  serio, tanto funcionales como no funcionales, así como los actores y casos de usos del sistema que establecieron las bases fundamentales para el desarrollo de la aplicación. Se realizó el diseño de la base de datos, abarcando tanto el modelo lógico como el físico, lo que aseguró una estructura robusta y eficiente para el almacenamiento de los datos. La manipulación de los datos se abordó de manera integral, desde la conexión con la base de datos hasta la persistencia de los resultados estadísticos. Se diseñó e implementó la comunicación con el pedal motorizado y la implementación de la interfaz gráfica para la representación de los datos EMG. Se definieron los escenarios de entrenamiento para las modalidades Ligero y Clínico, asegurando una cobertura completa de las necesidad de entrenamiento del usuario. Por último en el ámbito estadístico se desarrolló una serie de gráficos para el seguimiento de los resultados en las rutinas de entrenamiento.   
    
\end{thesischapter}

% EMG GRAPHIC
\begin{thesischapter}{2} {Diseño e implementación del Juego Serio}
En este capítulo se discuten los detalles de desarrollo de los aspectos citados en el capítulo anterior. Este comienza con una descripción y caracterización general del sistema, donde se  abordan cada uno de los componentes requeridos para su completo funcionamiento. Posteriormente se detalla la ingienría de software requerida en la etapa de conceptualización de la aplicación, se explican de forma detallada los aspectos teóricos y de implementación de la base de datos, el funcionamiento del protocolo de comunicación y por último los escenarios de juegos requeridos en las rutinas de entrenamiento ligero y clínico, y las estadísticas generadas por estos. Como herramienta de desarrollo se utilizó c\#.

% SYSTEM DESCRIPTION AND CHARACTERIZATION TO APPLY
\input{main/chapter2/section1/content.tex}
     
% SERIOUS GAME REQUIREMENTS
\input{main/chapter2/section2/content.tex}    

% USE CASE DEFINITION
\input{main/chapter2/section3/content.tex}

% USE CASE REALIZATION
\input{main/chapter2/section4/content.tex}

% DATABASE DESIGN 
\input{main/chapter2/section5/content.tex}

% DATA MANIPULATION
\input{main/chapter2/section6/content.tex}

% COMMUNICATION 
\input{main/chapter2/section7/content.tex}

% TRAINING SCENARIOS
\input{main/chapter2/section8/content.tex}

% EMG GRAPHIC
\input{main/chapter2/section9/content.tex}

% STATICAL REPORTS GENERATION
\input{main/chapter2/section10/content.tex}

\subthesischapter{Conclusiones del capítulo}
Se presentó una descripción del sistema de adquisición de datos para rehabilitación, sus componentes, características distintivas y su funcionamiento. Se identificaron y definieron los requisitos del juego  serio, tanto funcionales como no funcionales, así como los actores y casos de usos del sistema que establecieron las bases fundamentales para el desarrollo de la aplicación. Se realizó el diseño de la base de datos, abarcando tanto el modelo lógico como el físico, lo que aseguró una estructura robusta y eficiente para el almacenamiento de los datos. La manipulación de los datos se abordó de manera integral, desde la conexión con la base de datos hasta la persistencia de los resultados estadísticos. Se diseñó e implementó la comunicación con el pedal motorizado y la implementación de la interfaz gráfica para la representación de los datos EMG. Se definieron los escenarios de entrenamiento para las modalidades Ligero y Clínico, asegurando una cobertura completa de las necesidad de entrenamiento del usuario. Por último en el ámbito estadístico se desarrolló una serie de gráficos para el seguimiento de los resultados en las rutinas de entrenamiento.   
    
\end{thesischapter}

% STATICAL REPORTS GENERATION
\begin{thesischapter}{2} {Diseño e implementación del Juego Serio}
En este capítulo se discuten los detalles de desarrollo de los aspectos citados en el capítulo anterior. Este comienza con una descripción y caracterización general del sistema, donde se  abordan cada uno de los componentes requeridos para su completo funcionamiento. Posteriormente se detalla la ingienría de software requerida en la etapa de conceptualización de la aplicación, se explican de forma detallada los aspectos teóricos y de implementación de la base de datos, el funcionamiento del protocolo de comunicación y por último los escenarios de juegos requeridos en las rutinas de entrenamiento ligero y clínico, y las estadísticas generadas por estos. Como herramienta de desarrollo se utilizó c\#.

% SYSTEM DESCRIPTION AND CHARACTERIZATION TO APPLY
\input{main/chapter2/section1/content.tex}
     
% SERIOUS GAME REQUIREMENTS
\input{main/chapter2/section2/content.tex}    

% USE CASE DEFINITION
\input{main/chapter2/section3/content.tex}

% USE CASE REALIZATION
\input{main/chapter2/section4/content.tex}

% DATABASE DESIGN 
\input{main/chapter2/section5/content.tex}

% DATA MANIPULATION
\input{main/chapter2/section6/content.tex}

% COMMUNICATION 
\input{main/chapter2/section7/content.tex}

% TRAINING SCENARIOS
\input{main/chapter2/section8/content.tex}

% EMG GRAPHIC
\input{main/chapter2/section9/content.tex}

% STATICAL REPORTS GENERATION
\input{main/chapter2/section10/content.tex}

\subthesischapter{Conclusiones del capítulo}
Se presentó una descripción del sistema de adquisición de datos para rehabilitación, sus componentes, características distintivas y su funcionamiento. Se identificaron y definieron los requisitos del juego  serio, tanto funcionales como no funcionales, así como los actores y casos de usos del sistema que establecieron las bases fundamentales para el desarrollo de la aplicación. Se realizó el diseño de la base de datos, abarcando tanto el modelo lógico como el físico, lo que aseguró una estructura robusta y eficiente para el almacenamiento de los datos. La manipulación de los datos se abordó de manera integral, desde la conexión con la base de datos hasta la persistencia de los resultados estadísticos. Se diseñó e implementó la comunicación con el pedal motorizado y la implementación de la interfaz gráfica para la representación de los datos EMG. Se definieron los escenarios de entrenamiento para las modalidades Ligero y Clínico, asegurando una cobertura completa de las necesidad de entrenamiento del usuario. Por último en el ámbito estadístico se desarrolló una serie de gráficos para el seguimiento de los resultados en las rutinas de entrenamiento.   
    
\end{thesischapter}

\subthesischapter{Conclusiones del capítulo}
Se presentó una descripción del sistema de adquisición de datos para rehabilitación, sus componentes, características distintivas y su funcionamiento. Se identificaron y definieron los requisitos del juego  serio, tanto funcionales como no funcionales, así como los actores y casos de usos del sistema que establecieron las bases fundamentales para el desarrollo de la aplicación. Se realizó el diseño de la base de datos, abarcando tanto el modelo lógico como el físico, lo que aseguró una estructura robusta y eficiente para el almacenamiento de los datos. La manipulación de los datos se abordó de manera integral, desde la conexión con la base de datos hasta la persistencia de los resultados estadísticos. Se diseñó e implementó la comunicación con el pedal motorizado y la implementación de la interfaz gráfica para la representación de los datos EMG. Se definieron los escenarios de entrenamiento para las modalidades Ligero y Clínico, asegurando una cobertura completa de las necesidad de entrenamiento del usuario. Por último en el ámbito estadístico se desarrolló una serie de gráficos para el seguimiento de los resultados en las rutinas de entrenamiento.   
    
\end{thesischapter}

% DATABASE DESIGN 
\begin{thesischapter}{2} {Diseño e implementación del Juego Serio}
En este capítulo se discuten los detalles de desarrollo de los aspectos citados en el capítulo anterior. Este comienza con una descripción y caracterización general del sistema, donde se  abordan cada uno de los componentes requeridos para su completo funcionamiento. Posteriormente se detalla la ingienría de software requerida en la etapa de conceptualización de la aplicación, se explican de forma detallada los aspectos teóricos y de implementación de la base de datos, el funcionamiento del protocolo de comunicación y por último los escenarios de juegos requeridos en las rutinas de entrenamiento ligero y clínico, y las estadísticas generadas por estos. Como herramienta de desarrollo se utilizó c\#.

% SYSTEM DESCRIPTION AND CHARACTERIZATION TO APPLY
\begin{thesischapter}{2} {Diseño e implementación del Juego Serio}
En este capítulo se discuten los detalles de desarrollo de los aspectos citados en el capítulo anterior. Este comienza con una descripción y caracterización general del sistema, donde se  abordan cada uno de los componentes requeridos para su completo funcionamiento. Posteriormente se detalla la ingienría de software requerida en la etapa de conceptualización de la aplicación, se explican de forma detallada los aspectos teóricos y de implementación de la base de datos, el funcionamiento del protocolo de comunicación y por último los escenarios de juegos requeridos en las rutinas de entrenamiento ligero y clínico, y las estadísticas generadas por estos. Como herramienta de desarrollo se utilizó c\#.

% SYSTEM DESCRIPTION AND CHARACTERIZATION TO APPLY
\input{main/chapter2/section1/content.tex}
     
% SERIOUS GAME REQUIREMENTS
\input{main/chapter2/section2/content.tex}    

% USE CASE DEFINITION
\input{main/chapter2/section3/content.tex}

% USE CASE REALIZATION
\input{main/chapter2/section4/content.tex}

% DATABASE DESIGN 
\input{main/chapter2/section5/content.tex}

% DATA MANIPULATION
\input{main/chapter2/section6/content.tex}

% COMMUNICATION 
\input{main/chapter2/section7/content.tex}

% TRAINING SCENARIOS
\input{main/chapter2/section8/content.tex}

% EMG GRAPHIC
\input{main/chapter2/section9/content.tex}

% STATICAL REPORTS GENERATION
\input{main/chapter2/section10/content.tex}

\subthesischapter{Conclusiones del capítulo}
Se presentó una descripción del sistema de adquisición de datos para rehabilitación, sus componentes, características distintivas y su funcionamiento. Se identificaron y definieron los requisitos del juego  serio, tanto funcionales como no funcionales, así como los actores y casos de usos del sistema que establecieron las bases fundamentales para el desarrollo de la aplicación. Se realizó el diseño de la base de datos, abarcando tanto el modelo lógico como el físico, lo que aseguró una estructura robusta y eficiente para el almacenamiento de los datos. La manipulación de los datos se abordó de manera integral, desde la conexión con la base de datos hasta la persistencia de los resultados estadísticos. Se diseñó e implementó la comunicación con el pedal motorizado y la implementación de la interfaz gráfica para la representación de los datos EMG. Se definieron los escenarios de entrenamiento para las modalidades Ligero y Clínico, asegurando una cobertura completa de las necesidad de entrenamiento del usuario. Por último en el ámbito estadístico se desarrolló una serie de gráficos para el seguimiento de los resultados en las rutinas de entrenamiento.   
    
\end{thesischapter}
     
% SERIOUS GAME REQUIREMENTS
\begin{thesischapter}{2} {Diseño e implementación del Juego Serio}
En este capítulo se discuten los detalles de desarrollo de los aspectos citados en el capítulo anterior. Este comienza con una descripción y caracterización general del sistema, donde se  abordan cada uno de los componentes requeridos para su completo funcionamiento. Posteriormente se detalla la ingienría de software requerida en la etapa de conceptualización de la aplicación, se explican de forma detallada los aspectos teóricos y de implementación de la base de datos, el funcionamiento del protocolo de comunicación y por último los escenarios de juegos requeridos en las rutinas de entrenamiento ligero y clínico, y las estadísticas generadas por estos. Como herramienta de desarrollo se utilizó c\#.

% SYSTEM DESCRIPTION AND CHARACTERIZATION TO APPLY
\input{main/chapter2/section1/content.tex}
     
% SERIOUS GAME REQUIREMENTS
\input{main/chapter2/section2/content.tex}    

% USE CASE DEFINITION
\input{main/chapter2/section3/content.tex}

% USE CASE REALIZATION
\input{main/chapter2/section4/content.tex}

% DATABASE DESIGN 
\input{main/chapter2/section5/content.tex}

% DATA MANIPULATION
\input{main/chapter2/section6/content.tex}

% COMMUNICATION 
\input{main/chapter2/section7/content.tex}

% TRAINING SCENARIOS
\input{main/chapter2/section8/content.tex}

% EMG GRAPHIC
\input{main/chapter2/section9/content.tex}

% STATICAL REPORTS GENERATION
\input{main/chapter2/section10/content.tex}

\subthesischapter{Conclusiones del capítulo}
Se presentó una descripción del sistema de adquisición de datos para rehabilitación, sus componentes, características distintivas y su funcionamiento. Se identificaron y definieron los requisitos del juego  serio, tanto funcionales como no funcionales, así como los actores y casos de usos del sistema que establecieron las bases fundamentales para el desarrollo de la aplicación. Se realizó el diseño de la base de datos, abarcando tanto el modelo lógico como el físico, lo que aseguró una estructura robusta y eficiente para el almacenamiento de los datos. La manipulación de los datos se abordó de manera integral, desde la conexión con la base de datos hasta la persistencia de los resultados estadísticos. Se diseñó e implementó la comunicación con el pedal motorizado y la implementación de la interfaz gráfica para la representación de los datos EMG. Se definieron los escenarios de entrenamiento para las modalidades Ligero y Clínico, asegurando una cobertura completa de las necesidad de entrenamiento del usuario. Por último en el ámbito estadístico se desarrolló una serie de gráficos para el seguimiento de los resultados en las rutinas de entrenamiento.   
    
\end{thesischapter}    

% USE CASE DEFINITION
\begin{thesischapter}{2} {Diseño e implementación del Juego Serio}
En este capítulo se discuten los detalles de desarrollo de los aspectos citados en el capítulo anterior. Este comienza con una descripción y caracterización general del sistema, donde se  abordan cada uno de los componentes requeridos para su completo funcionamiento. Posteriormente se detalla la ingienría de software requerida en la etapa de conceptualización de la aplicación, se explican de forma detallada los aspectos teóricos y de implementación de la base de datos, el funcionamiento del protocolo de comunicación y por último los escenarios de juegos requeridos en las rutinas de entrenamiento ligero y clínico, y las estadísticas generadas por estos. Como herramienta de desarrollo se utilizó c\#.

% SYSTEM DESCRIPTION AND CHARACTERIZATION TO APPLY
\input{main/chapter2/section1/content.tex}
     
% SERIOUS GAME REQUIREMENTS
\input{main/chapter2/section2/content.tex}    

% USE CASE DEFINITION
\input{main/chapter2/section3/content.tex}

% USE CASE REALIZATION
\input{main/chapter2/section4/content.tex}

% DATABASE DESIGN 
\input{main/chapter2/section5/content.tex}

% DATA MANIPULATION
\input{main/chapter2/section6/content.tex}

% COMMUNICATION 
\input{main/chapter2/section7/content.tex}

% TRAINING SCENARIOS
\input{main/chapter2/section8/content.tex}

% EMG GRAPHIC
\input{main/chapter2/section9/content.tex}

% STATICAL REPORTS GENERATION
\input{main/chapter2/section10/content.tex}

\subthesischapter{Conclusiones del capítulo}
Se presentó una descripción del sistema de adquisición de datos para rehabilitación, sus componentes, características distintivas y su funcionamiento. Se identificaron y definieron los requisitos del juego  serio, tanto funcionales como no funcionales, así como los actores y casos de usos del sistema que establecieron las bases fundamentales para el desarrollo de la aplicación. Se realizó el diseño de la base de datos, abarcando tanto el modelo lógico como el físico, lo que aseguró una estructura robusta y eficiente para el almacenamiento de los datos. La manipulación de los datos se abordó de manera integral, desde la conexión con la base de datos hasta la persistencia de los resultados estadísticos. Se diseñó e implementó la comunicación con el pedal motorizado y la implementación de la interfaz gráfica para la representación de los datos EMG. Se definieron los escenarios de entrenamiento para las modalidades Ligero y Clínico, asegurando una cobertura completa de las necesidad de entrenamiento del usuario. Por último en el ámbito estadístico se desarrolló una serie de gráficos para el seguimiento de los resultados en las rutinas de entrenamiento.   
    
\end{thesischapter}

% USE CASE REALIZATION
\begin{thesischapter}{2} {Diseño e implementación del Juego Serio}
En este capítulo se discuten los detalles de desarrollo de los aspectos citados en el capítulo anterior. Este comienza con una descripción y caracterización general del sistema, donde se  abordan cada uno de los componentes requeridos para su completo funcionamiento. Posteriormente se detalla la ingienría de software requerida en la etapa de conceptualización de la aplicación, se explican de forma detallada los aspectos teóricos y de implementación de la base de datos, el funcionamiento del protocolo de comunicación y por último los escenarios de juegos requeridos en las rutinas de entrenamiento ligero y clínico, y las estadísticas generadas por estos. Como herramienta de desarrollo se utilizó c\#.

% SYSTEM DESCRIPTION AND CHARACTERIZATION TO APPLY
\input{main/chapter2/section1/content.tex}
     
% SERIOUS GAME REQUIREMENTS
\input{main/chapter2/section2/content.tex}    

% USE CASE DEFINITION
\input{main/chapter2/section3/content.tex}

% USE CASE REALIZATION
\input{main/chapter2/section4/content.tex}

% DATABASE DESIGN 
\input{main/chapter2/section5/content.tex}

% DATA MANIPULATION
\input{main/chapter2/section6/content.tex}

% COMMUNICATION 
\input{main/chapter2/section7/content.tex}

% TRAINING SCENARIOS
\input{main/chapter2/section8/content.tex}

% EMG GRAPHIC
\input{main/chapter2/section9/content.tex}

% STATICAL REPORTS GENERATION
\input{main/chapter2/section10/content.tex}

\subthesischapter{Conclusiones del capítulo}
Se presentó una descripción del sistema de adquisición de datos para rehabilitación, sus componentes, características distintivas y su funcionamiento. Se identificaron y definieron los requisitos del juego  serio, tanto funcionales como no funcionales, así como los actores y casos de usos del sistema que establecieron las bases fundamentales para el desarrollo de la aplicación. Se realizó el diseño de la base de datos, abarcando tanto el modelo lógico como el físico, lo que aseguró una estructura robusta y eficiente para el almacenamiento de los datos. La manipulación de los datos se abordó de manera integral, desde la conexión con la base de datos hasta la persistencia de los resultados estadísticos. Se diseñó e implementó la comunicación con el pedal motorizado y la implementación de la interfaz gráfica para la representación de los datos EMG. Se definieron los escenarios de entrenamiento para las modalidades Ligero y Clínico, asegurando una cobertura completa de las necesidad de entrenamiento del usuario. Por último en el ámbito estadístico se desarrolló una serie de gráficos para el seguimiento de los resultados en las rutinas de entrenamiento.   
    
\end{thesischapter}

% DATABASE DESIGN 
\begin{thesischapter}{2} {Diseño e implementación del Juego Serio}
En este capítulo se discuten los detalles de desarrollo de los aspectos citados en el capítulo anterior. Este comienza con una descripción y caracterización general del sistema, donde se  abordan cada uno de los componentes requeridos para su completo funcionamiento. Posteriormente se detalla la ingienría de software requerida en la etapa de conceptualización de la aplicación, se explican de forma detallada los aspectos teóricos y de implementación de la base de datos, el funcionamiento del protocolo de comunicación y por último los escenarios de juegos requeridos en las rutinas de entrenamiento ligero y clínico, y las estadísticas generadas por estos. Como herramienta de desarrollo se utilizó c\#.

% SYSTEM DESCRIPTION AND CHARACTERIZATION TO APPLY
\input{main/chapter2/section1/content.tex}
     
% SERIOUS GAME REQUIREMENTS
\input{main/chapter2/section2/content.tex}    

% USE CASE DEFINITION
\input{main/chapter2/section3/content.tex}

% USE CASE REALIZATION
\input{main/chapter2/section4/content.tex}

% DATABASE DESIGN 
\input{main/chapter2/section5/content.tex}

% DATA MANIPULATION
\input{main/chapter2/section6/content.tex}

% COMMUNICATION 
\input{main/chapter2/section7/content.tex}

% TRAINING SCENARIOS
\input{main/chapter2/section8/content.tex}

% EMG GRAPHIC
\input{main/chapter2/section9/content.tex}

% STATICAL REPORTS GENERATION
\input{main/chapter2/section10/content.tex}

\subthesischapter{Conclusiones del capítulo}
Se presentó una descripción del sistema de adquisición de datos para rehabilitación, sus componentes, características distintivas y su funcionamiento. Se identificaron y definieron los requisitos del juego  serio, tanto funcionales como no funcionales, así como los actores y casos de usos del sistema que establecieron las bases fundamentales para el desarrollo de la aplicación. Se realizó el diseño de la base de datos, abarcando tanto el modelo lógico como el físico, lo que aseguró una estructura robusta y eficiente para el almacenamiento de los datos. La manipulación de los datos se abordó de manera integral, desde la conexión con la base de datos hasta la persistencia de los resultados estadísticos. Se diseñó e implementó la comunicación con el pedal motorizado y la implementación de la interfaz gráfica para la representación de los datos EMG. Se definieron los escenarios de entrenamiento para las modalidades Ligero y Clínico, asegurando una cobertura completa de las necesidad de entrenamiento del usuario. Por último en el ámbito estadístico se desarrolló una serie de gráficos para el seguimiento de los resultados en las rutinas de entrenamiento.   
    
\end{thesischapter}

% DATA MANIPULATION
\begin{thesischapter}{2} {Diseño e implementación del Juego Serio}
En este capítulo se discuten los detalles de desarrollo de los aspectos citados en el capítulo anterior. Este comienza con una descripción y caracterización general del sistema, donde se  abordan cada uno de los componentes requeridos para su completo funcionamiento. Posteriormente se detalla la ingienría de software requerida en la etapa de conceptualización de la aplicación, se explican de forma detallada los aspectos teóricos y de implementación de la base de datos, el funcionamiento del protocolo de comunicación y por último los escenarios de juegos requeridos en las rutinas de entrenamiento ligero y clínico, y las estadísticas generadas por estos. Como herramienta de desarrollo se utilizó c\#.

% SYSTEM DESCRIPTION AND CHARACTERIZATION TO APPLY
\input{main/chapter2/section1/content.tex}
     
% SERIOUS GAME REQUIREMENTS
\input{main/chapter2/section2/content.tex}    

% USE CASE DEFINITION
\input{main/chapter2/section3/content.tex}

% USE CASE REALIZATION
\input{main/chapter2/section4/content.tex}

% DATABASE DESIGN 
\input{main/chapter2/section5/content.tex}

% DATA MANIPULATION
\input{main/chapter2/section6/content.tex}

% COMMUNICATION 
\input{main/chapter2/section7/content.tex}

% TRAINING SCENARIOS
\input{main/chapter2/section8/content.tex}

% EMG GRAPHIC
\input{main/chapter2/section9/content.tex}

% STATICAL REPORTS GENERATION
\input{main/chapter2/section10/content.tex}

\subthesischapter{Conclusiones del capítulo}
Se presentó una descripción del sistema de adquisición de datos para rehabilitación, sus componentes, características distintivas y su funcionamiento. Se identificaron y definieron los requisitos del juego  serio, tanto funcionales como no funcionales, así como los actores y casos de usos del sistema que establecieron las bases fundamentales para el desarrollo de la aplicación. Se realizó el diseño de la base de datos, abarcando tanto el modelo lógico como el físico, lo que aseguró una estructura robusta y eficiente para el almacenamiento de los datos. La manipulación de los datos se abordó de manera integral, desde la conexión con la base de datos hasta la persistencia de los resultados estadísticos. Se diseñó e implementó la comunicación con el pedal motorizado y la implementación de la interfaz gráfica para la representación de los datos EMG. Se definieron los escenarios de entrenamiento para las modalidades Ligero y Clínico, asegurando una cobertura completa de las necesidad de entrenamiento del usuario. Por último en el ámbito estadístico se desarrolló una serie de gráficos para el seguimiento de los resultados en las rutinas de entrenamiento.   
    
\end{thesischapter}

% COMMUNICATION 
\begin{thesischapter}{2} {Diseño e implementación del Juego Serio}
En este capítulo se discuten los detalles de desarrollo de los aspectos citados en el capítulo anterior. Este comienza con una descripción y caracterización general del sistema, donde se  abordan cada uno de los componentes requeridos para su completo funcionamiento. Posteriormente se detalla la ingienría de software requerida en la etapa de conceptualización de la aplicación, se explican de forma detallada los aspectos teóricos y de implementación de la base de datos, el funcionamiento del protocolo de comunicación y por último los escenarios de juegos requeridos en las rutinas de entrenamiento ligero y clínico, y las estadísticas generadas por estos. Como herramienta de desarrollo se utilizó c\#.

% SYSTEM DESCRIPTION AND CHARACTERIZATION TO APPLY
\input{main/chapter2/section1/content.tex}
     
% SERIOUS GAME REQUIREMENTS
\input{main/chapter2/section2/content.tex}    

% USE CASE DEFINITION
\input{main/chapter2/section3/content.tex}

% USE CASE REALIZATION
\input{main/chapter2/section4/content.tex}

% DATABASE DESIGN 
\input{main/chapter2/section5/content.tex}

% DATA MANIPULATION
\input{main/chapter2/section6/content.tex}

% COMMUNICATION 
\input{main/chapter2/section7/content.tex}

% TRAINING SCENARIOS
\input{main/chapter2/section8/content.tex}

% EMG GRAPHIC
\input{main/chapter2/section9/content.tex}

% STATICAL REPORTS GENERATION
\input{main/chapter2/section10/content.tex}

\subthesischapter{Conclusiones del capítulo}
Se presentó una descripción del sistema de adquisición de datos para rehabilitación, sus componentes, características distintivas y su funcionamiento. Se identificaron y definieron los requisitos del juego  serio, tanto funcionales como no funcionales, así como los actores y casos de usos del sistema que establecieron las bases fundamentales para el desarrollo de la aplicación. Se realizó el diseño de la base de datos, abarcando tanto el modelo lógico como el físico, lo que aseguró una estructura robusta y eficiente para el almacenamiento de los datos. La manipulación de los datos se abordó de manera integral, desde la conexión con la base de datos hasta la persistencia de los resultados estadísticos. Se diseñó e implementó la comunicación con el pedal motorizado y la implementación de la interfaz gráfica para la representación de los datos EMG. Se definieron los escenarios de entrenamiento para las modalidades Ligero y Clínico, asegurando una cobertura completa de las necesidad de entrenamiento del usuario. Por último en el ámbito estadístico se desarrolló una serie de gráficos para el seguimiento de los resultados en las rutinas de entrenamiento.   
    
\end{thesischapter}

% TRAINING SCENARIOS
\begin{thesischapter}{2} {Diseño e implementación del Juego Serio}
En este capítulo se discuten los detalles de desarrollo de los aspectos citados en el capítulo anterior. Este comienza con una descripción y caracterización general del sistema, donde se  abordan cada uno de los componentes requeridos para su completo funcionamiento. Posteriormente se detalla la ingienría de software requerida en la etapa de conceptualización de la aplicación, se explican de forma detallada los aspectos teóricos y de implementación de la base de datos, el funcionamiento del protocolo de comunicación y por último los escenarios de juegos requeridos en las rutinas de entrenamiento ligero y clínico, y las estadísticas generadas por estos. Como herramienta de desarrollo se utilizó c\#.

% SYSTEM DESCRIPTION AND CHARACTERIZATION TO APPLY
\input{main/chapter2/section1/content.tex}
     
% SERIOUS GAME REQUIREMENTS
\input{main/chapter2/section2/content.tex}    

% USE CASE DEFINITION
\input{main/chapter2/section3/content.tex}

% USE CASE REALIZATION
\input{main/chapter2/section4/content.tex}

% DATABASE DESIGN 
\input{main/chapter2/section5/content.tex}

% DATA MANIPULATION
\input{main/chapter2/section6/content.tex}

% COMMUNICATION 
\input{main/chapter2/section7/content.tex}

% TRAINING SCENARIOS
\input{main/chapter2/section8/content.tex}

% EMG GRAPHIC
\input{main/chapter2/section9/content.tex}

% STATICAL REPORTS GENERATION
\input{main/chapter2/section10/content.tex}

\subthesischapter{Conclusiones del capítulo}
Se presentó una descripción del sistema de adquisición de datos para rehabilitación, sus componentes, características distintivas y su funcionamiento. Se identificaron y definieron los requisitos del juego  serio, tanto funcionales como no funcionales, así como los actores y casos de usos del sistema que establecieron las bases fundamentales para el desarrollo de la aplicación. Se realizó el diseño de la base de datos, abarcando tanto el modelo lógico como el físico, lo que aseguró una estructura robusta y eficiente para el almacenamiento de los datos. La manipulación de los datos se abordó de manera integral, desde la conexión con la base de datos hasta la persistencia de los resultados estadísticos. Se diseñó e implementó la comunicación con el pedal motorizado y la implementación de la interfaz gráfica para la representación de los datos EMG. Se definieron los escenarios de entrenamiento para las modalidades Ligero y Clínico, asegurando una cobertura completa de las necesidad de entrenamiento del usuario. Por último en el ámbito estadístico se desarrolló una serie de gráficos para el seguimiento de los resultados en las rutinas de entrenamiento.   
    
\end{thesischapter}

% EMG GRAPHIC
\begin{thesischapter}{2} {Diseño e implementación del Juego Serio}
En este capítulo se discuten los detalles de desarrollo de los aspectos citados en el capítulo anterior. Este comienza con una descripción y caracterización general del sistema, donde se  abordan cada uno de los componentes requeridos para su completo funcionamiento. Posteriormente se detalla la ingienría de software requerida en la etapa de conceptualización de la aplicación, se explican de forma detallada los aspectos teóricos y de implementación de la base de datos, el funcionamiento del protocolo de comunicación y por último los escenarios de juegos requeridos en las rutinas de entrenamiento ligero y clínico, y las estadísticas generadas por estos. Como herramienta de desarrollo se utilizó c\#.

% SYSTEM DESCRIPTION AND CHARACTERIZATION TO APPLY
\input{main/chapter2/section1/content.tex}
     
% SERIOUS GAME REQUIREMENTS
\input{main/chapter2/section2/content.tex}    

% USE CASE DEFINITION
\input{main/chapter2/section3/content.tex}

% USE CASE REALIZATION
\input{main/chapter2/section4/content.tex}

% DATABASE DESIGN 
\input{main/chapter2/section5/content.tex}

% DATA MANIPULATION
\input{main/chapter2/section6/content.tex}

% COMMUNICATION 
\input{main/chapter2/section7/content.tex}

% TRAINING SCENARIOS
\input{main/chapter2/section8/content.tex}

% EMG GRAPHIC
\input{main/chapter2/section9/content.tex}

% STATICAL REPORTS GENERATION
\input{main/chapter2/section10/content.tex}

\subthesischapter{Conclusiones del capítulo}
Se presentó una descripción del sistema de adquisición de datos para rehabilitación, sus componentes, características distintivas y su funcionamiento. Se identificaron y definieron los requisitos del juego  serio, tanto funcionales como no funcionales, así como los actores y casos de usos del sistema que establecieron las bases fundamentales para el desarrollo de la aplicación. Se realizó el diseño de la base de datos, abarcando tanto el modelo lógico como el físico, lo que aseguró una estructura robusta y eficiente para el almacenamiento de los datos. La manipulación de los datos se abordó de manera integral, desde la conexión con la base de datos hasta la persistencia de los resultados estadísticos. Se diseñó e implementó la comunicación con el pedal motorizado y la implementación de la interfaz gráfica para la representación de los datos EMG. Se definieron los escenarios de entrenamiento para las modalidades Ligero y Clínico, asegurando una cobertura completa de las necesidad de entrenamiento del usuario. Por último en el ámbito estadístico se desarrolló una serie de gráficos para el seguimiento de los resultados en las rutinas de entrenamiento.   
    
\end{thesischapter}

% STATICAL REPORTS GENERATION
\begin{thesischapter}{2} {Diseño e implementación del Juego Serio}
En este capítulo se discuten los detalles de desarrollo de los aspectos citados en el capítulo anterior. Este comienza con una descripción y caracterización general del sistema, donde se  abordan cada uno de los componentes requeridos para su completo funcionamiento. Posteriormente se detalla la ingienría de software requerida en la etapa de conceptualización de la aplicación, se explican de forma detallada los aspectos teóricos y de implementación de la base de datos, el funcionamiento del protocolo de comunicación y por último los escenarios de juegos requeridos en las rutinas de entrenamiento ligero y clínico, y las estadísticas generadas por estos. Como herramienta de desarrollo se utilizó c\#.

% SYSTEM DESCRIPTION AND CHARACTERIZATION TO APPLY
\input{main/chapter2/section1/content.tex}
     
% SERIOUS GAME REQUIREMENTS
\input{main/chapter2/section2/content.tex}    

% USE CASE DEFINITION
\input{main/chapter2/section3/content.tex}

% USE CASE REALIZATION
\input{main/chapter2/section4/content.tex}

% DATABASE DESIGN 
\input{main/chapter2/section5/content.tex}

% DATA MANIPULATION
\input{main/chapter2/section6/content.tex}

% COMMUNICATION 
\input{main/chapter2/section7/content.tex}

% TRAINING SCENARIOS
\input{main/chapter2/section8/content.tex}

% EMG GRAPHIC
\input{main/chapter2/section9/content.tex}

% STATICAL REPORTS GENERATION
\input{main/chapter2/section10/content.tex}

\subthesischapter{Conclusiones del capítulo}
Se presentó una descripción del sistema de adquisición de datos para rehabilitación, sus componentes, características distintivas y su funcionamiento. Se identificaron y definieron los requisitos del juego  serio, tanto funcionales como no funcionales, así como los actores y casos de usos del sistema que establecieron las bases fundamentales para el desarrollo de la aplicación. Se realizó el diseño de la base de datos, abarcando tanto el modelo lógico como el físico, lo que aseguró una estructura robusta y eficiente para el almacenamiento de los datos. La manipulación de los datos se abordó de manera integral, desde la conexión con la base de datos hasta la persistencia de los resultados estadísticos. Se diseñó e implementó la comunicación con el pedal motorizado y la implementación de la interfaz gráfica para la representación de los datos EMG. Se definieron los escenarios de entrenamiento para las modalidades Ligero y Clínico, asegurando una cobertura completa de las necesidad de entrenamiento del usuario. Por último en el ámbito estadístico se desarrolló una serie de gráficos para el seguimiento de los resultados en las rutinas de entrenamiento.   
    
\end{thesischapter}

\subthesischapter{Conclusiones del capítulo}
Se presentó una descripción del sistema de adquisición de datos para rehabilitación, sus componentes, características distintivas y su funcionamiento. Se identificaron y definieron los requisitos del juego  serio, tanto funcionales como no funcionales, así como los actores y casos de usos del sistema que establecieron las bases fundamentales para el desarrollo de la aplicación. Se realizó el diseño de la base de datos, abarcando tanto el modelo lógico como el físico, lo que aseguró una estructura robusta y eficiente para el almacenamiento de los datos. La manipulación de los datos se abordó de manera integral, desde la conexión con la base de datos hasta la persistencia de los resultados estadísticos. Se diseñó e implementó la comunicación con el pedal motorizado y la implementación de la interfaz gráfica para la representación de los datos EMG. Se definieron los escenarios de entrenamiento para las modalidades Ligero y Clínico, asegurando una cobertura completa de las necesidad de entrenamiento del usuario. Por último en el ámbito estadístico se desarrolló una serie de gráficos para el seguimiento de los resultados en las rutinas de entrenamiento.   
    
\end{thesischapter}

% DATA MANIPULATION
\begin{thesischapter}{2} {Diseño e implementación del Juego Serio}
En este capítulo se discuten los detalles de desarrollo de los aspectos citados en el capítulo anterior. Este comienza con una descripción y caracterización general del sistema, donde se  abordan cada uno de los componentes requeridos para su completo funcionamiento. Posteriormente se detalla la ingienría de software requerida en la etapa de conceptualización de la aplicación, se explican de forma detallada los aspectos teóricos y de implementación de la base de datos, el funcionamiento del protocolo de comunicación y por último los escenarios de juegos requeridos en las rutinas de entrenamiento ligero y clínico, y las estadísticas generadas por estos. Como herramienta de desarrollo se utilizó c\#.

% SYSTEM DESCRIPTION AND CHARACTERIZATION TO APPLY
\begin{thesischapter}{2} {Diseño e implementación del Juego Serio}
En este capítulo se discuten los detalles de desarrollo de los aspectos citados en el capítulo anterior. Este comienza con una descripción y caracterización general del sistema, donde se  abordan cada uno de los componentes requeridos para su completo funcionamiento. Posteriormente se detalla la ingienría de software requerida en la etapa de conceptualización de la aplicación, se explican de forma detallada los aspectos teóricos y de implementación de la base de datos, el funcionamiento del protocolo de comunicación y por último los escenarios de juegos requeridos en las rutinas de entrenamiento ligero y clínico, y las estadísticas generadas por estos. Como herramienta de desarrollo se utilizó c\#.

% SYSTEM DESCRIPTION AND CHARACTERIZATION TO APPLY
\input{main/chapter2/section1/content.tex}
     
% SERIOUS GAME REQUIREMENTS
\input{main/chapter2/section2/content.tex}    

% USE CASE DEFINITION
\input{main/chapter2/section3/content.tex}

% USE CASE REALIZATION
\input{main/chapter2/section4/content.tex}

% DATABASE DESIGN 
\input{main/chapter2/section5/content.tex}

% DATA MANIPULATION
\input{main/chapter2/section6/content.tex}

% COMMUNICATION 
\input{main/chapter2/section7/content.tex}

% TRAINING SCENARIOS
\input{main/chapter2/section8/content.tex}

% EMG GRAPHIC
\input{main/chapter2/section9/content.tex}

% STATICAL REPORTS GENERATION
\input{main/chapter2/section10/content.tex}

\subthesischapter{Conclusiones del capítulo}
Se presentó una descripción del sistema de adquisición de datos para rehabilitación, sus componentes, características distintivas y su funcionamiento. Se identificaron y definieron los requisitos del juego  serio, tanto funcionales como no funcionales, así como los actores y casos de usos del sistema que establecieron las bases fundamentales para el desarrollo de la aplicación. Se realizó el diseño de la base de datos, abarcando tanto el modelo lógico como el físico, lo que aseguró una estructura robusta y eficiente para el almacenamiento de los datos. La manipulación de los datos se abordó de manera integral, desde la conexión con la base de datos hasta la persistencia de los resultados estadísticos. Se diseñó e implementó la comunicación con el pedal motorizado y la implementación de la interfaz gráfica para la representación de los datos EMG. Se definieron los escenarios de entrenamiento para las modalidades Ligero y Clínico, asegurando una cobertura completa de las necesidad de entrenamiento del usuario. Por último en el ámbito estadístico se desarrolló una serie de gráficos para el seguimiento de los resultados en las rutinas de entrenamiento.   
    
\end{thesischapter}
     
% SERIOUS GAME REQUIREMENTS
\begin{thesischapter}{2} {Diseño e implementación del Juego Serio}
En este capítulo se discuten los detalles de desarrollo de los aspectos citados en el capítulo anterior. Este comienza con una descripción y caracterización general del sistema, donde se  abordan cada uno de los componentes requeridos para su completo funcionamiento. Posteriormente se detalla la ingienría de software requerida en la etapa de conceptualización de la aplicación, se explican de forma detallada los aspectos teóricos y de implementación de la base de datos, el funcionamiento del protocolo de comunicación y por último los escenarios de juegos requeridos en las rutinas de entrenamiento ligero y clínico, y las estadísticas generadas por estos. Como herramienta de desarrollo se utilizó c\#.

% SYSTEM DESCRIPTION AND CHARACTERIZATION TO APPLY
\input{main/chapter2/section1/content.tex}
     
% SERIOUS GAME REQUIREMENTS
\input{main/chapter2/section2/content.tex}    

% USE CASE DEFINITION
\input{main/chapter2/section3/content.tex}

% USE CASE REALIZATION
\input{main/chapter2/section4/content.tex}

% DATABASE DESIGN 
\input{main/chapter2/section5/content.tex}

% DATA MANIPULATION
\input{main/chapter2/section6/content.tex}

% COMMUNICATION 
\input{main/chapter2/section7/content.tex}

% TRAINING SCENARIOS
\input{main/chapter2/section8/content.tex}

% EMG GRAPHIC
\input{main/chapter2/section9/content.tex}

% STATICAL REPORTS GENERATION
\input{main/chapter2/section10/content.tex}

\subthesischapter{Conclusiones del capítulo}
Se presentó una descripción del sistema de adquisición de datos para rehabilitación, sus componentes, características distintivas y su funcionamiento. Se identificaron y definieron los requisitos del juego  serio, tanto funcionales como no funcionales, así como los actores y casos de usos del sistema que establecieron las bases fundamentales para el desarrollo de la aplicación. Se realizó el diseño de la base de datos, abarcando tanto el modelo lógico como el físico, lo que aseguró una estructura robusta y eficiente para el almacenamiento de los datos. La manipulación de los datos se abordó de manera integral, desde la conexión con la base de datos hasta la persistencia de los resultados estadísticos. Se diseñó e implementó la comunicación con el pedal motorizado y la implementación de la interfaz gráfica para la representación de los datos EMG. Se definieron los escenarios de entrenamiento para las modalidades Ligero y Clínico, asegurando una cobertura completa de las necesidad de entrenamiento del usuario. Por último en el ámbito estadístico se desarrolló una serie de gráficos para el seguimiento de los resultados en las rutinas de entrenamiento.   
    
\end{thesischapter}    

% USE CASE DEFINITION
\begin{thesischapter}{2} {Diseño e implementación del Juego Serio}
En este capítulo se discuten los detalles de desarrollo de los aspectos citados en el capítulo anterior. Este comienza con una descripción y caracterización general del sistema, donde se  abordan cada uno de los componentes requeridos para su completo funcionamiento. Posteriormente se detalla la ingienría de software requerida en la etapa de conceptualización de la aplicación, se explican de forma detallada los aspectos teóricos y de implementación de la base de datos, el funcionamiento del protocolo de comunicación y por último los escenarios de juegos requeridos en las rutinas de entrenamiento ligero y clínico, y las estadísticas generadas por estos. Como herramienta de desarrollo se utilizó c\#.

% SYSTEM DESCRIPTION AND CHARACTERIZATION TO APPLY
\input{main/chapter2/section1/content.tex}
     
% SERIOUS GAME REQUIREMENTS
\input{main/chapter2/section2/content.tex}    

% USE CASE DEFINITION
\input{main/chapter2/section3/content.tex}

% USE CASE REALIZATION
\input{main/chapter2/section4/content.tex}

% DATABASE DESIGN 
\input{main/chapter2/section5/content.tex}

% DATA MANIPULATION
\input{main/chapter2/section6/content.tex}

% COMMUNICATION 
\input{main/chapter2/section7/content.tex}

% TRAINING SCENARIOS
\input{main/chapter2/section8/content.tex}

% EMG GRAPHIC
\input{main/chapter2/section9/content.tex}

% STATICAL REPORTS GENERATION
\input{main/chapter2/section10/content.tex}

\subthesischapter{Conclusiones del capítulo}
Se presentó una descripción del sistema de adquisición de datos para rehabilitación, sus componentes, características distintivas y su funcionamiento. Se identificaron y definieron los requisitos del juego  serio, tanto funcionales como no funcionales, así como los actores y casos de usos del sistema que establecieron las bases fundamentales para el desarrollo de la aplicación. Se realizó el diseño de la base de datos, abarcando tanto el modelo lógico como el físico, lo que aseguró una estructura robusta y eficiente para el almacenamiento de los datos. La manipulación de los datos se abordó de manera integral, desde la conexión con la base de datos hasta la persistencia de los resultados estadísticos. Se diseñó e implementó la comunicación con el pedal motorizado y la implementación de la interfaz gráfica para la representación de los datos EMG. Se definieron los escenarios de entrenamiento para las modalidades Ligero y Clínico, asegurando una cobertura completa de las necesidad de entrenamiento del usuario. Por último en el ámbito estadístico se desarrolló una serie de gráficos para el seguimiento de los resultados en las rutinas de entrenamiento.   
    
\end{thesischapter}

% USE CASE REALIZATION
\begin{thesischapter}{2} {Diseño e implementación del Juego Serio}
En este capítulo se discuten los detalles de desarrollo de los aspectos citados en el capítulo anterior. Este comienza con una descripción y caracterización general del sistema, donde se  abordan cada uno de los componentes requeridos para su completo funcionamiento. Posteriormente se detalla la ingienría de software requerida en la etapa de conceptualización de la aplicación, se explican de forma detallada los aspectos teóricos y de implementación de la base de datos, el funcionamiento del protocolo de comunicación y por último los escenarios de juegos requeridos en las rutinas de entrenamiento ligero y clínico, y las estadísticas generadas por estos. Como herramienta de desarrollo se utilizó c\#.

% SYSTEM DESCRIPTION AND CHARACTERIZATION TO APPLY
\input{main/chapter2/section1/content.tex}
     
% SERIOUS GAME REQUIREMENTS
\input{main/chapter2/section2/content.tex}    

% USE CASE DEFINITION
\input{main/chapter2/section3/content.tex}

% USE CASE REALIZATION
\input{main/chapter2/section4/content.tex}

% DATABASE DESIGN 
\input{main/chapter2/section5/content.tex}

% DATA MANIPULATION
\input{main/chapter2/section6/content.tex}

% COMMUNICATION 
\input{main/chapter2/section7/content.tex}

% TRAINING SCENARIOS
\input{main/chapter2/section8/content.tex}

% EMG GRAPHIC
\input{main/chapter2/section9/content.tex}

% STATICAL REPORTS GENERATION
\input{main/chapter2/section10/content.tex}

\subthesischapter{Conclusiones del capítulo}
Se presentó una descripción del sistema de adquisición de datos para rehabilitación, sus componentes, características distintivas y su funcionamiento. Se identificaron y definieron los requisitos del juego  serio, tanto funcionales como no funcionales, así como los actores y casos de usos del sistema que establecieron las bases fundamentales para el desarrollo de la aplicación. Se realizó el diseño de la base de datos, abarcando tanto el modelo lógico como el físico, lo que aseguró una estructura robusta y eficiente para el almacenamiento de los datos. La manipulación de los datos se abordó de manera integral, desde la conexión con la base de datos hasta la persistencia de los resultados estadísticos. Se diseñó e implementó la comunicación con el pedal motorizado y la implementación de la interfaz gráfica para la representación de los datos EMG. Se definieron los escenarios de entrenamiento para las modalidades Ligero y Clínico, asegurando una cobertura completa de las necesidad de entrenamiento del usuario. Por último en el ámbito estadístico se desarrolló una serie de gráficos para el seguimiento de los resultados en las rutinas de entrenamiento.   
    
\end{thesischapter}

% DATABASE DESIGN 
\begin{thesischapter}{2} {Diseño e implementación del Juego Serio}
En este capítulo se discuten los detalles de desarrollo de los aspectos citados en el capítulo anterior. Este comienza con una descripción y caracterización general del sistema, donde se  abordan cada uno de los componentes requeridos para su completo funcionamiento. Posteriormente se detalla la ingienría de software requerida en la etapa de conceptualización de la aplicación, se explican de forma detallada los aspectos teóricos y de implementación de la base de datos, el funcionamiento del protocolo de comunicación y por último los escenarios de juegos requeridos en las rutinas de entrenamiento ligero y clínico, y las estadísticas generadas por estos. Como herramienta de desarrollo se utilizó c\#.

% SYSTEM DESCRIPTION AND CHARACTERIZATION TO APPLY
\input{main/chapter2/section1/content.tex}
     
% SERIOUS GAME REQUIREMENTS
\input{main/chapter2/section2/content.tex}    

% USE CASE DEFINITION
\input{main/chapter2/section3/content.tex}

% USE CASE REALIZATION
\input{main/chapter2/section4/content.tex}

% DATABASE DESIGN 
\input{main/chapter2/section5/content.tex}

% DATA MANIPULATION
\input{main/chapter2/section6/content.tex}

% COMMUNICATION 
\input{main/chapter2/section7/content.tex}

% TRAINING SCENARIOS
\input{main/chapter2/section8/content.tex}

% EMG GRAPHIC
\input{main/chapter2/section9/content.tex}

% STATICAL REPORTS GENERATION
\input{main/chapter2/section10/content.tex}

\subthesischapter{Conclusiones del capítulo}
Se presentó una descripción del sistema de adquisición de datos para rehabilitación, sus componentes, características distintivas y su funcionamiento. Se identificaron y definieron los requisitos del juego  serio, tanto funcionales como no funcionales, así como los actores y casos de usos del sistema que establecieron las bases fundamentales para el desarrollo de la aplicación. Se realizó el diseño de la base de datos, abarcando tanto el modelo lógico como el físico, lo que aseguró una estructura robusta y eficiente para el almacenamiento de los datos. La manipulación de los datos se abordó de manera integral, desde la conexión con la base de datos hasta la persistencia de los resultados estadísticos. Se diseñó e implementó la comunicación con el pedal motorizado y la implementación de la interfaz gráfica para la representación de los datos EMG. Se definieron los escenarios de entrenamiento para las modalidades Ligero y Clínico, asegurando una cobertura completa de las necesidad de entrenamiento del usuario. Por último en el ámbito estadístico se desarrolló una serie de gráficos para el seguimiento de los resultados en las rutinas de entrenamiento.   
    
\end{thesischapter}

% DATA MANIPULATION
\begin{thesischapter}{2} {Diseño e implementación del Juego Serio}
En este capítulo se discuten los detalles de desarrollo de los aspectos citados en el capítulo anterior. Este comienza con una descripción y caracterización general del sistema, donde se  abordan cada uno de los componentes requeridos para su completo funcionamiento. Posteriormente se detalla la ingienría de software requerida en la etapa de conceptualización de la aplicación, se explican de forma detallada los aspectos teóricos y de implementación de la base de datos, el funcionamiento del protocolo de comunicación y por último los escenarios de juegos requeridos en las rutinas de entrenamiento ligero y clínico, y las estadísticas generadas por estos. Como herramienta de desarrollo se utilizó c\#.

% SYSTEM DESCRIPTION AND CHARACTERIZATION TO APPLY
\input{main/chapter2/section1/content.tex}
     
% SERIOUS GAME REQUIREMENTS
\input{main/chapter2/section2/content.tex}    

% USE CASE DEFINITION
\input{main/chapter2/section3/content.tex}

% USE CASE REALIZATION
\input{main/chapter2/section4/content.tex}

% DATABASE DESIGN 
\input{main/chapter2/section5/content.tex}

% DATA MANIPULATION
\input{main/chapter2/section6/content.tex}

% COMMUNICATION 
\input{main/chapter2/section7/content.tex}

% TRAINING SCENARIOS
\input{main/chapter2/section8/content.tex}

% EMG GRAPHIC
\input{main/chapter2/section9/content.tex}

% STATICAL REPORTS GENERATION
\input{main/chapter2/section10/content.tex}

\subthesischapter{Conclusiones del capítulo}
Se presentó una descripción del sistema de adquisición de datos para rehabilitación, sus componentes, características distintivas y su funcionamiento. Se identificaron y definieron los requisitos del juego  serio, tanto funcionales como no funcionales, así como los actores y casos de usos del sistema que establecieron las bases fundamentales para el desarrollo de la aplicación. Se realizó el diseño de la base de datos, abarcando tanto el modelo lógico como el físico, lo que aseguró una estructura robusta y eficiente para el almacenamiento de los datos. La manipulación de los datos se abordó de manera integral, desde la conexión con la base de datos hasta la persistencia de los resultados estadísticos. Se diseñó e implementó la comunicación con el pedal motorizado y la implementación de la interfaz gráfica para la representación de los datos EMG. Se definieron los escenarios de entrenamiento para las modalidades Ligero y Clínico, asegurando una cobertura completa de las necesidad de entrenamiento del usuario. Por último en el ámbito estadístico se desarrolló una serie de gráficos para el seguimiento de los resultados en las rutinas de entrenamiento.   
    
\end{thesischapter}

% COMMUNICATION 
\begin{thesischapter}{2} {Diseño e implementación del Juego Serio}
En este capítulo se discuten los detalles de desarrollo de los aspectos citados en el capítulo anterior. Este comienza con una descripción y caracterización general del sistema, donde se  abordan cada uno de los componentes requeridos para su completo funcionamiento. Posteriormente se detalla la ingienría de software requerida en la etapa de conceptualización de la aplicación, se explican de forma detallada los aspectos teóricos y de implementación de la base de datos, el funcionamiento del protocolo de comunicación y por último los escenarios de juegos requeridos en las rutinas de entrenamiento ligero y clínico, y las estadísticas generadas por estos. Como herramienta de desarrollo se utilizó c\#.

% SYSTEM DESCRIPTION AND CHARACTERIZATION TO APPLY
\input{main/chapter2/section1/content.tex}
     
% SERIOUS GAME REQUIREMENTS
\input{main/chapter2/section2/content.tex}    

% USE CASE DEFINITION
\input{main/chapter2/section3/content.tex}

% USE CASE REALIZATION
\input{main/chapter2/section4/content.tex}

% DATABASE DESIGN 
\input{main/chapter2/section5/content.tex}

% DATA MANIPULATION
\input{main/chapter2/section6/content.tex}

% COMMUNICATION 
\input{main/chapter2/section7/content.tex}

% TRAINING SCENARIOS
\input{main/chapter2/section8/content.tex}

% EMG GRAPHIC
\input{main/chapter2/section9/content.tex}

% STATICAL REPORTS GENERATION
\input{main/chapter2/section10/content.tex}

\subthesischapter{Conclusiones del capítulo}
Se presentó una descripción del sistema de adquisición de datos para rehabilitación, sus componentes, características distintivas y su funcionamiento. Se identificaron y definieron los requisitos del juego  serio, tanto funcionales como no funcionales, así como los actores y casos de usos del sistema que establecieron las bases fundamentales para el desarrollo de la aplicación. Se realizó el diseño de la base de datos, abarcando tanto el modelo lógico como el físico, lo que aseguró una estructura robusta y eficiente para el almacenamiento de los datos. La manipulación de los datos se abordó de manera integral, desde la conexión con la base de datos hasta la persistencia de los resultados estadísticos. Se diseñó e implementó la comunicación con el pedal motorizado y la implementación de la interfaz gráfica para la representación de los datos EMG. Se definieron los escenarios de entrenamiento para las modalidades Ligero y Clínico, asegurando una cobertura completa de las necesidad de entrenamiento del usuario. Por último en el ámbito estadístico se desarrolló una serie de gráficos para el seguimiento de los resultados en las rutinas de entrenamiento.   
    
\end{thesischapter}

% TRAINING SCENARIOS
\begin{thesischapter}{2} {Diseño e implementación del Juego Serio}
En este capítulo se discuten los detalles de desarrollo de los aspectos citados en el capítulo anterior. Este comienza con una descripción y caracterización general del sistema, donde se  abordan cada uno de los componentes requeridos para su completo funcionamiento. Posteriormente se detalla la ingienría de software requerida en la etapa de conceptualización de la aplicación, se explican de forma detallada los aspectos teóricos y de implementación de la base de datos, el funcionamiento del protocolo de comunicación y por último los escenarios de juegos requeridos en las rutinas de entrenamiento ligero y clínico, y las estadísticas generadas por estos. Como herramienta de desarrollo se utilizó c\#.

% SYSTEM DESCRIPTION AND CHARACTERIZATION TO APPLY
\input{main/chapter2/section1/content.tex}
     
% SERIOUS GAME REQUIREMENTS
\input{main/chapter2/section2/content.tex}    

% USE CASE DEFINITION
\input{main/chapter2/section3/content.tex}

% USE CASE REALIZATION
\input{main/chapter2/section4/content.tex}

% DATABASE DESIGN 
\input{main/chapter2/section5/content.tex}

% DATA MANIPULATION
\input{main/chapter2/section6/content.tex}

% COMMUNICATION 
\input{main/chapter2/section7/content.tex}

% TRAINING SCENARIOS
\input{main/chapter2/section8/content.tex}

% EMG GRAPHIC
\input{main/chapter2/section9/content.tex}

% STATICAL REPORTS GENERATION
\input{main/chapter2/section10/content.tex}

\subthesischapter{Conclusiones del capítulo}
Se presentó una descripción del sistema de adquisición de datos para rehabilitación, sus componentes, características distintivas y su funcionamiento. Se identificaron y definieron los requisitos del juego  serio, tanto funcionales como no funcionales, así como los actores y casos de usos del sistema que establecieron las bases fundamentales para el desarrollo de la aplicación. Se realizó el diseño de la base de datos, abarcando tanto el modelo lógico como el físico, lo que aseguró una estructura robusta y eficiente para el almacenamiento de los datos. La manipulación de los datos se abordó de manera integral, desde la conexión con la base de datos hasta la persistencia de los resultados estadísticos. Se diseñó e implementó la comunicación con el pedal motorizado y la implementación de la interfaz gráfica para la representación de los datos EMG. Se definieron los escenarios de entrenamiento para las modalidades Ligero y Clínico, asegurando una cobertura completa de las necesidad de entrenamiento del usuario. Por último en el ámbito estadístico se desarrolló una serie de gráficos para el seguimiento de los resultados en las rutinas de entrenamiento.   
    
\end{thesischapter}

% EMG GRAPHIC
\begin{thesischapter}{2} {Diseño e implementación del Juego Serio}
En este capítulo se discuten los detalles de desarrollo de los aspectos citados en el capítulo anterior. Este comienza con una descripción y caracterización general del sistema, donde se  abordan cada uno de los componentes requeridos para su completo funcionamiento. Posteriormente se detalla la ingienría de software requerida en la etapa de conceptualización de la aplicación, se explican de forma detallada los aspectos teóricos y de implementación de la base de datos, el funcionamiento del protocolo de comunicación y por último los escenarios de juegos requeridos en las rutinas de entrenamiento ligero y clínico, y las estadísticas generadas por estos. Como herramienta de desarrollo se utilizó c\#.

% SYSTEM DESCRIPTION AND CHARACTERIZATION TO APPLY
\input{main/chapter2/section1/content.tex}
     
% SERIOUS GAME REQUIREMENTS
\input{main/chapter2/section2/content.tex}    

% USE CASE DEFINITION
\input{main/chapter2/section3/content.tex}

% USE CASE REALIZATION
\input{main/chapter2/section4/content.tex}

% DATABASE DESIGN 
\input{main/chapter2/section5/content.tex}

% DATA MANIPULATION
\input{main/chapter2/section6/content.tex}

% COMMUNICATION 
\input{main/chapter2/section7/content.tex}

% TRAINING SCENARIOS
\input{main/chapter2/section8/content.tex}

% EMG GRAPHIC
\input{main/chapter2/section9/content.tex}

% STATICAL REPORTS GENERATION
\input{main/chapter2/section10/content.tex}

\subthesischapter{Conclusiones del capítulo}
Se presentó una descripción del sistema de adquisición de datos para rehabilitación, sus componentes, características distintivas y su funcionamiento. Se identificaron y definieron los requisitos del juego  serio, tanto funcionales como no funcionales, así como los actores y casos de usos del sistema que establecieron las bases fundamentales para el desarrollo de la aplicación. Se realizó el diseño de la base de datos, abarcando tanto el modelo lógico como el físico, lo que aseguró una estructura robusta y eficiente para el almacenamiento de los datos. La manipulación de los datos se abordó de manera integral, desde la conexión con la base de datos hasta la persistencia de los resultados estadísticos. Se diseñó e implementó la comunicación con el pedal motorizado y la implementación de la interfaz gráfica para la representación de los datos EMG. Se definieron los escenarios de entrenamiento para las modalidades Ligero y Clínico, asegurando una cobertura completa de las necesidad de entrenamiento del usuario. Por último en el ámbito estadístico se desarrolló una serie de gráficos para el seguimiento de los resultados en las rutinas de entrenamiento.   
    
\end{thesischapter}

% STATICAL REPORTS GENERATION
\begin{thesischapter}{2} {Diseño e implementación del Juego Serio}
En este capítulo se discuten los detalles de desarrollo de los aspectos citados en el capítulo anterior. Este comienza con una descripción y caracterización general del sistema, donde se  abordan cada uno de los componentes requeridos para su completo funcionamiento. Posteriormente se detalla la ingienría de software requerida en la etapa de conceptualización de la aplicación, se explican de forma detallada los aspectos teóricos y de implementación de la base de datos, el funcionamiento del protocolo de comunicación y por último los escenarios de juegos requeridos en las rutinas de entrenamiento ligero y clínico, y las estadísticas generadas por estos. Como herramienta de desarrollo se utilizó c\#.

% SYSTEM DESCRIPTION AND CHARACTERIZATION TO APPLY
\input{main/chapter2/section1/content.tex}
     
% SERIOUS GAME REQUIREMENTS
\input{main/chapter2/section2/content.tex}    

% USE CASE DEFINITION
\input{main/chapter2/section3/content.tex}

% USE CASE REALIZATION
\input{main/chapter2/section4/content.tex}

% DATABASE DESIGN 
\input{main/chapter2/section5/content.tex}

% DATA MANIPULATION
\input{main/chapter2/section6/content.tex}

% COMMUNICATION 
\input{main/chapter2/section7/content.tex}

% TRAINING SCENARIOS
\input{main/chapter2/section8/content.tex}

% EMG GRAPHIC
\input{main/chapter2/section9/content.tex}

% STATICAL REPORTS GENERATION
\input{main/chapter2/section10/content.tex}

\subthesischapter{Conclusiones del capítulo}
Se presentó una descripción del sistema de adquisición de datos para rehabilitación, sus componentes, características distintivas y su funcionamiento. Se identificaron y definieron los requisitos del juego  serio, tanto funcionales como no funcionales, así como los actores y casos de usos del sistema que establecieron las bases fundamentales para el desarrollo de la aplicación. Se realizó el diseño de la base de datos, abarcando tanto el modelo lógico como el físico, lo que aseguró una estructura robusta y eficiente para el almacenamiento de los datos. La manipulación de los datos se abordó de manera integral, desde la conexión con la base de datos hasta la persistencia de los resultados estadísticos. Se diseñó e implementó la comunicación con el pedal motorizado y la implementación de la interfaz gráfica para la representación de los datos EMG. Se definieron los escenarios de entrenamiento para las modalidades Ligero y Clínico, asegurando una cobertura completa de las necesidad de entrenamiento del usuario. Por último en el ámbito estadístico se desarrolló una serie de gráficos para el seguimiento de los resultados en las rutinas de entrenamiento.   
    
\end{thesischapter}

\subthesischapter{Conclusiones del capítulo}
Se presentó una descripción del sistema de adquisición de datos para rehabilitación, sus componentes, características distintivas y su funcionamiento. Se identificaron y definieron los requisitos del juego  serio, tanto funcionales como no funcionales, así como los actores y casos de usos del sistema que establecieron las bases fundamentales para el desarrollo de la aplicación. Se realizó el diseño de la base de datos, abarcando tanto el modelo lógico como el físico, lo que aseguró una estructura robusta y eficiente para el almacenamiento de los datos. La manipulación de los datos se abordó de manera integral, desde la conexión con la base de datos hasta la persistencia de los resultados estadísticos. Se diseñó e implementó la comunicación con el pedal motorizado y la implementación de la interfaz gráfica para la representación de los datos EMG. Se definieron los escenarios de entrenamiento para las modalidades Ligero y Clínico, asegurando una cobertura completa de las necesidad de entrenamiento del usuario. Por último en el ámbito estadístico se desarrolló una serie de gráficos para el seguimiento de los resultados en las rutinas de entrenamiento.   
    
\end{thesischapter}

% COMMUNICATION 
\begin{thesischapter}{2} {Diseño e implementación del Juego Serio}
En este capítulo se discuten los detalles de desarrollo de los aspectos citados en el capítulo anterior. Este comienza con una descripción y caracterización general del sistema, donde se  abordan cada uno de los componentes requeridos para su completo funcionamiento. Posteriormente se detalla la ingienría de software requerida en la etapa de conceptualización de la aplicación, se explican de forma detallada los aspectos teóricos y de implementación de la base de datos, el funcionamiento del protocolo de comunicación y por último los escenarios de juegos requeridos en las rutinas de entrenamiento ligero y clínico, y las estadísticas generadas por estos. Como herramienta de desarrollo se utilizó c\#.

% SYSTEM DESCRIPTION AND CHARACTERIZATION TO APPLY
\begin{thesischapter}{2} {Diseño e implementación del Juego Serio}
En este capítulo se discuten los detalles de desarrollo de los aspectos citados en el capítulo anterior. Este comienza con una descripción y caracterización general del sistema, donde se  abordan cada uno de los componentes requeridos para su completo funcionamiento. Posteriormente se detalla la ingienría de software requerida en la etapa de conceptualización de la aplicación, se explican de forma detallada los aspectos teóricos y de implementación de la base de datos, el funcionamiento del protocolo de comunicación y por último los escenarios de juegos requeridos en las rutinas de entrenamiento ligero y clínico, y las estadísticas generadas por estos. Como herramienta de desarrollo se utilizó c\#.

% SYSTEM DESCRIPTION AND CHARACTERIZATION TO APPLY
\input{main/chapter2/section1/content.tex}
     
% SERIOUS GAME REQUIREMENTS
\input{main/chapter2/section2/content.tex}    

% USE CASE DEFINITION
\input{main/chapter2/section3/content.tex}

% USE CASE REALIZATION
\input{main/chapter2/section4/content.tex}

% DATABASE DESIGN 
\input{main/chapter2/section5/content.tex}

% DATA MANIPULATION
\input{main/chapter2/section6/content.tex}

% COMMUNICATION 
\input{main/chapter2/section7/content.tex}

% TRAINING SCENARIOS
\input{main/chapter2/section8/content.tex}

% EMG GRAPHIC
\input{main/chapter2/section9/content.tex}

% STATICAL REPORTS GENERATION
\input{main/chapter2/section10/content.tex}

\subthesischapter{Conclusiones del capítulo}
Se presentó una descripción del sistema de adquisición de datos para rehabilitación, sus componentes, características distintivas y su funcionamiento. Se identificaron y definieron los requisitos del juego  serio, tanto funcionales como no funcionales, así como los actores y casos de usos del sistema que establecieron las bases fundamentales para el desarrollo de la aplicación. Se realizó el diseño de la base de datos, abarcando tanto el modelo lógico como el físico, lo que aseguró una estructura robusta y eficiente para el almacenamiento de los datos. La manipulación de los datos se abordó de manera integral, desde la conexión con la base de datos hasta la persistencia de los resultados estadísticos. Se diseñó e implementó la comunicación con el pedal motorizado y la implementación de la interfaz gráfica para la representación de los datos EMG. Se definieron los escenarios de entrenamiento para las modalidades Ligero y Clínico, asegurando una cobertura completa de las necesidad de entrenamiento del usuario. Por último en el ámbito estadístico se desarrolló una serie de gráficos para el seguimiento de los resultados en las rutinas de entrenamiento.   
    
\end{thesischapter}
     
% SERIOUS GAME REQUIREMENTS
\begin{thesischapter}{2} {Diseño e implementación del Juego Serio}
En este capítulo se discuten los detalles de desarrollo de los aspectos citados en el capítulo anterior. Este comienza con una descripción y caracterización general del sistema, donde se  abordan cada uno de los componentes requeridos para su completo funcionamiento. Posteriormente se detalla la ingienría de software requerida en la etapa de conceptualización de la aplicación, se explican de forma detallada los aspectos teóricos y de implementación de la base de datos, el funcionamiento del protocolo de comunicación y por último los escenarios de juegos requeridos en las rutinas de entrenamiento ligero y clínico, y las estadísticas generadas por estos. Como herramienta de desarrollo se utilizó c\#.

% SYSTEM DESCRIPTION AND CHARACTERIZATION TO APPLY
\input{main/chapter2/section1/content.tex}
     
% SERIOUS GAME REQUIREMENTS
\input{main/chapter2/section2/content.tex}    

% USE CASE DEFINITION
\input{main/chapter2/section3/content.tex}

% USE CASE REALIZATION
\input{main/chapter2/section4/content.tex}

% DATABASE DESIGN 
\input{main/chapter2/section5/content.tex}

% DATA MANIPULATION
\input{main/chapter2/section6/content.tex}

% COMMUNICATION 
\input{main/chapter2/section7/content.tex}

% TRAINING SCENARIOS
\input{main/chapter2/section8/content.tex}

% EMG GRAPHIC
\input{main/chapter2/section9/content.tex}

% STATICAL REPORTS GENERATION
\input{main/chapter2/section10/content.tex}

\subthesischapter{Conclusiones del capítulo}
Se presentó una descripción del sistema de adquisición de datos para rehabilitación, sus componentes, características distintivas y su funcionamiento. Se identificaron y definieron los requisitos del juego  serio, tanto funcionales como no funcionales, así como los actores y casos de usos del sistema que establecieron las bases fundamentales para el desarrollo de la aplicación. Se realizó el diseño de la base de datos, abarcando tanto el modelo lógico como el físico, lo que aseguró una estructura robusta y eficiente para el almacenamiento de los datos. La manipulación de los datos se abordó de manera integral, desde la conexión con la base de datos hasta la persistencia de los resultados estadísticos. Se diseñó e implementó la comunicación con el pedal motorizado y la implementación de la interfaz gráfica para la representación de los datos EMG. Se definieron los escenarios de entrenamiento para las modalidades Ligero y Clínico, asegurando una cobertura completa de las necesidad de entrenamiento del usuario. Por último en el ámbito estadístico se desarrolló una serie de gráficos para el seguimiento de los resultados en las rutinas de entrenamiento.   
    
\end{thesischapter}    

% USE CASE DEFINITION
\begin{thesischapter}{2} {Diseño e implementación del Juego Serio}
En este capítulo se discuten los detalles de desarrollo de los aspectos citados en el capítulo anterior. Este comienza con una descripción y caracterización general del sistema, donde se  abordan cada uno de los componentes requeridos para su completo funcionamiento. Posteriormente se detalla la ingienría de software requerida en la etapa de conceptualización de la aplicación, se explican de forma detallada los aspectos teóricos y de implementación de la base de datos, el funcionamiento del protocolo de comunicación y por último los escenarios de juegos requeridos en las rutinas de entrenamiento ligero y clínico, y las estadísticas generadas por estos. Como herramienta de desarrollo se utilizó c\#.

% SYSTEM DESCRIPTION AND CHARACTERIZATION TO APPLY
\input{main/chapter2/section1/content.tex}
     
% SERIOUS GAME REQUIREMENTS
\input{main/chapter2/section2/content.tex}    

% USE CASE DEFINITION
\input{main/chapter2/section3/content.tex}

% USE CASE REALIZATION
\input{main/chapter2/section4/content.tex}

% DATABASE DESIGN 
\input{main/chapter2/section5/content.tex}

% DATA MANIPULATION
\input{main/chapter2/section6/content.tex}

% COMMUNICATION 
\input{main/chapter2/section7/content.tex}

% TRAINING SCENARIOS
\input{main/chapter2/section8/content.tex}

% EMG GRAPHIC
\input{main/chapter2/section9/content.tex}

% STATICAL REPORTS GENERATION
\input{main/chapter2/section10/content.tex}

\subthesischapter{Conclusiones del capítulo}
Se presentó una descripción del sistema de adquisición de datos para rehabilitación, sus componentes, características distintivas y su funcionamiento. Se identificaron y definieron los requisitos del juego  serio, tanto funcionales como no funcionales, así como los actores y casos de usos del sistema que establecieron las bases fundamentales para el desarrollo de la aplicación. Se realizó el diseño de la base de datos, abarcando tanto el modelo lógico como el físico, lo que aseguró una estructura robusta y eficiente para el almacenamiento de los datos. La manipulación de los datos se abordó de manera integral, desde la conexión con la base de datos hasta la persistencia de los resultados estadísticos. Se diseñó e implementó la comunicación con el pedal motorizado y la implementación de la interfaz gráfica para la representación de los datos EMG. Se definieron los escenarios de entrenamiento para las modalidades Ligero y Clínico, asegurando una cobertura completa de las necesidad de entrenamiento del usuario. Por último en el ámbito estadístico se desarrolló una serie de gráficos para el seguimiento de los resultados en las rutinas de entrenamiento.   
    
\end{thesischapter}

% USE CASE REALIZATION
\begin{thesischapter}{2} {Diseño e implementación del Juego Serio}
En este capítulo se discuten los detalles de desarrollo de los aspectos citados en el capítulo anterior. Este comienza con una descripción y caracterización general del sistema, donde se  abordan cada uno de los componentes requeridos para su completo funcionamiento. Posteriormente se detalla la ingienría de software requerida en la etapa de conceptualización de la aplicación, se explican de forma detallada los aspectos teóricos y de implementación de la base de datos, el funcionamiento del protocolo de comunicación y por último los escenarios de juegos requeridos en las rutinas de entrenamiento ligero y clínico, y las estadísticas generadas por estos. Como herramienta de desarrollo se utilizó c\#.

% SYSTEM DESCRIPTION AND CHARACTERIZATION TO APPLY
\input{main/chapter2/section1/content.tex}
     
% SERIOUS GAME REQUIREMENTS
\input{main/chapter2/section2/content.tex}    

% USE CASE DEFINITION
\input{main/chapter2/section3/content.tex}

% USE CASE REALIZATION
\input{main/chapter2/section4/content.tex}

% DATABASE DESIGN 
\input{main/chapter2/section5/content.tex}

% DATA MANIPULATION
\input{main/chapter2/section6/content.tex}

% COMMUNICATION 
\input{main/chapter2/section7/content.tex}

% TRAINING SCENARIOS
\input{main/chapter2/section8/content.tex}

% EMG GRAPHIC
\input{main/chapter2/section9/content.tex}

% STATICAL REPORTS GENERATION
\input{main/chapter2/section10/content.tex}

\subthesischapter{Conclusiones del capítulo}
Se presentó una descripción del sistema de adquisición de datos para rehabilitación, sus componentes, características distintivas y su funcionamiento. Se identificaron y definieron los requisitos del juego  serio, tanto funcionales como no funcionales, así como los actores y casos de usos del sistema que establecieron las bases fundamentales para el desarrollo de la aplicación. Se realizó el diseño de la base de datos, abarcando tanto el modelo lógico como el físico, lo que aseguró una estructura robusta y eficiente para el almacenamiento de los datos. La manipulación de los datos se abordó de manera integral, desde la conexión con la base de datos hasta la persistencia de los resultados estadísticos. Se diseñó e implementó la comunicación con el pedal motorizado y la implementación de la interfaz gráfica para la representación de los datos EMG. Se definieron los escenarios de entrenamiento para las modalidades Ligero y Clínico, asegurando una cobertura completa de las necesidad de entrenamiento del usuario. Por último en el ámbito estadístico se desarrolló una serie de gráficos para el seguimiento de los resultados en las rutinas de entrenamiento.   
    
\end{thesischapter}

% DATABASE DESIGN 
\begin{thesischapter}{2} {Diseño e implementación del Juego Serio}
En este capítulo se discuten los detalles de desarrollo de los aspectos citados en el capítulo anterior. Este comienza con una descripción y caracterización general del sistema, donde se  abordan cada uno de los componentes requeridos para su completo funcionamiento. Posteriormente se detalla la ingienría de software requerida en la etapa de conceptualización de la aplicación, se explican de forma detallada los aspectos teóricos y de implementación de la base de datos, el funcionamiento del protocolo de comunicación y por último los escenarios de juegos requeridos en las rutinas de entrenamiento ligero y clínico, y las estadísticas generadas por estos. Como herramienta de desarrollo se utilizó c\#.

% SYSTEM DESCRIPTION AND CHARACTERIZATION TO APPLY
\input{main/chapter2/section1/content.tex}
     
% SERIOUS GAME REQUIREMENTS
\input{main/chapter2/section2/content.tex}    

% USE CASE DEFINITION
\input{main/chapter2/section3/content.tex}

% USE CASE REALIZATION
\input{main/chapter2/section4/content.tex}

% DATABASE DESIGN 
\input{main/chapter2/section5/content.tex}

% DATA MANIPULATION
\input{main/chapter2/section6/content.tex}

% COMMUNICATION 
\input{main/chapter2/section7/content.tex}

% TRAINING SCENARIOS
\input{main/chapter2/section8/content.tex}

% EMG GRAPHIC
\input{main/chapter2/section9/content.tex}

% STATICAL REPORTS GENERATION
\input{main/chapter2/section10/content.tex}

\subthesischapter{Conclusiones del capítulo}
Se presentó una descripción del sistema de adquisición de datos para rehabilitación, sus componentes, características distintivas y su funcionamiento. Se identificaron y definieron los requisitos del juego  serio, tanto funcionales como no funcionales, así como los actores y casos de usos del sistema que establecieron las bases fundamentales para el desarrollo de la aplicación. Se realizó el diseño de la base de datos, abarcando tanto el modelo lógico como el físico, lo que aseguró una estructura robusta y eficiente para el almacenamiento de los datos. La manipulación de los datos se abordó de manera integral, desde la conexión con la base de datos hasta la persistencia de los resultados estadísticos. Se diseñó e implementó la comunicación con el pedal motorizado y la implementación de la interfaz gráfica para la representación de los datos EMG. Se definieron los escenarios de entrenamiento para las modalidades Ligero y Clínico, asegurando una cobertura completa de las necesidad de entrenamiento del usuario. Por último en el ámbito estadístico se desarrolló una serie de gráficos para el seguimiento de los resultados en las rutinas de entrenamiento.   
    
\end{thesischapter}

% DATA MANIPULATION
\begin{thesischapter}{2} {Diseño e implementación del Juego Serio}
En este capítulo se discuten los detalles de desarrollo de los aspectos citados en el capítulo anterior. Este comienza con una descripción y caracterización general del sistema, donde se  abordan cada uno de los componentes requeridos para su completo funcionamiento. Posteriormente se detalla la ingienría de software requerida en la etapa de conceptualización de la aplicación, se explican de forma detallada los aspectos teóricos y de implementación de la base de datos, el funcionamiento del protocolo de comunicación y por último los escenarios de juegos requeridos en las rutinas de entrenamiento ligero y clínico, y las estadísticas generadas por estos. Como herramienta de desarrollo se utilizó c\#.

% SYSTEM DESCRIPTION AND CHARACTERIZATION TO APPLY
\input{main/chapter2/section1/content.tex}
     
% SERIOUS GAME REQUIREMENTS
\input{main/chapter2/section2/content.tex}    

% USE CASE DEFINITION
\input{main/chapter2/section3/content.tex}

% USE CASE REALIZATION
\input{main/chapter2/section4/content.tex}

% DATABASE DESIGN 
\input{main/chapter2/section5/content.tex}

% DATA MANIPULATION
\input{main/chapter2/section6/content.tex}

% COMMUNICATION 
\input{main/chapter2/section7/content.tex}

% TRAINING SCENARIOS
\input{main/chapter2/section8/content.tex}

% EMG GRAPHIC
\input{main/chapter2/section9/content.tex}

% STATICAL REPORTS GENERATION
\input{main/chapter2/section10/content.tex}

\subthesischapter{Conclusiones del capítulo}
Se presentó una descripción del sistema de adquisición de datos para rehabilitación, sus componentes, características distintivas y su funcionamiento. Se identificaron y definieron los requisitos del juego  serio, tanto funcionales como no funcionales, así como los actores y casos de usos del sistema que establecieron las bases fundamentales para el desarrollo de la aplicación. Se realizó el diseño de la base de datos, abarcando tanto el modelo lógico como el físico, lo que aseguró una estructura robusta y eficiente para el almacenamiento de los datos. La manipulación de los datos se abordó de manera integral, desde la conexión con la base de datos hasta la persistencia de los resultados estadísticos. Se diseñó e implementó la comunicación con el pedal motorizado y la implementación de la interfaz gráfica para la representación de los datos EMG. Se definieron los escenarios de entrenamiento para las modalidades Ligero y Clínico, asegurando una cobertura completa de las necesidad de entrenamiento del usuario. Por último en el ámbito estadístico se desarrolló una serie de gráficos para el seguimiento de los resultados en las rutinas de entrenamiento.   
    
\end{thesischapter}

% COMMUNICATION 
\begin{thesischapter}{2} {Diseño e implementación del Juego Serio}
En este capítulo se discuten los detalles de desarrollo de los aspectos citados en el capítulo anterior. Este comienza con una descripción y caracterización general del sistema, donde se  abordan cada uno de los componentes requeridos para su completo funcionamiento. Posteriormente se detalla la ingienría de software requerida en la etapa de conceptualización de la aplicación, se explican de forma detallada los aspectos teóricos y de implementación de la base de datos, el funcionamiento del protocolo de comunicación y por último los escenarios de juegos requeridos en las rutinas de entrenamiento ligero y clínico, y las estadísticas generadas por estos. Como herramienta de desarrollo se utilizó c\#.

% SYSTEM DESCRIPTION AND CHARACTERIZATION TO APPLY
\input{main/chapter2/section1/content.tex}
     
% SERIOUS GAME REQUIREMENTS
\input{main/chapter2/section2/content.tex}    

% USE CASE DEFINITION
\input{main/chapter2/section3/content.tex}

% USE CASE REALIZATION
\input{main/chapter2/section4/content.tex}

% DATABASE DESIGN 
\input{main/chapter2/section5/content.tex}

% DATA MANIPULATION
\input{main/chapter2/section6/content.tex}

% COMMUNICATION 
\input{main/chapter2/section7/content.tex}

% TRAINING SCENARIOS
\input{main/chapter2/section8/content.tex}

% EMG GRAPHIC
\input{main/chapter2/section9/content.tex}

% STATICAL REPORTS GENERATION
\input{main/chapter2/section10/content.tex}

\subthesischapter{Conclusiones del capítulo}
Se presentó una descripción del sistema de adquisición de datos para rehabilitación, sus componentes, características distintivas y su funcionamiento. Se identificaron y definieron los requisitos del juego  serio, tanto funcionales como no funcionales, así como los actores y casos de usos del sistema que establecieron las bases fundamentales para el desarrollo de la aplicación. Se realizó el diseño de la base de datos, abarcando tanto el modelo lógico como el físico, lo que aseguró una estructura robusta y eficiente para el almacenamiento de los datos. La manipulación de los datos se abordó de manera integral, desde la conexión con la base de datos hasta la persistencia de los resultados estadísticos. Se diseñó e implementó la comunicación con el pedal motorizado y la implementación de la interfaz gráfica para la representación de los datos EMG. Se definieron los escenarios de entrenamiento para las modalidades Ligero y Clínico, asegurando una cobertura completa de las necesidad de entrenamiento del usuario. Por último en el ámbito estadístico se desarrolló una serie de gráficos para el seguimiento de los resultados en las rutinas de entrenamiento.   
    
\end{thesischapter}

% TRAINING SCENARIOS
\begin{thesischapter}{2} {Diseño e implementación del Juego Serio}
En este capítulo se discuten los detalles de desarrollo de los aspectos citados en el capítulo anterior. Este comienza con una descripción y caracterización general del sistema, donde se  abordan cada uno de los componentes requeridos para su completo funcionamiento. Posteriormente se detalla la ingienría de software requerida en la etapa de conceptualización de la aplicación, se explican de forma detallada los aspectos teóricos y de implementación de la base de datos, el funcionamiento del protocolo de comunicación y por último los escenarios de juegos requeridos en las rutinas de entrenamiento ligero y clínico, y las estadísticas generadas por estos. Como herramienta de desarrollo se utilizó c\#.

% SYSTEM DESCRIPTION AND CHARACTERIZATION TO APPLY
\input{main/chapter2/section1/content.tex}
     
% SERIOUS GAME REQUIREMENTS
\input{main/chapter2/section2/content.tex}    

% USE CASE DEFINITION
\input{main/chapter2/section3/content.tex}

% USE CASE REALIZATION
\input{main/chapter2/section4/content.tex}

% DATABASE DESIGN 
\input{main/chapter2/section5/content.tex}

% DATA MANIPULATION
\input{main/chapter2/section6/content.tex}

% COMMUNICATION 
\input{main/chapter2/section7/content.tex}

% TRAINING SCENARIOS
\input{main/chapter2/section8/content.tex}

% EMG GRAPHIC
\input{main/chapter2/section9/content.tex}

% STATICAL REPORTS GENERATION
\input{main/chapter2/section10/content.tex}

\subthesischapter{Conclusiones del capítulo}
Se presentó una descripción del sistema de adquisición de datos para rehabilitación, sus componentes, características distintivas y su funcionamiento. Se identificaron y definieron los requisitos del juego  serio, tanto funcionales como no funcionales, así como los actores y casos de usos del sistema que establecieron las bases fundamentales para el desarrollo de la aplicación. Se realizó el diseño de la base de datos, abarcando tanto el modelo lógico como el físico, lo que aseguró una estructura robusta y eficiente para el almacenamiento de los datos. La manipulación de los datos se abordó de manera integral, desde la conexión con la base de datos hasta la persistencia de los resultados estadísticos. Se diseñó e implementó la comunicación con el pedal motorizado y la implementación de la interfaz gráfica para la representación de los datos EMG. Se definieron los escenarios de entrenamiento para las modalidades Ligero y Clínico, asegurando una cobertura completa de las necesidad de entrenamiento del usuario. Por último en el ámbito estadístico se desarrolló una serie de gráficos para el seguimiento de los resultados en las rutinas de entrenamiento.   
    
\end{thesischapter}

% EMG GRAPHIC
\begin{thesischapter}{2} {Diseño e implementación del Juego Serio}
En este capítulo se discuten los detalles de desarrollo de los aspectos citados en el capítulo anterior. Este comienza con una descripción y caracterización general del sistema, donde se  abordan cada uno de los componentes requeridos para su completo funcionamiento. Posteriormente se detalla la ingienría de software requerida en la etapa de conceptualización de la aplicación, se explican de forma detallada los aspectos teóricos y de implementación de la base de datos, el funcionamiento del protocolo de comunicación y por último los escenarios de juegos requeridos en las rutinas de entrenamiento ligero y clínico, y las estadísticas generadas por estos. Como herramienta de desarrollo se utilizó c\#.

% SYSTEM DESCRIPTION AND CHARACTERIZATION TO APPLY
\input{main/chapter2/section1/content.tex}
     
% SERIOUS GAME REQUIREMENTS
\input{main/chapter2/section2/content.tex}    

% USE CASE DEFINITION
\input{main/chapter2/section3/content.tex}

% USE CASE REALIZATION
\input{main/chapter2/section4/content.tex}

% DATABASE DESIGN 
\input{main/chapter2/section5/content.tex}

% DATA MANIPULATION
\input{main/chapter2/section6/content.tex}

% COMMUNICATION 
\input{main/chapter2/section7/content.tex}

% TRAINING SCENARIOS
\input{main/chapter2/section8/content.tex}

% EMG GRAPHIC
\input{main/chapter2/section9/content.tex}

% STATICAL REPORTS GENERATION
\input{main/chapter2/section10/content.tex}

\subthesischapter{Conclusiones del capítulo}
Se presentó una descripción del sistema de adquisición de datos para rehabilitación, sus componentes, características distintivas y su funcionamiento. Se identificaron y definieron los requisitos del juego  serio, tanto funcionales como no funcionales, así como los actores y casos de usos del sistema que establecieron las bases fundamentales para el desarrollo de la aplicación. Se realizó el diseño de la base de datos, abarcando tanto el modelo lógico como el físico, lo que aseguró una estructura robusta y eficiente para el almacenamiento de los datos. La manipulación de los datos se abordó de manera integral, desde la conexión con la base de datos hasta la persistencia de los resultados estadísticos. Se diseñó e implementó la comunicación con el pedal motorizado y la implementación de la interfaz gráfica para la representación de los datos EMG. Se definieron los escenarios de entrenamiento para las modalidades Ligero y Clínico, asegurando una cobertura completa de las necesidad de entrenamiento del usuario. Por último en el ámbito estadístico se desarrolló una serie de gráficos para el seguimiento de los resultados en las rutinas de entrenamiento.   
    
\end{thesischapter}

% STATICAL REPORTS GENERATION
\begin{thesischapter}{2} {Diseño e implementación del Juego Serio}
En este capítulo se discuten los detalles de desarrollo de los aspectos citados en el capítulo anterior. Este comienza con una descripción y caracterización general del sistema, donde se  abordan cada uno de los componentes requeridos para su completo funcionamiento. Posteriormente se detalla la ingienría de software requerida en la etapa de conceptualización de la aplicación, se explican de forma detallada los aspectos teóricos y de implementación de la base de datos, el funcionamiento del protocolo de comunicación y por último los escenarios de juegos requeridos en las rutinas de entrenamiento ligero y clínico, y las estadísticas generadas por estos. Como herramienta de desarrollo se utilizó c\#.

% SYSTEM DESCRIPTION AND CHARACTERIZATION TO APPLY
\input{main/chapter2/section1/content.tex}
     
% SERIOUS GAME REQUIREMENTS
\input{main/chapter2/section2/content.tex}    

% USE CASE DEFINITION
\input{main/chapter2/section3/content.tex}

% USE CASE REALIZATION
\input{main/chapter2/section4/content.tex}

% DATABASE DESIGN 
\input{main/chapter2/section5/content.tex}

% DATA MANIPULATION
\input{main/chapter2/section6/content.tex}

% COMMUNICATION 
\input{main/chapter2/section7/content.tex}

% TRAINING SCENARIOS
\input{main/chapter2/section8/content.tex}

% EMG GRAPHIC
\input{main/chapter2/section9/content.tex}

% STATICAL REPORTS GENERATION
\input{main/chapter2/section10/content.tex}

\subthesischapter{Conclusiones del capítulo}
Se presentó una descripción del sistema de adquisición de datos para rehabilitación, sus componentes, características distintivas y su funcionamiento. Se identificaron y definieron los requisitos del juego  serio, tanto funcionales como no funcionales, así como los actores y casos de usos del sistema que establecieron las bases fundamentales para el desarrollo de la aplicación. Se realizó el diseño de la base de datos, abarcando tanto el modelo lógico como el físico, lo que aseguró una estructura robusta y eficiente para el almacenamiento de los datos. La manipulación de los datos se abordó de manera integral, desde la conexión con la base de datos hasta la persistencia de los resultados estadísticos. Se diseñó e implementó la comunicación con el pedal motorizado y la implementación de la interfaz gráfica para la representación de los datos EMG. Se definieron los escenarios de entrenamiento para las modalidades Ligero y Clínico, asegurando una cobertura completa de las necesidad de entrenamiento del usuario. Por último en el ámbito estadístico se desarrolló una serie de gráficos para el seguimiento de los resultados en las rutinas de entrenamiento.   
    
\end{thesischapter}

\subthesischapter{Conclusiones del capítulo}
Se presentó una descripción del sistema de adquisición de datos para rehabilitación, sus componentes, características distintivas y su funcionamiento. Se identificaron y definieron los requisitos del juego  serio, tanto funcionales como no funcionales, así como los actores y casos de usos del sistema que establecieron las bases fundamentales para el desarrollo de la aplicación. Se realizó el diseño de la base de datos, abarcando tanto el modelo lógico como el físico, lo que aseguró una estructura robusta y eficiente para el almacenamiento de los datos. La manipulación de los datos se abordó de manera integral, desde la conexión con la base de datos hasta la persistencia de los resultados estadísticos. Se diseñó e implementó la comunicación con el pedal motorizado y la implementación de la interfaz gráfica para la representación de los datos EMG. Se definieron los escenarios de entrenamiento para las modalidades Ligero y Clínico, asegurando una cobertura completa de las necesidad de entrenamiento del usuario. Por último en el ámbito estadístico se desarrolló una serie de gráficos para el seguimiento de los resultados en las rutinas de entrenamiento.   
    
\end{thesischapter}

% TRAINING SCENARIOS
\begin{thesischapter}{2} {Diseño e implementación del Juego Serio}
En este capítulo se discuten los detalles de desarrollo de los aspectos citados en el capítulo anterior. Este comienza con una descripción y caracterización general del sistema, donde se  abordan cada uno de los componentes requeridos para su completo funcionamiento. Posteriormente se detalla la ingienría de software requerida en la etapa de conceptualización de la aplicación, se explican de forma detallada los aspectos teóricos y de implementación de la base de datos, el funcionamiento del protocolo de comunicación y por último los escenarios de juegos requeridos en las rutinas de entrenamiento ligero y clínico, y las estadísticas generadas por estos. Como herramienta de desarrollo se utilizó c\#.

% SYSTEM DESCRIPTION AND CHARACTERIZATION TO APPLY
\begin{thesischapter}{2} {Diseño e implementación del Juego Serio}
En este capítulo se discuten los detalles de desarrollo de los aspectos citados en el capítulo anterior. Este comienza con una descripción y caracterización general del sistema, donde se  abordan cada uno de los componentes requeridos para su completo funcionamiento. Posteriormente se detalla la ingienría de software requerida en la etapa de conceptualización de la aplicación, se explican de forma detallada los aspectos teóricos y de implementación de la base de datos, el funcionamiento del protocolo de comunicación y por último los escenarios de juegos requeridos en las rutinas de entrenamiento ligero y clínico, y las estadísticas generadas por estos. Como herramienta de desarrollo se utilizó c\#.

% SYSTEM DESCRIPTION AND CHARACTERIZATION TO APPLY
\input{main/chapter2/section1/content.tex}
     
% SERIOUS GAME REQUIREMENTS
\input{main/chapter2/section2/content.tex}    

% USE CASE DEFINITION
\input{main/chapter2/section3/content.tex}

% USE CASE REALIZATION
\input{main/chapter2/section4/content.tex}

% DATABASE DESIGN 
\input{main/chapter2/section5/content.tex}

% DATA MANIPULATION
\input{main/chapter2/section6/content.tex}

% COMMUNICATION 
\input{main/chapter2/section7/content.tex}

% TRAINING SCENARIOS
\input{main/chapter2/section8/content.tex}

% EMG GRAPHIC
\input{main/chapter2/section9/content.tex}

% STATICAL REPORTS GENERATION
\input{main/chapter2/section10/content.tex}

\subthesischapter{Conclusiones del capítulo}
Se presentó una descripción del sistema de adquisición de datos para rehabilitación, sus componentes, características distintivas y su funcionamiento. Se identificaron y definieron los requisitos del juego  serio, tanto funcionales como no funcionales, así como los actores y casos de usos del sistema que establecieron las bases fundamentales para el desarrollo de la aplicación. Se realizó el diseño de la base de datos, abarcando tanto el modelo lógico como el físico, lo que aseguró una estructura robusta y eficiente para el almacenamiento de los datos. La manipulación de los datos se abordó de manera integral, desde la conexión con la base de datos hasta la persistencia de los resultados estadísticos. Se diseñó e implementó la comunicación con el pedal motorizado y la implementación de la interfaz gráfica para la representación de los datos EMG. Se definieron los escenarios de entrenamiento para las modalidades Ligero y Clínico, asegurando una cobertura completa de las necesidad de entrenamiento del usuario. Por último en el ámbito estadístico se desarrolló una serie de gráficos para el seguimiento de los resultados en las rutinas de entrenamiento.   
    
\end{thesischapter}
     
% SERIOUS GAME REQUIREMENTS
\begin{thesischapter}{2} {Diseño e implementación del Juego Serio}
En este capítulo se discuten los detalles de desarrollo de los aspectos citados en el capítulo anterior. Este comienza con una descripción y caracterización general del sistema, donde se  abordan cada uno de los componentes requeridos para su completo funcionamiento. Posteriormente se detalla la ingienría de software requerida en la etapa de conceptualización de la aplicación, se explican de forma detallada los aspectos teóricos y de implementación de la base de datos, el funcionamiento del protocolo de comunicación y por último los escenarios de juegos requeridos en las rutinas de entrenamiento ligero y clínico, y las estadísticas generadas por estos. Como herramienta de desarrollo se utilizó c\#.

% SYSTEM DESCRIPTION AND CHARACTERIZATION TO APPLY
\input{main/chapter2/section1/content.tex}
     
% SERIOUS GAME REQUIREMENTS
\input{main/chapter2/section2/content.tex}    

% USE CASE DEFINITION
\input{main/chapter2/section3/content.tex}

% USE CASE REALIZATION
\input{main/chapter2/section4/content.tex}

% DATABASE DESIGN 
\input{main/chapter2/section5/content.tex}

% DATA MANIPULATION
\input{main/chapter2/section6/content.tex}

% COMMUNICATION 
\input{main/chapter2/section7/content.tex}

% TRAINING SCENARIOS
\input{main/chapter2/section8/content.tex}

% EMG GRAPHIC
\input{main/chapter2/section9/content.tex}

% STATICAL REPORTS GENERATION
\input{main/chapter2/section10/content.tex}

\subthesischapter{Conclusiones del capítulo}
Se presentó una descripción del sistema de adquisición de datos para rehabilitación, sus componentes, características distintivas y su funcionamiento. Se identificaron y definieron los requisitos del juego  serio, tanto funcionales como no funcionales, así como los actores y casos de usos del sistema que establecieron las bases fundamentales para el desarrollo de la aplicación. Se realizó el diseño de la base de datos, abarcando tanto el modelo lógico como el físico, lo que aseguró una estructura robusta y eficiente para el almacenamiento de los datos. La manipulación de los datos se abordó de manera integral, desde la conexión con la base de datos hasta la persistencia de los resultados estadísticos. Se diseñó e implementó la comunicación con el pedal motorizado y la implementación de la interfaz gráfica para la representación de los datos EMG. Se definieron los escenarios de entrenamiento para las modalidades Ligero y Clínico, asegurando una cobertura completa de las necesidad de entrenamiento del usuario. Por último en el ámbito estadístico se desarrolló una serie de gráficos para el seguimiento de los resultados en las rutinas de entrenamiento.   
    
\end{thesischapter}    

% USE CASE DEFINITION
\begin{thesischapter}{2} {Diseño e implementación del Juego Serio}
En este capítulo se discuten los detalles de desarrollo de los aspectos citados en el capítulo anterior. Este comienza con una descripción y caracterización general del sistema, donde se  abordan cada uno de los componentes requeridos para su completo funcionamiento. Posteriormente se detalla la ingienría de software requerida en la etapa de conceptualización de la aplicación, se explican de forma detallada los aspectos teóricos y de implementación de la base de datos, el funcionamiento del protocolo de comunicación y por último los escenarios de juegos requeridos en las rutinas de entrenamiento ligero y clínico, y las estadísticas generadas por estos. Como herramienta de desarrollo se utilizó c\#.

% SYSTEM DESCRIPTION AND CHARACTERIZATION TO APPLY
\input{main/chapter2/section1/content.tex}
     
% SERIOUS GAME REQUIREMENTS
\input{main/chapter2/section2/content.tex}    

% USE CASE DEFINITION
\input{main/chapter2/section3/content.tex}

% USE CASE REALIZATION
\input{main/chapter2/section4/content.tex}

% DATABASE DESIGN 
\input{main/chapter2/section5/content.tex}

% DATA MANIPULATION
\input{main/chapter2/section6/content.tex}

% COMMUNICATION 
\input{main/chapter2/section7/content.tex}

% TRAINING SCENARIOS
\input{main/chapter2/section8/content.tex}

% EMG GRAPHIC
\input{main/chapter2/section9/content.tex}

% STATICAL REPORTS GENERATION
\input{main/chapter2/section10/content.tex}

\subthesischapter{Conclusiones del capítulo}
Se presentó una descripción del sistema de adquisición de datos para rehabilitación, sus componentes, características distintivas y su funcionamiento. Se identificaron y definieron los requisitos del juego  serio, tanto funcionales como no funcionales, así como los actores y casos de usos del sistema que establecieron las bases fundamentales para el desarrollo de la aplicación. Se realizó el diseño de la base de datos, abarcando tanto el modelo lógico como el físico, lo que aseguró una estructura robusta y eficiente para el almacenamiento de los datos. La manipulación de los datos se abordó de manera integral, desde la conexión con la base de datos hasta la persistencia de los resultados estadísticos. Se diseñó e implementó la comunicación con el pedal motorizado y la implementación de la interfaz gráfica para la representación de los datos EMG. Se definieron los escenarios de entrenamiento para las modalidades Ligero y Clínico, asegurando una cobertura completa de las necesidad de entrenamiento del usuario. Por último en el ámbito estadístico se desarrolló una serie de gráficos para el seguimiento de los resultados en las rutinas de entrenamiento.   
    
\end{thesischapter}

% USE CASE REALIZATION
\begin{thesischapter}{2} {Diseño e implementación del Juego Serio}
En este capítulo se discuten los detalles de desarrollo de los aspectos citados en el capítulo anterior. Este comienza con una descripción y caracterización general del sistema, donde se  abordan cada uno de los componentes requeridos para su completo funcionamiento. Posteriormente se detalla la ingienría de software requerida en la etapa de conceptualización de la aplicación, se explican de forma detallada los aspectos teóricos y de implementación de la base de datos, el funcionamiento del protocolo de comunicación y por último los escenarios de juegos requeridos en las rutinas de entrenamiento ligero y clínico, y las estadísticas generadas por estos. Como herramienta de desarrollo se utilizó c\#.

% SYSTEM DESCRIPTION AND CHARACTERIZATION TO APPLY
\input{main/chapter2/section1/content.tex}
     
% SERIOUS GAME REQUIREMENTS
\input{main/chapter2/section2/content.tex}    

% USE CASE DEFINITION
\input{main/chapter2/section3/content.tex}

% USE CASE REALIZATION
\input{main/chapter2/section4/content.tex}

% DATABASE DESIGN 
\input{main/chapter2/section5/content.tex}

% DATA MANIPULATION
\input{main/chapter2/section6/content.tex}

% COMMUNICATION 
\input{main/chapter2/section7/content.tex}

% TRAINING SCENARIOS
\input{main/chapter2/section8/content.tex}

% EMG GRAPHIC
\input{main/chapter2/section9/content.tex}

% STATICAL REPORTS GENERATION
\input{main/chapter2/section10/content.tex}

\subthesischapter{Conclusiones del capítulo}
Se presentó una descripción del sistema de adquisición de datos para rehabilitación, sus componentes, características distintivas y su funcionamiento. Se identificaron y definieron los requisitos del juego  serio, tanto funcionales como no funcionales, así como los actores y casos de usos del sistema que establecieron las bases fundamentales para el desarrollo de la aplicación. Se realizó el diseño de la base de datos, abarcando tanto el modelo lógico como el físico, lo que aseguró una estructura robusta y eficiente para el almacenamiento de los datos. La manipulación de los datos se abordó de manera integral, desde la conexión con la base de datos hasta la persistencia de los resultados estadísticos. Se diseñó e implementó la comunicación con el pedal motorizado y la implementación de la interfaz gráfica para la representación de los datos EMG. Se definieron los escenarios de entrenamiento para las modalidades Ligero y Clínico, asegurando una cobertura completa de las necesidad de entrenamiento del usuario. Por último en el ámbito estadístico se desarrolló una serie de gráficos para el seguimiento de los resultados en las rutinas de entrenamiento.   
    
\end{thesischapter}

% DATABASE DESIGN 
\begin{thesischapter}{2} {Diseño e implementación del Juego Serio}
En este capítulo se discuten los detalles de desarrollo de los aspectos citados en el capítulo anterior. Este comienza con una descripción y caracterización general del sistema, donde se  abordan cada uno de los componentes requeridos para su completo funcionamiento. Posteriormente se detalla la ingienría de software requerida en la etapa de conceptualización de la aplicación, se explican de forma detallada los aspectos teóricos y de implementación de la base de datos, el funcionamiento del protocolo de comunicación y por último los escenarios de juegos requeridos en las rutinas de entrenamiento ligero y clínico, y las estadísticas generadas por estos. Como herramienta de desarrollo se utilizó c\#.

% SYSTEM DESCRIPTION AND CHARACTERIZATION TO APPLY
\input{main/chapter2/section1/content.tex}
     
% SERIOUS GAME REQUIREMENTS
\input{main/chapter2/section2/content.tex}    

% USE CASE DEFINITION
\input{main/chapter2/section3/content.tex}

% USE CASE REALIZATION
\input{main/chapter2/section4/content.tex}

% DATABASE DESIGN 
\input{main/chapter2/section5/content.tex}

% DATA MANIPULATION
\input{main/chapter2/section6/content.tex}

% COMMUNICATION 
\input{main/chapter2/section7/content.tex}

% TRAINING SCENARIOS
\input{main/chapter2/section8/content.tex}

% EMG GRAPHIC
\input{main/chapter2/section9/content.tex}

% STATICAL REPORTS GENERATION
\input{main/chapter2/section10/content.tex}

\subthesischapter{Conclusiones del capítulo}
Se presentó una descripción del sistema de adquisición de datos para rehabilitación, sus componentes, características distintivas y su funcionamiento. Se identificaron y definieron los requisitos del juego  serio, tanto funcionales como no funcionales, así como los actores y casos de usos del sistema que establecieron las bases fundamentales para el desarrollo de la aplicación. Se realizó el diseño de la base de datos, abarcando tanto el modelo lógico como el físico, lo que aseguró una estructura robusta y eficiente para el almacenamiento de los datos. La manipulación de los datos se abordó de manera integral, desde la conexión con la base de datos hasta la persistencia de los resultados estadísticos. Se diseñó e implementó la comunicación con el pedal motorizado y la implementación de la interfaz gráfica para la representación de los datos EMG. Se definieron los escenarios de entrenamiento para las modalidades Ligero y Clínico, asegurando una cobertura completa de las necesidad de entrenamiento del usuario. Por último en el ámbito estadístico se desarrolló una serie de gráficos para el seguimiento de los resultados en las rutinas de entrenamiento.   
    
\end{thesischapter}

% DATA MANIPULATION
\begin{thesischapter}{2} {Diseño e implementación del Juego Serio}
En este capítulo se discuten los detalles de desarrollo de los aspectos citados en el capítulo anterior. Este comienza con una descripción y caracterización general del sistema, donde se  abordan cada uno de los componentes requeridos para su completo funcionamiento. Posteriormente se detalla la ingienría de software requerida en la etapa de conceptualización de la aplicación, se explican de forma detallada los aspectos teóricos y de implementación de la base de datos, el funcionamiento del protocolo de comunicación y por último los escenarios de juegos requeridos en las rutinas de entrenamiento ligero y clínico, y las estadísticas generadas por estos. Como herramienta de desarrollo se utilizó c\#.

% SYSTEM DESCRIPTION AND CHARACTERIZATION TO APPLY
\input{main/chapter2/section1/content.tex}
     
% SERIOUS GAME REQUIREMENTS
\input{main/chapter2/section2/content.tex}    

% USE CASE DEFINITION
\input{main/chapter2/section3/content.tex}

% USE CASE REALIZATION
\input{main/chapter2/section4/content.tex}

% DATABASE DESIGN 
\input{main/chapter2/section5/content.tex}

% DATA MANIPULATION
\input{main/chapter2/section6/content.tex}

% COMMUNICATION 
\input{main/chapter2/section7/content.tex}

% TRAINING SCENARIOS
\input{main/chapter2/section8/content.tex}

% EMG GRAPHIC
\input{main/chapter2/section9/content.tex}

% STATICAL REPORTS GENERATION
\input{main/chapter2/section10/content.tex}

\subthesischapter{Conclusiones del capítulo}
Se presentó una descripción del sistema de adquisición de datos para rehabilitación, sus componentes, características distintivas y su funcionamiento. Se identificaron y definieron los requisitos del juego  serio, tanto funcionales como no funcionales, así como los actores y casos de usos del sistema que establecieron las bases fundamentales para el desarrollo de la aplicación. Se realizó el diseño de la base de datos, abarcando tanto el modelo lógico como el físico, lo que aseguró una estructura robusta y eficiente para el almacenamiento de los datos. La manipulación de los datos se abordó de manera integral, desde la conexión con la base de datos hasta la persistencia de los resultados estadísticos. Se diseñó e implementó la comunicación con el pedal motorizado y la implementación de la interfaz gráfica para la representación de los datos EMG. Se definieron los escenarios de entrenamiento para las modalidades Ligero y Clínico, asegurando una cobertura completa de las necesidad de entrenamiento del usuario. Por último en el ámbito estadístico se desarrolló una serie de gráficos para el seguimiento de los resultados en las rutinas de entrenamiento.   
    
\end{thesischapter}

% COMMUNICATION 
\begin{thesischapter}{2} {Diseño e implementación del Juego Serio}
En este capítulo se discuten los detalles de desarrollo de los aspectos citados en el capítulo anterior. Este comienza con una descripción y caracterización general del sistema, donde se  abordan cada uno de los componentes requeridos para su completo funcionamiento. Posteriormente se detalla la ingienría de software requerida en la etapa de conceptualización de la aplicación, se explican de forma detallada los aspectos teóricos y de implementación de la base de datos, el funcionamiento del protocolo de comunicación y por último los escenarios de juegos requeridos en las rutinas de entrenamiento ligero y clínico, y las estadísticas generadas por estos. Como herramienta de desarrollo se utilizó c\#.

% SYSTEM DESCRIPTION AND CHARACTERIZATION TO APPLY
\input{main/chapter2/section1/content.tex}
     
% SERIOUS GAME REQUIREMENTS
\input{main/chapter2/section2/content.tex}    

% USE CASE DEFINITION
\input{main/chapter2/section3/content.tex}

% USE CASE REALIZATION
\input{main/chapter2/section4/content.tex}

% DATABASE DESIGN 
\input{main/chapter2/section5/content.tex}

% DATA MANIPULATION
\input{main/chapter2/section6/content.tex}

% COMMUNICATION 
\input{main/chapter2/section7/content.tex}

% TRAINING SCENARIOS
\input{main/chapter2/section8/content.tex}

% EMG GRAPHIC
\input{main/chapter2/section9/content.tex}

% STATICAL REPORTS GENERATION
\input{main/chapter2/section10/content.tex}

\subthesischapter{Conclusiones del capítulo}
Se presentó una descripción del sistema de adquisición de datos para rehabilitación, sus componentes, características distintivas y su funcionamiento. Se identificaron y definieron los requisitos del juego  serio, tanto funcionales como no funcionales, así como los actores y casos de usos del sistema que establecieron las bases fundamentales para el desarrollo de la aplicación. Se realizó el diseño de la base de datos, abarcando tanto el modelo lógico como el físico, lo que aseguró una estructura robusta y eficiente para el almacenamiento de los datos. La manipulación de los datos se abordó de manera integral, desde la conexión con la base de datos hasta la persistencia de los resultados estadísticos. Se diseñó e implementó la comunicación con el pedal motorizado y la implementación de la interfaz gráfica para la representación de los datos EMG. Se definieron los escenarios de entrenamiento para las modalidades Ligero y Clínico, asegurando una cobertura completa de las necesidad de entrenamiento del usuario. Por último en el ámbito estadístico se desarrolló una serie de gráficos para el seguimiento de los resultados en las rutinas de entrenamiento.   
    
\end{thesischapter}

% TRAINING SCENARIOS
\begin{thesischapter}{2} {Diseño e implementación del Juego Serio}
En este capítulo se discuten los detalles de desarrollo de los aspectos citados en el capítulo anterior. Este comienza con una descripción y caracterización general del sistema, donde se  abordan cada uno de los componentes requeridos para su completo funcionamiento. Posteriormente se detalla la ingienría de software requerida en la etapa de conceptualización de la aplicación, se explican de forma detallada los aspectos teóricos y de implementación de la base de datos, el funcionamiento del protocolo de comunicación y por último los escenarios de juegos requeridos en las rutinas de entrenamiento ligero y clínico, y las estadísticas generadas por estos. Como herramienta de desarrollo se utilizó c\#.

% SYSTEM DESCRIPTION AND CHARACTERIZATION TO APPLY
\input{main/chapter2/section1/content.tex}
     
% SERIOUS GAME REQUIREMENTS
\input{main/chapter2/section2/content.tex}    

% USE CASE DEFINITION
\input{main/chapter2/section3/content.tex}

% USE CASE REALIZATION
\input{main/chapter2/section4/content.tex}

% DATABASE DESIGN 
\input{main/chapter2/section5/content.tex}

% DATA MANIPULATION
\input{main/chapter2/section6/content.tex}

% COMMUNICATION 
\input{main/chapter2/section7/content.tex}

% TRAINING SCENARIOS
\input{main/chapter2/section8/content.tex}

% EMG GRAPHIC
\input{main/chapter2/section9/content.tex}

% STATICAL REPORTS GENERATION
\input{main/chapter2/section10/content.tex}

\subthesischapter{Conclusiones del capítulo}
Se presentó una descripción del sistema de adquisición de datos para rehabilitación, sus componentes, características distintivas y su funcionamiento. Se identificaron y definieron los requisitos del juego  serio, tanto funcionales como no funcionales, así como los actores y casos de usos del sistema que establecieron las bases fundamentales para el desarrollo de la aplicación. Se realizó el diseño de la base de datos, abarcando tanto el modelo lógico como el físico, lo que aseguró una estructura robusta y eficiente para el almacenamiento de los datos. La manipulación de los datos se abordó de manera integral, desde la conexión con la base de datos hasta la persistencia de los resultados estadísticos. Se diseñó e implementó la comunicación con el pedal motorizado y la implementación de la interfaz gráfica para la representación de los datos EMG. Se definieron los escenarios de entrenamiento para las modalidades Ligero y Clínico, asegurando una cobertura completa de las necesidad de entrenamiento del usuario. Por último en el ámbito estadístico se desarrolló una serie de gráficos para el seguimiento de los resultados en las rutinas de entrenamiento.   
    
\end{thesischapter}

% EMG GRAPHIC
\begin{thesischapter}{2} {Diseño e implementación del Juego Serio}
En este capítulo se discuten los detalles de desarrollo de los aspectos citados en el capítulo anterior. Este comienza con una descripción y caracterización general del sistema, donde se  abordan cada uno de los componentes requeridos para su completo funcionamiento. Posteriormente se detalla la ingienría de software requerida en la etapa de conceptualización de la aplicación, se explican de forma detallada los aspectos teóricos y de implementación de la base de datos, el funcionamiento del protocolo de comunicación y por último los escenarios de juegos requeridos en las rutinas de entrenamiento ligero y clínico, y las estadísticas generadas por estos. Como herramienta de desarrollo se utilizó c\#.

% SYSTEM DESCRIPTION AND CHARACTERIZATION TO APPLY
\input{main/chapter2/section1/content.tex}
     
% SERIOUS GAME REQUIREMENTS
\input{main/chapter2/section2/content.tex}    

% USE CASE DEFINITION
\input{main/chapter2/section3/content.tex}

% USE CASE REALIZATION
\input{main/chapter2/section4/content.tex}

% DATABASE DESIGN 
\input{main/chapter2/section5/content.tex}

% DATA MANIPULATION
\input{main/chapter2/section6/content.tex}

% COMMUNICATION 
\input{main/chapter2/section7/content.tex}

% TRAINING SCENARIOS
\input{main/chapter2/section8/content.tex}

% EMG GRAPHIC
\input{main/chapter2/section9/content.tex}

% STATICAL REPORTS GENERATION
\input{main/chapter2/section10/content.tex}

\subthesischapter{Conclusiones del capítulo}
Se presentó una descripción del sistema de adquisición de datos para rehabilitación, sus componentes, características distintivas y su funcionamiento. Se identificaron y definieron los requisitos del juego  serio, tanto funcionales como no funcionales, así como los actores y casos de usos del sistema que establecieron las bases fundamentales para el desarrollo de la aplicación. Se realizó el diseño de la base de datos, abarcando tanto el modelo lógico como el físico, lo que aseguró una estructura robusta y eficiente para el almacenamiento de los datos. La manipulación de los datos se abordó de manera integral, desde la conexión con la base de datos hasta la persistencia de los resultados estadísticos. Se diseñó e implementó la comunicación con el pedal motorizado y la implementación de la interfaz gráfica para la representación de los datos EMG. Se definieron los escenarios de entrenamiento para las modalidades Ligero y Clínico, asegurando una cobertura completa de las necesidad de entrenamiento del usuario. Por último en el ámbito estadístico se desarrolló una serie de gráficos para el seguimiento de los resultados en las rutinas de entrenamiento.   
    
\end{thesischapter}

% STATICAL REPORTS GENERATION
\begin{thesischapter}{2} {Diseño e implementación del Juego Serio}
En este capítulo se discuten los detalles de desarrollo de los aspectos citados en el capítulo anterior. Este comienza con una descripción y caracterización general del sistema, donde se  abordan cada uno de los componentes requeridos para su completo funcionamiento. Posteriormente se detalla la ingienría de software requerida en la etapa de conceptualización de la aplicación, se explican de forma detallada los aspectos teóricos y de implementación de la base de datos, el funcionamiento del protocolo de comunicación y por último los escenarios de juegos requeridos en las rutinas de entrenamiento ligero y clínico, y las estadísticas generadas por estos. Como herramienta de desarrollo se utilizó c\#.

% SYSTEM DESCRIPTION AND CHARACTERIZATION TO APPLY
\input{main/chapter2/section1/content.tex}
     
% SERIOUS GAME REQUIREMENTS
\input{main/chapter2/section2/content.tex}    

% USE CASE DEFINITION
\input{main/chapter2/section3/content.tex}

% USE CASE REALIZATION
\input{main/chapter2/section4/content.tex}

% DATABASE DESIGN 
\input{main/chapter2/section5/content.tex}

% DATA MANIPULATION
\input{main/chapter2/section6/content.tex}

% COMMUNICATION 
\input{main/chapter2/section7/content.tex}

% TRAINING SCENARIOS
\input{main/chapter2/section8/content.tex}

% EMG GRAPHIC
\input{main/chapter2/section9/content.tex}

% STATICAL REPORTS GENERATION
\input{main/chapter2/section10/content.tex}

\subthesischapter{Conclusiones del capítulo}
Se presentó una descripción del sistema de adquisición de datos para rehabilitación, sus componentes, características distintivas y su funcionamiento. Se identificaron y definieron los requisitos del juego  serio, tanto funcionales como no funcionales, así como los actores y casos de usos del sistema que establecieron las bases fundamentales para el desarrollo de la aplicación. Se realizó el diseño de la base de datos, abarcando tanto el modelo lógico como el físico, lo que aseguró una estructura robusta y eficiente para el almacenamiento de los datos. La manipulación de los datos se abordó de manera integral, desde la conexión con la base de datos hasta la persistencia de los resultados estadísticos. Se diseñó e implementó la comunicación con el pedal motorizado y la implementación de la interfaz gráfica para la representación de los datos EMG. Se definieron los escenarios de entrenamiento para las modalidades Ligero y Clínico, asegurando una cobertura completa de las necesidad de entrenamiento del usuario. Por último en el ámbito estadístico se desarrolló una serie de gráficos para el seguimiento de los resultados en las rutinas de entrenamiento.   
    
\end{thesischapter}

\subthesischapter{Conclusiones del capítulo}
Se presentó una descripción del sistema de adquisición de datos para rehabilitación, sus componentes, características distintivas y su funcionamiento. Se identificaron y definieron los requisitos del juego  serio, tanto funcionales como no funcionales, así como los actores y casos de usos del sistema que establecieron las bases fundamentales para el desarrollo de la aplicación. Se realizó el diseño de la base de datos, abarcando tanto el modelo lógico como el físico, lo que aseguró una estructura robusta y eficiente para el almacenamiento de los datos. La manipulación de los datos se abordó de manera integral, desde la conexión con la base de datos hasta la persistencia de los resultados estadísticos. Se diseñó e implementó la comunicación con el pedal motorizado y la implementación de la interfaz gráfica para la representación de los datos EMG. Se definieron los escenarios de entrenamiento para las modalidades Ligero y Clínico, asegurando una cobertura completa de las necesidad de entrenamiento del usuario. Por último en el ámbito estadístico se desarrolló una serie de gráficos para el seguimiento de los resultados en las rutinas de entrenamiento.   
    
\end{thesischapter}

% EMG GRAPHIC
\begin{thesischapter}{2} {Diseño e implementación del Juego Serio}
En este capítulo se discuten los detalles de desarrollo de los aspectos citados en el capítulo anterior. Este comienza con una descripción y caracterización general del sistema, donde se  abordan cada uno de los componentes requeridos para su completo funcionamiento. Posteriormente se detalla la ingienría de software requerida en la etapa de conceptualización de la aplicación, se explican de forma detallada los aspectos teóricos y de implementación de la base de datos, el funcionamiento del protocolo de comunicación y por último los escenarios de juegos requeridos en las rutinas de entrenamiento ligero y clínico, y las estadísticas generadas por estos. Como herramienta de desarrollo se utilizó c\#.

% SYSTEM DESCRIPTION AND CHARACTERIZATION TO APPLY
\begin{thesischapter}{2} {Diseño e implementación del Juego Serio}
En este capítulo se discuten los detalles de desarrollo de los aspectos citados en el capítulo anterior. Este comienza con una descripción y caracterización general del sistema, donde se  abordan cada uno de los componentes requeridos para su completo funcionamiento. Posteriormente se detalla la ingienría de software requerida en la etapa de conceptualización de la aplicación, se explican de forma detallada los aspectos teóricos y de implementación de la base de datos, el funcionamiento del protocolo de comunicación y por último los escenarios de juegos requeridos en las rutinas de entrenamiento ligero y clínico, y las estadísticas generadas por estos. Como herramienta de desarrollo se utilizó c\#.

% SYSTEM DESCRIPTION AND CHARACTERIZATION TO APPLY
\input{main/chapter2/section1/content.tex}
     
% SERIOUS GAME REQUIREMENTS
\input{main/chapter2/section2/content.tex}    

% USE CASE DEFINITION
\input{main/chapter2/section3/content.tex}

% USE CASE REALIZATION
\input{main/chapter2/section4/content.tex}

% DATABASE DESIGN 
\input{main/chapter2/section5/content.tex}

% DATA MANIPULATION
\input{main/chapter2/section6/content.tex}

% COMMUNICATION 
\input{main/chapter2/section7/content.tex}

% TRAINING SCENARIOS
\input{main/chapter2/section8/content.tex}

% EMG GRAPHIC
\input{main/chapter2/section9/content.tex}

% STATICAL REPORTS GENERATION
\input{main/chapter2/section10/content.tex}

\subthesischapter{Conclusiones del capítulo}
Se presentó una descripción del sistema de adquisición de datos para rehabilitación, sus componentes, características distintivas y su funcionamiento. Se identificaron y definieron los requisitos del juego  serio, tanto funcionales como no funcionales, así como los actores y casos de usos del sistema que establecieron las bases fundamentales para el desarrollo de la aplicación. Se realizó el diseño de la base de datos, abarcando tanto el modelo lógico como el físico, lo que aseguró una estructura robusta y eficiente para el almacenamiento de los datos. La manipulación de los datos se abordó de manera integral, desde la conexión con la base de datos hasta la persistencia de los resultados estadísticos. Se diseñó e implementó la comunicación con el pedal motorizado y la implementación de la interfaz gráfica para la representación de los datos EMG. Se definieron los escenarios de entrenamiento para las modalidades Ligero y Clínico, asegurando una cobertura completa de las necesidad de entrenamiento del usuario. Por último en el ámbito estadístico se desarrolló una serie de gráficos para el seguimiento de los resultados en las rutinas de entrenamiento.   
    
\end{thesischapter}
     
% SERIOUS GAME REQUIREMENTS
\begin{thesischapter}{2} {Diseño e implementación del Juego Serio}
En este capítulo se discuten los detalles de desarrollo de los aspectos citados en el capítulo anterior. Este comienza con una descripción y caracterización general del sistema, donde se  abordan cada uno de los componentes requeridos para su completo funcionamiento. Posteriormente se detalla la ingienría de software requerida en la etapa de conceptualización de la aplicación, se explican de forma detallada los aspectos teóricos y de implementación de la base de datos, el funcionamiento del protocolo de comunicación y por último los escenarios de juegos requeridos en las rutinas de entrenamiento ligero y clínico, y las estadísticas generadas por estos. Como herramienta de desarrollo se utilizó c\#.

% SYSTEM DESCRIPTION AND CHARACTERIZATION TO APPLY
\input{main/chapter2/section1/content.tex}
     
% SERIOUS GAME REQUIREMENTS
\input{main/chapter2/section2/content.tex}    

% USE CASE DEFINITION
\input{main/chapter2/section3/content.tex}

% USE CASE REALIZATION
\input{main/chapter2/section4/content.tex}

% DATABASE DESIGN 
\input{main/chapter2/section5/content.tex}

% DATA MANIPULATION
\input{main/chapter2/section6/content.tex}

% COMMUNICATION 
\input{main/chapter2/section7/content.tex}

% TRAINING SCENARIOS
\input{main/chapter2/section8/content.tex}

% EMG GRAPHIC
\input{main/chapter2/section9/content.tex}

% STATICAL REPORTS GENERATION
\input{main/chapter2/section10/content.tex}

\subthesischapter{Conclusiones del capítulo}
Se presentó una descripción del sistema de adquisición de datos para rehabilitación, sus componentes, características distintivas y su funcionamiento. Se identificaron y definieron los requisitos del juego  serio, tanto funcionales como no funcionales, así como los actores y casos de usos del sistema que establecieron las bases fundamentales para el desarrollo de la aplicación. Se realizó el diseño de la base de datos, abarcando tanto el modelo lógico como el físico, lo que aseguró una estructura robusta y eficiente para el almacenamiento de los datos. La manipulación de los datos se abordó de manera integral, desde la conexión con la base de datos hasta la persistencia de los resultados estadísticos. Se diseñó e implementó la comunicación con el pedal motorizado y la implementación de la interfaz gráfica para la representación de los datos EMG. Se definieron los escenarios de entrenamiento para las modalidades Ligero y Clínico, asegurando una cobertura completa de las necesidad de entrenamiento del usuario. Por último en el ámbito estadístico se desarrolló una serie de gráficos para el seguimiento de los resultados en las rutinas de entrenamiento.   
    
\end{thesischapter}    

% USE CASE DEFINITION
\begin{thesischapter}{2} {Diseño e implementación del Juego Serio}
En este capítulo se discuten los detalles de desarrollo de los aspectos citados en el capítulo anterior. Este comienza con una descripción y caracterización general del sistema, donde se  abordan cada uno de los componentes requeridos para su completo funcionamiento. Posteriormente se detalla la ingienría de software requerida en la etapa de conceptualización de la aplicación, se explican de forma detallada los aspectos teóricos y de implementación de la base de datos, el funcionamiento del protocolo de comunicación y por último los escenarios de juegos requeridos en las rutinas de entrenamiento ligero y clínico, y las estadísticas generadas por estos. Como herramienta de desarrollo se utilizó c\#.

% SYSTEM DESCRIPTION AND CHARACTERIZATION TO APPLY
\input{main/chapter2/section1/content.tex}
     
% SERIOUS GAME REQUIREMENTS
\input{main/chapter2/section2/content.tex}    

% USE CASE DEFINITION
\input{main/chapter2/section3/content.tex}

% USE CASE REALIZATION
\input{main/chapter2/section4/content.tex}

% DATABASE DESIGN 
\input{main/chapter2/section5/content.tex}

% DATA MANIPULATION
\input{main/chapter2/section6/content.tex}

% COMMUNICATION 
\input{main/chapter2/section7/content.tex}

% TRAINING SCENARIOS
\input{main/chapter2/section8/content.tex}

% EMG GRAPHIC
\input{main/chapter2/section9/content.tex}

% STATICAL REPORTS GENERATION
\input{main/chapter2/section10/content.tex}

\subthesischapter{Conclusiones del capítulo}
Se presentó una descripción del sistema de adquisición de datos para rehabilitación, sus componentes, características distintivas y su funcionamiento. Se identificaron y definieron los requisitos del juego  serio, tanto funcionales como no funcionales, así como los actores y casos de usos del sistema que establecieron las bases fundamentales para el desarrollo de la aplicación. Se realizó el diseño de la base de datos, abarcando tanto el modelo lógico como el físico, lo que aseguró una estructura robusta y eficiente para el almacenamiento de los datos. La manipulación de los datos se abordó de manera integral, desde la conexión con la base de datos hasta la persistencia de los resultados estadísticos. Se diseñó e implementó la comunicación con el pedal motorizado y la implementación de la interfaz gráfica para la representación de los datos EMG. Se definieron los escenarios de entrenamiento para las modalidades Ligero y Clínico, asegurando una cobertura completa de las necesidad de entrenamiento del usuario. Por último en el ámbito estadístico se desarrolló una serie de gráficos para el seguimiento de los resultados en las rutinas de entrenamiento.   
    
\end{thesischapter}

% USE CASE REALIZATION
\begin{thesischapter}{2} {Diseño e implementación del Juego Serio}
En este capítulo se discuten los detalles de desarrollo de los aspectos citados en el capítulo anterior. Este comienza con una descripción y caracterización general del sistema, donde se  abordan cada uno de los componentes requeridos para su completo funcionamiento. Posteriormente se detalla la ingienría de software requerida en la etapa de conceptualización de la aplicación, se explican de forma detallada los aspectos teóricos y de implementación de la base de datos, el funcionamiento del protocolo de comunicación y por último los escenarios de juegos requeridos en las rutinas de entrenamiento ligero y clínico, y las estadísticas generadas por estos. Como herramienta de desarrollo se utilizó c\#.

% SYSTEM DESCRIPTION AND CHARACTERIZATION TO APPLY
\input{main/chapter2/section1/content.tex}
     
% SERIOUS GAME REQUIREMENTS
\input{main/chapter2/section2/content.tex}    

% USE CASE DEFINITION
\input{main/chapter2/section3/content.tex}

% USE CASE REALIZATION
\input{main/chapter2/section4/content.tex}

% DATABASE DESIGN 
\input{main/chapter2/section5/content.tex}

% DATA MANIPULATION
\input{main/chapter2/section6/content.tex}

% COMMUNICATION 
\input{main/chapter2/section7/content.tex}

% TRAINING SCENARIOS
\input{main/chapter2/section8/content.tex}

% EMG GRAPHIC
\input{main/chapter2/section9/content.tex}

% STATICAL REPORTS GENERATION
\input{main/chapter2/section10/content.tex}

\subthesischapter{Conclusiones del capítulo}
Se presentó una descripción del sistema de adquisición de datos para rehabilitación, sus componentes, características distintivas y su funcionamiento. Se identificaron y definieron los requisitos del juego  serio, tanto funcionales como no funcionales, así como los actores y casos de usos del sistema que establecieron las bases fundamentales para el desarrollo de la aplicación. Se realizó el diseño de la base de datos, abarcando tanto el modelo lógico como el físico, lo que aseguró una estructura robusta y eficiente para el almacenamiento de los datos. La manipulación de los datos se abordó de manera integral, desde la conexión con la base de datos hasta la persistencia de los resultados estadísticos. Se diseñó e implementó la comunicación con el pedal motorizado y la implementación de la interfaz gráfica para la representación de los datos EMG. Se definieron los escenarios de entrenamiento para las modalidades Ligero y Clínico, asegurando una cobertura completa de las necesidad de entrenamiento del usuario. Por último en el ámbito estadístico se desarrolló una serie de gráficos para el seguimiento de los resultados en las rutinas de entrenamiento.   
    
\end{thesischapter}

% DATABASE DESIGN 
\begin{thesischapter}{2} {Diseño e implementación del Juego Serio}
En este capítulo se discuten los detalles de desarrollo de los aspectos citados en el capítulo anterior. Este comienza con una descripción y caracterización general del sistema, donde se  abordan cada uno de los componentes requeridos para su completo funcionamiento. Posteriormente se detalla la ingienría de software requerida en la etapa de conceptualización de la aplicación, se explican de forma detallada los aspectos teóricos y de implementación de la base de datos, el funcionamiento del protocolo de comunicación y por último los escenarios de juegos requeridos en las rutinas de entrenamiento ligero y clínico, y las estadísticas generadas por estos. Como herramienta de desarrollo se utilizó c\#.

% SYSTEM DESCRIPTION AND CHARACTERIZATION TO APPLY
\input{main/chapter2/section1/content.tex}
     
% SERIOUS GAME REQUIREMENTS
\input{main/chapter2/section2/content.tex}    

% USE CASE DEFINITION
\input{main/chapter2/section3/content.tex}

% USE CASE REALIZATION
\input{main/chapter2/section4/content.tex}

% DATABASE DESIGN 
\input{main/chapter2/section5/content.tex}

% DATA MANIPULATION
\input{main/chapter2/section6/content.tex}

% COMMUNICATION 
\input{main/chapter2/section7/content.tex}

% TRAINING SCENARIOS
\input{main/chapter2/section8/content.tex}

% EMG GRAPHIC
\input{main/chapter2/section9/content.tex}

% STATICAL REPORTS GENERATION
\input{main/chapter2/section10/content.tex}

\subthesischapter{Conclusiones del capítulo}
Se presentó una descripción del sistema de adquisición de datos para rehabilitación, sus componentes, características distintivas y su funcionamiento. Se identificaron y definieron los requisitos del juego  serio, tanto funcionales como no funcionales, así como los actores y casos de usos del sistema que establecieron las bases fundamentales para el desarrollo de la aplicación. Se realizó el diseño de la base de datos, abarcando tanto el modelo lógico como el físico, lo que aseguró una estructura robusta y eficiente para el almacenamiento de los datos. La manipulación de los datos se abordó de manera integral, desde la conexión con la base de datos hasta la persistencia de los resultados estadísticos. Se diseñó e implementó la comunicación con el pedal motorizado y la implementación de la interfaz gráfica para la representación de los datos EMG. Se definieron los escenarios de entrenamiento para las modalidades Ligero y Clínico, asegurando una cobertura completa de las necesidad de entrenamiento del usuario. Por último en el ámbito estadístico se desarrolló una serie de gráficos para el seguimiento de los resultados en las rutinas de entrenamiento.   
    
\end{thesischapter}

% DATA MANIPULATION
\begin{thesischapter}{2} {Diseño e implementación del Juego Serio}
En este capítulo se discuten los detalles de desarrollo de los aspectos citados en el capítulo anterior. Este comienza con una descripción y caracterización general del sistema, donde se  abordan cada uno de los componentes requeridos para su completo funcionamiento. Posteriormente se detalla la ingienría de software requerida en la etapa de conceptualización de la aplicación, se explican de forma detallada los aspectos teóricos y de implementación de la base de datos, el funcionamiento del protocolo de comunicación y por último los escenarios de juegos requeridos en las rutinas de entrenamiento ligero y clínico, y las estadísticas generadas por estos. Como herramienta de desarrollo se utilizó c\#.

% SYSTEM DESCRIPTION AND CHARACTERIZATION TO APPLY
\input{main/chapter2/section1/content.tex}
     
% SERIOUS GAME REQUIREMENTS
\input{main/chapter2/section2/content.tex}    

% USE CASE DEFINITION
\input{main/chapter2/section3/content.tex}

% USE CASE REALIZATION
\input{main/chapter2/section4/content.tex}

% DATABASE DESIGN 
\input{main/chapter2/section5/content.tex}

% DATA MANIPULATION
\input{main/chapter2/section6/content.tex}

% COMMUNICATION 
\input{main/chapter2/section7/content.tex}

% TRAINING SCENARIOS
\input{main/chapter2/section8/content.tex}

% EMG GRAPHIC
\input{main/chapter2/section9/content.tex}

% STATICAL REPORTS GENERATION
\input{main/chapter2/section10/content.tex}

\subthesischapter{Conclusiones del capítulo}
Se presentó una descripción del sistema de adquisición de datos para rehabilitación, sus componentes, características distintivas y su funcionamiento. Se identificaron y definieron los requisitos del juego  serio, tanto funcionales como no funcionales, así como los actores y casos de usos del sistema que establecieron las bases fundamentales para el desarrollo de la aplicación. Se realizó el diseño de la base de datos, abarcando tanto el modelo lógico como el físico, lo que aseguró una estructura robusta y eficiente para el almacenamiento de los datos. La manipulación de los datos se abordó de manera integral, desde la conexión con la base de datos hasta la persistencia de los resultados estadísticos. Se diseñó e implementó la comunicación con el pedal motorizado y la implementación de la interfaz gráfica para la representación de los datos EMG. Se definieron los escenarios de entrenamiento para las modalidades Ligero y Clínico, asegurando una cobertura completa de las necesidad de entrenamiento del usuario. Por último en el ámbito estadístico se desarrolló una serie de gráficos para el seguimiento de los resultados en las rutinas de entrenamiento.   
    
\end{thesischapter}

% COMMUNICATION 
\begin{thesischapter}{2} {Diseño e implementación del Juego Serio}
En este capítulo se discuten los detalles de desarrollo de los aspectos citados en el capítulo anterior. Este comienza con una descripción y caracterización general del sistema, donde se  abordan cada uno de los componentes requeridos para su completo funcionamiento. Posteriormente se detalla la ingienría de software requerida en la etapa de conceptualización de la aplicación, se explican de forma detallada los aspectos teóricos y de implementación de la base de datos, el funcionamiento del protocolo de comunicación y por último los escenarios de juegos requeridos en las rutinas de entrenamiento ligero y clínico, y las estadísticas generadas por estos. Como herramienta de desarrollo se utilizó c\#.

% SYSTEM DESCRIPTION AND CHARACTERIZATION TO APPLY
\input{main/chapter2/section1/content.tex}
     
% SERIOUS GAME REQUIREMENTS
\input{main/chapter2/section2/content.tex}    

% USE CASE DEFINITION
\input{main/chapter2/section3/content.tex}

% USE CASE REALIZATION
\input{main/chapter2/section4/content.tex}

% DATABASE DESIGN 
\input{main/chapter2/section5/content.tex}

% DATA MANIPULATION
\input{main/chapter2/section6/content.tex}

% COMMUNICATION 
\input{main/chapter2/section7/content.tex}

% TRAINING SCENARIOS
\input{main/chapter2/section8/content.tex}

% EMG GRAPHIC
\input{main/chapter2/section9/content.tex}

% STATICAL REPORTS GENERATION
\input{main/chapter2/section10/content.tex}

\subthesischapter{Conclusiones del capítulo}
Se presentó una descripción del sistema de adquisición de datos para rehabilitación, sus componentes, características distintivas y su funcionamiento. Se identificaron y definieron los requisitos del juego  serio, tanto funcionales como no funcionales, así como los actores y casos de usos del sistema que establecieron las bases fundamentales para el desarrollo de la aplicación. Se realizó el diseño de la base de datos, abarcando tanto el modelo lógico como el físico, lo que aseguró una estructura robusta y eficiente para el almacenamiento de los datos. La manipulación de los datos se abordó de manera integral, desde la conexión con la base de datos hasta la persistencia de los resultados estadísticos. Se diseñó e implementó la comunicación con el pedal motorizado y la implementación de la interfaz gráfica para la representación de los datos EMG. Se definieron los escenarios de entrenamiento para las modalidades Ligero y Clínico, asegurando una cobertura completa de las necesidad de entrenamiento del usuario. Por último en el ámbito estadístico se desarrolló una serie de gráficos para el seguimiento de los resultados en las rutinas de entrenamiento.   
    
\end{thesischapter}

% TRAINING SCENARIOS
\begin{thesischapter}{2} {Diseño e implementación del Juego Serio}
En este capítulo se discuten los detalles de desarrollo de los aspectos citados en el capítulo anterior. Este comienza con una descripción y caracterización general del sistema, donde se  abordan cada uno de los componentes requeridos para su completo funcionamiento. Posteriormente se detalla la ingienría de software requerida en la etapa de conceptualización de la aplicación, se explican de forma detallada los aspectos teóricos y de implementación de la base de datos, el funcionamiento del protocolo de comunicación y por último los escenarios de juegos requeridos en las rutinas de entrenamiento ligero y clínico, y las estadísticas generadas por estos. Como herramienta de desarrollo se utilizó c\#.

% SYSTEM DESCRIPTION AND CHARACTERIZATION TO APPLY
\input{main/chapter2/section1/content.tex}
     
% SERIOUS GAME REQUIREMENTS
\input{main/chapter2/section2/content.tex}    

% USE CASE DEFINITION
\input{main/chapter2/section3/content.tex}

% USE CASE REALIZATION
\input{main/chapter2/section4/content.tex}

% DATABASE DESIGN 
\input{main/chapter2/section5/content.tex}

% DATA MANIPULATION
\input{main/chapter2/section6/content.tex}

% COMMUNICATION 
\input{main/chapter2/section7/content.tex}

% TRAINING SCENARIOS
\input{main/chapter2/section8/content.tex}

% EMG GRAPHIC
\input{main/chapter2/section9/content.tex}

% STATICAL REPORTS GENERATION
\input{main/chapter2/section10/content.tex}

\subthesischapter{Conclusiones del capítulo}
Se presentó una descripción del sistema de adquisición de datos para rehabilitación, sus componentes, características distintivas y su funcionamiento. Se identificaron y definieron los requisitos del juego  serio, tanto funcionales como no funcionales, así como los actores y casos de usos del sistema que establecieron las bases fundamentales para el desarrollo de la aplicación. Se realizó el diseño de la base de datos, abarcando tanto el modelo lógico como el físico, lo que aseguró una estructura robusta y eficiente para el almacenamiento de los datos. La manipulación de los datos se abordó de manera integral, desde la conexión con la base de datos hasta la persistencia de los resultados estadísticos. Se diseñó e implementó la comunicación con el pedal motorizado y la implementación de la interfaz gráfica para la representación de los datos EMG. Se definieron los escenarios de entrenamiento para las modalidades Ligero y Clínico, asegurando una cobertura completa de las necesidad de entrenamiento del usuario. Por último en el ámbito estadístico se desarrolló una serie de gráficos para el seguimiento de los resultados en las rutinas de entrenamiento.   
    
\end{thesischapter}

% EMG GRAPHIC
\begin{thesischapter}{2} {Diseño e implementación del Juego Serio}
En este capítulo se discuten los detalles de desarrollo de los aspectos citados en el capítulo anterior. Este comienza con una descripción y caracterización general del sistema, donde se  abordan cada uno de los componentes requeridos para su completo funcionamiento. Posteriormente se detalla la ingienría de software requerida en la etapa de conceptualización de la aplicación, se explican de forma detallada los aspectos teóricos y de implementación de la base de datos, el funcionamiento del protocolo de comunicación y por último los escenarios de juegos requeridos en las rutinas de entrenamiento ligero y clínico, y las estadísticas generadas por estos. Como herramienta de desarrollo se utilizó c\#.

% SYSTEM DESCRIPTION AND CHARACTERIZATION TO APPLY
\input{main/chapter2/section1/content.tex}
     
% SERIOUS GAME REQUIREMENTS
\input{main/chapter2/section2/content.tex}    

% USE CASE DEFINITION
\input{main/chapter2/section3/content.tex}

% USE CASE REALIZATION
\input{main/chapter2/section4/content.tex}

% DATABASE DESIGN 
\input{main/chapter2/section5/content.tex}

% DATA MANIPULATION
\input{main/chapter2/section6/content.tex}

% COMMUNICATION 
\input{main/chapter2/section7/content.tex}

% TRAINING SCENARIOS
\input{main/chapter2/section8/content.tex}

% EMG GRAPHIC
\input{main/chapter2/section9/content.tex}

% STATICAL REPORTS GENERATION
\input{main/chapter2/section10/content.tex}

\subthesischapter{Conclusiones del capítulo}
Se presentó una descripción del sistema de adquisición de datos para rehabilitación, sus componentes, características distintivas y su funcionamiento. Se identificaron y definieron los requisitos del juego  serio, tanto funcionales como no funcionales, así como los actores y casos de usos del sistema que establecieron las bases fundamentales para el desarrollo de la aplicación. Se realizó el diseño de la base de datos, abarcando tanto el modelo lógico como el físico, lo que aseguró una estructura robusta y eficiente para el almacenamiento de los datos. La manipulación de los datos se abordó de manera integral, desde la conexión con la base de datos hasta la persistencia de los resultados estadísticos. Se diseñó e implementó la comunicación con el pedal motorizado y la implementación de la interfaz gráfica para la representación de los datos EMG. Se definieron los escenarios de entrenamiento para las modalidades Ligero y Clínico, asegurando una cobertura completa de las necesidad de entrenamiento del usuario. Por último en el ámbito estadístico se desarrolló una serie de gráficos para el seguimiento de los resultados en las rutinas de entrenamiento.   
    
\end{thesischapter}

% STATICAL REPORTS GENERATION
\begin{thesischapter}{2} {Diseño e implementación del Juego Serio}
En este capítulo se discuten los detalles de desarrollo de los aspectos citados en el capítulo anterior. Este comienza con una descripción y caracterización general del sistema, donde se  abordan cada uno de los componentes requeridos para su completo funcionamiento. Posteriormente se detalla la ingienría de software requerida en la etapa de conceptualización de la aplicación, se explican de forma detallada los aspectos teóricos y de implementación de la base de datos, el funcionamiento del protocolo de comunicación y por último los escenarios de juegos requeridos en las rutinas de entrenamiento ligero y clínico, y las estadísticas generadas por estos. Como herramienta de desarrollo se utilizó c\#.

% SYSTEM DESCRIPTION AND CHARACTERIZATION TO APPLY
\input{main/chapter2/section1/content.tex}
     
% SERIOUS GAME REQUIREMENTS
\input{main/chapter2/section2/content.tex}    

% USE CASE DEFINITION
\input{main/chapter2/section3/content.tex}

% USE CASE REALIZATION
\input{main/chapter2/section4/content.tex}

% DATABASE DESIGN 
\input{main/chapter2/section5/content.tex}

% DATA MANIPULATION
\input{main/chapter2/section6/content.tex}

% COMMUNICATION 
\input{main/chapter2/section7/content.tex}

% TRAINING SCENARIOS
\input{main/chapter2/section8/content.tex}

% EMG GRAPHIC
\input{main/chapter2/section9/content.tex}

% STATICAL REPORTS GENERATION
\input{main/chapter2/section10/content.tex}

\subthesischapter{Conclusiones del capítulo}
Se presentó una descripción del sistema de adquisición de datos para rehabilitación, sus componentes, características distintivas y su funcionamiento. Se identificaron y definieron los requisitos del juego  serio, tanto funcionales como no funcionales, así como los actores y casos de usos del sistema que establecieron las bases fundamentales para el desarrollo de la aplicación. Se realizó el diseño de la base de datos, abarcando tanto el modelo lógico como el físico, lo que aseguró una estructura robusta y eficiente para el almacenamiento de los datos. La manipulación de los datos se abordó de manera integral, desde la conexión con la base de datos hasta la persistencia de los resultados estadísticos. Se diseñó e implementó la comunicación con el pedal motorizado y la implementación de la interfaz gráfica para la representación de los datos EMG. Se definieron los escenarios de entrenamiento para las modalidades Ligero y Clínico, asegurando una cobertura completa de las necesidad de entrenamiento del usuario. Por último en el ámbito estadístico se desarrolló una serie de gráficos para el seguimiento de los resultados en las rutinas de entrenamiento.   
    
\end{thesischapter}

\subthesischapter{Conclusiones del capítulo}
Se presentó una descripción del sistema de adquisición de datos para rehabilitación, sus componentes, características distintivas y su funcionamiento. Se identificaron y definieron los requisitos del juego  serio, tanto funcionales como no funcionales, así como los actores y casos de usos del sistema que establecieron las bases fundamentales para el desarrollo de la aplicación. Se realizó el diseño de la base de datos, abarcando tanto el modelo lógico como el físico, lo que aseguró una estructura robusta y eficiente para el almacenamiento de los datos. La manipulación de los datos se abordó de manera integral, desde la conexión con la base de datos hasta la persistencia de los resultados estadísticos. Se diseñó e implementó la comunicación con el pedal motorizado y la implementación de la interfaz gráfica para la representación de los datos EMG. Se definieron los escenarios de entrenamiento para las modalidades Ligero y Clínico, asegurando una cobertura completa de las necesidad de entrenamiento del usuario. Por último en el ámbito estadístico se desarrolló una serie de gráficos para el seguimiento de los resultados en las rutinas de entrenamiento.   
    
\end{thesischapter}

% STATICAL REPORTS GENERATION
\begin{thesischapter}{2} {Diseño e implementación del Juego Serio}
En este capítulo se discuten los detalles de desarrollo de los aspectos citados en el capítulo anterior. Este comienza con una descripción y caracterización general del sistema, donde se  abordan cada uno de los componentes requeridos para su completo funcionamiento. Posteriormente se detalla la ingienría de software requerida en la etapa de conceptualización de la aplicación, se explican de forma detallada los aspectos teóricos y de implementación de la base de datos, el funcionamiento del protocolo de comunicación y por último los escenarios de juegos requeridos en las rutinas de entrenamiento ligero y clínico, y las estadísticas generadas por estos. Como herramienta de desarrollo se utilizó c\#.

% SYSTEM DESCRIPTION AND CHARACTERIZATION TO APPLY
\begin{thesischapter}{2} {Diseño e implementación del Juego Serio}
En este capítulo se discuten los detalles de desarrollo de los aspectos citados en el capítulo anterior. Este comienza con una descripción y caracterización general del sistema, donde se  abordan cada uno de los componentes requeridos para su completo funcionamiento. Posteriormente se detalla la ingienría de software requerida en la etapa de conceptualización de la aplicación, se explican de forma detallada los aspectos teóricos y de implementación de la base de datos, el funcionamiento del protocolo de comunicación y por último los escenarios de juegos requeridos en las rutinas de entrenamiento ligero y clínico, y las estadísticas generadas por estos. Como herramienta de desarrollo se utilizó c\#.

% SYSTEM DESCRIPTION AND CHARACTERIZATION TO APPLY
\input{main/chapter2/section1/content.tex}
     
% SERIOUS GAME REQUIREMENTS
\input{main/chapter2/section2/content.tex}    

% USE CASE DEFINITION
\input{main/chapter2/section3/content.tex}

% USE CASE REALIZATION
\input{main/chapter2/section4/content.tex}

% DATABASE DESIGN 
\input{main/chapter2/section5/content.tex}

% DATA MANIPULATION
\input{main/chapter2/section6/content.tex}

% COMMUNICATION 
\input{main/chapter2/section7/content.tex}

% TRAINING SCENARIOS
\input{main/chapter2/section8/content.tex}

% EMG GRAPHIC
\input{main/chapter2/section9/content.tex}

% STATICAL REPORTS GENERATION
\input{main/chapter2/section10/content.tex}

\subthesischapter{Conclusiones del capítulo}
Se presentó una descripción del sistema de adquisición de datos para rehabilitación, sus componentes, características distintivas y su funcionamiento. Se identificaron y definieron los requisitos del juego  serio, tanto funcionales como no funcionales, así como los actores y casos de usos del sistema que establecieron las bases fundamentales para el desarrollo de la aplicación. Se realizó el diseño de la base de datos, abarcando tanto el modelo lógico como el físico, lo que aseguró una estructura robusta y eficiente para el almacenamiento de los datos. La manipulación de los datos se abordó de manera integral, desde la conexión con la base de datos hasta la persistencia de los resultados estadísticos. Se diseñó e implementó la comunicación con el pedal motorizado y la implementación de la interfaz gráfica para la representación de los datos EMG. Se definieron los escenarios de entrenamiento para las modalidades Ligero y Clínico, asegurando una cobertura completa de las necesidad de entrenamiento del usuario. Por último en el ámbito estadístico se desarrolló una serie de gráficos para el seguimiento de los resultados en las rutinas de entrenamiento.   
    
\end{thesischapter}
     
% SERIOUS GAME REQUIREMENTS
\begin{thesischapter}{2} {Diseño e implementación del Juego Serio}
En este capítulo se discuten los detalles de desarrollo de los aspectos citados en el capítulo anterior. Este comienza con una descripción y caracterización general del sistema, donde se  abordan cada uno de los componentes requeridos para su completo funcionamiento. Posteriormente se detalla la ingienría de software requerida en la etapa de conceptualización de la aplicación, se explican de forma detallada los aspectos teóricos y de implementación de la base de datos, el funcionamiento del protocolo de comunicación y por último los escenarios de juegos requeridos en las rutinas de entrenamiento ligero y clínico, y las estadísticas generadas por estos. Como herramienta de desarrollo se utilizó c\#.

% SYSTEM DESCRIPTION AND CHARACTERIZATION TO APPLY
\input{main/chapter2/section1/content.tex}
     
% SERIOUS GAME REQUIREMENTS
\input{main/chapter2/section2/content.tex}    

% USE CASE DEFINITION
\input{main/chapter2/section3/content.tex}

% USE CASE REALIZATION
\input{main/chapter2/section4/content.tex}

% DATABASE DESIGN 
\input{main/chapter2/section5/content.tex}

% DATA MANIPULATION
\input{main/chapter2/section6/content.tex}

% COMMUNICATION 
\input{main/chapter2/section7/content.tex}

% TRAINING SCENARIOS
\input{main/chapter2/section8/content.tex}

% EMG GRAPHIC
\input{main/chapter2/section9/content.tex}

% STATICAL REPORTS GENERATION
\input{main/chapter2/section10/content.tex}

\subthesischapter{Conclusiones del capítulo}
Se presentó una descripción del sistema de adquisición de datos para rehabilitación, sus componentes, características distintivas y su funcionamiento. Se identificaron y definieron los requisitos del juego  serio, tanto funcionales como no funcionales, así como los actores y casos de usos del sistema que establecieron las bases fundamentales para el desarrollo de la aplicación. Se realizó el diseño de la base de datos, abarcando tanto el modelo lógico como el físico, lo que aseguró una estructura robusta y eficiente para el almacenamiento de los datos. La manipulación de los datos se abordó de manera integral, desde la conexión con la base de datos hasta la persistencia de los resultados estadísticos. Se diseñó e implementó la comunicación con el pedal motorizado y la implementación de la interfaz gráfica para la representación de los datos EMG. Se definieron los escenarios de entrenamiento para las modalidades Ligero y Clínico, asegurando una cobertura completa de las necesidad de entrenamiento del usuario. Por último en el ámbito estadístico se desarrolló una serie de gráficos para el seguimiento de los resultados en las rutinas de entrenamiento.   
    
\end{thesischapter}    

% USE CASE DEFINITION
\begin{thesischapter}{2} {Diseño e implementación del Juego Serio}
En este capítulo se discuten los detalles de desarrollo de los aspectos citados en el capítulo anterior. Este comienza con una descripción y caracterización general del sistema, donde se  abordan cada uno de los componentes requeridos para su completo funcionamiento. Posteriormente se detalla la ingienría de software requerida en la etapa de conceptualización de la aplicación, se explican de forma detallada los aspectos teóricos y de implementación de la base de datos, el funcionamiento del protocolo de comunicación y por último los escenarios de juegos requeridos en las rutinas de entrenamiento ligero y clínico, y las estadísticas generadas por estos. Como herramienta de desarrollo se utilizó c\#.

% SYSTEM DESCRIPTION AND CHARACTERIZATION TO APPLY
\input{main/chapter2/section1/content.tex}
     
% SERIOUS GAME REQUIREMENTS
\input{main/chapter2/section2/content.tex}    

% USE CASE DEFINITION
\input{main/chapter2/section3/content.tex}

% USE CASE REALIZATION
\input{main/chapter2/section4/content.tex}

% DATABASE DESIGN 
\input{main/chapter2/section5/content.tex}

% DATA MANIPULATION
\input{main/chapter2/section6/content.tex}

% COMMUNICATION 
\input{main/chapter2/section7/content.tex}

% TRAINING SCENARIOS
\input{main/chapter2/section8/content.tex}

% EMG GRAPHIC
\input{main/chapter2/section9/content.tex}

% STATICAL REPORTS GENERATION
\input{main/chapter2/section10/content.tex}

\subthesischapter{Conclusiones del capítulo}
Se presentó una descripción del sistema de adquisición de datos para rehabilitación, sus componentes, características distintivas y su funcionamiento. Se identificaron y definieron los requisitos del juego  serio, tanto funcionales como no funcionales, así como los actores y casos de usos del sistema que establecieron las bases fundamentales para el desarrollo de la aplicación. Se realizó el diseño de la base de datos, abarcando tanto el modelo lógico como el físico, lo que aseguró una estructura robusta y eficiente para el almacenamiento de los datos. La manipulación de los datos se abordó de manera integral, desde la conexión con la base de datos hasta la persistencia de los resultados estadísticos. Se diseñó e implementó la comunicación con el pedal motorizado y la implementación de la interfaz gráfica para la representación de los datos EMG. Se definieron los escenarios de entrenamiento para las modalidades Ligero y Clínico, asegurando una cobertura completa de las necesidad de entrenamiento del usuario. Por último en el ámbito estadístico se desarrolló una serie de gráficos para el seguimiento de los resultados en las rutinas de entrenamiento.   
    
\end{thesischapter}

% USE CASE REALIZATION
\begin{thesischapter}{2} {Diseño e implementación del Juego Serio}
En este capítulo se discuten los detalles de desarrollo de los aspectos citados en el capítulo anterior. Este comienza con una descripción y caracterización general del sistema, donde se  abordan cada uno de los componentes requeridos para su completo funcionamiento. Posteriormente se detalla la ingienría de software requerida en la etapa de conceptualización de la aplicación, se explican de forma detallada los aspectos teóricos y de implementación de la base de datos, el funcionamiento del protocolo de comunicación y por último los escenarios de juegos requeridos en las rutinas de entrenamiento ligero y clínico, y las estadísticas generadas por estos. Como herramienta de desarrollo se utilizó c\#.

% SYSTEM DESCRIPTION AND CHARACTERIZATION TO APPLY
\input{main/chapter2/section1/content.tex}
     
% SERIOUS GAME REQUIREMENTS
\input{main/chapter2/section2/content.tex}    

% USE CASE DEFINITION
\input{main/chapter2/section3/content.tex}

% USE CASE REALIZATION
\input{main/chapter2/section4/content.tex}

% DATABASE DESIGN 
\input{main/chapter2/section5/content.tex}

% DATA MANIPULATION
\input{main/chapter2/section6/content.tex}

% COMMUNICATION 
\input{main/chapter2/section7/content.tex}

% TRAINING SCENARIOS
\input{main/chapter2/section8/content.tex}

% EMG GRAPHIC
\input{main/chapter2/section9/content.tex}

% STATICAL REPORTS GENERATION
\input{main/chapter2/section10/content.tex}

\subthesischapter{Conclusiones del capítulo}
Se presentó una descripción del sistema de adquisición de datos para rehabilitación, sus componentes, características distintivas y su funcionamiento. Se identificaron y definieron los requisitos del juego  serio, tanto funcionales como no funcionales, así como los actores y casos de usos del sistema que establecieron las bases fundamentales para el desarrollo de la aplicación. Se realizó el diseño de la base de datos, abarcando tanto el modelo lógico como el físico, lo que aseguró una estructura robusta y eficiente para el almacenamiento de los datos. La manipulación de los datos se abordó de manera integral, desde la conexión con la base de datos hasta la persistencia de los resultados estadísticos. Se diseñó e implementó la comunicación con el pedal motorizado y la implementación de la interfaz gráfica para la representación de los datos EMG. Se definieron los escenarios de entrenamiento para las modalidades Ligero y Clínico, asegurando una cobertura completa de las necesidad de entrenamiento del usuario. Por último en el ámbito estadístico se desarrolló una serie de gráficos para el seguimiento de los resultados en las rutinas de entrenamiento.   
    
\end{thesischapter}

% DATABASE DESIGN 
\begin{thesischapter}{2} {Diseño e implementación del Juego Serio}
En este capítulo se discuten los detalles de desarrollo de los aspectos citados en el capítulo anterior. Este comienza con una descripción y caracterización general del sistema, donde se  abordan cada uno de los componentes requeridos para su completo funcionamiento. Posteriormente se detalla la ingienría de software requerida en la etapa de conceptualización de la aplicación, se explican de forma detallada los aspectos teóricos y de implementación de la base de datos, el funcionamiento del protocolo de comunicación y por último los escenarios de juegos requeridos en las rutinas de entrenamiento ligero y clínico, y las estadísticas generadas por estos. Como herramienta de desarrollo se utilizó c\#.

% SYSTEM DESCRIPTION AND CHARACTERIZATION TO APPLY
\input{main/chapter2/section1/content.tex}
     
% SERIOUS GAME REQUIREMENTS
\input{main/chapter2/section2/content.tex}    

% USE CASE DEFINITION
\input{main/chapter2/section3/content.tex}

% USE CASE REALIZATION
\input{main/chapter2/section4/content.tex}

% DATABASE DESIGN 
\input{main/chapter2/section5/content.tex}

% DATA MANIPULATION
\input{main/chapter2/section6/content.tex}

% COMMUNICATION 
\input{main/chapter2/section7/content.tex}

% TRAINING SCENARIOS
\input{main/chapter2/section8/content.tex}

% EMG GRAPHIC
\input{main/chapter2/section9/content.tex}

% STATICAL REPORTS GENERATION
\input{main/chapter2/section10/content.tex}

\subthesischapter{Conclusiones del capítulo}
Se presentó una descripción del sistema de adquisición de datos para rehabilitación, sus componentes, características distintivas y su funcionamiento. Se identificaron y definieron los requisitos del juego  serio, tanto funcionales como no funcionales, así como los actores y casos de usos del sistema que establecieron las bases fundamentales para el desarrollo de la aplicación. Se realizó el diseño de la base de datos, abarcando tanto el modelo lógico como el físico, lo que aseguró una estructura robusta y eficiente para el almacenamiento de los datos. La manipulación de los datos se abordó de manera integral, desde la conexión con la base de datos hasta la persistencia de los resultados estadísticos. Se diseñó e implementó la comunicación con el pedal motorizado y la implementación de la interfaz gráfica para la representación de los datos EMG. Se definieron los escenarios de entrenamiento para las modalidades Ligero y Clínico, asegurando una cobertura completa de las necesidad de entrenamiento del usuario. Por último en el ámbito estadístico se desarrolló una serie de gráficos para el seguimiento de los resultados en las rutinas de entrenamiento.   
    
\end{thesischapter}

% DATA MANIPULATION
\begin{thesischapter}{2} {Diseño e implementación del Juego Serio}
En este capítulo se discuten los detalles de desarrollo de los aspectos citados en el capítulo anterior. Este comienza con una descripción y caracterización general del sistema, donde se  abordan cada uno de los componentes requeridos para su completo funcionamiento. Posteriormente se detalla la ingienría de software requerida en la etapa de conceptualización de la aplicación, se explican de forma detallada los aspectos teóricos y de implementación de la base de datos, el funcionamiento del protocolo de comunicación y por último los escenarios de juegos requeridos en las rutinas de entrenamiento ligero y clínico, y las estadísticas generadas por estos. Como herramienta de desarrollo se utilizó c\#.

% SYSTEM DESCRIPTION AND CHARACTERIZATION TO APPLY
\input{main/chapter2/section1/content.tex}
     
% SERIOUS GAME REQUIREMENTS
\input{main/chapter2/section2/content.tex}    

% USE CASE DEFINITION
\input{main/chapter2/section3/content.tex}

% USE CASE REALIZATION
\input{main/chapter2/section4/content.tex}

% DATABASE DESIGN 
\input{main/chapter2/section5/content.tex}

% DATA MANIPULATION
\input{main/chapter2/section6/content.tex}

% COMMUNICATION 
\input{main/chapter2/section7/content.tex}

% TRAINING SCENARIOS
\input{main/chapter2/section8/content.tex}

% EMG GRAPHIC
\input{main/chapter2/section9/content.tex}

% STATICAL REPORTS GENERATION
\input{main/chapter2/section10/content.tex}

\subthesischapter{Conclusiones del capítulo}
Se presentó una descripción del sistema de adquisición de datos para rehabilitación, sus componentes, características distintivas y su funcionamiento. Se identificaron y definieron los requisitos del juego  serio, tanto funcionales como no funcionales, así como los actores y casos de usos del sistema que establecieron las bases fundamentales para el desarrollo de la aplicación. Se realizó el diseño de la base de datos, abarcando tanto el modelo lógico como el físico, lo que aseguró una estructura robusta y eficiente para el almacenamiento de los datos. La manipulación de los datos se abordó de manera integral, desde la conexión con la base de datos hasta la persistencia de los resultados estadísticos. Se diseñó e implementó la comunicación con el pedal motorizado y la implementación de la interfaz gráfica para la representación de los datos EMG. Se definieron los escenarios de entrenamiento para las modalidades Ligero y Clínico, asegurando una cobertura completa de las necesidad de entrenamiento del usuario. Por último en el ámbito estadístico se desarrolló una serie de gráficos para el seguimiento de los resultados en las rutinas de entrenamiento.   
    
\end{thesischapter}

% COMMUNICATION 
\begin{thesischapter}{2} {Diseño e implementación del Juego Serio}
En este capítulo se discuten los detalles de desarrollo de los aspectos citados en el capítulo anterior. Este comienza con una descripción y caracterización general del sistema, donde se  abordan cada uno de los componentes requeridos para su completo funcionamiento. Posteriormente se detalla la ingienría de software requerida en la etapa de conceptualización de la aplicación, se explican de forma detallada los aspectos teóricos y de implementación de la base de datos, el funcionamiento del protocolo de comunicación y por último los escenarios de juegos requeridos en las rutinas de entrenamiento ligero y clínico, y las estadísticas generadas por estos. Como herramienta de desarrollo se utilizó c\#.

% SYSTEM DESCRIPTION AND CHARACTERIZATION TO APPLY
\input{main/chapter2/section1/content.tex}
     
% SERIOUS GAME REQUIREMENTS
\input{main/chapter2/section2/content.tex}    

% USE CASE DEFINITION
\input{main/chapter2/section3/content.tex}

% USE CASE REALIZATION
\input{main/chapter2/section4/content.tex}

% DATABASE DESIGN 
\input{main/chapter2/section5/content.tex}

% DATA MANIPULATION
\input{main/chapter2/section6/content.tex}

% COMMUNICATION 
\input{main/chapter2/section7/content.tex}

% TRAINING SCENARIOS
\input{main/chapter2/section8/content.tex}

% EMG GRAPHIC
\input{main/chapter2/section9/content.tex}

% STATICAL REPORTS GENERATION
\input{main/chapter2/section10/content.tex}

\subthesischapter{Conclusiones del capítulo}
Se presentó una descripción del sistema de adquisición de datos para rehabilitación, sus componentes, características distintivas y su funcionamiento. Se identificaron y definieron los requisitos del juego  serio, tanto funcionales como no funcionales, así como los actores y casos de usos del sistema que establecieron las bases fundamentales para el desarrollo de la aplicación. Se realizó el diseño de la base de datos, abarcando tanto el modelo lógico como el físico, lo que aseguró una estructura robusta y eficiente para el almacenamiento de los datos. La manipulación de los datos se abordó de manera integral, desde la conexión con la base de datos hasta la persistencia de los resultados estadísticos. Se diseñó e implementó la comunicación con el pedal motorizado y la implementación de la interfaz gráfica para la representación de los datos EMG. Se definieron los escenarios de entrenamiento para las modalidades Ligero y Clínico, asegurando una cobertura completa de las necesidad de entrenamiento del usuario. Por último en el ámbito estadístico se desarrolló una serie de gráficos para el seguimiento de los resultados en las rutinas de entrenamiento.   
    
\end{thesischapter}

% TRAINING SCENARIOS
\begin{thesischapter}{2} {Diseño e implementación del Juego Serio}
En este capítulo se discuten los detalles de desarrollo de los aspectos citados en el capítulo anterior. Este comienza con una descripción y caracterización general del sistema, donde se  abordan cada uno de los componentes requeridos para su completo funcionamiento. Posteriormente se detalla la ingienría de software requerida en la etapa de conceptualización de la aplicación, se explican de forma detallada los aspectos teóricos y de implementación de la base de datos, el funcionamiento del protocolo de comunicación y por último los escenarios de juegos requeridos en las rutinas de entrenamiento ligero y clínico, y las estadísticas generadas por estos. Como herramienta de desarrollo se utilizó c\#.

% SYSTEM DESCRIPTION AND CHARACTERIZATION TO APPLY
\input{main/chapter2/section1/content.tex}
     
% SERIOUS GAME REQUIREMENTS
\input{main/chapter2/section2/content.tex}    

% USE CASE DEFINITION
\input{main/chapter2/section3/content.tex}

% USE CASE REALIZATION
\input{main/chapter2/section4/content.tex}

% DATABASE DESIGN 
\input{main/chapter2/section5/content.tex}

% DATA MANIPULATION
\input{main/chapter2/section6/content.tex}

% COMMUNICATION 
\input{main/chapter2/section7/content.tex}

% TRAINING SCENARIOS
\input{main/chapter2/section8/content.tex}

% EMG GRAPHIC
\input{main/chapter2/section9/content.tex}

% STATICAL REPORTS GENERATION
\input{main/chapter2/section10/content.tex}

\subthesischapter{Conclusiones del capítulo}
Se presentó una descripción del sistema de adquisición de datos para rehabilitación, sus componentes, características distintivas y su funcionamiento. Se identificaron y definieron los requisitos del juego  serio, tanto funcionales como no funcionales, así como los actores y casos de usos del sistema que establecieron las bases fundamentales para el desarrollo de la aplicación. Se realizó el diseño de la base de datos, abarcando tanto el modelo lógico como el físico, lo que aseguró una estructura robusta y eficiente para el almacenamiento de los datos. La manipulación de los datos se abordó de manera integral, desde la conexión con la base de datos hasta la persistencia de los resultados estadísticos. Se diseñó e implementó la comunicación con el pedal motorizado y la implementación de la interfaz gráfica para la representación de los datos EMG. Se definieron los escenarios de entrenamiento para las modalidades Ligero y Clínico, asegurando una cobertura completa de las necesidad de entrenamiento del usuario. Por último en el ámbito estadístico se desarrolló una serie de gráficos para el seguimiento de los resultados en las rutinas de entrenamiento.   
    
\end{thesischapter}

% EMG GRAPHIC
\begin{thesischapter}{2} {Diseño e implementación del Juego Serio}
En este capítulo se discuten los detalles de desarrollo de los aspectos citados en el capítulo anterior. Este comienza con una descripción y caracterización general del sistema, donde se  abordan cada uno de los componentes requeridos para su completo funcionamiento. Posteriormente se detalla la ingienría de software requerida en la etapa de conceptualización de la aplicación, se explican de forma detallada los aspectos teóricos y de implementación de la base de datos, el funcionamiento del protocolo de comunicación y por último los escenarios de juegos requeridos en las rutinas de entrenamiento ligero y clínico, y las estadísticas generadas por estos. Como herramienta de desarrollo se utilizó c\#.

% SYSTEM DESCRIPTION AND CHARACTERIZATION TO APPLY
\input{main/chapter2/section1/content.tex}
     
% SERIOUS GAME REQUIREMENTS
\input{main/chapter2/section2/content.tex}    

% USE CASE DEFINITION
\input{main/chapter2/section3/content.tex}

% USE CASE REALIZATION
\input{main/chapter2/section4/content.tex}

% DATABASE DESIGN 
\input{main/chapter2/section5/content.tex}

% DATA MANIPULATION
\input{main/chapter2/section6/content.tex}

% COMMUNICATION 
\input{main/chapter2/section7/content.tex}

% TRAINING SCENARIOS
\input{main/chapter2/section8/content.tex}

% EMG GRAPHIC
\input{main/chapter2/section9/content.tex}

% STATICAL REPORTS GENERATION
\input{main/chapter2/section10/content.tex}

\subthesischapter{Conclusiones del capítulo}
Se presentó una descripción del sistema de adquisición de datos para rehabilitación, sus componentes, características distintivas y su funcionamiento. Se identificaron y definieron los requisitos del juego  serio, tanto funcionales como no funcionales, así como los actores y casos de usos del sistema que establecieron las bases fundamentales para el desarrollo de la aplicación. Se realizó el diseño de la base de datos, abarcando tanto el modelo lógico como el físico, lo que aseguró una estructura robusta y eficiente para el almacenamiento de los datos. La manipulación de los datos se abordó de manera integral, desde la conexión con la base de datos hasta la persistencia de los resultados estadísticos. Se diseñó e implementó la comunicación con el pedal motorizado y la implementación de la interfaz gráfica para la representación de los datos EMG. Se definieron los escenarios de entrenamiento para las modalidades Ligero y Clínico, asegurando una cobertura completa de las necesidad de entrenamiento del usuario. Por último en el ámbito estadístico se desarrolló una serie de gráficos para el seguimiento de los resultados en las rutinas de entrenamiento.   
    
\end{thesischapter}

% STATICAL REPORTS GENERATION
\begin{thesischapter}{2} {Diseño e implementación del Juego Serio}
En este capítulo se discuten los detalles de desarrollo de los aspectos citados en el capítulo anterior. Este comienza con una descripción y caracterización general del sistema, donde se  abordan cada uno de los componentes requeridos para su completo funcionamiento. Posteriormente se detalla la ingienría de software requerida en la etapa de conceptualización de la aplicación, se explican de forma detallada los aspectos teóricos y de implementación de la base de datos, el funcionamiento del protocolo de comunicación y por último los escenarios de juegos requeridos en las rutinas de entrenamiento ligero y clínico, y las estadísticas generadas por estos. Como herramienta de desarrollo se utilizó c\#.

% SYSTEM DESCRIPTION AND CHARACTERIZATION TO APPLY
\input{main/chapter2/section1/content.tex}
     
% SERIOUS GAME REQUIREMENTS
\input{main/chapter2/section2/content.tex}    

% USE CASE DEFINITION
\input{main/chapter2/section3/content.tex}

% USE CASE REALIZATION
\input{main/chapter2/section4/content.tex}

% DATABASE DESIGN 
\input{main/chapter2/section5/content.tex}

% DATA MANIPULATION
\input{main/chapter2/section6/content.tex}

% COMMUNICATION 
\input{main/chapter2/section7/content.tex}

% TRAINING SCENARIOS
\input{main/chapter2/section8/content.tex}

% EMG GRAPHIC
\input{main/chapter2/section9/content.tex}

% STATICAL REPORTS GENERATION
\input{main/chapter2/section10/content.tex}

\subthesischapter{Conclusiones del capítulo}
Se presentó una descripción del sistema de adquisición de datos para rehabilitación, sus componentes, características distintivas y su funcionamiento. Se identificaron y definieron los requisitos del juego  serio, tanto funcionales como no funcionales, así como los actores y casos de usos del sistema que establecieron las bases fundamentales para el desarrollo de la aplicación. Se realizó el diseño de la base de datos, abarcando tanto el modelo lógico como el físico, lo que aseguró una estructura robusta y eficiente para el almacenamiento de los datos. La manipulación de los datos se abordó de manera integral, desde la conexión con la base de datos hasta la persistencia de los resultados estadísticos. Se diseñó e implementó la comunicación con el pedal motorizado y la implementación de la interfaz gráfica para la representación de los datos EMG. Se definieron los escenarios de entrenamiento para las modalidades Ligero y Clínico, asegurando una cobertura completa de las necesidad de entrenamiento del usuario. Por último en el ámbito estadístico se desarrolló una serie de gráficos para el seguimiento de los resultados en las rutinas de entrenamiento.   
    
\end{thesischapter}

\subthesischapter{Conclusiones del capítulo}
Se presentó una descripción del sistema de adquisición de datos para rehabilitación, sus componentes, características distintivas y su funcionamiento. Se identificaron y definieron los requisitos del juego  serio, tanto funcionales como no funcionales, así como los actores y casos de usos del sistema que establecieron las bases fundamentales para el desarrollo de la aplicación. Se realizó el diseño de la base de datos, abarcando tanto el modelo lógico como el físico, lo que aseguró una estructura robusta y eficiente para el almacenamiento de los datos. La manipulación de los datos se abordó de manera integral, desde la conexión con la base de datos hasta la persistencia de los resultados estadísticos. Se diseñó e implementó la comunicación con el pedal motorizado y la implementación de la interfaz gráfica para la representación de los datos EMG. Se definieron los escenarios de entrenamiento para las modalidades Ligero y Clínico, asegurando una cobertura completa de las necesidad de entrenamiento del usuario. Por último en el ámbito estadístico se desarrolló una serie de gráficos para el seguimiento de los resultados en las rutinas de entrenamiento.   
    
\end{thesischapter}

\subthesischapter{Conclusiones del capítulo}
Se presentó una descripción del sistema de adquisición de datos para rehabilitación, sus componentes, características distintivas y su funcionamiento. Se identificaron y definieron los requisitos del juego  serio, tanto funcionales como no funcionales, así como los actores y casos de usos del sistema que establecieron las bases fundamentales para el desarrollo de la aplicación. Se realizó el diseño de la base de datos, abarcando tanto el modelo lógico como el físico, lo que aseguró una estructura robusta y eficiente para el almacenamiento de los datos. La manipulación de los datos se abordó de manera integral, desde la conexión con la base de datos hasta la persistencia de los resultados estadísticos. Se diseñó e implementó la comunicación con el pedal motorizado y la implementación de la interfaz gráfica para la representación de los datos EMG. Se definieron los escenarios de entrenamiento para las modalidades Ligero y Clínico, asegurando una cobertura completa de las necesidad de entrenamiento del usuario. Por último en el ámbito estadístico se desarrolló una serie de gráficos para el seguimiento de los resultados en las rutinas de entrenamiento.   
    
\end{thesischapter}

% CONCLUSIONS
\subthesischapter{Conclusiones del capítulo}
En este capítulo se expusieron los requerimientos de software y hardware del sistema, junto con una evaluación exhaustiva del impacto social de la aplicación. Se presentó meticulosamente los datos estadísticos relacionados con la prevalencia de discapacidades físicas el cual ha permitido una comprensión profunda del contexto en el que se implementará el sistema, resaltando su relevancia y necesidad en la sociedad actual. Se realizó una descripción minuciosa del flujo de trabajo estándar, respaldada por representaciones visuales claras. Los resultados obtenidos de estas evaluaciones arrojan luz sobre el cumplimiento de los objetivos planteados. Además, las valiosas observaciones proporcionadas por los especialistas han revelado posibles aplicaciones adicionales del sistema en el proceso de rehabilitación. Estas aportaciones cualitativas han enriquecido la comprensión del potencial de la aplicación, destacando su versatilidad y capacidad para adaptarse a diversas necesidades dentro del ámbito de la rehabilitación.

\end{thesischapter}