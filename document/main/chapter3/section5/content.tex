\subthesischapter{Impacto socio-económico}

Se considera un impacto social cuando surge un cambio favorable en la vida de la población. Esto puede incluso extenderse a generaciones futuras. Por tanto, los tipos más destacados suelen ser los relacionados con cambios en el nivel educativo, el estado de salud o el nivel de ingresos. En este caso, el software está orientado principalmente a mejorar la salud, sin embargo, también hay un fuerte componente económico. Teniendo esto en cuenta, se podría decir que el proyecto de integración del software a un sistema de rehabilitación tiene una propuesta de valor ciertamente robusta. Por un lado y como aspecto relevante de impacto es el potencial beneficio que puede brindar a las personas que han sufrido un ACV. 

Las estimaciones más recientes de la GDB (Global Burden of Disease por sus siglas en inglés) en 2019 muestran que el ictus sigue siendo la segunda causa principal de muerte y la tercera causa de muerte y discapacidad combinadas en el mundo. El coste mundial estimado para tratar el ictus es superior a los \$891 billones de dólares (1,12\% del PIB mundial). Entre 1990 y 2019, la carga (en términos de número absoluto de casos) aumentó sustancialmente (70,0\% de aumento de los ictus incidentes, 43,0\% de muertes por ictus, 102,0\% de ictus prevalentes)~\cite{feigin2022world}. Además, según las estadísticas descritas por la Organización Mundial de Accidentes Cerebrovasculares~\cite{lindsay2019world}, cada año se producen más de 12,2 millones de nuevos accidentes cerebrovasculares. En todo el mundo, una de cada cuatro personas mayores de 25 años sufrirá un ictus a lo largo de su vida. Cada año, más del 16\% de todos los accidentes cerebrovasculares se producen en personas de 15 a 49 años. Cada año, más del 62\% de todos los accidentes cerebrovasculares se producen en personas menores de 70 años. Cada año, el 47\% de los ictus se producen en hombres y el 53\% en mujeres. Estas cifras muestran la magnitud del problema de salud pública que representa el ACV, así como la necesidad de contar con estrategias de prevención, tratamiento y rehabilitación eficaces y accesibles. 

Por otra parte la aplicación de juego serio, al integrarse al sistema de rehabilitación muscular por pedal motorizado, ofrece una solución innovadora y atractiva para mejorar la recuperación funcional y la calidad de vida de las personas que han sufrido un ACV. Al estimular la actividad física, la motivación, el entretenimiento y el aprendizaje, la aplicación puede contribuir a reducir el impacto negativo del ACV en la salud, la economía y la sociedad, al ser un recurso de bajo costo que permite un ahorro sustancial en costes a la institución médica que lo utilice y por ende mayor capacidad de rehabilitación. Además esta mejora no se remite solamente a personas que han sufrido un ACV, cualquier persona con problemas de movilidad podría usarlo como método de rehabilitación. Los beneficios de los videojuegos para la rehabilitación más importantes que se han conseguido hasta la fecha son: mejoras en la marcha, resistencia, equilibrio, coordinación, fortalecimiento de los músculos y entrenamiento de habilidades cognitivas. La actividad física en combinación con los videojuegos es una estrategia factible y efectiva para fortalecer los miembros superiores e inferiores del cuerpo, la flexibilidad, el equilibrio y la agilidad.  La inserción de nuevas tecnologías en ambientes realistas en los que se realiza la actividad física permite, además, dinamizar la relación entre el ejercicio y los resultados.  

La utilidad de los videojuegos desde su concepción es la de servir como herramienta para hacer ejercicio desde la comodidad del hogar. Con la implementación de esta tecnología no solo se mejora la salud, sino también se logra el mantenimiento físico de los adultos mayores, en pro de prolongar una existencia con calidad, demostrando incrementos en el consumo de oxígeno, gasto energético, frecuencia cardíaca y tasa de esfuerzo percibido lo cual deriva en beneficios importantes para la salud y en un impacto para contrarrestar el sedentarismo y sus enfermedades derivadas. Si a esto se le añade el indudable atractivo para los jóvenes, por tratarse de un videojuego, y la saciedad en la curiosidad de los adultos por probar una tecnología novedosa, podría llegar a ser un producto muy solicitado. Por tanto, se obtiene un producto comercializable integrado a un sistema de rehabilitación por pedal motorizado, con valor agregado, centrado en mejorar la salud de las personas mientras mantiene un costo relativamente pequeño en comparación con el mercado mundial, que puede oscilar entre los \$10,000 a \$200,000 según \cite{Costodeh3}.