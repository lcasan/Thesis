\begin{thesischapter}{1} {Marco teórico}
    En este capítulo se explican algunos de los conceptos claves utilizados en este
    trabajo. Además, se presenta un resumido estado del arte y las herramientas que
    dan solución a algunos de los aspectos más importantes en el sistema y se exponen
    las principales características de estas que son de utilidad. Al final del capítulo se
    hace un breve resumen de la metodología de trabajo para realizar el sistema.

    %% Conceptos teóricos
    \subthesischapter{Aspectos conceptuales en la Rehabilitación Muscular}
    
    \vspace{10pt}
    \subsubthesischapter{Accidentes Cerebrovasculares}
    Se define como accidente cerebrovascular (ACV) o Ictus a todo episodio de
    instauración súbita, aguda o subaguda, en el que, a causa de una lesión primaria o
    secundaria localizable en cualquier punto del sistema cardiovascular, se produce un
    déficit neurológico, permanente o transitorio, en relación con la zona afectada.~\cite{ictus}

    \vspace{10pt}
    \subsubthesischapter{Tipos de afecciones motoras y cognitiva}
    Los tipos y grados de discapacidad que siguen a un derrame dependen de qué área
    del cerebro esté dañada fig: \ref{fig: cerebralcortex}. Generalmente, el accidente cerebrovascular puede
    causar cinco tipos de discapacidades: parálisis o problemas para controlar el movimiento;
    Trastornos sensoriales, incluyendo dolor; Problemas en el uso o comprensión del
    lenguaje; Problemas con el pensamiento y la memoria; Y trastornos emocionales.~\cite{post-strok} 
    \begin{figure}[ht]
        \centering
        \includegraphics[scale=0.5]{images/brain.jpg}
        \caption{Áreas funcionales de la corteza cerebral}
        Tomado de: ~\cite{areacereabral}
        \label{fig: cerebralcortex}
    \end{figure}

    \begin{itemize}
        \item Parálisis o problemas para controlar el movimiento: \\
        La parálisis es una de las discapacidades más comunes causadas por un accidente
        cerebrovascular. Se suele dar en el lado del cuerpo opuesto al lado del cerebro dañado y
        puede afectar a la cara, un brazo, una pierna o a todo un lado del cuerpo. Esta parálisis
        unilateral se denomina hemiplejia (si la parálisis no es completa se denomina
        hemiparesia). Los pacientes con hemiparesia o hemiplejia pueden tener dificultades con
        las actividades cotidianas, como caminar o agarrar objetos.~\cite{cuidadosalpacienteadulto}
        \item Alteraciones sensoriales, incluyendo dolor:\\
        Los pacientes con accidente cerebrovascular pueden perder la capacidad de sentir
        el tacto, dolor, temperatura o posición. También pueden tener dificultad para reconocer
        objetos que están sosteniendo, e incluso pueden ser lo suficientemente graves como para
        causar la pérdida del reconocimiento de su propia extremidad. Algunos pacientes con
        apoplejía llegan a experimentan dolor, entumecimiento o sensaciones extrañas de
        hormigueo o picor en las extremidades paralizadas o debilitadas, un síntoma conocido
        como parestesias.~\cite{post-strok}
        \item Problemas para usar o comprender el lenguaje (afasia):\\
        Los sobrevivientes de accidentes cerebrovasculares experimentan deficiencias en el lenguaje, 
        que involucran la capacidad de hablar, escribir y comprender el lenguaje hablado y escrito. 
        En las personas diestras, estos accidentes cerebrovasculares generalmente involucran el lado 
        izquierdo del cerebro. Una lesión inducida por un accidente cerebrovascular en cualquiera de 
        los centros de control del lenguaje del cerebro puede afectar gravemente la comunicación verbal. 
        Hay varios tipos de afasia:~\cite{post-strok}
        \begin{itemize}
            \item afasia expresiva , en la que las personas pierden la capacidad de hablar o escribir las palabras que están pensando y de juntar palabras en oraciones coherentes y gramaticalmente correctas.
            \item afasia receptiva , en la que las personas tienen dificultad para comprender el lenguaje hablado o escrito y, a menudo, tienen un habla incoherente. Aunque estos individuos pueden formar oraciones gramaticalmente correctas, sus expresiones a menudo carecen de significado.
            \item afasia global , en la que las personas pierden casi todas sus habilidades lingüísticas; no pueden entender el lenguaje o usarlo para transmitir pensamientos.  
        \end{itemize}
        \item Problemas con el pensamiento y la memoria:\\
        Un accidente cerebrovascular puede causar daño a partes del cerebro responsables
        de la memoria, el aprendizaje y la conciencia. Los pacientes pueden haber reducido
        dramáticamente su capacidad de atención o pueden experimentar déficit en la memoria a
        corto plazo. También pueden perder su capacidad de hacer planes, comprender el
        significado de las cosas, aprender nuevas tareas, o participar en otras actividades mentales
        complejas.~\cite{post-strok}
        \item Trastornos emocionales:\\
        Muchas personas tras sobrevivir a un derrame sienten miedo, ansiedad,
        frustración, ira, tristeza y un sentimiento de dolor por sus pérdidas físicas y mentales.
        Estos sentimientos son una respuesta natural al trauma psicológico del ictus. Algunos
        trastornos emocionales y cambios de personalidad son causados por los efectos físicos
        del daño cerebral.~\cite{post-strok} 
    \end{itemize}

    \vspace{10pt}
    \subsubthesischapter{Activación muscular durante el pedaleo}
    Muchos estudios de ciclismo han utilizado las señales de EMGs para comprender mejor
    cómo funcionan los músculos durante las diferentes fases del movimiento. Una de las
    particularidades observadas ocurre durante la fase de propulsión del pedaleo, donde la
    mayoría de los pares de músculos agonistas/antagonistas se activan juntos para generar
    torques durante esta fase. El análisis está basado en los tiempos de inicio y final de la
    contracción muscular y los niveles de amplitud alcanzados, sincronizados con mediciones
    cinemáticas o dinámicas del movimiento como el ángulo, la aceleración o la velocidad
    angular, torque, potencia o de marcadores de tiempo con el uso de codificadores en la
    manivela de la bicicleta o del pedal motorizado. Con ellos se ha demostrado que las
    personas sanas tienen patrones de coordinación intermusculares predecibles durante el
    pedaleo. Los cuales también están relacionados con la cadencia o revoluciones por
    minutos, la posición del cuerpo, así como con la carga de entrenamiento a la cual es
    sometido el usuario. Estos patrones suelen ser alterados en pacientes con
    discapacidades motoras debido a cambios que sufren los músculos, como el
    acortamiento ~\cite{johnston2007biomechanical}. De ahí que la señales de EMGs permiten 
    detectar los grupos musculares dañados y enfocar el entrenamiento hacia estos 
    ~\cite{hug2009electromyographic, kautz1998relationships}

    \vspace{10pt}
    \subsubthesischapter{Tecnologías de la Rehabilitación}
    El término $"$tecnología de rehabilitación$"$ remite a las herramientas que ayudan a las personas 
    a recuperar sus funciones después de una lesión o una enfermedad. Las "tecnologías de asistencia" 
    pueden ser tan simples como una lupa para mejorar la visión, o tan complejas como un sistema de 
    comunicación digital. Algunas de estas tecnologías son

    \vspace{10pt}
    \textbf{Rutinas de Rehabilitación} \\ 

    \vspace{10pt}
    \textbf{Modalidades de Rehabilitación} \\ 
    La terapia con dispositivos róboticos de pedaleo se clasifica, al igual que las terapias
    convencionales, en modo pasivo, activo y activo asistido [19]. Estos diferentes enfoques
    se preescriben según el grado de las secuelas que padece el paciente, el nivel de
    conciencia de este y las diferentes fases del proceso de rehabilitación en que se
    encuentran. El ciclismo con estos fines consiste que los usuarios ejecuten movimientos
    circulares, similares a los realizados en una bicicleta.

    \vspace{5pt}
    En el ciclismo activo, la persona al pisar o agarrar el pedal es capaz de continuar el
    ejercicio por sí solo. Durante esta modalidad, el paciente debe ir venciendo gradualmente
    las resistencias opuestas al movimiento. De esta manera el tratamiento es denominado
    como resistido y su objetivo principal es el fortalecimiento de los músculos involucrados.
    En el modo pasivo, los miembros afectados siguen la trayectoria y velocidad programada
    para los pedales [20]. Este modo es también empleado en la rehabilitación activa asistida,
    la cual es aplicada a los pacientes que debido a la disminución de la fuerza muscular,
    inician el movimiento activo pero no son capaces de recorrer todo el arco articular [21].
    En este caso las tecnologías de pedaleo permiten la transición de forma automática o
    manual del entrenamiento activo al pasivo y viceversa.

    \vspace{5pt}
    El ciclismo pasivo es adecuado al principio de la rehabilitación, pues contribuye a la
    regulación del tono muscular, relajación de la musculatura y prevención de las dificultades
    que trae consigo estar inactivo o inmovilizado, disminuyendo así las probabilidades de
    complicaciones a largo plazo [21]. El inicio temprano de intervenciones de ejercicios de
    pedaleo desde la cama en pacientes críticamente enfermos puede ayudar a reducir la
    sarcopenia y recuperar la fuerza muscular [22]. También es indicado para reducir la
    inflamación, aliviar el dolor y recuperar el rango de movimiento. Además, se ha
    comprobado un mejoramiento en el sistema cardiovascular y respiratorio después de la
    terapia pasiva [21, 23].
    Esta modalidad es particularmente útil en pacientes con déficit severo que no tienen
    control o potencia motriz suficiente para participar en movimientos activos, por ejemplo,
    sujetos con LM completa [23, 24]. Phadke et al [23] reflejaron en una revisión sistemática
    notables efectos del ciclismo pasivo en miembros inferiores para personas con esta
    condición. En las intervenciones clínicas analizadas por estos autores, los efectos
    positivos fueron más evidentes tras la realización de múltiples sesiones. Entre los efectos
    cardiovasculares, se observó un incremento del flujo sanguíneo en las piernas con la
    consecuente disminución de la resistencia vascular periférica. En el sistema
    músculoesquelético, hubo un mantenimiento e incremento del RDM de las articulaciones
    y una atenuación de la atrofia muscular. Además, fue comprobada la influencia de los
    ejercicios pasivos en el SNC, donde los indicadores mostraron una disminución
    significativa de la espasticidad (evaluada tras la medición de la amplitud del reflejo H) y
    de la inhibición intracortical. En los pacientes con EM, el ciclismo pasivo también ha
    producido estos efectos antiespásticos [25, 26]. Todas estas repercusiones generalmente
    son apreciadas a partir de la realización de pedaleo activo [27], sobre todo en sujetos
    sanos, de ahí la importancia de conocer que el pedaleo pasivo también produzca efectos
    similares.

    \vspace{5pt}
    El ciclismo activo-asistido ha producido mejoras en la función motora de los pacientes
    con EP, particularmente el temblor y la bradicinesia [28, 29]. Esta modalidad es
    aprovechada para su uso con terapias complementarias como la estimulación eléctrica
    funcional (EEF). Esta técnica ha sido considerada desde hace varios años como un
    método bien establecido y estandarizada en la rehabilitación de pacientes con
    enfermedades neurológicas como las ACV [30]. El ciclismo asistido por EEF es un
    enfoque utilizado con fines de re-habilitación que contribuye a restaurar el trofismo
    muscular y aumentar la fuerza muscular [3, 31]. El principio de funcionamento se basa
    en estimular las unidades motoras para activar los músculos paralizados o paréticos
    durante la realización de tareas funcionales. Diferentes estudios han evaluado la eficacia
    de la EEF combinada con el ciclismo en la recuperación motora, principalmente en
    pacientes con ACV [32, 33], LM [34] y EM [35]. También se ha evidenciado beneficios
    positivos sobre la capacidad aeróbica máxima, el control postural, la espasticidad y la
    coordinación motora [3, 30, 36, 37].
    


    \subthesischapter{Aspectos conceptuales en sistemas informáticos.}
    \vspace{10pt}
    \subsubthesischapter{Arquitectura Cliente-Servidor}
    Según ~\cite{moyano2020arquitectura} la arquitectura Cliente - Servidor, es 
    un modelo de una aplicación distribuida en el cual se basa en dos actores:
    Uno con rol de proveedor de recursos y otro con rol consultor sobre los recursos.
    \begin{itemize}
        \item Cliente: Programa ejecutable que participa activamente en el establecimiento de las conexiones. Envía una petición al servidor y se queda
        esperando por una respuesta. Su tiempo de vida es finito una vez que son
        servidas sus solicitudes, termina el trabajo.
        \item Servidor: Es un programa que ofrece un servicio que se puede obtener
        en una red. Acepta la petición desde la red, realiza el servicio y devuelve
        el resultado al solicitante. Al ser posible implantarlo como aplicaciones de
        programas, puede ejecutarse en cualquier sistema donde exista TCP/IP y
        junto con otros programas de aplicación. El servidor comienza su ejecución
        antes de comenzar la interacción con el cliente.
    \end{itemize}

    \vspace{10pt}
    \subsubthesischapter{Motores Gráficos}

    \vspace{2pt}
    Un motor de videojuegos es una aplicación de software que ofrece
    todas las herramientas necesarias para el diseño y desarrollo completo de un videojuego, disponiendo
    de un motor de renderizado para gráficos 2D y 3D, detector de colisiones, sonidos, scripting,
    animación, inteligencia artificial, redes, streaming, administración de memoria y mucho más.~\cite{arce2011desarrollo} 
    
    \vspace{10pt}
    \subsubthesischapter{Protocolos de Comunicación TCP/IP}
    TCP/IP es un conjunto de protocolos que especifican estándares de comunicaciones entre sistemas y detallan los convenios 
    para el direccionamiento y la interconexión de redes. TCP/IP permite las comunicaciones entre varios sistemas (llamados 
    sistemas principales) conectados en una red. A su vez, cada red puede estar conectada a otra para comunicarse con los 
    sistemas principales de dicha red. Aunque existen muchos tipos de tecnologías de red, muchas de las cuales utilizan el 
    transporte en modalidad continua y por conmutación de paquetes, TCP/IP ofrece una ventaja importante: la independencia 
    de hardware ~\cite{protocolo-tcp-ip}.

    \vspace{10pt}
    \subsubthesischapter{Base de Datos}
    Una Base de Datos (BD) consiste en una colección organizada de datos. La creación de una
    base de datos tiene ventajas en cuanto a la mejor disponibilidad y mayor eficiencia en la
    entrada de los datos en el sistema, mayor valor informativo, mejor documentación de la
    información, acceso más rápido y sencillo a los usuarios finales, mayores facilidades para
    compartir los datos y flexibilidad para atender a demandas cambiantes [32]

    \vspace{10pt}
    \textbf{Sistemas de gestión de Bases de Datos}\\
    Un Sistema de Gestión de Bases de Datos (SGBD) es el software que permite el control y la
    utilización y/o la actualización de los datos almacenados en una (o varias) base(s) de dato(s).
    Los datos son almacenados en tablas, las cuales están organizadas en registros (filas) y
    campos (columnas). Los sistemas actuales se le denominan relacionales, pues utilizan tablas
    de datos relacionadas por un campo en común [32]. Un Sistema de Bases de Datos (SBD)
    representa la interrelación entre una base de datos y el SGBD que la administra, permitiendo
    a los usuarios acceder y modificar estos datos o archivos. Uno de los propósitos principales
    de un sistema de bases de datos es proporcionar a los usuarios una visión abstracta de los
    datos. Es decir, el sistema esconde ciertos detalles de cómo se almacenan y mantienen los
    datos [44].
    Entre los SGBD relacionales más modernos se encuentran [33]:
    • MySQL: Es un sistema relacional de código abierto, multihilo y multiusuario. Código
    abierto significa que todo el mundo puede acceder a1 código fuente, es decir, a1 código de
    programación del SGBD, en este caso MySQL [34]. Presenta como ventajas principales: alta
    velocidad al realizar las operaciones, bajo costo en requerimientos para la elaboración de
    bases de dato, facilidad de configuración, instalación, usabilidad y administración. MySQL
    puede ejecutarse en la inmensa mayoría de sistemas operativos y tiene compatibilidad en su
    mayor parte con los lenguajes de programación ANSI C y C++.
    • Microsoft SQL Server: Es un sistema basado en el lenguaje Transact-SQL, capaz de poner
    a disposición de muchos usuarios grandes cantidades de datos de manera simultánea. Es un
    sistema propietario de Microsoft. Sus principales características son: soporte de
    transacciones, alta escalabilidad, estabilidad y seguridad. Incluye también un potente entorno
    gráfico de administración. Su principal desventaja es el precio, aunque cuenta con una
    versión EXPRESS que permite usarlo en entornos pequeños.
    • Oracle: Es un sistema multiplataforma fabricado por Oracle Corporation. Tradicionalmente
    ha sido el SGBD por excelencia, considerado como el más completo y robusto; destacado
    por su soporte de transacciones, alta estabilidad y escalabilidad. También siempre ha sido
    considerado de los más caros, por lo que no se ha estandarizado su uso como otras
    aplicaciones. Al igual que SQL Server, Oracle cuenta con una versión EXPRESS gratis para
    pequeñas instalaciones o usuarios personales.
    • Microsoft Access: Es un sistema creado por Microsoft para uso personal de pequeñas
    organizaciones. Una posibilidad adicional es la de crear ficheros con bases de datos que
    pueden ser consultados por otros programas. Este SGBD permite crear tablas de datos
    indexadas, modificar tablas de datos, creación de consultas, formularios, informes, vistas de
    diseño y consultas referencias cruzadas.
    • PostgreSQL: Es uno de los líderes de los sistemas de gestión de base de datos (SGBD)
    relacionales de código abierto. Este gestor de BD es uno de los almacenes de datos más rápido
    y potente del mercado debido a sus características avanzadas (extensibilidad, seguridad y
    estabilidad). Dado a que no tiene restricción en su entrada a datos, usa multiprocesos en vez
    de multihilos para garantizar la seguridad del sistema, un fallo en uno de los procesos no
    afectará el resto y el sistema continuará funcionando. Ofrece, al igual que MySQL, un sistema
    de contraseñas y privilegios seguros mediante verificación basada en el host, y el tráfico de
    contraseñas está cifrado al conectarse a un servidor. Además, este gestor de BD soporta gran
    cantidad y variedad de datos y tiene compatibilidad en su mayor parte con los lenguajes de
    programación Java, C y C++, entre otros

    \vspace{10pt}
    \subthesischapter{Juegos Serios}
    Los videojuegos interactivos han surgido como nuevos enfoques de tratamiento en
    la rehabilitación del accidente cerebrovascular. Estos enfoques pueden ser
    ventajosos ya que dan la oportunidad de practicar actividades que no se pueden
    realizar dentro del entorno clínico. Además, los programas de realidad virtual están
    diseñados para ser más interesantes y agradables que las terapias tradicionales,
    alentando que el paciente realice un mayor número de repeticiones de los ejercicios~\cite{laver2018virtual,alfageme2002aprendiendo}

    \vspace{10pt}
    En el área de la rehabilitación %dirigida a pacientes con afecciones cerebrovasculares 
    existen diferentes contribuciones dirigidas a la recuperación y rehabilitación de las diferentes habilidades y competencias psicomotrices. 
    A continuación se presentan las propuestas más relevantes que existen actualmente usando juegos serio:
    \begin{itemize}
        \item En el 2019, ~\cite{rodriguez2019design} presentó un juego serio para propósitos de rehabilitación a personas con ACV. El juego 
        utiliza el software de desarrollo para entornos Unity 3D en su versión free. El sensor que realiza el seguimiento de los movimientos 
        de la persona es el sensor Kinect. El objetivo del juego consiste en que el usuario agarre esferas de un color específico. Los puntajes 
        alcanzados en el juego y los parámetros de configuración colocados por el kinesiólogo son almacenados junto con la información del paciente. 
        De esta manera el fisioterapeuta puede recuperar los datos almacenados y realizar un análisis de la evaluación de la evolución de la persona 
        en su rehabilitación.
        \item Alberto y Edwin Daniel ~\cite{morales2019desarrollo} proponen una aplicación de juego serio para tratar de aumentar y de evaluar 
        el límite de estabilidad de la población envejecida.Tras esta idea, se hacen uso del sensor Kinect V1 y el 
        motor de juego Unity3D, compatible mediante el uso de la librería Kinect with MS-SDK, para poder registrar los movimientos del paciente durante 
        la terapia. El juego basará su fundamento clínico en diversos test de evaluación del balance, como son el test Fugl-Meyer o el Choice Stepping 
        Reaction Time, y tomará como mecánica de referencia la de uno de los clásicos de los videojuegos, el Tetris. Blocks Rehab, la adaptación clínica 
        de este juego ochentero, consistirá en tres modalidades de juego (Bloques, Puzzle y E.T. Tris), cumpliendo cada uno de ellos una función específca. 
        El paciente deberá desplazarse de manera frontal, lateral y mantener el equilibrio sobre una pierna para llevar a cabo los objetivos que el 
        videojuego les plantea, con el propósito de extraer métri- cas de salida que evalúen la actuación del usuario en términos de estabilidad. 
        \item ~\cite{doi:10.1177/1545968314535985} y colaboradores (2015), compararon el efecto de realizar el entrenamiento con un
        soporte de brazo, combinado con ejercicios mediante un juego digital y la rehabilitación
        convencional en pacientes con ictus.\\
        El dispositivo que utiliza el soporte para el brazo se llama ArmeoBoom. Está compuesto de
        un sistema de suspensión en cabestrillo elevado, que proporciona un soporte para la
        muñeca. Este elemento proporciona un buen soporte en un espacio tridimensional
        que te permite realizar movimientos funcionales sin ninguna restricción. El mecanismo se
        adapta a la situación física de cada paciente. ArmeoBoom está compuesto por una webcam
        y un ordenador portátil que permiten al usuario interaccionar con los videojuegos
        incorporados en el ordenador portátil y jugar moviendo la extremidad afectada en un
        ambiente tridimensional ajustado al grado de movimiento funcional de cada paciente.
        Los movimientos, tanto en el plano horizontal como el plano vertical son controlados y
        grabados con la finalidad de premiar con puntos dependiendo de la actuación del usuario
        y del tiempo de ejecución.
        
        \vspace{2pt}
        Entrenar con el soporte de brazo consiste en realizar los movimientos con el brazo
        afectado, con el objetivo de maximizar la habilidad de los ejercicios usando el mínimo
        soporte de brazo posible.
        
        \vspace{2pt}
        El entrenamiento de rehabilitación convencional consiste en la realización de unos
        ejercicios ejecutados con los brazos, dirigidos por el terapeuta, para reflejar la terapia física
        y ocupacional.
        El objetivo de este entrenamiento es que el paciente incremente el rango de movimiento
        del brazo, principalmente del hombro y el codo con el mínimo soporte posible de una
        superficie como una mesa. Todos los ejercicios son análogos en esencia y consisten en alcanzar 
        objetos colocados en una mesa, en mover o apilar vasos, colocar discos o transportar bloques de 
        pinzas sin la ayuda de ningún ordenador ni soporte informático. Conforme el paciente va superando los juegos
        propuestos, los movimientos se van complicando un poco más.

        \vspace{2pt}
        Un total de 70 pacientes que habían padecido un ictus hemorrágico o isquémico en las
        últimas 12 semanas y que estaban estables con su medicación fueron elegidos para la
        rehabilitación. Los participantes pasaron por un programa de entrenamiento de seis
        semanas, divididas en tres sesiones de 30 minutos cada una.
        
        \vspace{2pt}
        Un grupo de 35 pacientes se formó para la rehabilitación de las extremidades superiores
        en combinación con el soporte de brazo y el ejercicio, mientras que 33 pacientes realizaron
        la rehabilitación con los ejercicios convencionales.
        
        \item En el 2014  autores como ~\cite{10.3233/NRE-141105}, comparan el efecto
        que se produce al entrenar con un videojuego de realidad virtual Kinect Xbox 360 Paddle
        Panic Mini Game, en pacientes que han padecido un ictus. El objetivo es mejorar el
        movimiento de la extremidad superior, observando el hemisferio cerebral afectado de
        estos pacientes.\\
        La evaluación se lleva a cabo mediante la acción de beber un vaso de agua antes y después
        de entrenar

        \vspace{5pt}
        Se seleccionaron un total de 40 participantes para realizar el estudio. El grupo inicial se
        subdividió en cuatro grupos atendiendo a los siguientes criterios:
        \begin{itemize}
            \item Pacientes con afectación derecha del cerebro (hemiparesia izquierda)
            \item Pacientes con afectación izquierda del cerebro (hemiparesia derecha)
            \item Pacientes sanos que entrenaron con la extremidad superior derecha en el videojuego
            \item Pacientes sanos que entrenaron con la extremidad superior izquierda en el videojuego
        \end{itemize}
        Los resultados del estudio muestran que los pacientes con la afectación de la extremidad
        superior por ictus presentaron una mejoría del entrenamiento con el videojuego,
        demostrado un incremento de la puntuación en cada jugada.
    \end{itemize}

    \vspace{5pt}
    Las terapias con juegos serios son una alternativa complementaria a las terapias
    tradicionales. Tanto los estudios que se apoyan en tecnologías de realidad virtual, como Kinect y
    Nintendo, las cuales no tienen propósitos clínicos, como los que se centran en juegos
    serios diseñados específicamente para la rehabilitación física de diferentes patologías,
    indican que estos juegos tienen el potencial de mejorar el equilibrio, el control postural y
    la condición física de los pacientes, sin embargo los estudios son pocos y demasiado
    pequeños para llegar a una conclusión completamente fiable. La falta de eventos adversos
    reportados (como mareos, dolor de cabeza o náuseas) sugiere que este enfoque de la
    terapia es relativamente seguro, aunque esto puede variar dependiendo de las
    características de la persona, el hardware y el software de la realidad virtual y la tarea. 

    %%-----------------------------------------------------------
    \subthesischapter{Tecnologías y herramientas}
    \subsubthesischapter{Motor gráfico(Unity3D)}
    Actualmente podemos encontrar una amplia gama de motores de videojuegos, con diferentes tipos de
    licencias y orientados a cumplir distintos tipos de propósitos. Se puede encontrar motores comerciales
    y gratuitos, con metodologías 2D o 3D, inclusive que brindan soluciones de juegos a variadas
    plataformas (Windows, Linux, android, etc.).\\ 
    Afortunadamente existe una nueva tendencia por parte de algunas empresas que desarrollan motores de 
    videojuegos, que los impulsa a colocar en el mercado versiones gratuitas (generalmente limitadas en 
    algún aspecto) de sus herramientas para que las personas que deseen aprender, puedan hacerlo sin tener 
    que comprar una versión full. Entre estas empresas podemos encontrar las conocidas Epic Games
    y Unity Technologies, las cuales ofrecen versiones gratuitas y descargables de sus famosos motores de videojuegos.~\cite{arce2011desarrollo} 
    
    \begin{center}
        Diferencias generales ~\cite{chavez2021diferencias}\\
        \includegraphics[height=5in]{images/unityvsunreal.png}
    \end{center}
    
    Según lo antes expuesto podemos concluir en que los dos motores de videojuegos son pieza indispensable a la hora del desarrollo
    de videojuego tanto en 2d, 3d o RV, cada uno de los motores poseen características únicas que los diferencian entre sí.
    
    \vspace{2pt}
    Para el proposito de la tesis considero Unity como unos de los motores mas populares a la hora del desarrollo
    de videojuegos, para las distintas plataformas, una gran ventaja que tiene unity a la hora de
    desarrollar un videojuego para Android, es su modo de configurar el entorno, porque solo se necesita descargar los módulos de 
    Android, y ya está listo para crear juegos.
     
    \vspace{5pt}
    Principales características de este motor gráfico que son de utilidad en el sistema:~\cite{unity3d}
    \begin{itemize}
        \item Programación con C\#.
        \item Soporte parcial de .NET (incluye soporte a .NET Sockets).
        \item Soporte de plugins para código nativo.
        \item Incluye el motor físico PhysX de Nvidia.
        \item Carga de modelados y texturas de varios formatos de programas externos como Blender, Maya, Adobe Suite, 3D Max, Cinema 4D, entre otros.
        \item Despliegue gratis sobre Android.
        \item Inspector para clases personalizadas en tiempo de ejecución.
        \item Animación a través de cinemática directa e inversa.
    \end{itemize}

    \subsubthesischapter{Visual Studio Code}
    Visual Studio Code es un editor de código fuente desarrollado por Microsoft para
    Windows, Linux, macOS y Web. Incluye soporte para la depuración, control
    integrado de Git, resaltado de sintaxis, finalización inteligente de código, fragmentos
    y refactorización de código, características que lo convierten en una herramienta
    perfecta para manipular los script en unity.~\cite{vscode}
    \subsubthesischapter{NET Framework 4.7}
    El framework .NET provee un modelo de programación global que permite el
    desarrollo de todo tipo de aplicaciones desde móviles a web a escritorio12.
    Técnicamente el framework .NET es un ambiente de ejecución que administra las
    aplicaciones que corren sobre este. En él se brinda un conjunto extensivo de
    bibliotecas para dar solución a las principales áreas del desarrollo de software.\\
    El framework en sí consiste de dos grandes componentes: motor de ejecución CLR
    (por sus siglas del inglés common language runtime) el cuál se encarga de la
    ejecución de las aplicaciones; y la biblioteca de clases (nombre en inglés .NET
    Framework Class Library)~\cite{netframework}

    \vspace{2pt}
    Características principales:
    \begin{itemize}
        \item Manejo de memoria automático.
        \item Sistema común de tipado. O sea, los tipos de datos primitivos son definidos por el framework, lo cual facilita las operaciones entre distintos lenguajes que usan este ambiente de ejecución.
        \item Una biblioteca bien vasta para operaciones generales de bajo nivel.
        \item Operatividad entre lenguajes. O sea, se puede acceder a rutinas y bibliotecas (dinámicas y estáticas) escritas en otros lenguajes si fueron compiladas con compiladores para esos lenguajes que soporten el framework .NET.
        \item Gran número de lenguajes disponibles: Visual Basic, C\#, Visual F\#, C++, entre otros.
        \item Es posible crear compilados que funcionen en múltiples plataformas de la .NET.
    \end{itemize}

    \subthesischapter{Metodología de trabajo}
    Existen diferentes modelos y metodologías que han sido utilizados en los últimos años como 
    herramientas de apoyo para el desarrollo del software. La interrogante principal está en 
    conocer cuál modelo utilizar en el proceso de desarrollo de software en un proyecto.~\cite{DELGADOOLIVERA2021}
    
    \vspace{2pt}
    Para el desarrollo de nuestra aplicación se hará uso del modelo RAD que permite la construcción de software basada en
    módulos, utilizando herramientas de software que permitan de forma ágil y efectiva realizar una
    aplicación con altos estándares de calidad en un corto período de tiempo. Esto se debe a que un sistema completamente integrado e interdependiente entre cada una de sus
    partes era muy difícil de adaptar a los distintos cambios de requisitos. Además,
    el diseño modular del mismo brinda la posibilidad de migrar el sistema a otras
    tecnología distintas o más modernas, con relativa facilidad.
    
    \vspace{2pt}
    El sistema se desarrollará de manera iterativa, donde en cada iteración se
    realizará un acercamiento a una versión más definitiva de un componente o
    módulo en específico. De esta forma siempre habrá una sección del producto
    parcialmente terminada que podía ser mostrada, para adaptarla a las
    sugerencia y peticiones de los usuarios.

    \subthesischapter{Conclusiones del capítulo}
\end{thesischapter}