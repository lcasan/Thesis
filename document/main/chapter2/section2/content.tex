\subthesischapter{Requisitos del Juego Serio}
Un requisito es una restricción que el sistema debe cumplir. Tiene como propósito expresar un comportamiento o propiedad que el sistema debe poseer~\cite{jacobson2000uml}. El objetivo esencial de un requisito es el de definir un comportamiento específico de manera clara y comprensible. Pueden clasificarse en funcionales y no funcionales.

\subsubthesischapter{Requisitos funcionales}
La integración de tecnologías de pedaleo en el campo de la rehabilitación neuromuscular ha abierto nuevas oportunidades para mejorar la calidad y eficacia de las terapias. En particular, los juegos serios han demostrado ser una herramienta efectiva para motivar a los pacientes durante el proceso de rehabilitación. Tomando como punto de partida lo anteriormente abordado se hace necesario el desarrollo de una aplicación android que cumpliera con los siguientes requerimientos funcionales:    

\begin{enumerate}
    \item Iniciar sesión en el sistema
    \item Cerrar sesión en el sistema
    \item Registrarse en el sistema
    \item Eliminar cuenta de usuario
    \item Editar los datos del usuario
    \item Enviar datos al pedal
    \item Recibir datos del pedal
    \item Establecer comunicación con el pedal
    \item Finalizar comunicación con el pedal
    \item Configurar rutina
    \item Iniciar rutina
    \item Pausar o reanudar el rutina
    \item Finalizar rutina
    \item Mostrar rutina
    \item Mostrar estadísticas rutina
    \item Calibrar sistema
    \item Guardar registro de comportamiento en archivo \textbf{logs}\footnote{Archivo de texto en el que constan cronológicamente los acontecimientos que han ido afectando a un sistema informático}
    \item Mostrar notificaciones 
\end{enumerate}


\subsubthesischapter{Requisitos no funcionales}
Los requisitos no funcionales son propiedades o cualidades que el producto debe tener. Enuncian los requisitos del sistema que no pueden ser expresados como funcionalidades en respuesta a alguna acción de un usuario. En nuestro caso se identificaron los siguientes requisitos no funcionales:
        
\begin{itemize}
    \item \underline{Requisitos de usabilidad}: La interfaz de trabajo del usuario con la herramienta deberá ser simple, interactiva e intuitiva y deberá notificar los diferentes errores que se produzcan durante su ejecución.
    \item \underline{Requisitos de escalabilidad}: La aplicación debería ser capaz de escalar para manejar un aumento en el número de usuarios o la cantidad de datos sin perder rendimiento.     
    \item \underline{Interoperabilidad}: Si la aplicación interactúa con otros sistemas, debe ser capaz de hacerlo de manera eficiente y sin errores significativos. En otras palabras, no debería haber interrupciones, fallos, o malentendidos en la comunicación entre la aplicación y los sistemas con los que interactúa.
\end{itemize}

