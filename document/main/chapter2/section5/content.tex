\newpage
\subthesischapter{Diseño de la base de datos}
Existen dos tipos prominentes de sistemas de gestión de bases de datos ampliamente reconocidos: los conocido como Lenguaje de Consulta Estructurado (SQL, por sus siglas en inglés) organizan los datos en estructuras tabulares, mientras que los otros, denominado No-SQL, prescinden de las tablas y optan por almacenar datos en forma de objetos. Contrariamente a SQL, el enfoque No-SQL no sigue una estructura rígida, permitiendo mayor flexibilidad en la organización de los datos. No obstante, se ha observado que los sistemas No-SQL tienden a requerir una mayor cantidad de recursos computacionales en comparación con las implementaciones SQL. En el contexto de aplicaciones móviles, se ha preferido el uso de SQL debido a su eficiencia y menor consumo de recursos, lo que contribuye a una experiencia de usuario más fluida y eficaz.

\subsubthesischapter{Modelo lógico}
La descripción que surge de esta fase de diseño sirve como base para especificar la estructura conceptual de la base de datos. 
La siguiente lista describe los principales requisitos para el registro de datos en el sistema:
\begin{itemize}
    \item Los usuarios son reconocidos a través de un identificador único. En el sistema, se almacenan los detalles de identificación del usuario, los cuales se especifican en los requisitos de registro e incluyen el nombre completo, su dirección de correo electrónico y una contraseña asociada.
    \item Cada usuario tiene la capacidad de establecer múltiples rutinas de entrenamiento personalizadas. Estas rutinas están compuestas por un identificador exclusivo, el tipo de ejercicio elegido y los parámetros específicos que han sido configurados por el usuario. 
    \item Luego de terminada una rutina se almacena en el sistema los resultados obtenidos y la fecha de ejecución. 
\end{itemize}   
    
\textbf{Designación de los conjuntos de entidades y de relaciones}\\
Desde la especificación listada anteriormente se comienzan a
identificar los conjuntos de entidades y relaciones, así como sus atributos:
\begin{itemize}
    \item La entidad \underline{usuario}, con los atributos id\_usuario, nombre, correo, contraseña.
    \item La entidad \underline{rutina} con los atributos id\_rutina, modalidad, distancia, tiempo, \textbf{mcv\textsubscript{1}, mcv\textsubscript{2}, mcv\textsubscript{3}, mcv\textsubscript{4}}\footnote{mcv\textsubscript{i}: Valor porcentual definido de MCV en el canal i luego de la etapa de calibración.}.
    \item La relación muchos a muchos de \underline{usuario} a \underline{rutinas} denominado \underline{ejecuta}, es decir un usuario puede ejecutar 
    varias rutinas y la misma rutina puede ser ejecutada por distintos usuarios. Dicha relación cuenta con los atributos velocidad, mcv\_promedio y fecha\_de\_ejecución. 
\end{itemize}
    
\subsubthesischapter{Modelo físico}
El modelo físico de la base de datos mostrado en la Figura \ref{fig: logic-model-db} consta de tres tablas principales: \textbf{User, Rutine y Do}\footnote{User, Rutine y Do hace referencias a las tablas usuario, rutina y ejecuta respectivamente.} en las que se han definido las siguientes restricciones para mantener la integridad de los datos:

\begin{itemize}
    \item En la tabla USER, se ha establecido un campo \textbf{id} como clave primaria, que se genera automáticamente mediante autoincremento cada vez que se inserta una nueva fila. Los campos \textbf{fullname}, \textbf{email} y \textbf{password} almacenan la información del usuario.

    \item La tabla \textbf{Rutine} contiene información sobre las rutinas de ejercicios. Se ha definido un campo ID como clave primaria con autoincremento. Los campos \textbf{modality}, \textbf{time}, \textbf{distance}, \textbf{mc1}, \textbf{mc2}, \textbf{mc3} y \textbf{mc4} almacenan los detalles de la rutina.

    \item En la tabla \textbf{Do}, se ha establecido un campo \textbf{id} como clave primaria con autoincremento. Los campos \textbf{user\_id} y \textbf{rutine\_id} actúan como claves foráneas, haciendo referencia a las tablas \textbf{Rutine} y \textbf{User} respectivamente. El campo \textbf{velocity} y \textbf{mcv\_average} almacena los resultados de velocidad y el promedio de MCV obtenidos al finalizar la rutina, y el campo \textbf{date} registra la fecha de la realización de la rutina.
\end{itemize}

\begin{figure}[ht]
    \centering
    \includegraphics[scale=0.6]{images/logic-model-db.png}
    \caption{Diagrama del modelo físico de la base de datos.}
    \label{fig: logic-model-db}
\end{figure} 