\subthesischapter{Realización de los casos de uso}
La realización de los casos de uso en el análisis es una colaboración que describe cómo se lleva a cabo y ejecuta un caso de uso determinado en término de las clases del análisis y de sus objetos en interacción, por lo tanto, se centra en los requisitos funcionales. Los casos de usos del sistema se encuentran descrito en el Manual de Casos de Uso~\cite{casosdeuso}.A continuación se describe el caso de uso de mayor relevancia.

\begin{table}
    \vspace*{-60pt}
    \hspace*{-48pt}
    \begin{tabularx}{1.2\textwidth}{|X|X|}
        \hline
        \textbf{Caso de uso:} & Configurar rutina \\\hline
        \textbf{Actores:}     & Usuario \\\hline
        
        \multicolumn{2}{|X|}{        
        \begin{minipage}[t]{0.950\columnwidth}
            \textbf{Descripción:}

            Permite configurar una rutina de entrenamiento para una de las modalidades disponibles en el sistema.
        \end{minipage}} \\\hline

        \textbf{Requisitos funcionales asociados:} &  RF10\\\hline
        \textbf{Precondiciones:} & Haber iniciado sesión en el sistema y tener conectado el módulo de comunicación del dispositivo móvil a la red wifi del pedal. \\\hline
        \textbf{Poscondiciones:} & La configuración de la nueva rutina quedarán registrada en el sistema \\\hline
        
        % SECTION 1
        \multicolumn{2}{|X|}{        
        \begin{minipage}[t]{0.925\columnwidth}
            \begin{center}
                \textbf{Sección principal}
            \end{center}
        \end{minipage}} \\\hline
        
        Acción del actor & Acción del sistema \\\hline
        1. El usuario selecciona la opción de entrenamiento en la vista principal, ver anexo 3. & 2. El sistema muestra la interfaz para la selección de la modalidad. \\\hline
        3. El usuario selecciona la opción que desea:
        
        Caso <<Modalidad ligero>>: ir al curso alterno <<Modalidad ligero>>
        
        Caso <<Modalidad clínico>>: ir al curso alterno <<Modalidad clínico>> & \\\hline
        
        % SECTION 2
        \multicolumn{2}{|X|}{        
        \begin{minipage}[t]{0.925\columnwidth}
            \begin{center}
                \textbf{Sección de cursos alternos}
            \end{center}
        \end{minipage}} \\\hline
        \multicolumn{2}{|X|}{        
        \begin{minipage}[t]{0.925\columnwidth}
                \textbf{Curso alterno: Modalidad ligero}
        \end{minipage}} \\\hline
        
        Acción del actor & Acción del sistema \\\hline
        & 
        1. El sistema construye la interfaz con los parámetros generales de la modalidad seleccionada, (ver Figura \ref{annex: 3}(a)). \\\hline
        2. El usuario inserta los valores de distancia y tiempo según la dificultad del entrenamiento que desee. 
        
        3. Termina el CU. 
        & \\\hline
        \multicolumn{2}{|X|}{        
        \begin{minipage}[t]{0.925\columnwidth}
                \textbf{Curso alterno: Modalidad clínico}
        \end{minipage}} \\\hline
        
        Acción del actor & Acción del sistema \\\hline
        & 
        1. El sistema construye la interfaz correspondiente a la sección de calibración, (ver Figura \ref{annex: 3}(c)). \\\hline
        2. El usuario presiona en los botones iniciar medición, calcular línea base y calcular MCV siguiendo el protocol médico de calibración.
        
        3. El usuario presiona en el botón de modalidad clínico.
        &
        4. El sistema construye la interfaz con los parámetros generales de la modalidad seleccionada, (ver Figura \ref{annex: 3}(b)). \\\hline
        5. El usuario inserta los valores de distancia, tiempo y porciento de MCV según la dificultad del entrenamiento que desee y la calibración obtenida. 
        
        6. Termina el CU. 
        &\\\hline
    \end{tabularx}
    \caption{Descripción del CU: Configurar rutina.}
\end{table}