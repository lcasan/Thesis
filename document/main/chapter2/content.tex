\begin{thesischapter}{2} {Análisis, diseño e implementación del Juego Serio}
En este capítulo se discuten los detalles de desarrollo de los aspectos citados en el capítulo anterior. Este comienza con una descripción y caracterización general del sistema, donde se  abordan cada uno de los componentes requeridos para su completo funcionamiento. Posteriormente se detalla la ingienría de software requerida en la etapa de conceptualización de la aplicación, se explican de forma detallada los aspectos teóricos y de implementación de la base de datos, el funcionamiento del protocolo de comunicación y por último los escenarios de juegos requeridos en las rutinas de entrenamiento ligero y clínico, y las estadísticas generadas por estos. Como herramienta de desarrollo se utilizó c\#.

% SYSTEM DESCRIPTION AND CHARACTERIZATION TO APPLY
\subthesischapter{Implementación de la interfaz gráfica para la representación de datos EMG}
En el contexto del desarrollo de la interfaz de calibración clínica para la rehabilitación neuromuscular de pacientes con enfermedades cerebrovasculares fue necesario la implementación de la clase \underline{CoordinateSystem}. Esta clase, diseñada para extender \textbf{VisualElement}\footnote{Es un elemento visual de Unity3D} en Unity, representa el núcleo de la representación gráfica de los datos EMG en la aplicación móvil. Uno de los aspectos clave de esta implementación radica en su capacidad para generar un sistema de coordenadas dinámico y animado que se ajusta a las variaciones en los datos recibidos.

En el constructor de la clase \underline{CoordinateSystem}, se establece la estructura básica del sistema de coordenadas. Se crean y configuran elementos visuales esenciales, como el marco (frame), los ejes (axes), la ventana de visualización (window), y las etiquetas de los ejes x e y (xAxisLabel y yAxisLabel). Para proporcionar contexto a los datos representados, se han desarrollado métodos para agregar etiquetas a los ejes x e y. Estas etiquetas, definidas como objetos Label, muestran los valores correspondientes a las coordenadas en el sistema. La inclusión de estas etiquetas es esencial para que los profesionales de la salud y los pacientes comprendan el significado de los datos EMG presentados, mejorando así la utilidad clínica de la interfaz. 
    
Estos elementos son fundamentales para la presentación adecuada de los datos EMG en el contexto clínico, pero además, se aplican estilos \textbf{USS}\footnote{Son archivos de texto inspirados en las hojas de estilo en cascada (CSS) de HTML} a estos, definidos en archivos específicos (SystemCoordinateUSS.uss), para asegurar una presentación visual coherente y atractiva, ver figura~\ref{fig: graph-emg}.

\begin{figure}[ht]
    \centering
    \begin{tikzpicture}   
        \begin{axis}[
            ytick distance=0.5,
        ]
            \addplot [mark=none] table [x=x, y=y, col sep=comma] {data/emg.csv};
        \end{axis}
    \end{tikzpicture}
    \caption{Gráfico para la representación de los datos EMG.}
    \label{fig: graph-emg}
\end{figure}

Adicional a lo anteriormente mencionado la clase proporciona métodos para trazar puntos y funciones en el sistema de coordenadas. A través de los métodos PlotPoint() y PlotFunction(), la aplicación puede representar los datos EMG de forma gráfica y dinámica. Además, se ha implementado una funcionalidad de animación que permite mostrar la progresión de los datos en el tiempo. El método DrawFunction() utiliza técnicas de interpolación para crear una animación fluida, mejorando así la comprensión de la obtención de los datos en la línea de tiempo.

     
% SERIOUS GAME REQUIREMENTS
\subthesischapter{Implementación de la interfaz gráfica para la representación de datos EMG}
En el contexto del desarrollo de la interfaz de calibración clínica para la rehabilitación neuromuscular de pacientes con enfermedades cerebrovasculares fue necesario la implementación de la clase \underline{CoordinateSystem}. Esta clase, diseñada para extender \textbf{VisualElement}\footnote{Es un elemento visual de Unity3D} en Unity, representa el núcleo de la representación gráfica de los datos EMG en la aplicación móvil. Uno de los aspectos clave de esta implementación radica en su capacidad para generar un sistema de coordenadas dinámico y animado que se ajusta a las variaciones en los datos recibidos.

En el constructor de la clase \underline{CoordinateSystem}, se establece la estructura básica del sistema de coordenadas. Se crean y configuran elementos visuales esenciales, como el marco (frame), los ejes (axes), la ventana de visualización (window), y las etiquetas de los ejes x e y (xAxisLabel y yAxisLabel). Para proporcionar contexto a los datos representados, se han desarrollado métodos para agregar etiquetas a los ejes x e y. Estas etiquetas, definidas como objetos Label, muestran los valores correspondientes a las coordenadas en el sistema. La inclusión de estas etiquetas es esencial para que los profesionales de la salud y los pacientes comprendan el significado de los datos EMG presentados, mejorando así la utilidad clínica de la interfaz. 
    
Estos elementos son fundamentales para la presentación adecuada de los datos EMG en el contexto clínico, pero además, se aplican estilos \textbf{USS}\footnote{Son archivos de texto inspirados en las hojas de estilo en cascada (CSS) de HTML} a estos, definidos en archivos específicos (SystemCoordinateUSS.uss), para asegurar una presentación visual coherente y atractiva, ver figura~\ref{fig: graph-emg}.

\begin{figure}[ht]
    \centering
    \begin{tikzpicture}   
        \begin{axis}[
            ytick distance=0.5,
        ]
            \addplot [mark=none] table [x=x, y=y, col sep=comma] {data/emg.csv};
        \end{axis}
    \end{tikzpicture}
    \caption{Gráfico para la representación de los datos EMG.}
    \label{fig: graph-emg}
\end{figure}

Adicional a lo anteriormente mencionado la clase proporciona métodos para trazar puntos y funciones en el sistema de coordenadas. A través de los métodos PlotPoint() y PlotFunction(), la aplicación puede representar los datos EMG de forma gráfica y dinámica. Además, se ha implementado una funcionalidad de animación que permite mostrar la progresión de los datos en el tiempo. El método DrawFunction() utiliza técnicas de interpolación para crear una animación fluida, mejorando así la comprensión de la obtención de los datos en la línea de tiempo.
    

% USE CASE DEFINITION
\subthesischapter{Implementación de la interfaz gráfica para la representación de datos EMG}
En el contexto del desarrollo de la interfaz de calibración clínica para la rehabilitación neuromuscular de pacientes con enfermedades cerebrovasculares fue necesario la implementación de la clase \underline{CoordinateSystem}. Esta clase, diseñada para extender \textbf{VisualElement}\footnote{Es un elemento visual de Unity3D} en Unity, representa el núcleo de la representación gráfica de los datos EMG en la aplicación móvil. Uno de los aspectos clave de esta implementación radica en su capacidad para generar un sistema de coordenadas dinámico y animado que se ajusta a las variaciones en los datos recibidos.

En el constructor de la clase \underline{CoordinateSystem}, se establece la estructura básica del sistema de coordenadas. Se crean y configuran elementos visuales esenciales, como el marco (frame), los ejes (axes), la ventana de visualización (window), y las etiquetas de los ejes x e y (xAxisLabel y yAxisLabel). Para proporcionar contexto a los datos representados, se han desarrollado métodos para agregar etiquetas a los ejes x e y. Estas etiquetas, definidas como objetos Label, muestran los valores correspondientes a las coordenadas en el sistema. La inclusión de estas etiquetas es esencial para que los profesionales de la salud y los pacientes comprendan el significado de los datos EMG presentados, mejorando así la utilidad clínica de la interfaz. 
    
Estos elementos son fundamentales para la presentación adecuada de los datos EMG en el contexto clínico, pero además, se aplican estilos \textbf{USS}\footnote{Son archivos de texto inspirados en las hojas de estilo en cascada (CSS) de HTML} a estos, definidos en archivos específicos (SystemCoordinateUSS.uss), para asegurar una presentación visual coherente y atractiva, ver figura~\ref{fig: graph-emg}.

\begin{figure}[ht]
    \centering
    \begin{tikzpicture}   
        \begin{axis}[
            ytick distance=0.5,
        ]
            \addplot [mark=none] table [x=x, y=y, col sep=comma] {data/emg.csv};
        \end{axis}
    \end{tikzpicture}
    \caption{Gráfico para la representación de los datos EMG.}
    \label{fig: graph-emg}
\end{figure}

Adicional a lo anteriormente mencionado la clase proporciona métodos para trazar puntos y funciones en el sistema de coordenadas. A través de los métodos PlotPoint() y PlotFunction(), la aplicación puede representar los datos EMG de forma gráfica y dinámica. Además, se ha implementado una funcionalidad de animación que permite mostrar la progresión de los datos en el tiempo. El método DrawFunction() utiliza técnicas de interpolación para crear una animación fluida, mejorando así la comprensión de la obtención de los datos en la línea de tiempo.


% USE CASE REALIZATION
\subthesischapter{Implementación de la interfaz gráfica para la representación de datos EMG}
En el contexto del desarrollo de la interfaz de calibración clínica para la rehabilitación neuromuscular de pacientes con enfermedades cerebrovasculares fue necesario la implementación de la clase \underline{CoordinateSystem}. Esta clase, diseñada para extender \textbf{VisualElement}\footnote{Es un elemento visual de Unity3D} en Unity, representa el núcleo de la representación gráfica de los datos EMG en la aplicación móvil. Uno de los aspectos clave de esta implementación radica en su capacidad para generar un sistema de coordenadas dinámico y animado que se ajusta a las variaciones en los datos recibidos.

En el constructor de la clase \underline{CoordinateSystem}, se establece la estructura básica del sistema de coordenadas. Se crean y configuran elementos visuales esenciales, como el marco (frame), los ejes (axes), la ventana de visualización (window), y las etiquetas de los ejes x e y (xAxisLabel y yAxisLabel). Para proporcionar contexto a los datos representados, se han desarrollado métodos para agregar etiquetas a los ejes x e y. Estas etiquetas, definidas como objetos Label, muestran los valores correspondientes a las coordenadas en el sistema. La inclusión de estas etiquetas es esencial para que los profesionales de la salud y los pacientes comprendan el significado de los datos EMG presentados, mejorando así la utilidad clínica de la interfaz. 
    
Estos elementos son fundamentales para la presentación adecuada de los datos EMG en el contexto clínico, pero además, se aplican estilos \textbf{USS}\footnote{Son archivos de texto inspirados en las hojas de estilo en cascada (CSS) de HTML} a estos, definidos en archivos específicos (SystemCoordinateUSS.uss), para asegurar una presentación visual coherente y atractiva, ver figura~\ref{fig: graph-emg}.

\begin{figure}[ht]
    \centering
    \begin{tikzpicture}   
        \begin{axis}[
            ytick distance=0.5,
        ]
            \addplot [mark=none] table [x=x, y=y, col sep=comma] {data/emg.csv};
        \end{axis}
    \end{tikzpicture}
    \caption{Gráfico para la representación de los datos EMG.}
    \label{fig: graph-emg}
\end{figure}

Adicional a lo anteriormente mencionado la clase proporciona métodos para trazar puntos y funciones en el sistema de coordenadas. A través de los métodos PlotPoint() y PlotFunction(), la aplicación puede representar los datos EMG de forma gráfica y dinámica. Además, se ha implementado una funcionalidad de animación que permite mostrar la progresión de los datos en el tiempo. El método DrawFunction() utiliza técnicas de interpolación para crear una animación fluida, mejorando así la comprensión de la obtención de los datos en la línea de tiempo.


% DATABASE DESIGN 
\subthesischapter{Implementación de la interfaz gráfica para la representación de datos EMG}
En el contexto del desarrollo de la interfaz de calibración clínica para la rehabilitación neuromuscular de pacientes con enfermedades cerebrovasculares fue necesario la implementación de la clase \underline{CoordinateSystem}. Esta clase, diseñada para extender \textbf{VisualElement}\footnote{Es un elemento visual de Unity3D} en Unity, representa el núcleo de la representación gráfica de los datos EMG en la aplicación móvil. Uno de los aspectos clave de esta implementación radica en su capacidad para generar un sistema de coordenadas dinámico y animado que se ajusta a las variaciones en los datos recibidos.

En el constructor de la clase \underline{CoordinateSystem}, se establece la estructura básica del sistema de coordenadas. Se crean y configuran elementos visuales esenciales, como el marco (frame), los ejes (axes), la ventana de visualización (window), y las etiquetas de los ejes x e y (xAxisLabel y yAxisLabel). Para proporcionar contexto a los datos representados, se han desarrollado métodos para agregar etiquetas a los ejes x e y. Estas etiquetas, definidas como objetos Label, muestran los valores correspondientes a las coordenadas en el sistema. La inclusión de estas etiquetas es esencial para que los profesionales de la salud y los pacientes comprendan el significado de los datos EMG presentados, mejorando así la utilidad clínica de la interfaz. 
    
Estos elementos son fundamentales para la presentación adecuada de los datos EMG en el contexto clínico, pero además, se aplican estilos \textbf{USS}\footnote{Son archivos de texto inspirados en las hojas de estilo en cascada (CSS) de HTML} a estos, definidos en archivos específicos (SystemCoordinateUSS.uss), para asegurar una presentación visual coherente y atractiva, ver figura~\ref{fig: graph-emg}.

\begin{figure}[ht]
    \centering
    \begin{tikzpicture}   
        \begin{axis}[
            ytick distance=0.5,
        ]
            \addplot [mark=none] table [x=x, y=y, col sep=comma] {data/emg.csv};
        \end{axis}
    \end{tikzpicture}
    \caption{Gráfico para la representación de los datos EMG.}
    \label{fig: graph-emg}
\end{figure}

Adicional a lo anteriormente mencionado la clase proporciona métodos para trazar puntos y funciones en el sistema de coordenadas. A través de los métodos PlotPoint() y PlotFunction(), la aplicación puede representar los datos EMG de forma gráfica y dinámica. Además, se ha implementado una funcionalidad de animación que permite mostrar la progresión de los datos en el tiempo. El método DrawFunction() utiliza técnicas de interpolación para crear una animación fluida, mejorando así la comprensión de la obtención de los datos en la línea de tiempo.


% DATA MANIPULATION
\subthesischapter{Implementación de la interfaz gráfica para la representación de datos EMG}
En el contexto del desarrollo de la interfaz de calibración clínica para la rehabilitación neuromuscular de pacientes con enfermedades cerebrovasculares fue necesario la implementación de la clase \underline{CoordinateSystem}. Esta clase, diseñada para extender \textbf{VisualElement}\footnote{Es un elemento visual de Unity3D} en Unity, representa el núcleo de la representación gráfica de los datos EMG en la aplicación móvil. Uno de los aspectos clave de esta implementación radica en su capacidad para generar un sistema de coordenadas dinámico y animado que se ajusta a las variaciones en los datos recibidos.

En el constructor de la clase \underline{CoordinateSystem}, se establece la estructura básica del sistema de coordenadas. Se crean y configuran elementos visuales esenciales, como el marco (frame), los ejes (axes), la ventana de visualización (window), y las etiquetas de los ejes x e y (xAxisLabel y yAxisLabel). Para proporcionar contexto a los datos representados, se han desarrollado métodos para agregar etiquetas a los ejes x e y. Estas etiquetas, definidas como objetos Label, muestran los valores correspondientes a las coordenadas en el sistema. La inclusión de estas etiquetas es esencial para que los profesionales de la salud y los pacientes comprendan el significado de los datos EMG presentados, mejorando así la utilidad clínica de la interfaz. 
    
Estos elementos son fundamentales para la presentación adecuada de los datos EMG en el contexto clínico, pero además, se aplican estilos \textbf{USS}\footnote{Son archivos de texto inspirados en las hojas de estilo en cascada (CSS) de HTML} a estos, definidos en archivos específicos (SystemCoordinateUSS.uss), para asegurar una presentación visual coherente y atractiva, ver figura~\ref{fig: graph-emg}.

\begin{figure}[ht]
    \centering
    \begin{tikzpicture}   
        \begin{axis}[
            ytick distance=0.5,
        ]
            \addplot [mark=none] table [x=x, y=y, col sep=comma] {data/emg.csv};
        \end{axis}
    \end{tikzpicture}
    \caption{Gráfico para la representación de los datos EMG.}
    \label{fig: graph-emg}
\end{figure}

Adicional a lo anteriormente mencionado la clase proporciona métodos para trazar puntos y funciones en el sistema de coordenadas. A través de los métodos PlotPoint() y PlotFunction(), la aplicación puede representar los datos EMG de forma gráfica y dinámica. Además, se ha implementado una funcionalidad de animación que permite mostrar la progresión de los datos en el tiempo. El método DrawFunction() utiliza técnicas de interpolación para crear una animación fluida, mejorando así la comprensión de la obtención de los datos en la línea de tiempo.


% COMMUNICATION 
\subthesischapter{Implementación de la interfaz gráfica para la representación de datos EMG}
En el contexto del desarrollo de la interfaz de calibración clínica para la rehabilitación neuromuscular de pacientes con enfermedades cerebrovasculares fue necesario la implementación de la clase \underline{CoordinateSystem}. Esta clase, diseñada para extender \textbf{VisualElement}\footnote{Es un elemento visual de Unity3D} en Unity, representa el núcleo de la representación gráfica de los datos EMG en la aplicación móvil. Uno de los aspectos clave de esta implementación radica en su capacidad para generar un sistema de coordenadas dinámico y animado que se ajusta a las variaciones en los datos recibidos.

En el constructor de la clase \underline{CoordinateSystem}, se establece la estructura básica del sistema de coordenadas. Se crean y configuran elementos visuales esenciales, como el marco (frame), los ejes (axes), la ventana de visualización (window), y las etiquetas de los ejes x e y (xAxisLabel y yAxisLabel). Para proporcionar contexto a los datos representados, se han desarrollado métodos para agregar etiquetas a los ejes x e y. Estas etiquetas, definidas como objetos Label, muestran los valores correspondientes a las coordenadas en el sistema. La inclusión de estas etiquetas es esencial para que los profesionales de la salud y los pacientes comprendan el significado de los datos EMG presentados, mejorando así la utilidad clínica de la interfaz. 
    
Estos elementos son fundamentales para la presentación adecuada de los datos EMG en el contexto clínico, pero además, se aplican estilos \textbf{USS}\footnote{Son archivos de texto inspirados en las hojas de estilo en cascada (CSS) de HTML} a estos, definidos en archivos específicos (SystemCoordinateUSS.uss), para asegurar una presentación visual coherente y atractiva, ver figura~\ref{fig: graph-emg}.

\begin{figure}[ht]
    \centering
    \begin{tikzpicture}   
        \begin{axis}[
            ytick distance=0.5,
        ]
            \addplot [mark=none] table [x=x, y=y, col sep=comma] {data/emg.csv};
        \end{axis}
    \end{tikzpicture}
    \caption{Gráfico para la representación de los datos EMG.}
    \label{fig: graph-emg}
\end{figure}

Adicional a lo anteriormente mencionado la clase proporciona métodos para trazar puntos y funciones en el sistema de coordenadas. A través de los métodos PlotPoint() y PlotFunction(), la aplicación puede representar los datos EMG de forma gráfica y dinámica. Además, se ha implementado una funcionalidad de animación que permite mostrar la progresión de los datos en el tiempo. El método DrawFunction() utiliza técnicas de interpolación para crear una animación fluida, mejorando así la comprensión de la obtención de los datos en la línea de tiempo.


% TRAINING SCENARIOS
\subthesischapter{Implementación de la interfaz gráfica para la representación de datos EMG}
En el contexto del desarrollo de la interfaz de calibración clínica para la rehabilitación neuromuscular de pacientes con enfermedades cerebrovasculares fue necesario la implementación de la clase \underline{CoordinateSystem}. Esta clase, diseñada para extender \textbf{VisualElement}\footnote{Es un elemento visual de Unity3D} en Unity, representa el núcleo de la representación gráfica de los datos EMG en la aplicación móvil. Uno de los aspectos clave de esta implementación radica en su capacidad para generar un sistema de coordenadas dinámico y animado que se ajusta a las variaciones en los datos recibidos.

En el constructor de la clase \underline{CoordinateSystem}, se establece la estructura básica del sistema de coordenadas. Se crean y configuran elementos visuales esenciales, como el marco (frame), los ejes (axes), la ventana de visualización (window), y las etiquetas de los ejes x e y (xAxisLabel y yAxisLabel). Para proporcionar contexto a los datos representados, se han desarrollado métodos para agregar etiquetas a los ejes x e y. Estas etiquetas, definidas como objetos Label, muestran los valores correspondientes a las coordenadas en el sistema. La inclusión de estas etiquetas es esencial para que los profesionales de la salud y los pacientes comprendan el significado de los datos EMG presentados, mejorando así la utilidad clínica de la interfaz. 
    
Estos elementos son fundamentales para la presentación adecuada de los datos EMG en el contexto clínico, pero además, se aplican estilos \textbf{USS}\footnote{Son archivos de texto inspirados en las hojas de estilo en cascada (CSS) de HTML} a estos, definidos en archivos específicos (SystemCoordinateUSS.uss), para asegurar una presentación visual coherente y atractiva, ver figura~\ref{fig: graph-emg}.

\begin{figure}[ht]
    \centering
    \begin{tikzpicture}   
        \begin{axis}[
            ytick distance=0.5,
        ]
            \addplot [mark=none] table [x=x, y=y, col sep=comma] {data/emg.csv};
        \end{axis}
    \end{tikzpicture}
    \caption{Gráfico para la representación de los datos EMG.}
    \label{fig: graph-emg}
\end{figure}

Adicional a lo anteriormente mencionado la clase proporciona métodos para trazar puntos y funciones en el sistema de coordenadas. A través de los métodos PlotPoint() y PlotFunction(), la aplicación puede representar los datos EMG de forma gráfica y dinámica. Además, se ha implementado una funcionalidad de animación que permite mostrar la progresión de los datos en el tiempo. El método DrawFunction() utiliza técnicas de interpolación para crear una animación fluida, mejorando así la comprensión de la obtención de los datos en la línea de tiempo.


% EMG GRAPHIC
\subthesischapter{Implementación de la interfaz gráfica para la representación de datos EMG}
En el contexto del desarrollo de la interfaz de calibración clínica para la rehabilitación neuromuscular de pacientes con enfermedades cerebrovasculares fue necesario la implementación de la clase \underline{CoordinateSystem}. Esta clase, diseñada para extender \textbf{VisualElement}\footnote{Es un elemento visual de Unity3D} en Unity, representa el núcleo de la representación gráfica de los datos EMG en la aplicación móvil. Uno de los aspectos clave de esta implementación radica en su capacidad para generar un sistema de coordenadas dinámico y animado que se ajusta a las variaciones en los datos recibidos.

En el constructor de la clase \underline{CoordinateSystem}, se establece la estructura básica del sistema de coordenadas. Se crean y configuran elementos visuales esenciales, como el marco (frame), los ejes (axes), la ventana de visualización (window), y las etiquetas de los ejes x e y (xAxisLabel y yAxisLabel). Para proporcionar contexto a los datos representados, se han desarrollado métodos para agregar etiquetas a los ejes x e y. Estas etiquetas, definidas como objetos Label, muestran los valores correspondientes a las coordenadas en el sistema. La inclusión de estas etiquetas es esencial para que los profesionales de la salud y los pacientes comprendan el significado de los datos EMG presentados, mejorando así la utilidad clínica de la interfaz. 
    
Estos elementos son fundamentales para la presentación adecuada de los datos EMG en el contexto clínico, pero además, se aplican estilos \textbf{USS}\footnote{Son archivos de texto inspirados en las hojas de estilo en cascada (CSS) de HTML} a estos, definidos en archivos específicos (SystemCoordinateUSS.uss), para asegurar una presentación visual coherente y atractiva, ver figura~\ref{fig: graph-emg}.

\begin{figure}[ht]
    \centering
    \begin{tikzpicture}   
        \begin{axis}[
            ytick distance=0.5,
        ]
            \addplot [mark=none] table [x=x, y=y, col sep=comma] {data/emg.csv};
        \end{axis}
    \end{tikzpicture}
    \caption{Gráfico para la representación de los datos EMG.}
    \label{fig: graph-emg}
\end{figure}

Adicional a lo anteriormente mencionado la clase proporciona métodos para trazar puntos y funciones en el sistema de coordenadas. A través de los métodos PlotPoint() y PlotFunction(), la aplicación puede representar los datos EMG de forma gráfica y dinámica. Además, se ha implementado una funcionalidad de animación que permite mostrar la progresión de los datos en el tiempo. El método DrawFunction() utiliza técnicas de interpolación para crear una animación fluida, mejorando así la comprensión de la obtención de los datos en la línea de tiempo.


% STATICAL REPORTS GENERATION
\subthesischapter{Implementación de la interfaz gráfica para la representación de datos EMG}
En el contexto del desarrollo de la interfaz de calibración clínica para la rehabilitación neuromuscular de pacientes con enfermedades cerebrovasculares fue necesario la implementación de la clase \underline{CoordinateSystem}. Esta clase, diseñada para extender \textbf{VisualElement}\footnote{Es un elemento visual de Unity3D} en Unity, representa el núcleo de la representación gráfica de los datos EMG en la aplicación móvil. Uno de los aspectos clave de esta implementación radica en su capacidad para generar un sistema de coordenadas dinámico y animado que se ajusta a las variaciones en los datos recibidos.

En el constructor de la clase \underline{CoordinateSystem}, se establece la estructura básica del sistema de coordenadas. Se crean y configuran elementos visuales esenciales, como el marco (frame), los ejes (axes), la ventana de visualización (window), y las etiquetas de los ejes x e y (xAxisLabel y yAxisLabel). Para proporcionar contexto a los datos representados, se han desarrollado métodos para agregar etiquetas a los ejes x e y. Estas etiquetas, definidas como objetos Label, muestran los valores correspondientes a las coordenadas en el sistema. La inclusión de estas etiquetas es esencial para que los profesionales de la salud y los pacientes comprendan el significado de los datos EMG presentados, mejorando así la utilidad clínica de la interfaz. 
    
Estos elementos son fundamentales para la presentación adecuada de los datos EMG en el contexto clínico, pero además, se aplican estilos \textbf{USS}\footnote{Son archivos de texto inspirados en las hojas de estilo en cascada (CSS) de HTML} a estos, definidos en archivos específicos (SystemCoordinateUSS.uss), para asegurar una presentación visual coherente y atractiva, ver figura~\ref{fig: graph-emg}.

\begin{figure}[ht]
    \centering
    \begin{tikzpicture}   
        \begin{axis}[
            ytick distance=0.5,
        ]
            \addplot [mark=none] table [x=x, y=y, col sep=comma] {data/emg.csv};
        \end{axis}
    \end{tikzpicture}
    \caption{Gráfico para la representación de los datos EMG.}
    \label{fig: graph-emg}
\end{figure}

Adicional a lo anteriormente mencionado la clase proporciona métodos para trazar puntos y funciones en el sistema de coordenadas. A través de los métodos PlotPoint() y PlotFunction(), la aplicación puede representar los datos EMG de forma gráfica y dinámica. Además, se ha implementado una funcionalidad de animación que permite mostrar la progresión de los datos en el tiempo. El método DrawFunction() utiliza técnicas de interpolación para crear una animación fluida, mejorando así la comprensión de la obtención de los datos en la línea de tiempo.


\subthesischapter{Conclusiones del capítulo}
Se presentó una descripción del sistema de adquisición de datos para rehabilitación, sus componentes, características distintivas y su funcionamiento. Se identificaron y definieron los requisitos del juego  serio, tanto funcionales como no funcionales, así como los actores y casos de usos del sistema que establecieron las bases fundamentales para el desarrollo de la aplicación. Se realizó el diseño de la base de datos, abarcando tanto el modelo lógico como el físico, lo que aseguró una estructura robusta y eficiente para el almacenamiento de los datos. La manipulación de los datos se abordó de manera integral, desde la conexión con la base de datos hasta la persistencia de los resultados estadísticos. Se diseñó e implementó la comunicación con el pedal motorizado y la implementación de la interfaz gráfica para la representación de los datos EMG. Se definieron los escenarios de entrenamiento para las modalidades Ligero y Clínico, asegurando una cobertura completa de las necesidad de entrenamiento del usuario. Por último en el ámbito estadístico se desarrolló una serie de gráficos para el seguimiento de los resultados en las rutinas de entrenamiento.   
    
\end{thesischapter}