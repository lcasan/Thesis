\begin{abstract}
There is a close relationship between health sciences and new information technologies. A clear example is the application of video games in the clinical setting, which have emerged as new treatment approaches in stroke rehabilitation. 

The main purpose of this study is the development of a serious game application integrated to a neuromuscular rehabilitation system equipped with a motorized pedal, specifically designed for patients affected by cerebrovascular diseases. This approach aims to improve treatment adherence by providing immediate feedback on performance, stimulating motor and cognitive skills, customizing treatment to individual needs, monitoring patient progress, and mitigating stress and anxiety associated with rehabilitation.

Implementation of this approach required an initial phase of analysis of the state of the art in clinical rehabilitation and serious play technologies. This analysis served as the foundation for the definition of the functional and non-functional requirements essential for the development of the system, also taking into account the functionalities provided by the motorized pedal. In collaboration with experts in the field, game routines aimed at enhancing key physical capabilities such as strength, endurance and speed were designed, selecting Free-type routines for the initial version of the application. The integration of the data collected by the pedal with the application was achieved through the development of a network client-server model and simultaneously, a database was designed to monitor the development of the rehabilitation process and control patient data.

In terms of evaluation, the testing strategy employed for both functional and non-functional requirements of the software and a social-economic analysis demonstrating the benefits that the application of serious gaming integrated to a neuromuscular rehabilitation system with motorized pedal brings to patients affected by cerebrovascular diseases are presented.

\textbf{Key words}: serious games, stroke, rehabilitation system, rehabilitation technologies.
\end{abstract}