\begin{resumen}
Existe una estrecha relación entre las ciencias para la salud y las nuevas tecnologías de la información. Un claro ejemplo es la aplicación de los videojuegos en el entorno clínico, los cuales han surgido como nuevos enfoques de tratamiento en la rehabilitación de accidentes cerebrovasculares. 

El propósito principal de este estudio es el desarrollo de una aplicación de juego serio integrada a un sistema de rehabilitación neuromuscular equipado con un pedal motorizado, específicamente diseñada para pacientes afectados por enfermedades cerebrovasculares. Este enfoque pretende mejorar la adherencia al tratamiento mediante la provisión de retroalimentación inmediata sobre el rendimiento, la estimulación de habilidades motoras y cognitivas, la personalización del tratamiento según las necesidades individuales, el monitoreo del progreso del paciente, y la mitigación del estrés y la ansiedad asociados con la rehabilitación.

La implementación de este enfoque requirió una fase inicial de análisis del estado del arte en las tecnologías de rehabilitación clínica y de juego serio. Este análisis sirvió como fundamento para la definición de los requisitos funcionales y no funcionales esenciales para el desarrollo del sistema, teniendo en cuenta además las funcionalidades proporcionadas por el pedal motorizado. En colaboración con expertos en el campo, se diseñaron rutinas de juegos destinadas a potenciar capacidades físicas clave como fuerza, resistencia y velocidad, seleccionando las rutinas de tipo Libre para la versión inicial de la aplicación. La integración de los datos recopilados por el pedal con la aplicación se logró mediante el desarrollo de un modelo cliente-servidor de red y simultáneamente, se diseñó una base de datos para supervisar el desarrollo del proceso de rehabilitación y controlar los datos del paciente.

En términos de evaluación, se exponen la estrategia de pruebas empleadas tanto para los requisitos funcionales como los no funcionales del software y un análisis económico-social que demuestra los beneficios que la aplicación de juego serio integrada a un sistema de rehabilitación neuromuscular con pedal motorizado aporta a los pacientes afectados por enfermedades cerebrovasculares.

\textbf{Palabras claves}: juegos serio, accidentes cerebrovasculares, sistema de rehabilitación, tecnologías de rehabilitación.
\end{resumen}
