\begin{conclusions}
En este trabajo se desarrolló una aplicación de juego serio integrada a un sistema de rehabilitación neuromuscular con pedal motorizado para pacientes con enfermedades cerebrovasculares. Para ello:
\begin{itemize}
    \item Se elaboró el marco teórico relacionado con las tecnologías de rehabilitación clínica y de juego serio.
    \item Se realizó el análisis, diseño e implementación del sistema.
    \item Se diseñó e implementó  un modelo cliente-servidor de red que permita la
    recepción de las señales capturadas por los sensores del pedal motorizado
    y la supervisión a un cliente remoto ejecutado sobre plataforma Android.
    \item Se diseñó en conjunto con expertos, las rutinas de juegos que
    puedan desarrollar las capacidades físicas: fuerza, resistencia, velocidad, y que
    permitan la supervisión del desarrollo de la rehabilitación.
    \item Se diseñó una base de datos para gestionar la información relacionada con
    el paciente, las rutinas configuradas, y el proceso de rehabilitación.
\end{itemize}
Todos estos resultados fueron validados mediante pruebas al sistemas y encuestas a especialistas.
\end{conclusions}